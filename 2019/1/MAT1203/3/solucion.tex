\documentclass[12pt]{article}

\usepackage{fullpage}
\usepackage{graphicx}
\usepackage{amssymb}
\usepackage{amsmath}
\usepackage[none]{hyphenat}
\usepackage{parskip}
\usepackage[spanish]{babel}
\usepackage[utf8]{inputenc}
\usepackage{hyperref}
\usepackage{fancyhdr}
\usepackage{tasks}
\usepackage{mdframed}
\usepackage{xcolor}
\usepackage{pgfplots}
\usepackage[makeroom]{cancel}
\usepackage{multicol}
\usepackage[shortlabels]{enumitem}
\usepackage{stackrel}

\setlength{\headheight}{10pt}
\setlength{\headsep}{10pt}
\pagestyle{fancy}
\rhead{\ayudantia \ - \alumno}

\newcommand*{\mybox}[2]{\colorbox{#1!30}{\parbox{.98\linewidth}{#2}}}

\newenvironment{solucion}
{\begin{mdframed}[backgroundcolor=black!10]
		{\bf Solución:}\\
	}
	{
	\end{mdframed}
}

\newenvironment{alternativas}[1]
{\begin{multicols}{#1}
		\begin{enumerate}[a)]
		}
		{
		\end{enumerate}
	\end{multicols}
}

\newenvironment{preguntas}
{\begin{enumerate}\itemsep12pt
	}
	{
	\end{enumerate}
}

\newcommand{\ayudantia}{{\sc Ayudantía 3}}
\newcommand{\tituloayu}{Independencia lineal y transformaciones lineales}
\newcommand{\fecha}{28 de marzo de 2019}
\newcommand{\sigla}{MAT1203}
\newcommand{\nombre}{Álgebra Lineal}
\newcommand{\profesor}{Camilo Perez}
\newcommand{\ano}{2019}
\newcommand{\semestre}{1}
\newcommand{\mail}{mat1203@ifcastaneda.cl}
\newcommand{\alumno}{Ignacio Castañeda - \mail}

\newcommand{\ev}{\Big|}
\newcommand{\ra}{\rightarrow}
\newcommand{\lra}{\leftrightarrow}
\newcommand{\N}{\mathbb{N}}
\newcommand{\R}{\mathbb{R}}
\newcommand{\Exp}[1]{\mathcal{E}_{#1}}
\newcommand{\List}[1]{\mathcal{L}_{#1}}
\newcommand{\EN}{\Exp{\N}}
\newcommand{\LN}{\List{\N}}
\newcommand{\comment}[1]{}
\newcommand{\lb}{\\~\\}
\newcommand{\eop}{_{\square}}
\newcommand{\hsig}{\hat{\sigma}}
\newcommand{\widesim}[2][1.5]{
	\mathrel{\overset{#2}{\scalebox{#1}[1]{$\sim$}}}
}
\newcommand{\wsim}{\widesim{}}

\begin{document}
\thispagestyle{empty}

\begin{minipage}{2cm}
	\includegraphics[width=2cm]{../../../../img/logo.pdf}
	\vspace{0.5cm}
\end{minipage}
\begin{minipage}{\linewidth}
	\begin{tabular}{lrl}
		{\scriptsize\sc Pontificia Universidad Catolica de Chile} & \hspace*{0.7in}Curso: &
		\sigla  - \nombre\\
		{\sc Facultad de Matemáticas}&
		Profesor: & \profesor \\
		{\sc Semestre \ano-\semestre} & Ayudante: & {Ignacio Castañeda}\\
		& {Mail:} & \texttt{\mail}
	\end{tabular}
\end{minipage}

\vspace{-10mm}
\begin{center}
	{\LARGE\bf \ayudantia}\\
	\vspace{0.1cm}
	{\tituloayu}\\
	\vspace{0.1cm}
	\fecha\\
	\vspace{0.4cm}
\end{center}

\begin{preguntas}
\item Sean $\{u, v, w\}$ un conjunto de vectores linealmente independientes. Demuestre que el conjunto $\{u+v, u+2w, v+3u+w\}$ es linealmente independiente.
\begin{solucion}
$$P.D. \quad \{y, v, w\}\quad L.I. \ra \{u+v, u+2w, v+3u+w\}\quad L.I.$$
		Para que $\{u+v, u+2w, v+3u+w\}$ sea L.I., tiene que cumplirse que el sistema
		$$\alpha(u+v) + \beta(u+2w) + \gamma(v+3u+w) = 0$$
		tenga solucion única $\begin{pmatrix}
		\alpha\\ \beta \\ \gamma
		\end{pmatrix} = \begin{pmatrix} 0\\0\\0\end{pmatrix}$\\
		Trabajemos entonces con el sistema
		$$\alpha(u+v) + \beta(u+2w) + \gamma(v+3u+w) = 0$$
		$$\alpha u+ \alpha v + \beta u+2\beta w + \gamma v+3\gamma u+\gamma w = 0$$
		$$(\alpha + \beta + 3 \gamma)u + (\alpha + \gamma)v + (2\beta + \gamma)w = 0$$
		Recordemos que ${u, v, w}$ es L.I, por lo que debe cumplirse que
		$$\begin{array}{rcl}
		\alpha + \beta + 3 \gamma & = & 0\\
		\alpha + \gamma & = & 0\\
		2\beta + \gamma & = & 0
		\end{array}$$
		Este sistema lo podemos expresar de forma matricial,
		$$\begin{bmatrix}
		1 & 1 & 3\\
		1 & 0 & 1 \\
		0 & 2 & 1
		\end{bmatrix} \stackrel{F.E.}{\sim} \begin{bmatrix}
		1 & 0 & 1\\
		0 & 1 & -2\\
		0 & 0 & 1
		\end{bmatrix}$$
		Cuya solución es
		$$\begin{array}{rcl}
		\alpha & = & 0\\
		\beta & = & 0\\
		\gamma & = & 0
		\end{array}$$
		$$q.e.d$$
	
\end{solucion}
\item Sea $L: P_2(\R) \ra \R^2$ una transformación lineal tal que
	$$L(1+x) = \begin{pmatrix}
	1\\
	1
	\end{pmatrix}, \quad L(1+x+x^2) = \begin{pmatrix}
	1\\
	-1
	\end{pmatrix}\ y \ L(1+2x) = \begin{pmatrix}
	1\\
	2
	\end{pmatrix}$$
	determine $L(a+bx+cx^2)$ para todo $a,b,c \in \R$.
\begin{solucion}
Recordemos la siguientes propiedades de las transformaciones lineales
		$$L(v_1) + L(v_2) = L(v_1 + v_2), \quad \alpha L(v) = L(\alpha v)$$
		Debemos buscar 
		$$L(1), \quad L(x), \quad L(x^2)$$
		Para esto, debemos jugar con la información que nos dan hasta encontrar cada uno de ellos.
		\begin{center}\rule{14.5cm}{0.1pt}\end{center}
		$$L(1+x+x^2) - L(1+x) = \begin{pmatrix}
		1\\-1
		\end{pmatrix} - \begin{pmatrix}
		1\\1
		\end{pmatrix}$$
		$$L(x^2) = \begin{pmatrix}
		0\\-2
		\end{pmatrix}$$
		
		\begin{center}\rule{14.5cm}{0.1pt}\end{center}
		$$L(1+2x) - L(1+x) = \begin{pmatrix}
		1\\2
		\end{pmatrix} - \begin{pmatrix}
		1\\1
		\end{pmatrix}$$
		$$L(x) = \begin{pmatrix}
		0\\1
		\end{pmatrix}$$
		
		\begin{center}\rule{14.5cm}{0.1pt}\end{center}
		$$L(1+x) - L(x) = \begin{pmatrix}
		1\\1
		\end{pmatrix} - \begin{pmatrix}
		0\\1
		\end{pmatrix}$$
		$$L(x) = \begin{pmatrix}
		1\\0
		\end{pmatrix}$$
		
		Luego, 
		$$L(a+bx+cx^2) = aL(1) + bL(x) + cL(x^2)$$
		$$L(a+bx+cx^2) = aL\begin{pmatrix}1\\0\end{pmatrix} + b\begin{pmatrix}0\\1\end{pmatrix} + c\begin{pmatrix}0\\-2\end{pmatrix}$$
		$$L(a+bx+cx^2) = \begin{pmatrix}a\\b-2c\end{pmatrix}$$
\end{solucion}
\item Sea $T$ una transformación lineal tal que
	$$T\begin{bmatrix}2 \\ 1\end{bmatrix} = \begin{bmatrix}1 \\ 0 \\ 1 \\ 0\end{bmatrix} 
	\quad y \quad
	T \begin{bmatrix}3 \\ 1\end{bmatrix}  =\begin{bmatrix}0 \\ 1 \\ 0 \\ 1\end{bmatrix}$$
	Determine la matriz que representa a $T$
\begin{solucion}
Estamos buscando la matriz $\begin{bmatrix}
		T\begin{bmatrix}
		1 \\ 0
		\end{bmatrix}
		&
		
		T\begin{bmatrix}
		0 \\ 1
		\end{bmatrix}
		\end{bmatrix}$.\\
		Entonces, lo que necesitamos es buscar $T\begin{bmatrix}
		1 \\ 0
		\end{bmatrix}$ y $T\begin{bmatrix}
			0 \\ 1
		\end{bmatrix}$ y para ello, debemos escribir cada uno como una combinación lineal de $T\begin{bmatrix}
		2 \\ 1
		\end{bmatrix}$ y $T\begin{bmatrix}
		3 \\ 1
		\end{bmatrix}$, es decir
		$$\begin{bmatrix}
		1 \\ 0
		\end{bmatrix} = \alpha \begin{bmatrix}
		2 \\ 1
		\end{bmatrix} +  \beta \begin{bmatrix}
		3 \\ 1
		\end{bmatrix} \quad y \quad
		\begin{bmatrix}
		0 \\ 1
		\end{bmatrix} = \gamma \begin{bmatrix}
		2 \\ 1
		\end{bmatrix} +  \delta \begin{bmatrix}
		3 \\ 1
		\end{bmatrix}$$
		O sea, debemos resolver los sistemas
		$$\begin{array}{rcl}
		2\alpha + 3\beta & = & 1\\
		\alpha + \beta & = & 0
		\end{array} \qquad \qquad \qquad
		\begin{array}{rcl}
		2\gamma + 3\delta & = & 0\\
		\gamma + \delta & = & 1
		\end{array}$$
		La solución de estos sistemas es
		$$\alpha = -1, \quad \beta = 1, \quad \gamma = 3, \quad \delta = -2$$
		Por lo que
		$$\begin{bmatrix}
		1 \\ 0
		\end{bmatrix} = \begin{bmatrix}
		3 \\ 1
		\end{bmatrix}- \begin{bmatrix}
		2 \\ 1
		\end{bmatrix}  \quad y \quad
		\begin{bmatrix}
		0 \\ 1
		\end{bmatrix} = 3 \begin{bmatrix}
		2 \\ 1
		\end{bmatrix} - 2 \begin{bmatrix}
		3 \\ 1
		\end{bmatrix}$$
		Luego,
		$$T\begin{bmatrix}
		1 \\ 0
		\end{bmatrix} = T\begin{bmatrix}
		3 \\ 1
		\end{bmatrix}- T\begin{bmatrix}
		2 \\ 1
		\end{bmatrix} = \begin{bmatrix}0 \\ 1 \\ 0 \\ 1\end{bmatrix} - \begin{bmatrix}1 \\0 \\ 1 \\ 0\end{bmatrix} = \begin{bmatrix}-1 \\ 1 \\ -1 \\ 1\end{bmatrix}$$
		y
		$$T\begin{bmatrix}
		0 \\ 1
		\end{bmatrix} = 3 T\begin{bmatrix}
		2 \\ 1
		\end{bmatrix} - 2 T\begin{bmatrix}
		3 \\ 1
		\end{bmatrix} = 3 \begin{bmatrix}1 \\ 0 \\ 1 \\ 0\end{bmatrix} - 2 \begin{bmatrix}0 \\ 1 \\ 0 \\ 1\end{bmatrix} = \begin{bmatrix}3 \\ -2 \\ 3 \\ -2\end{bmatrix}$$
		Finalmente, la matriz que representa $T$ es
		$$\begin{bmatrix}
		T\begin{bmatrix}
		1 \\ 0
		\end{bmatrix}
		&
		
		T\begin{bmatrix}
		0 \\ 1
		\end{bmatrix}
		\end{bmatrix} =
		\begin{bmatrix}
		-1 & 3 \\
		1 & -2 \\
		-1 & 3 \\
		1 & -2
		\end{bmatrix}$$
\end{solucion}
\item Sea la transformación lineal $T$, dada por la matriz
	$$T = \begin{bmatrix}
	1 & 2 & -1\\
	2 & 4 & -2
	\end{bmatrix}$$
	Determine la imagen del plano de ecuación $x_1 + x_3 = 1$ por T.
\begin{solucion}
En primer lugar, escribimos el plano en su forma paramétrica
		$$\begin{array}{rcl}
		x_1 & = &1-x_3\\
		x_2 & = &x_2\\
		x_3 & = &x_3
		\end{array}$$
		Vemos entonces que 3 puntos arbitrarios pertenecienctes al plano son
		$$P_1(1,0,0), \quad P_2(1,1,0), \quad P_3(0,0,1) $$
		Luego, dos vectores directores que definen al plano son
		$$\vec{d_1} = \vec{P_1P_2} = (0,1,0), \quad =\vec{d_2} = \vec{P_1P_3} = (-1,0,1)$$
		Con esto, podemos expresar el plano en su forma vectorial
		$$\vec{P}= \vec{P_1} + \alpha \vec{d_1}  + \beta \vec{d_2}$$
		Para encontrar la imagen del plano por $T$, basta con hacer
		$$T\vec{P}= T\vec{P_1} + \alpha T\vec{d_1}  + \beta T\vec{d_2}$$
		Como
		$$T\vec{P_1} = \begin{bmatrix}
		1 & 2 & -1\\
		2 & 4 & -2
		\end{bmatrix} \begin{bmatrix}
		1 \\ 0 \\ 0
		\end{bmatrix} = \begin{bmatrix}
		1 \\ 2
		\end{bmatrix}$$
		$$T\vec{d_1} = \begin{bmatrix}
		1 & 2 & -1\\
		2 & 4 & -2
		\end{bmatrix} \begin{bmatrix}
		0 \\ 1 \\ 0
		\end{bmatrix} = \begin{bmatrix}
		2 \\ 4
		\end{bmatrix}$$
		$$T\vec{d_2} = \begin{bmatrix}
		1 & 2 & -1\\
		2 & 4 & -2
		\end{bmatrix} \begin{bmatrix}
		-1 \\ 0 \\ 1
		\end{bmatrix} = \begin{bmatrix}
		-2 \\ -4
		\end{bmatrix}$$
		Tenemos que
		$$TP =  \begin{bmatrix}
		1 \\ 2
		\end{bmatrix} + \alpha  \begin{bmatrix}
		2 \\ 4
		\end{bmatrix} + 
		\beta  \begin{bmatrix}
		-2 \\ -4
		\end{bmatrix} =  \begin{bmatrix}
		1 + 2\alpha  - 2\beta \\
		2 + 4\alpha - 5\beta
		\end{bmatrix} = 
		(1+2\alpha - 2\beta)  \begin{bmatrix}
		1 \\ 2
		\end{bmatrix}$$
		En otras palabras, la imagen del plano por la traslación $T$ corresponde a los ponderados de $\begin{bmatrix}
			1 \\ 2
		\end{bmatrix}$, es decir
		$$T\vec{P} = Gen\left\{ \begin{bmatrix}
		1 \\ 2
		\end{bmatrix} \right\}$$
\end{solucion}
\end{preguntas}
\end{document}