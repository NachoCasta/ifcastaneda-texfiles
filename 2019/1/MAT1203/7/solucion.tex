\documentclass[12pt]{article}

\usepackage{fullpage}
\usepackage{graphicx}
\usepackage{amssymb}
\usepackage{amsmath}
\usepackage[none]{hyphenat}
\usepackage{parskip}
\usepackage[spanish]{babel}
\usepackage[utf8]{inputenc}
\usepackage{hyperref}
\usepackage{fancyhdr}
\usepackage{tasks}
\usepackage{mdframed}
\usepackage{xcolor}
\usepackage{pgfplots}
\usepackage[makeroom]{cancel}
\usepackage{multicol}
\usepackage[shortlabels]{enumitem}
\usepackage{stackrel}
\usepackage{tkz-tab}
\usepackage{xpatch}
\xpatchcmd{\tkzTabLine}{$0$}{$\bullet$}{}{}

\setlength{\headheight}{10pt}
\setlength{\headsep}{10pt}
\pagestyle{fancy}
\rhead{\ayudantia \ - \alumno}
\tikzset{t style/.style={style=solid}}

\newcommand*{\mybox}[2]{\colorbox{#1!30}{\parbox{.98\linewidth}{#2}}}

\newenvironment{solucion}
{\begin{mdframed}[backgroundcolor=black!10]
		{\bf Solución:}\\
	}
	{
	\end{mdframed}
}

\newenvironment{alternativas}[1]
{\begin{multicols}{#1}
		\begin{enumerate}[a)]
		}
		{
		\end{enumerate}
	\end{multicols}
}

\newenvironment{preguntas}
{\begin{enumerate}\itemsep12pt
	}
	{
	\end{enumerate}
}

\newcommand{\ayudantia}{{\sc Ayudantía 7}}
\newcommand{\tituloayu}{Determinantes y subespacios vectoriales}
\newcommand{\fecha}{25 de abril de 2019}
\newcommand{\sigla}{MAT1203}
\newcommand{\nombre}{Álgebra Lineal}
\newcommand{\profesor}{Camilo Perez}
\newcommand{\ano}{2019}
\newcommand{\semestre}{1}
\newcommand{\mail}{mat1203@ifcastaneda.cl}
\newcommand{\alumno}{Ignacio Castañeda - \mail}

\newcommand{\ev}{\Big|}
\newcommand{\ra}{\rightarrow}
\newcommand{\lra}{\leftrightarrow}
\newcommand{\N}{\mathbb{N}}
\newcommand{\R}{\mathbb{R}}
\newcommand{\Exp}[1]{\mathcal{E}_{#1}}
\newcommand{\List}[1]{\mathcal{L}_{#1}}
\newcommand{\EN}{\Exp{\N}}
\newcommand{\LN}{\List{\N}}
\newcommand{\comment}[1]{}
\newcommand{\lb}{\\~\\}
\newcommand{\eop}{_{\square}}
\newcommand{\hsig}{\hat{\sigma}}
\newcommand{\widesim}[2][1.5]{
	\mathrel{\overset{#2}{\scalebox{#1}[1]{$\sim$}}}
}
\newcommand{\wsim}{\widesim{}}
\newcommand{\lh}{\stackrel{L'H}{=}}

\begin{document}
\thispagestyle{empty}

\begin{minipage}{2cm}
	\includegraphics[width=2cm]{../../../../img/logo.pdf}
	\vspace{0.5cm}
\end{minipage}
\begin{minipage}{\linewidth}
	\begin{tabular}{lrl}
		{\scriptsize\sc Pontificia Universidad Catolica de Chile} & \hspace*{0.7in}Curso: &
		\sigla  - \nombre\\
		{\sc Facultad de Matemáticas}&
		Profesor: & \profesor \\
		{\sc Semestre \ano-\semestre} & Ayudante: & {Ignacio Castañeda}\\
		& {Mail:} & \texttt{\mail}
	\end{tabular}
\end{minipage}

\vspace{-10mm}
\begin{center}
	{\LARGE\bf \ayudantia}\\
	\vspace{0.1cm}
	{\tituloayu}\\
	\vspace{0.1cm}
	\fecha\\
	\vspace{0.4cm}
\end{center}

\begin{preguntas}
\item Sea $A$ una matriz de $4\times 4$, tal que $det(A) = \alpha \neq 0$
\begin{enumerate}[a)]
\item Calcule  $det(5A)+det(3A^{-1})$ en terminos de $\alpha$.
\item Se sabe que $det(Adj(A))=8$. Calcule $\alpha$.
\end{enumerate}
\begin{solucion}

\begin{enumerate}[a)]
\item Calcule  $det(5A)+det(3A^{-1})$ en terminos de $\alpha$.
			
			En primer lugar,
			$$det(5A) = 5^4det(a) = 5^4\alpha = 625 \alpha$$
			En segundo lugar,
			$$det(3A^{-1}) = 3^4det(A^{-1}) = 3^4 \dfrac{1}{\alpha} = \dfrac{81}{\alpha}$$
			Finalmente,
			$$det(5A)+det(3A^{-1}) = 625 \alpha + \dfrac{81}{\alpha}$$
\item Se sabe que $det(Adj(A))=8$. Calcule $\alpha$.
			
			Recordemos que
			$$A\ Adj(A) = det(A) I$$
			Apliquemos $det$ a ambos lados de la igualdad,
			$$det(A\ Adj(A)) = det(det(A) I)$$
			$$det(A)\ det(Adj(A)) = det(A)^4\ det(I)$$
			$$det(Adj(A)) = det(A)^3$$
			$$det(A) = \sqrt[3]{det(Adj(A))}$$
			$$det(A) = \sqrt[3]{8}$$
			$$det(A) = 2$$
\end{enumerate}
\end{solucion}
\item Sea $P$ el paralelepípedo con un vertice en el origen y vertices adyacentes en $(1,4,0)$, $(-2,-5,2)$, $(-1,2,-1)$
\begin{enumerate}[a)]
\item Determinar el volumen de $P$.
\item Se define $T: \R^3 \ra \R^3$ como la transformación lineal definida por $$T(x,y,z)=(x+y,y+zx-z)$$
		Calcule el volumen de $T(P)$.
\end{enumerate}
\begin{solucion}

\begin{enumerate}[a)]
\item Determinar el volumen de $P$.\\
			\\
			Los vectores de las aristas son 
			$$v_1 = \begin{pmatrix}
			1 \\ 4 \\ 0
			\end{pmatrix}, \quad
			v_2 = \begin{pmatrix}
			-2 \\ -5 \\ 2
			\end{pmatrix}, \quad
			v_3 = \begin{pmatrix}
			-1 \\ 2 \\ -1
			\end{pmatrix}$$
			Podemos representar a $P$ en forma matricial de la siguiente forma:
			$$A_P = \begin{bmatrix}
			1 & -2 & -1 \\
			4 & -5 & 2 \\
			0 & 2 & -1
			\end{bmatrix}$$
			Luego, el volumen de $P$ se calcula como $|det(A_P)|$, esto es
			$$\left| det \left(\begin{bmatrix}
			1 & -2 & -1 \\
			4 & -5 & 2 \\
			0 & 2 & -1
			\end{bmatrix}\right)\right| =| 0 -2(2+4) - 1(-5+8)| = |-12 - 3| =| -15| = 15$$
			Por lo que el volumen de $P$ es $15$.
\item Se define $T: \R^3 \ra \R^3$ como la transformación lineal definida por $$T(x,y,z)=(x+y,y+z,x-z)$$
			Calcule el volumen de $T(P)$.\\
			\\
			En primer lugar, encontremos la matriz de la transformación lineal
			$$T(x,y,z)=(x+y,y+z,x-z) = \begin{pmatrix}
			x+y \\ 
			y+z \\ 
			x-z
			\end{pmatrix} = \begin{bmatrix}
			1 & 1 & 0 \\
			0 & 1 & 1 \\
			1 & 0 & -1 
			\end{bmatrix} \begin{pmatrix}
			x \\ y \\ z
			\end{pmatrix}$$
			Por lo que
			$$A_T = \begin{bmatrix}
			1 & 1 & 0 \\
			0 & 1 & 1 \\
			1 & 0 & -1 
			\end{bmatrix}$$
			Luego, el volumen de $T(P)$ será $|det(A_P)| \cdot |det(A_T)|$
			$$|det(A_T)| = \left| det \left(\begin{bmatrix}
			1 & 1 & 0 \\
			0 & 1 & 1 \\
			1 & 0 & -1 
			\end{bmatrix}\right)\right| = |1(-1-0) - (0 - 1)| = |1-1| = 0$$
			Finalmente,
			$$V(T(P)) = |det(A_P)| \cdot |det(A_T)| = 15 \cdot 0 = 0$$
\end{enumerate}
\end{solucion}
\item Sea $V = \mathbb{P}_2$ el espacio vectorial de los polinomios con coeficientes reales de grado menor o igual a 2. Demuestre que el conjunto
	$$U=\{p(x) \in V : p(1) + p(0) = p(-1)\}$$
	Demuestre que $U$ es subespacio de $V$.
\begin{solucion}
Para demostrar que $U$ es un subespacio vectorial, debemos ver que sea no vacío, cerrado en la suma y cerrado en la multiplicación.
		
		No vacío:
		
		Sea el polinomio $p(x) = 0$, vemos que
		$$p(1) + p(0) =p(-1)$$
		$$0 = 0$$
		Por lo que $p(x) \in U$, asi que no es vacío.
		
		Suma:
		
		Sean $p, q \in U$,
		
		P.D.
		$$ \begin{array}{rcl}
		(p+q)(1) + (p+q)(0) &= &(p+q)(-1)\\
		p(1) + q(1)+ p(0)+q(0) &= &(p+q)(-1)\\
		p(1) + p(0)+ q(1) + q(0) &= &(p+q)(-1)\\
		p(-1)+ q(-1) &= &(p+q)(-1)\\
		(p+q)(-1) &= &(p+q)(-1)
		\end{array}$$
		Por lo que es cerrado en la suma
		
		Multiplicación:
		
		Sea $p \in U$ y $\alpha \in \R$,
		P.D.
		$$ \begin{array}{rcl}
		(\alpha p)(1) + (\alpha p)(0) = (\alpha p)(-1) \\
		\alpha p(1) + \alpha p(0) = (\alpha p)(-1) \\
		\alpha ( p(1) +  p(0)) = (\alpha p)(-1) \\
		\alpha p(-1) = (\alpha p)(-1)\\
		(\alpha  p)(-1) = (\alpha p)(-1)
		\end{array}$$
		 
		Por lo que es cerrado en la multiplicación
		
		Entonces, tenemos que $U$ es un subespacio de $V$.
		$$\blacksquare$$
\end{solucion}
\end{preguntas}
\end{document}