\documentclass[12pt]{article}

\usepackage{fullpage}
\usepackage{graphicx}
\usepackage{amssymb}
\usepackage{amsmath}
\usepackage[none]{hyphenat}
\usepackage{parskip}
\usepackage[spanish]{babel}
\usepackage[utf8]{inputenc}
\usepackage{hyperref}
\usepackage{fancyhdr}
\usepackage{tasks}
\usepackage{mdframed}
\usepackage{xcolor}
\usepackage{pgfplots}
\usepackage[makeroom]{cancel}
\usepackage{multicol}
\usepackage[shortlabels]{enumitem}
\usepackage{tabto}

\setlength{\headheight}{10pt}
\setlength{\headsep}{10pt}
\pagestyle{fancy}
\rhead{\ayudantia \ - \alumno}

\newcommand*{\mybox}[2]{\colorbox{#1!30}{\parbox{.98\linewidth}{#2}}}

\newenvironment{solucion}
{\begin{mdframed}[backgroundcolor=black!10]
		{\bf Solución:}\\
	}
	{
	\end{mdframed}
}

\newenvironment{alternativas}[1]
{\begin{multicols}{#1}
		\begin{enumerate}[a)]
		}
		{
		\end{enumerate}
	\end{multicols}
}

\newenvironment{preguntas}
{\begin{enumerate}\itemsep12pt
	}
	{
	\end{enumerate}
}

\newcommand{\ayudantia}{{\sc Ayudantía 2}}
\newcommand{\tituloayu}{Sistemas de ecuaciones lineales, formas escalonadas y conjunto solución}
\newcommand{\fecha}{21 de marzo de 2019}
\newcommand{\sigla}{MAT1203}
\newcommand{\nombre}{Álgebra Lineal}
\newcommand{\profesor}{Camilo Perez}
\newcommand{\ano}{2019}
\newcommand{\semestre}{1}
\newcommand{\mail}{mat1203@ifcastaneda.cl}
\newcommand{\alumno}{Ignacio Castañeda - \mail}

\newcommand{\ev}{\Big|}
\newcommand{\ra}{\rightarrow}
\newcommand{\lra}{\leftrightarrow}
\newcommand{\N}{\mathbb{N}}
\newcommand{\R}{\mathbb{R}}
\newcommand{\Exp}[1]{\mathcal{E}_{#1}}
\newcommand{\List}[1]{\mathcal{L}_{#1}}
\newcommand{\EN}{\Exp{\N}}
\newcommand{\LN}{\List{\N}}
\newcommand{\comment}[1]{}
\newcommand{\lb}{\\~\\}
\newcommand{\eop}{_{\square}}
\newcommand{\hsig}{\hat{\sigma}}

\begin{document}
\thispagestyle{empty}

\begin{minipage}{2cm}
	\includegraphics[width=2cm]{../../../../img/logo.pdf}
	\vspace{0.5cm}
\end{minipage}
\begin{minipage}{\linewidth}
	\begin{tabular}{lrl}
		{\scriptsize\sc Pontificia Universidad Catolica de Chile} & \hspace*{0.7in}Curso: &
		\sigla  - \nombre\\
		{\sc Facultad de Matemáticas}&
		Profesor: & \profesor \\
		{\sc Semestre \ano-\semestre} & Ayudante: & {Ignacio Castañeda}\\
		& {Mail:} & \texttt{\mail}
	\end{tabular}
\end{minipage}

\vspace{-10mm}
\begin{center}
	{\LARGE\bf \ayudantia}\\
	\vspace{0.1cm}
	{\tituloayu}\\
	\vspace{0.1cm}
	\fecha\\
	\vspace{0.4cm}
\end{center}

\begin{preguntas}
\item Escribe los siguientes sistemas de ecuaciones en notación matricial.
\begin{tasks}(2)
\task $
		\begin{array}{rcr}
		x_1 -x_2 + 3x_3 & = & 2\\
		3x_2 - 2x_3 & = & 0\\
		4x_1 +2x_2 & = & -1
		\end{array}
		$
\task $
		\begin{array}{rcr}
		3x_1 & = & 1\\
		x_1-x_2& = & 3\\
		-2x_1+x_2 & = & -8\\
		x_2 & = & 0
		\end{array}
		$
\task $
		\begin{array}{rcr}
		2x_2 -x_3 +2x_4& = & 5\\
		x_1+x_2 - x_3 & = & 2\\
		-4x_1 +x_3-x_4& = & -4
		\end{array}
		$
\task $
		\begin{array}{rcr}
		x_1 +x_2-3x_3& = & 0\\
		x_2+4x_4& = & -1\\
		x_1-2x_2+x_3-5x_4 & = & 2\\
		x_3-x_4 & = & 5
		\end{array}
		$
\end{tasks}
\begin{solucion}

\begin{enumerate}[a)]
\item $
			\begin{array}{rcr}
			x_1 -x_2 + 3x_3 & = & 2\\
			3x_2 - 2x_3 & = & 0\\
			4x_1 +2x_2 & = & -1
			\end{array}
			\Longrightarrow 
			\left[
			\begin{array}{ccc|c}
				1 & -1 & 3 & 2\\
				0 & 3 & -2 & 0\\
				4 & 2 & 0 & 1
			\end{array}
			\right]
			$
\item $
			\begin{array}{rcr}
			3x_1 & = & 1\\
			x_1-x_2& = & 3\\
			-2x_1+x_2 & = & -8\\
			x_2 & = & 0
			\end{array}
			\Longrightarrow 
			\left[
			\begin{array}{cc|c}
			3 & 0 & 1\\
			1 &-1 & 3\\
			-2 & 1 & -8\\
			0 & 1 & 0
			\end{array}
			\right]
			$
\item $
			\begin{array}{rcr}
			2x_2 -x_3 +2x_4& = & 5\\
			x_1+x_2 - x_3 & = & 2\\
			-4x_1 +x_3-x_4& = & -4
			\end{array}
			\Longrightarrow 
			\left[
			\begin{array}{cccc|c}
			0 &2 &-1&2&5\\
			1 &1 & -1 & 0 &2\\
			-4 &0 & 1 & -1 &-4
			\end{array}
			\right]
			$
\item $
			\begin{array}{rcr}
			x_1 +x_2-3x_3& = & 0\\
			x_2+4x_4& = & -1\\
			x_1-2x_2+x_3-5x_4 & = & 2\\
			x_3-x_4 & = & 5
			\end{array}
			\Longrightarrow 
			\left[
			\begin{array}{cccc|c}
			1 & 1 & -3 & 0 & 0\\
			0 & 1 & 0 & 4 & -1\\
			1 & -2 & 1 & -5 & 2\\
			0 & 0 & 1 & -1 & 5
			\end{array}
			\right]
			$
\end{enumerate}
\end{solucion}
\item Dadas las siguientes matrices aumentadas, escribir el sistema de ecuaciones asociado.
\begin{tasks}(2)
\task $
		\begin{bmatrix}
		-4 & 2 & 1 &  0\\
		2  & 1 & 2 & -1\\
		-3 & 5 &-6& -2
		\end{bmatrix}
		$
\task $
		\begin{bmatrix}
		2 & 1 & -3 & 0\\
		1  & -2 & 0 & 12\\
		4 & 2 &1& 8
		\end{bmatrix}
		$
\end{tasks}
\begin{solucion}

\begin{enumerate}[a)]
\item $
			\begin{bmatrix}
			-4 & 2 & 1 &  0\\
			2  & 1 & 2 & -1\\
			-3 & 5 &-6& -2
			\end{bmatrix}
			\Longrightarrow
			\begin{array}{rcr}
			-4x_1 + 2x_2 + x_3& = & 0\\
			2x_1 + x_2 + 2x_3& = & -1\\
			-3x_1 + 5x_2 - 6x_3 & = & -2
			\end{array}
			$
\item $
			\begin{bmatrix}
			2 & 1 & -3 & 0\\
			1  & -2 & 0 & 12\\
			4 & 2 &1& 8
			\end{bmatrix}
			\Longrightarrow
			\begin{array}{rcr}
			2x_1 + x_2 - 3x_3& = & 0\\
			x_1-2x_2& = & 12\\
			4x_1+2x_2+x_3 & = & 8
			\end{array}
			$
\end{enumerate}
\end{solucion}
\item Determinar si el siguiente sistema es consistente o no
	$$
	\begin{array}{rcr}
	x_1 -6x_2& = & 5\\
	x_2-4x_3+x_4& = & 0\\
	-x_1+6x_2+x_3+5x_4& = & 3\\
	-x_2+5x_3+4x_4 & = & 0
	\end{array}
	$$
\begin{solucion}
En primer lugar, escribiremos el sistema en notación matricial
		$$
		\left[
		\begin{array}{cccc|c}
		1 & -6 & 0 & 0 & 5\\
		0 & 1 & -4 & 1 & 0\\
		-1& 6 & 1 & 5 & 3 \\
		0 & -1 & 5 & 4 &0
		\end{array}
		\right]$$
		A continuación, utilizando operaciones de fila, llevaremos esta matriz a su forma escalonada
		$$\left[
		\begin{array}{cccc|c}
			1 & -6 & 0 & 0 & 5\\
			0 & 1 & -4 & 1 & 0\\
			-1& 6 & 1 & 5 & 3 \\
			0 & -1 & 5 & 4 &0
		\end{array}
		\right] \stackbin[F_4+F_2]{F_3 +F_1}{\wsim}
		\left[
		\begin{array}{cccc|c}
		1 & -6 & 0 & 0 & 5\\
		0 & 1 & -4 & 1 & 0\\
		0 & 0 & 1 & 5 & 8 \\
		0 & 0 & 1 & 5 & 0
		\end{array}
		\right] \stackbin[]{F_4-F_3}{\wsim}
		\left[
		\begin{array}{cccc|c}
		1 & -6 & 0 & 0 & 5\\
		0 & 1 & -4 & 1 & 0\\
		0 & 0 & 1 & 5 & 8 \\
		0 & 0 & 0 & 0 & -8
		\end{array}
		\right] $$
		Aqui podemos apreciar que en el la última fila, todas las variables tienen coeficiente cero y la columna de coeficientes aumentados tiene valor, lo que no es posible, ya que esto significaría que se cumple la igualdad $0 = -8$, lo que no es cierto. Dicho esto, el sistema no es consistente.
\end{solucion}
\item Lleve las siguientes matrices a su forma escalonada reducida y determine la existencia y unicidad de las soluciones del sistema.
\begin{tasks}(2)
\task $
		\begin{bmatrix}
		1 & 2 & 4 & 8\\
		2 & 4 & 6 & 8\\
		3 & 6 & 9 & 12
		\end{bmatrix}
		$
\task $
		\begin{bmatrix}
		1 & 2 & 4 & 5\\
		2 & 4 & 5& 4\\
		4 & 5 & 4 & 2
		\end{bmatrix}
		$
\end{tasks}
\begin{solucion}

\begin{enumerate}[a)]
\item $
			\begin{bmatrix}
			1 & 2 & 4 & 8\\
			2 & 4 & 6 & 8\\
			3 & 6 & 9 & 12
			\end{bmatrix}
			\stackbin[F_3 - 3F_1]{F_2-2F_1}{\wsim}
			\begin{bmatrix}
			1 & 2 & 4 & 8\\
			0 & 0 & -2 & -8\\
			0 & 0 & -3 & -12
			\end{bmatrix}
			\stackbin[F_3 \leftarrow -\frac{1}{3}F_3]{F_2 \leftarrow -\frac{1}{2}F_2}{\wsim}
			\begin{bmatrix}
			1 & 2 & 4 & 8\\
			0 & 0 & 1 & 4\\
			0 & 0 & 1 & 4
			\end{bmatrix}$\\\\
			$
			\stackbin[]{F_3- F_2}{\wsim}
			\begin{bmatrix}
			1 & 2 & 4 & 8\\
			0 & 0 & 1 & 4\\
			0 & 0 & 0 & 0
			\end{bmatrix}
			\stackbin[]{F_1- 4F_2}{\wsim}
			\begin{bmatrix}
			1 & 2 & 0 & -8\\
			0 & 0 & 1 & 4\\
			0 & 0 & 0 & 0
			\end{bmatrix}
			$\\\\
			Como hay una variable libre (2 pivotes y 3 filas), existen infinitas soluciones.
\item $
			\begin{bmatrix}
			1 & 2 & 4 & 5\\
			2 & 4 & 5& 4\\
			4 & 5 & 4 & 2
			\end{bmatrix}
			\stackbin[F_3 - 4F_1]{F_2 - 2F_1}{\wsim}
			\begin{bmatrix}
			1 & 2 & 4 & 5\\
			0 & 0 & -3& -6\\
			0 & -3 & -12 & -18
			\end{bmatrix}
			\stackbin[F_3 \leftarrow -\frac{1}{3} F_2]{F_2 \leftarrow -\frac{1}{3}F_3}{\wsim}
			\begin{bmatrix}
			1 & 2 & 4 & 5\\
			0 & 1 & 4& 6\\
			0 & 0 & 1 & 2
			\end{bmatrix}
			$\\\\
			$
			\stackbin[F_2 - 4F_3]{F_1 - 4F_3}{\wsim}
			\begin{bmatrix}
			1 & 2 & 0 & -3\\
			0 & 1 & 0& -2\\
			0 & 0 & 1 & 2
			\end{bmatrix}
			\stackbin[]{F_1 - 2F_2}{\wsim}
			\begin{bmatrix}
			1 & 0 & 0 & 1\\
			0 & 1 & 0& -2\\
			0 & 0 & 1 & 2
			\end{bmatrix}$\\\\
			Como tiene 3 pivotes y 3 filas, el sistema posee solución única
\end{enumerate}
\end{solucion}
\item Considere el siguiente sistema de ecuaciones, donde $a$ es una constante.
  $$\begin{array}{llll}
   x_1&+x_2&+x_3&=1 \\
    x_1&+x_2&+ax_3&=1 \\
     ax_1&+ax_2&+x_3&=a\\
      x_1&-ax_2&+ax_3&=0  
  \end{array}$$
\begin{enumerate}[a)]
\item Determine valores de $a$ para los cuales el sistema es inconsistente.
\item Determine valores de $a$ para los cuales el sistema es consistente, y encuentre la solución.
\end{enumerate}
\begin{solucion}

\begin{enumerate}[a)]
\item 
\item 
\end{enumerate}
\end{solucion}
\item Sea la matriz $A=
	\begin{bmatrix}
	3 & 5 & -4\\
	-3 & -2 & 4\\
	6 & 1 & -8
	\end{bmatrix}$
\begin{enumerate}[a)]
\item Determinar el conjunto solución de su sistema homogeneo.
\item Describir todas las soluciones de $Ax=b$ con $b=
		\begin{pmatrix}
		7\\
		-1\\
		-4
		\end{pmatrix}$ 
\end{enumerate}
\begin{solucion}

\begin{enumerate}[a)]
\item Determinar el conjunto solución de su sistema homogeneo.\\\\
			El sistema homogeneo corresponde a la ecuación $Ax=\vec{0}$, por lo que nuestra matriz aumentada sería
			$$\left[
			\begin{array}{ccc|c}
			3 & 5 & -4 & 0\\
			-3 & -2 & 4 & 0\\
			6 & 1 & -8 & 0
			\end{array}
			\right] \sim \left[
			\begin{array}{ccc|c}
			3 & 5 & -4 & 0\\
			0 & 3 & 0 & 0\\
			0 & -9 & 0 & 0
			\end{array}
			\right] \sim \left[
			\begin{array}{ccc|c}
			3 & 5 & -4 & 0\\
			0 & 3 & 0 & 0\\
			0 & 0 & 0 & 0
			\end{array}
			\right]$$
			$$\sim \left[
			\begin{array}{ccc|c}
			1 & \frac{5}{3} & -\frac{4}{3} & 0\\
			0 & 1 & 0 & 0\\
			0 & 0 & 0 & 0
			\end{array}
			\right] \sim \left[
			\begin{array}{ccc|c}
			1 & 0 & -\frac{4}{3} & 0\\
			0 & 1 & 0 & 0\\
			0 & 0 & 0 & 0
			\end{array}
			\right]$$
			Esto corresponde al sistema
			$$\begin{array}{rcl}
			x_1 - \frac{4}{3}x_3 & = & 0\\
			x_2 & = & 0\\
			0 & = & 0
			\end{array} \ra \begin{array}{rcl}
			x_1 & = & \frac{4}{3}x_3\\
			x_2 & = & 0\\
			x_3 & = & x_3
			\end{array} \ra x = \begin{pmatrix}
			\frac{4}{3}x_3\\
			0\\
			x_3
			\end{pmatrix} = x_3\begin{pmatrix}
			\frac{4}{3}\\
			0\\
			1
			\end{pmatrix}$$
			Luego, la solución del sistema homogeneo es
			$$S = Gen\left\{\begin{pmatrix}
			\frac{4}{3}\\
			0\\
			1
			\end{pmatrix}\right\}$$
\item Describir todas las soluciones de $Ax=b$ con $b=
			\begin{pmatrix}
			7\\
			-1\\
			-4
			\end{pmatrix}$ \\\\
			Para esto, hacemos lo mismo que en la parte a), pero aumentando la matriz por el vector $b$, es decir
			$$\left[
			\begin{array}{ccc|c}
			3 & 5 & -4 & 7\\
			-3 & -2 & 4 & -1\\
			6 & 1 & -8 & -4
			\end{array}
			\right] \sim \left[
			\begin{array}{ccc|c}
			3 & 5 & -4 & 7\\
			0 & 3 & 0 & 6\\
			0 & -9 & 0 & -18
			\end{array}
			\right] \sim \left[
			\begin{array}{ccc|c}
			3 & 5 & -4 & 7\\
			0 & 3 & 0 & 6\\
			0 & 0 & 0 & 0
			\end{array}
			\right]$$
			$$\sim \left[
			\begin{array}{ccc|c}
			1 & \frac{5}{3} & -\frac{4}{3} & \frac{7}{3}\\
			0 & 1 & 0 & 2\\
			0 & 0 & 0 & 0
			\end{array}
			\right] \sim \left[
			\begin{array}{ccc|c}
			1 & 0 & -\frac{4}{3} & -1\\
			0 & 1 & 0 & 2\\
			0 & 0 & 0 & 0
			\end{array}
			\right]$$
			Esto corresponde al sistema
			$$\begin{array}{rcl}
			x_1 - \frac{4}{3}x_3 & = & -1\\
			x_2 & = & 2\\
			0 & = & 0
			\end{array} \ra \begin{array}{rcl}
			x_1 & = & -1 + \frac{4}{3}x_3\\
			x_2 & = & 2\\
			x_3 & = & x_3
			\end{array}$$
			$$ x = \begin{pmatrix}
			-1 + \frac{4}{3}x_3\\
			2\\
			x_3
			\end{pmatrix} = \begin{pmatrix}
			-1\\2\\0
			\end{pmatrix} + x_3\begin{pmatrix}
			\frac{4}{3}\\
			0\\
			1
			\end{pmatrix}$$
			Luego, la solución del sistema $Ax = b$ es
			$$S = \begin{pmatrix}
			-1\\2\\0
			\end{pmatrix} + Gen\left\{\begin{pmatrix}
			\frac{4}{3}\\
			0\\
			1
			\end{pmatrix}\right\}$$
			Finalmente, podriamos amplificar el vector del generado por 3, para que quede más bonito, con lo que
			$$S = \begin{pmatrix}
			-1\\2\\0
			\end{pmatrix} + Gen\left\{\begin{pmatrix}
			4\\
			0\\
			3
			\end{pmatrix}\right\}$$
\end{enumerate}
\end{solucion}
\item Describa todas las soluciones del siguiente sistema de ecuaciones y comparelas con las de su sistema homogeneo
	$$
	\begin{array}{rcr}
	x_1 +2x_2-3x_3& = & 5\\
	2x_1 + x_2 - 3x_3& = & 13\\
	-x_1 + x_2 & = & -8
	\end{array}$$
\begin{solucion}
La matriz aumentada asociada es
		$$\left[
		\begin{array}{ccc|c}
		1 & 2 & -3 & 5\\
		2 & 1 & -3 & 13\\
		-1 & 1 & 0 & -8
		\end{array}
		\right] \sim \left[
		\begin{array}{ccc|c}
		1 & 2 & -3 & 5\\
		0 & -3 & 3 & 3\\
		0 & 3 & -3 & -3
		\end{array}
		\right] \sim \left[
		\begin{array}{ccc|c}
		1 & 2 & -3 & 5\\
		0 & -3 & 3 & 3\\
		0 & 0 & 0 & 0
		\end{array}
		\right]$$
		$$ \sim \left[
		\begin{array}{ccc|c}
		1 & 2 & -3 & 5\\
		0 & 1 & -1 & -1\\
		0 & 0 & 0 & 0
		\end{array}
		\right]  \sim \left[
		\begin{array}{ccc|c}
		1 & 0 & -1 & 7\\
		0 & 1 & -1 & -1\\
		0 & 0 & 0 & 0
		\end{array}
		\right] $$
		Esto corresponde al sistema
		$$\begin{array}{rcl}
		x_1 - x_3 & = & 7\\
		x_2 - x_3 & = & -1\\
		0 & = & 0
		\end{array} \ra \begin{array}{rcl}
		x_1 & = & x_3 + 7\\
		x_2 & = & x_3 - 1\\
		x_3 & = & x_3
		\end{array} $$
		Luego,
		$$x = \begin{pmatrix}
		x_3 + 7\\
		x_3 - 1\\
		x_3
		\end{pmatrix} = \begin{pmatrix}
		7\\
		- 1\\
		0
		\end{pmatrix} + x_3 \begin{pmatrix}
		1\\
		1\\
		1
		\end{pmatrix}$$
		Por último, la solución es
		$$S = \begin{pmatrix}
		7\\
		- 1\\
		0
		\end{pmatrix} + Gen\left\{\begin{pmatrix}
		1\\
		1\\
		1
		\end{pmatrix}\right\}$$
		Notemos que la solución del sistema homogeneo será lo mismo pero quitando el vector que no forma parte del generado, es decir
		$$S_H = Gen\left\{\begin{pmatrix}
		1\\
		1\\
		1
		\end{pmatrix}\right\}$$
\end{solucion}
\end{preguntas}
\end{document}