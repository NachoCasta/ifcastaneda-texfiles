\documentclass[12pt]{article}

\usepackage{fullpage}
\usepackage{graphicx}
\usepackage{amssymb}
\usepackage{amsmath}
\usepackage[none]{hyphenat}
\usepackage{parskip}
\usepackage[spanish]{babel}
\usepackage[utf8]{inputenc}
\usepackage{hyperref}
\usepackage{fancyhdr}
\usepackage{tasks}
\usepackage{mdframed}
\usepackage{xcolor}
\usepackage{pgfplots}
\usepackage[makeroom]{cancel}
\usepackage{multicol}
\usepackage[shortlabels]{enumitem}
\usepackage{tabto}

\setlength{\headheight}{10pt}
\setlength{\headsep}{10pt}
\pagestyle{fancy}
\rhead{\ayudantia \ - \alumno}

\newcommand*{\mybox}[2]{\colorbox{#1!30}{\parbox{.98\linewidth}{#2}}}

\newenvironment{solucion}
{\begin{mdframed}[backgroundcolor=black!10]
		{\bf Solución:}\\
	}
	{
	\end{mdframed}
}

\newenvironment{alternativas}[1]
{\begin{multicols}{#1}
		\begin{enumerate}[a)]
		}
		{
		\end{enumerate}
	\end{multicols}
}

\newenvironment{preguntas}
{\begin{enumerate}\itemsep12pt
	}
	{
	\end{enumerate}
}

\newcommand{\ayudantia}{{\sc Ayudantía 2}}
\newcommand{\tituloayu}{Sistemas de ecuaciones lineales, formas escalonadas y conjunto solución}
\newcommand{\fecha}{21 de marzo de 2019}
\newcommand{\sigla}{MAT1203}
\newcommand{\nombre}{Álgebra Lineal}
\newcommand{\profesor}{Camilo Perez}
\newcommand{\ano}{2019}
\newcommand{\semestre}{1}
\newcommand{\mail}{mat1203@ifcastaneda.cl}
\newcommand{\alumno}{Ignacio Castañeda - \mail}

\newcommand{\ev}{\Big|}
\newcommand{\ra}{\rightarrow}
\newcommand{\lra}{\leftrightarrow}
\newcommand{\N}{\mathbb{N}}
\newcommand{\R}{\mathbb{R}}
\newcommand{\Exp}[1]{\mathcal{E}_{#1}}
\newcommand{\List}[1]{\mathcal{L}_{#1}}
\newcommand{\EN}{\Exp{\N}}
\newcommand{\LN}{\List{\N}}
\newcommand{\comment}[1]{}
\newcommand{\lb}{\\~\\}
\newcommand{\eop}{_{\square}}
\newcommand{\hsig}{\hat{\sigma}}

\begin{document}
\thispagestyle{empty}

\begin{minipage}{2cm}
	\includegraphics[width=2cm]{../../../../img/logo.pdf}
	\vspace{0.5cm}
\end{minipage}
\begin{minipage}{\linewidth}
	\begin{tabular}{lrl}
		{\scriptsize\sc Pontificia Universidad Catolica de Chile} & \hspace*{0.7in}Curso: &
		\sigla  - \nombre\\
		{\sc Facultad de Matemáticas}&
		Profesor: & \profesor \\
		{\sc Semestre \ano-\semestre} & Ayudante: & {Ignacio Castañeda}\\
		& {Mail:} & \texttt{\mail}
	\end{tabular}
\end{minipage}

\vspace{-10mm}
\begin{center}
	{\LARGE\bf \ayudantia}\\
	\vspace{0.1cm}
	{\tituloayu}\\
	\vspace{0.1cm}
	\fecha\\
	\vspace{0.4cm}
\end{center}

\begin{preguntas}
\item Escribe los siguientes sistemas de ecuaciones en notación matricial.
\begin{tasks}(2)
\task $
		\begin{array}{rcr}
		x_1 -x_2 + 3x_3 & = & 2\\
		3x_2 - 2x_3 & = & 0\\
		4x_1 +2x_2 & = & -1
		\end{array}
		$
\task $
		\begin{array}{rcr}
		3x_1 & = & 1\\
		x_1-x_2& = & 3\\
		-2x_1+x_2 & = & -8\\
		x_2 & = & 0
		\end{array}
		$
\task $
		\begin{array}{rcr}
		2x_2 -x_3 +2x_4& = & 5\\
		x_1+x_2 - x_3 & = & 2\\
		-4x_1 +x_3-x_4& = & -4
		\end{array}
		$
\task $
		\begin{array}{rcr}
		x_1 +x_2-3x_3& = & 0\\
		x_2+4x_4& = & -1\\
		x_1-2x_2+x_3-5x_4 & = & 2\\
		x_3-x_4 & = & 5
		\end{array}
		$
\end{tasks}
\item Dadas las siguientes matrices aumentadas, escribir el sistema de ecuaciones asociado.
\begin{tasks}(2)
\task $
		\begin{bmatrix}
		-4 & 2 & 1 &  0\\
		2  & 1 & 2 & -1\\
		-3 & 5 &-6& -2
		\end{bmatrix}
		$
\task $
		\begin{bmatrix}
		2 & 1 & -3 & 0\\
		1  & -2 & 0 & 12\\
		4 & 2 &1& 8
		\end{bmatrix}
		$
\end{tasks}
\item Determinar si el siguiente sistema es consistente o no
	$$
	\begin{array}{rcr}
	x_1 -6x_2& = & 5\\
	x_2-4x_3+x_4& = & 0\\
	-x_1+6x_2+x_3+5x_4& = & 3\\
	-x_2+5x_3+4x_4 & = & 0
	\end{array}
	$$
\item Lleve las siguientes matrices a su forma escalonada reducida y determine la existencia y unicidad de las soluciones del sistema.
\begin{tasks}(2)
\task $
		\begin{bmatrix}
		1 & 2 & 4 & 8\\
		2 & 4 & 6 & 8\\
		3 & 6 & 9 & 12
		\end{bmatrix}
		$
\task $
		\begin{bmatrix}
		1 & 2 & 4 & 5\\
		2 & 4 & 5& 4\\
		4 & 5 & 4 & 2
		\end{bmatrix}
		$
\end{tasks}
\item Sea la matriz $A=
	\begin{bmatrix}
	3 & 5 & -4\\
	-3 & -2 & 4\\
	6 & 1 & -8
	\end{bmatrix}$
\begin{enumerate}[a)]
\item Determinar el conjunto solución de su sistema homogeneo.
\item Describir todas las soluciones de $Ax=b$ con $b=
		\begin{pmatrix}
		7\\
		-1\\
		-4
		\end{pmatrix}$ 
\end{enumerate}
\item Describa todas las soluciones del siguiente sistema de ecuaciones y comparelas con las de su sistema homogeneo
	$$
	\begin{array}{rcr}
	x_1 +2x_2-3x_3& = & 5\\
	2x_1 + x_2 - 3x_3& = & 13\\
	-x_1 + x_2 & = & -8
	\end{array}$$
\end{preguntas}
\end{document}