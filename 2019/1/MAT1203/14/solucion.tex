\documentclass[12pt]{article}

\usepackage{fullpage}
\usepackage{graphicx}
\usepackage{amssymb}
\usepackage{amsmath}
\usepackage[none]{hyphenat}
\usepackage{parskip}
\usepackage[spanish]{babel}
\usepackage[utf8]{inputenc}
\usepackage{hyperref}
\usepackage{fancyhdr}
\usepackage{tasks}
\usepackage{mdframed}
\usepackage{xcolor}
\usepackage{pgfplots}
\usepackage[makeroom]{cancel}
\usepackage{multicol}
\usepackage[shortlabels]{enumitem}
\usepackage{stackrel}
\usepackage{tkz-tab}
\usepackage{xpatch}
\usepackage{tkz-euclide}
\usetkzobj{all}
\xpatchcmd{\tkzTabLine}{$0$}{$\bullet$}{}{}

\setlength{\headheight}{10pt}
\setlength{\headsep}{10pt}
\pagestyle{fancy}
\rhead{\ayudantia \ - \alumno}
\tikzset{t style/.style={style=solid}}

\newcommand*{\mybox}[2]{\colorbox{#1!30}{\parbox{.98\linewidth}{#2}}}

\newenvironment{solucion}
{\begin{mdframed}[backgroundcolor=black!10]
		{\bf Solución:}\\
	}
	{
	\end{mdframed}
}

\newenvironment{alternativas}[1]
{\begin{multicols}{#1}
		\begin{enumerate}[a)]
		}
		{
		\end{enumerate}
	\end{multicols}
}

\newenvironment{preguntas}
{\begin{enumerate}\itemsep12pt
	}
	{
	\end{enumerate}
}

\newcommand{\ayudantia}{{\sc Ayudantía 14}}
\newcommand{\tituloayu}{Descomposición en valores singulares}
\newcommand{\fecha}{20 de junio de 2019}
\newcommand{\sigla}{MAT1203}
\newcommand{\nombre}{Álgebra Lineal}
\newcommand{\profesor}{Camilo Perez}
\newcommand{\ano}{2019}
\newcommand{\semestre}{1}
\newcommand{\mail}{mat1203@ifcastaneda.cl}
\newcommand{\alumno}{Ignacio Castañeda - \mail}

\newcommand{\ev}{\Big|}
\newcommand{\ra}{\rightarrow}
\newcommand{\lra}{\leftrightarrow}
\newcommand{\N}{\mathbb{N}}
\newcommand{\R}{\mathbb{R}}
\newcommand{\Exp}[1]{\mathcal{E}_{#1}}
\newcommand{\List}[1]{\mathcal{L}_{#1}}
\newcommand{\EN}{\Exp{\N}}
\newcommand{\LN}{\List{\N}}
\newcommand{\comment}[1]{}
\newcommand{\lb}{\\~\\}
\newcommand{\eop}{_{\square}}
\newcommand{\hsig}{\hat{\sigma}}
\newcommand{\widesim}[2][1.5]{
	\mathrel{\overset{#2}{\scalebox{#1}[1]{$\sim$}}}
}
\newcommand{\wsim}{\widesim{}}
\newcommand{\lh}{\stackrel{L'H}{=}}

\begin{document}
\thispagestyle{empty}

\begin{minipage}{2cm}
	\includegraphics[width=2cm]{../../../../img/logo.pdf}
	\vspace{0.5cm}
\end{minipage}
\begin{minipage}{\linewidth}
	\begin{tabular}{lrl}
		{\scriptsize\sc Pontificia Universidad Catolica de Chile} & \hspace*{0.7in}Curso: &
		\sigla  - \nombre\\
		{\sc Facultad de Matemáticas}&
		Profesor: & \profesor \\
		{\sc Semestre \ano-\semestre} & Ayudante: & {Ignacio Castañeda}\\
		& {Mail:} & \texttt{\mail}
	\end{tabular}
\end{minipage}

\vspace{-10mm}
\begin{center}
	{\LARGE\bf \ayudantia}\\
	\vspace{0.1cm}
	{\tituloayu}\\
	\vspace{0.1cm}
	\fecha\\
	\vspace{0.4cm}
\end{center}

\begin{preguntas}
\item Sea $A = \begin{bmatrix}1 & -3 \\ -3 & 9\end{bmatrix}$
\begin{enumerate}[a)]
\item Encuentre una matriz $L$ cuadrada, triangular inferior con numeros 1 en la diagonal, y una matriz diagonal $D$ tal que $A = LDL^T$.
\item Encuentre la segunda descomposición de Cholesky de $A$
\item Realice un cambio de variable $x=Py$ que transforme la forma cuadrática
		$$Q\left(\begin{bmatrix}x_1\\x_2\end{bmatrix}\right) = \begin{bmatrix}x_1& x_2\end{bmatrix}A\begin{bmatrix}x_1\\ x_2\end{bmatrix}$$
		en una sin términos con producto cruzado.
\end{enumerate}
\begin{solucion}

\begin{enumerate}[a)]
\item 
\item 
\item 
\end{enumerate}
\end{solucion}
\item Calcule la descomposición en valores singulares de la matriz $A = \begin{bmatrix}4 & 11 & 14\\ 8 & 7 & -2\end{bmatrix}$
\begin{solucion}
Sea $A_{m \times n}$ de rango $r$, debemos buscar una matriz $\Sigma_{m \times n}$, una matriz $U_{m \times m}$ y una matriz $V_{n \times n}$ tal que
		$$A = U \Sigma V^T$$
		La matriz $\Sigma$ será de la forma
		$$\Sigma = \begin{bmatrix}
		D & \dots & 0\\
		\vdots & \ddots & 0\\
		0 & \dots & 0
		\end{bmatrix}$$
		donde $D$ es una matriz diagonal con los primeros $r$ valores singulares de $A$ ordenados en de manera decreciente.
		
		$U$ y $V$ serán matrices ortogonales compuestas por los vectores singulares izquierdos y derechos de $A$, respectivamente.
		
		Los valores singulares de $A$ corresponden a las raices de los valores propios de la matriz $A^TA$. Estos los denominaremos como $\sigma_n = \sqrt[]{\lambda_n}$ y los ordenaremos de mayor a menor, por lo que siempre $\sigma_{n+1} \geq \sigma_n$.
		
		Los vectores singulares derechos de $A$ corresponderán a los vectores propios unitarios de $A^TA$, por lo que la matriz $V$ será de la forma 
		$$V = \begin{bmatrix}
		v_1 & v_2 & \dots & v_n
		\end{bmatrix}$$
		donde cada uno esta asociado a uno de valores singulares (recuerden que deben estar ordenados de mayor a menor).
		
		Los vectores singulares izquierdos corresponderán a $u_n = \dfrac{1}{\sigma_n}Av_n$ donde $v_n$ son los vectores propios unitarios de $A^TA$. Luego, la matriz $U$ es de la forma
		$$U = \begin{bmatrix}
		u_1 & u_2 & \dots & u_n
		\end{bmatrix}$$
		
		Veamos ahora que ocurre en el ejercicio.
		
		En primer lugar, debemos determinar los datos propios de $A^TA$ para obtener los valores y vectores singulares.
		$$A^TA = \begin{bmatrix}
		80 & 100 & 40 \\
		100 & 170 & 200 \\
		40 & 140 & 200
		\end{bmatrix}$$
		Luego,
		$$|A^TA - \lambda I| = 0$$
		con lo que
		$$\lambda_1 = 360 \ra \sigma_1 = 6 \ \sqrt[]{10}$$
		$$\ \lambda_2 = 90  \ra \sigma_2 = 3 \ \sqrt[]{10}$$
		$$ \ \lambda_3 = 0  \ra \sigma_3 = 0$$
		Los vectores propios normalizados asociados son
		$$v_1 = \begin{pmatrix}
		\dfrac{1}{3}\\\\
		\dfrac{2}{3}\\\\
		\dfrac{2}{3}
		\end{pmatrix},
		v_2 = \begin{pmatrix}
		-\dfrac{2}{3}\\\\
		-\dfrac{1}{3}\\\\
		\dfrac{2}{3}
		\end{pmatrix},
		v_3 = \begin{pmatrix}
		\dfrac{2}{3}\\\\
		-\dfrac{2}{3}\\\\
		\dfrac{1}{3}
		\end{pmatrix}$$
		A continuación debemos armar las matrices de la descomposición.
		
		Pivoteando podemos notar que el rango de $A$ es 2, por lo que la matriz $D$ será
		$$D = \begin{bmatrix}
		\sigma_1 & 0\\
		0 & \sigma_2
		\end{bmatrix} 
		= \begin{bmatrix}
		6 \ \sqrt[]{10} & 0\\
		0 & 3 \ \sqrt[]{10}
		\end{bmatrix} $$
		Luego,
		$$\Sigma = \begin{bmatrix}
		6 \ \sqrt[]{10} & 0 & 0\\
		0 & 3 \ \sqrt[]{10} & 0
		\end{bmatrix}$$
		Con los vectores propios podemos determinar
		$$V = \begin{bmatrix}
		\dfrac{1}{3} & -\dfrac{2}{3} & \dfrac{2}{3} \\\\
		\dfrac{2}{3} & -\dfrac{1}{3} & -\dfrac{2}{3}  \\\\
		\dfrac{2}{3} & \dfrac{2}{3}  & \dfrac{1}{3} 
		\end{bmatrix}$$
		Para determinar $U$ necesitamos primero encontrar los vectores singulares izquierdos, esto es
		$$u_1 = \dfrac{1}{\sigma_1} A v_1 = \dfrac{1}{6 \ \sqrt[]{10}} \begin{pmatrix}
		18 \\ 6
		\end{pmatrix} = 
		\begin{pmatrix}
		\dfrac{3}{\sqrt[]{10}} \\\\
		\dfrac{1}{\sqrt[]{10}}
		\end{pmatrix}$$
		$$u_2 = \dfrac{1}{\sigma_2} A v_2 = \dfrac{1}{3 \ \sqrt[]{10}} \begin{pmatrix}
		3 \\ -9
		\end{pmatrix} = 
		\begin{pmatrix}
		\dfrac{1}{\sqrt[]{10}} \\\\
		\dfrac{-3}{\sqrt[]{10}}
		\end{pmatrix}$$
		Luego,
		$$U = \begin{bmatrix}
		\dfrac{3}{\sqrt[]{10}} & \dfrac{1}{\sqrt[]{10}} \\\\
		\dfrac{1}{\sqrt[]{10}} & \dfrac{-3}{\sqrt[]{10}}
		\end{bmatrix}$$
		Finalmente,
		$$A = U \Sigma V^T = \begin{bmatrix}
		\dfrac{3}{\sqrt[]{10}} & \dfrac{1}{\sqrt[]{10}} \\\\
		\dfrac{1}{\sqrt[]{10}} & \dfrac{-3}{\sqrt[]{10}}
		\end{bmatrix} 
		\begin{bmatrix}
		6 \ \sqrt[]{10} & 0 & 0\\
		0 & 3 \ \sqrt[]{10} & 0
		\end{bmatrix}
		\begin{bmatrix}
		\dfrac{1}{3} & \dfrac{2}{3} & \dfrac{2}{3} \\\\
		-\dfrac{2}{3} & -\dfrac{1}{3} & \dfrac{2}{3}  \\\\
		\dfrac{2}{3} & -\dfrac{2}{3}  & \dfrac{1}{3} 
		\end{bmatrix}
		$$
\end{solucion}
\item Sea la matriz
	$A = \begin{bmatrix}
	1 & 0 \\1 & 1 \\
	-1 & 1
	\end{bmatrix}$
	, determine su descomposición en valores singulares.
\begin{solucion}

\end{solucion}
\end{preguntas}
\end{document}