\documentclass[12pt]{article}

\usepackage{fullpage}
\usepackage{graphicx}
\usepackage{amssymb}
\usepackage{amsmath}
\usepackage[none]{hyphenat}
\usepackage{parskip}
\usepackage[spanish]{babel}
\usepackage[utf8]{inputenc}
\usepackage{hyperref}
\usepackage{fancyhdr}
\usepackage{tasks}
\usepackage{mdframed}
\usepackage{xcolor}
\usepackage{pgfplots}
\usepackage[makeroom]{cancel}
\usepackage{multicol}
\usepackage[shortlabels]{enumitem}
\usepackage{stackrel}
\usepackage{tkz-tab}
\usepackage{xpatch}
\xpatchcmd{\tkzTabLine}{$0$}{$\bullet$}{}{}

\setlength{\headheight}{10pt}
\setlength{\headsep}{10pt}
\pagestyle{fancy}
\rhead{\ayudantia \ - \alumno}
\tikzset{t style/.style={style=solid}}

\newcommand*{\mybox}[2]{\colorbox{#1!30}{\parbox{.98\linewidth}{#2}}}

\newenvironment{solucion}
{\begin{mdframed}[backgroundcolor=black!10]
		{\bf Solución:}\\
	}
	{
	\end{mdframed}
}

\newenvironment{alternativas}[1]
{\begin{multicols}{#1}
		\begin{enumerate}[a)]
		}
		{
		\end{enumerate}
	\end{multicols}
}

\newenvironment{preguntas}
{\begin{enumerate}\itemsep12pt
	}
	{
	\end{enumerate}
}

\newcommand{\ayudantia}{{\sc Ayudantía 12}}
\newcommand{\tituloayu}{Repaso I3}
\newcommand{\fecha}{6 de junio de 2019}
\newcommand{\sigla}{MAT1203}
\newcommand{\nombre}{Álgebra Lineal}
\newcommand{\profesor}{Camilo Perez}
\newcommand{\ano}{2019}
\newcommand{\semestre}{1}
\newcommand{\mail}{mat1203@ifcastaneda.cl}
\newcommand{\alumno}{Ignacio Castañeda - \mail}

\newcommand{\ev}{\Big|}
\newcommand{\ra}{\rightarrow}
\newcommand{\lra}{\leftrightarrow}
\newcommand{\N}{\mathbb{N}}
\newcommand{\R}{\mathbb{R}}
\newcommand{\Exp}[1]{\mathcal{E}_{#1}}
\newcommand{\List}[1]{\mathcal{L}_{#1}}
\newcommand{\EN}{\Exp{\N}}
\newcommand{\LN}{\List{\N}}
\newcommand{\comment}[1]{}
\newcommand{\lb}{\\~\\}
\newcommand{\eop}{_{\square}}
\newcommand{\hsig}{\hat{\sigma}}
\newcommand{\widesim}[2][1.5]{
	\mathrel{\overset{#2}{\scalebox{#1}[1]{$\sim$}}}
}
\newcommand{\wsim}{\widesim{}}
\newcommand{\lh}{\stackrel{L'H}{=}}

\begin{document}
\thispagestyle{empty}

\begin{minipage}{2cm}
	\includegraphics[width=2cm]{../../../../img/logo.pdf}
	\vspace{0.5cm}
\end{minipage}
\begin{minipage}{\linewidth}
	\begin{tabular}{lrl}
		{\scriptsize\sc Pontificia Universidad Catolica de Chile} & \hspace*{0.7in}Curso: &
		\sigla  - \nombre\\
		{\sc Facultad de Matemáticas}&
		Profesor: & \profesor \\
		{\sc Semestre \ano-\semestre} & Ayudante: & {Ignacio Castañeda}\\
		& {Mail:} & \texttt{\mail}
	\end{tabular}
\end{minipage}

\vspace{-10mm}
\begin{center}
	{\LARGE\bf \ayudantia}\\
	\vspace{0.1cm}
	{\tituloayu}\\
	\vspace{0.1cm}
	\fecha\\
	\vspace{0.4cm}
\end{center}

\begin{preguntas}
\item Sean $u, v$ dos vectores ortogonales en $\R^n$ tales que $||u|| = 1, ||v|| = \sqrt[]{3/2}$. Demuestre que el conjunto 
	$$B = \{u-v, 3u+2v\}$$
	es ortogonal y encuentre las coordenadas del vector $4u - 9v$ respecto al conjunto $B$.
\begin{solucion}
Para demostrar que $B$ es ortogonal, debemos demostrar que todos sus elementos son ortogonales entre si, es decir, debemos demoestrar que
		$$(u-v) \cdot (3u+2v) = 0$$
		Notemos que como $u$ y $v$ son ortogonales entre si, $u \cdot v = 0$. Luego,
		$$\begin{array}{rcl}
		(u-v) \cdot (3u+2v) & = & 3u \cdot u + 2 u \cdot v - 3u \cdot v - 2v \cdot v\\
		& = & 3||u||^2 + 2 (u \cdot v) - 3(u \cdot v) - 2||v||^2 \\
		& = & 3 + 2 \cdot 0 - 3 \cdot 0 - 2(\ \sqrt[]{3/2})^2 \\
		& = & 3 - 3 \\
		& = & 0
		\end{array}$$
		$$\blacksquare$$
		Ahora, para buscar las coordenadas de $4u - 9v$ respecto al conjunto $B$, debemos buscar $\alpha, \beta \in \R$, tal que
		$$\alpha(u-v) + \beta(3u+2v) = 4u-9v$$
		Reordenando,
		$$(\alpha+3\beta)u + (2\beta -\alpha)v= 4u -9v$$
		Luego, debemos resolver el sistema
		$$\begin{array}{rcl}
		\alpha+3\beta & = & 4\\
		-\alpha+2\beta & = & -9
		\end{array}$$
		Resolviendolo, obtenemos que
		$$\alpha = 7 \quad y \quad \beta = -1$$
		Finalmente,
		$$[4u-9v]_B = \begin{pmatrix}
		7 \\ -1
		\end{pmatrix}$$
\end{solucion}
\item Demuestre que si $P$ es una matriz ortogonal de  $n \times n$, entonces para todo $x, y \in \R^n$ se tiene que $Px \cdot Py = x \cdot y$
\begin{solucion}
Como $P$ es ortogonal, se cumple que $P^TP = I$. Luego,
		$$Px \cdot Py = (Px)^T(Py) = x^TP^TPy = x^TIx= x^Ty = x \cdot y$$
\end{solucion}
\item Diagonalice ortogonalmente
	$$M = \begin{bmatrix}
	1 & 0 & 1 \\
	0 & 2 & 0 \\
	1 & 0 & 1
	\end{bmatrix}$$
\begin{solucion}
Para hacer esto debemos hacer un procedimiento similar al que realizamos cuando queremos diagonalizar una matriz, es decir, buscar $P$ y $D$ tal que $M = PDP^{-1}$, con la diferencia de que en este caso $P$ debe ser ortogonal.
		
		Comenzamos buscando los valores propios, con lo que obtendremos
		$$\lambda_1 = 0 \ra \text{multiplicidad 1}$$
		$$\lambda_2 = 2 \ra \text{multiplicidad 2}$$
		Luego, buscamos los vectores propios asociados a cada uno de estos valores propios, los que son
		$$\lambda_1 = 0 \ra v_1 = \begin{pmatrix}
		1 \\ 0 \\ -1
		\end{pmatrix}$$
		$$\lambda_2 = 0 \ra v_2 = \begin{pmatrix}
		0 \\ 1 \\ 0
		\end{pmatrix}, v_3 = \begin{pmatrix}
		1 \\ 0 \\ 1
		\end{pmatrix}$$
		Notemos que todos estos vectores son ortogonales entre si, sin embargo, necesitamos que sean ortonormales. Recordemos que un vector propio será cualquier multiplo de los vectores calculados anteriormente, por lo que para obtener vectores ortonormales, basta con dividir cada uno por su modulo. De esta forma, nuestros nuevos vectores serán
		$$v_1 = \begin{pmatrix}
		1/\ \sqrt[]{2} \\ 0 \\ -1/\ \sqrt[]{2}
		\end{pmatrix}, v_2 = \begin{pmatrix}
		0 \\ 1 \\ 0
		\end{pmatrix}, v_3 = \begin{pmatrix}
		1/\ \sqrt[]{2} \\ 0 \\ 1/\ \sqrt[]{2}
		\end{pmatrix}$$
		Luego, $M$ se diagonaliza con
		$$P = \begin{bmatrix}
		1/\ \sqrt[]{2} & 0 & 1/\ \sqrt[]{2} \\
		0 & 1 & 0 \\
		-1/\ \sqrt[]{2} & 0 & 1/\ \sqrt[]{2}
		\end{bmatrix}, \quad D = \begin{bmatrix}
		0 & 0 & 0 \\
		0 & 2 & 0 \\
		0 & 0 & 2
		\end{bmatrix}$$
\end{solucion}
\end{preguntas}
\end{document}