\documentclass[12pt]{article}

\usepackage{fullpage}
\usepackage{graphicx}
\usepackage{amssymb}
\usepackage{amsmath}
\usepackage[none]{hyphenat}
\usepackage{parskip}
\usepackage[spanish]{babel}
\usepackage[utf8]{inputenc}
\usepackage{hyperref}
\usepackage{fancyhdr}
\usepackage{tasks}
\usepackage{mdframed}
\usepackage{xcolor}
\usepackage{pgfplots}
\usepackage[makeroom]{cancel}
\usepackage{multicol}
\usepackage[shortlabels]{enumitem}
\usepackage{tabto}

\setlength{\headheight}{10pt}
\setlength{\headsep}{10pt}
\pagestyle{fancy}
\rhead{\ayudantia \ - \alumno}

\newcommand*{\mybox}[2]{\colorbox{#1!30}{\parbox{.98\linewidth}{#2}}}

\newenvironment{solucion}
{\begin{mdframed}[backgroundcolor=black!10]
		{\bf Solución:}\\
	}
	{
	\end{mdframed}
}

\newenvironment{alternativas}[1]
{\begin{multicols}{#1}
		\begin{enumerate}[a)]
		}
		{
		\end{enumerate}
	\end{multicols}
}

\newenvironment{preguntas}
{\begin{enumerate}\itemsep12pt
	}
	{
	\end{enumerate}
}

\newcommand{\ayudantia}{{\sc Ayudantía 1}}
\newcommand{\tituloayu}{Vectores, rectas y planos en el espacio}
\newcommand{\fecha}{14 de marzo de 2019}
\newcommand{\sigla}{MAT1203}
\newcommand{\nombre}{Álgebra Lineal}
\newcommand{\profesor}{Camilo Perez}
\newcommand{\ano}{2019}
\newcommand{\semestre}{1}
\newcommand{\mail}{mat1203@ifcastaneda.cl}
\newcommand{\alumno}{Ignacio Castañeda - \mail}

\newcommand{\ev}{\Big|}
\newcommand{\ra}{\rightarrow}
\newcommand{\lra}{\leftrightarrow}
\newcommand{\N}{\mathbb{N}}
\newcommand{\R}{\mathbb{R}}
\newcommand{\Exp}[1]{\mathcal{E}_{#1}}
\newcommand{\List}[1]{\mathcal{L}_{#1}}
\newcommand{\EN}{\Exp{\N}}
\newcommand{\LN}{\List{\N}}
\newcommand{\comment}[1]{}
\newcommand{\lb}{\\~\\}
\newcommand{\eop}{_{\square}}
\newcommand{\hsig}{\hat{\sigma}}

\begin{document}
\thispagestyle{empty}

\begin{minipage}{2cm}
	\includegraphics[width=2cm]{../../../../img/logo.pdf}
	\vspace{0.5cm}
\end{minipage}
\begin{minipage}{\linewidth}
	\begin{tabular}{lrl}
		{\scriptsize\sc Pontificia Universidad Catolica de Chile} & \hspace*{0.7in}Curso: &
		\sigla  - \nombre\\
		{\sc Facultad de Matemáticas}&
		Profesor: & \profesor \\
		{\sc Semestre \ano-\semestre} & Ayudante: & {Ignacio Castañeda}\\
		& {Mail:} & \texttt{\mail}
	\end{tabular}
\end{minipage}

\vspace{-10mm}
\begin{center}
	{\LARGE\bf \ayudantia}\\
	\vspace{0.1cm}
	{\tituloayu}\\
	\vspace{0.1cm}
	\fecha\\
	\vspace{0.4cm}
\end{center}

\begin{preguntas}
\item Realiza las siguientes operaciones con los vectores dados
	$$
	v_1 = \begin{pmatrix}
	1\\
	7\\
	8
\end{pmatrix};\qquad
	v_2 = \begin{pmatrix}
	-4\\
	2\\
	-3
\end{pmatrix}; \qquad
	v_3 = \begin{pmatrix}
	0\\
	1\\
	3
\end{pmatrix}; \qquad
	v_4 = \begin{pmatrix}
	13\\
	-3\\
	1
\end{pmatrix},
	 $$
\begin{tasks}(4)
\task $v_1 + v_2$
\task $v_3 - v_4$
\task $v_2 \cdot v_3$
\task $v_1 \times v_2$
\task $v_2 \times v_1$
\task $(v_1 \times v_2) \cdot v_2$
\task $3 - ((2v_1) \cdot v_2)$
\task $v_2 - v_3 \times v_1$
\end{tasks}
\item Verificar si los siguientes puntos son o no colineares entre si
\begin{tasks}(2)
\task $ 
\begin{pmatrix}
	3\\
	2\\
	5
\end{pmatrix}; \quad
\begin{pmatrix}
	0\\
	1/2\\
	-1
\end{pmatrix}; \quad
\begin{pmatrix}
	5\\
	3\\
	9
\end{pmatrix}$
\task $ 
\begin{pmatrix}
	1\\
	1\\
	4
\end{pmatrix}; \quad
\begin{pmatrix}
	-2\\
	-3\\
	-8
\end{pmatrix}; \quad
\begin{pmatrix}
	4\\
	4\\
	16
\end{pmatrix}$
\end{tasks}
\item Sean los vectores
		$$
	v_1 = \begin{pmatrix}
	2\\
	1\\
	5
\end{pmatrix};\qquad
	v_2 = \begin{pmatrix}
	-1\\
	-3\\
	0
\end{pmatrix}$$
\begin{enumerate}[a)]
\item Buscar un vector $v_3$ que sea perpendicular a $v_1$ y $v_2$ y verificar que efectivamente sea perpendicular a ambos.
\item Buscar un vector paralelo a $(v_2-v_3)$ y verificar que lo sea
\item Encontrar un vector perpendicular a $(v_3 + 3v_1)$
\end{enumerate}
\item Determinar el plano que pasa por los puntos $P_1(4,-1,-2)$, $P_2(0,0,1)$ y $P_3(2,-3,0)$.
\item Dado $P_1(0,2,-3)$ y $P_2(1,0,3)$, determinar la recta que pasa por $P_1$ y $P_2$.
\item Encontrar las ecuaciones de dos planos diferentes cuya intersección sea la recta que pasa por el punto $P_1(1,3,-2)$ y $P_2(2,0,4)$.
\item Encontrar un plano que sea perpendicular al plano cuya ecuación es $3x -7y +2z = 5$ y que pase por el punto $P(0,2,-1)$
\item Determinar la ecuación de un plano que pase por $P(4,2,1)$ y que sea paralelo al plano de ecuación $2x-5y+z=6$
\end{preguntas}
\end{document}