\documentclass[12pt]{article}

\usepackage{fullpage}
\usepackage{graphicx}
\usepackage{amssymb}
\usepackage{amsmath}
\usepackage[none]{hyphenat}
\usepackage{parskip}
\usepackage[spanish]{babel}
\usepackage[utf8]{inputenc}
\usepackage{hyperref}
\usepackage{fancyhdr}
\usepackage{tasks}
\usepackage{mdframed}
\usepackage{xcolor}
\usepackage{pgfplots}
\usepackage[makeroom]{cancel}
\usepackage{multicol}
\usepackage[shortlabels]{enumitem}
\usepackage{stackrel}
\usepackage{tkz-tab}
\usepackage{xpatch}
\xpatchcmd{\tkzTabLine}{$0$}{$\bullet$}{}{}

\setlength{\headheight}{10pt}
\setlength{\headsep}{10pt}
\pagestyle{fancy}
\rhead{\ayudantia \ - \alumno}
\tikzset{t style/.style={style=solid}}

\newcommand*{\mybox}[2]{\colorbox{#1!30}{\parbox{.98\linewidth}{#2}}}

\newenvironment{solucion}
{\begin{mdframed}[backgroundcolor=black!10]
		{\bf Solución:}\\
	}
	{
	\end{mdframed}
}

\newenvironment{alternativas}[1]
{\begin{multicols}{#1}
		\begin{enumerate}[a)]
		}
		{
		\end{enumerate}
	\end{multicols}
}

\newenvironment{preguntas}
{\begin{enumerate}\itemsep12pt
	}
	{
	\end{enumerate}
}

\newcommand{\ayudantia}{{\sc Ayudantía 9}}
\newcommand{\tituloayu}{Bases y coordenadas}
\newcommand{\fecha}{8 de mayo de 2019}
\newcommand{\sigla}{MAT1203}
\newcommand{\nombre}{Álgebra Lineal}
\newcommand{\profesor}{Camilo Perez}
\newcommand{\ano}{2019}
\newcommand{\semestre}{1}
\newcommand{\mail}{mat1203@ifcastaneda.cl}
\newcommand{\alumno}{Ignacio Castañeda - \mail}

\newcommand{\ev}{\Big|}
\newcommand{\ra}{\rightarrow}
\newcommand{\lra}{\leftrightarrow}
\newcommand{\N}{\mathbb{N}}
\newcommand{\R}{\mathbb{R}}
\newcommand{\Exp}[1]{\mathcal{E}_{#1}}
\newcommand{\List}[1]{\mathcal{L}_{#1}}
\newcommand{\EN}{\Exp{\N}}
\newcommand{\LN}{\List{\N}}
\newcommand{\comment}[1]{}
\newcommand{\lb}{\\~\\}
\newcommand{\eop}{_{\square}}
\newcommand{\hsig}{\hat{\sigma}}
\newcommand{\widesim}[2][1.5]{
	\mathrel{\overset{#2}{\scalebox{#1}[1]{$\sim$}}}
}
\newcommand{\wsim}{\widesim{}}
\newcommand{\lh}{\stackrel{L'H}{=}}

\begin{document}
\thispagestyle{empty}

\begin{minipage}{2cm}
	\includegraphics[width=2cm]{../../../../img/logo.pdf}
	\vspace{0.5cm}
\end{minipage}
\begin{minipage}{\linewidth}
	\begin{tabular}{lrl}
		{\scriptsize\sc Pontificia Universidad Catolica de Chile} & \hspace*{0.7in}Curso: &
		\sigla  - \nombre\\
		{\sc Facultad de Matemáticas}&
		Profesor: & \profesor \\
		{\sc Semestre \ano-\semestre} & Ayudante: & {Ignacio Castañeda}\\
		& {Mail:} & \texttt{\mail}
	\end{tabular}
\end{minipage}

\vspace{-10mm}
\begin{center}
	{\LARGE\bf \ayudantia}\\
	\vspace{0.1cm}
	{\tituloayu}\\
	\vspace{0.1cm}
	\fecha\\
	\vspace{0.4cm}
\end{center}

\begin{preguntas}
\item Sea 
	$$U=\{p(x) \in \mathbb{P}_2 : p(1) + p(0) = p(-1)\}$$
	Determinar una base de $U$.
\begin{solucion}
			Sea $p(x) \in \mathbb{P}_2$ un polinomio de la forma $p(x) = a + bx + cx^2$\\
			\\
			Para que pertenezca a $U$, debe cumplirse que
			$$p(1) + p(0) = p(-1)$$
			$$a+b+c+a=a-b+c$$
			$$a = -2b$$
			Luego,
			$$\begin{array}{rcl}
			a & = & -2b\\
			b & = & b\\
			c & = & c
			\end{array}$$
			Entonces, podemos reescribir el polinomio como
			$$p(x) = -2b + bc + cx^2$$
			$$p(x) = b(x-2) + c(x^2)$$
			Finalmente,
			$$U = Gen\{x-2, x^2\}$$
\end{solucion}
\item Use vectores de coordenadas para probar la independencia lineal del conjunto de polinomios $\{1-2t^2-t^3,t+2t^3,1+t-2t^2\}$.
\begin{solucion}
La base que utilizaremos será
		$$B = \{1,t,t^2,t^3\}$$
		Luego, podemos reescribir cada elemento del conjunto de la siguiente forma
		$$\begin{array}{lcl}
		1-2t-t^3 & \ra & (1,\ \ 0,-2,-1)\\
		t+2t^3 & \ra & (0,\ \ 1,\ \ 0,\ \ 2)\\
		1+t-2t^3 & \ra & (1,\ \ 1,-2,\ \ 0)
		\end{array}$$
		Ahora, para probar que los elementos del conjunto son $L.I.$, basta con probar que los vectores de coordenadas lo son. Para esto, ponemos los vectores en una matriz y pivoteamos:
		$$\begin{bmatrix}
		1 & 0 & 1\\
		0 & 1 & 1\\
		-2 & 0 & -2\\
		-1 & 2 & 0
		\end{bmatrix} \sim
		\begin{bmatrix}
		1 & 0 & 1\\
		0 & 1 & 1\\
		0 & 0 & 1\\
		0 & 0 & 0
		\end{bmatrix}$$
		Con lo que concluimos que son $L.I.$
		$$\blacksquare$$
\end{solucion}
\item Sean $p_1(t) = 1-t^2, p_2(t) = 1+t, p_3(t) = 1+t+t^2$. Se sabe que $\{p_1(t), p_2(t), p_3(t)\}$ es una base de $\mathbb{P}_2$.
\begin{enumerate}[a)]
\item Exprese los polinomios $f(t) = 3-5t + 2t^2$ y $g(t) = 1-3t$ como combinaciones lineales de $\{p_1(t), p_2(t), p_3(t)\}$.
\item Use los vectores de coordenadas encontrados en la parte anterior para determinar si el conjunto $\{p_1(t), f(t), g(t)\}$ es L.I. o L.D.
\end{enumerate}
\begin{solucion}

\begin{enumerate}[a)]
\item Exprese los polinomios $f(t) = 3-5t + 2t^2$ y $g(t) = 1-3t$ como combinaciones lineales de $\{p_1(t), p_2(t), p_3(t)\}$.\\
			\\
			Sea una base 
			$$B = \{p_1(t), p_2(t), p_3(t)\} = $$, lo que estamos buscando son las coordenadas de $f$ y $g$ en la base $B$.\\
			\\
			En primer lugar, veamos $f(t)$. Debemos buscar $a$, $b$, $c$, tal que
			$$f(t) = 3-5t + 2t^2 = ap_1(t) + bp_2(t) + cp_3(t)$$
			Reemplazando y reagrupando,
			$$3-5t + 2t^2 = a(1-t^2) + b(1+t) + c(1+t+t^2)$$
			$$3-5t + 2t^2 = (a+b+c) + (b+c)t + (c-a)t^2$$
			Por lo que
			$$\begin{array}{rl}
			a + b +c & = 3\\
			b + c & =-5\\
			-a+c & =2
			\end{array} \Longrightarrow
			\begin{array}{rl}
			a & = 8\\
			b & =-15\\
			c & = 10
			\end{array}$$
			Luego, las coordenadas de $f(t)$ en $B$ son $(8,-15,10)$.\\
			\\
			Veamos ahora que ocurre con $g(t)$. Nuevamente, buscamos $a$, $b$, $c$, tal que
			$$g(t) = 1-3t =  ap_1(t) + bp_2(t) + cp_3(t)$$
			Esto es,
			$$1-3t = a(1-t^2) + b(1+t) + c(1+t+t^2)$$
			$$1-3t = (a+b+c) + (b+c)t + (c-a)t^2$$
			Lo que se nos lleva al sistema
			$$\begin{array}{rl}
			a + b +c & = 1\\
			b + c & =-3\\
			-a+c & =0
			\end{array} \Longrightarrow
			\begin{array}{rl}
			a & = 4\\
			b & =-7\\
			c & = 4
			\end{array}$$
			Con lo que las coordenadas de $g(t)$ en $B$ son $(4,-7,4)$.
\item Use los vectores de coordenadas encontrados en la parte anterior para determinar si el conjunto $\{p_1(t), f(t), g(t)\}$ es L.I. o L.D.\\
			\\
			Para determinar esto, podemos usar las coordenadas de estos elementos en cualquier base. Por conveniencia, usaremos la base $B$. Aquí, los vectores de coordenadas son
			$$p_1(t) = (1,0,0)$$
			$$f(t) = (8,-15,10)$$
			$$g(t) = (4,-7,4)$$
			Armamos una matriz con ellos y pivoteamos,
			$$\begin{bmatrix}
			1 & 0 & 0 \\
			8 & -15 & 10 \\
			4 & -7 & 4
			\end{bmatrix}
			\sim
			\begin{bmatrix}
			1 & 0 & 0 \\
			0 & 1 & 0 \\
			0 & 0 & 1
			\end{bmatrix}$$
			Por lo que sus el conjunto $\{p_1(t), f(t), g(t)\}$ es $L.I.$
\end{enumerate}
\end{solucion}
\item Sea
$$T\left(\left[ \begin{array}{c}
a\\ b\\ c \end{array} \right] \right) = (a+b+c) + (a-b+2c)x +
(3b-c)x^2
$$
Determine una base para $Nul(T) $ y para $Im (T) $ y las dimensiones de estos subespacios.
\begin{solucion}

\end{solucion}
\end{preguntas}
\end{document}