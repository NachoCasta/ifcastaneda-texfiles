\documentclass[12pt]{article}

\usepackage{fullpage}
\usepackage{graphicx}
\usepackage{amssymb}
\usepackage{amsmath}
\usepackage[none]{hyphenat}
\usepackage{parskip}
\usepackage[spanish]{babel}
\usepackage[utf8]{inputenc}
\usepackage{hyperref}
\usepackage{fancyhdr}
\usepackage{tasks}
\usepackage{mdframed}
\usepackage{xcolor}
\usepackage{pgfplots}
\usepackage[makeroom]{cancel}
\usepackage{multicol}
\usepackage[shortlabels]{enumitem}
\usepackage{stackrel}

\setlength{\headheight}{10pt}
\setlength{\headsep}{10pt}
\pagestyle{fancy}
\rhead{\ayudantia \ - \alumno}

\newcommand*{\mybox}[2]{\colorbox{#1!30}{\parbox{.98\linewidth}{#2}}}

\newenvironment{solucion}
{\begin{mdframed}[backgroundcolor=black!10]
		{\bf Solución:}\\
	}
	{
	\end{mdframed}
}

\newenvironment{alternativas}[1]
{\begin{multicols}{#1}
		\begin{enumerate}[a)]
		}
		{
		\end{enumerate}
	\end{multicols}
}

\newenvironment{preguntas}
{\begin{enumerate}\itemsep12pt
	}
	{
	\end{enumerate}
}

\newcommand{\ayudantia}{{\sc Ayudantía 4}}
\newcommand{\tituloayu}{Matrices inversas}
\newcommand{\fecha}{3 de abril de 2019}
\newcommand{\sigla}{MAT1203}
\newcommand{\nombre}{Álgebra Lineal}
\newcommand{\profesor}{Camilo Perez}
\newcommand{\ano}{2019}
\newcommand{\semestre}{1}
\newcommand{\mail}{mat1203@ifcastaneda.cl}
\newcommand{\alumno}{Ignacio Castañeda - \mail}

\newcommand{\ev}{\Big|}
\newcommand{\ra}{\rightarrow}
\newcommand{\lra}{\leftrightarrow}
\newcommand{\N}{\mathbb{N}}
\newcommand{\R}{\mathbb{R}}
\newcommand{\Exp}[1]{\mathcal{E}_{#1}}
\newcommand{\List}[1]{\mathcal{L}_{#1}}
\newcommand{\EN}{\Exp{\N}}
\newcommand{\LN}{\List{\N}}
\newcommand{\comment}[1]{}
\newcommand{\lb}{\\~\\}
\newcommand{\eop}{_{\square}}
\newcommand{\hsig}{\hat{\sigma}}
\newcommand{\widesim}[2][1.5]{
	\mathrel{\overset{#2}{\scalebox{#1}[1]{$\sim$}}}
}
\newcommand{\wsim}{\widesim{}}

\begin{document}
\thispagestyle{empty}

\begin{minipage}{2cm}
	\includegraphics[width=2cm]{../../../../img/logo.pdf}
	\vspace{0.5cm}
\end{minipage}
\begin{minipage}{\linewidth}
	\begin{tabular}{lrl}
		{\scriptsize\sc Pontificia Universidad Catolica de Chile} & \hspace*{0.7in}Curso: &
		\sigla  - \nombre\\
		{\sc Facultad de Matemáticas}&
		Profesor: & \profesor \\
		{\sc Semestre \ano-\semestre} & Ayudante: & {Ignacio Castañeda}\\
		& {Mail:} & \texttt{\mail}
	\end{tabular}
\end{minipage}

\vspace{-10mm}
\begin{center}
	{\LARGE\bf \ayudantia}\\
	\vspace{0.1cm}
	{\tituloayu}\\
	\vspace{0.1cm}
	\fecha\\
	\vspace{0.4cm}
\end{center}

\begin{preguntas}
\item Sea
	$$A = \begin{bmatrix}
	1 & 0 & 6\\
	-1 & 2 & 0\\
	0 & 5 & -1
	\end{bmatrix}$$
	Escriba $A$ como un producto de matrices elementales
\begin{solucion}
Para esto, escalonamos la matriz $A$ hasta reducirla a la matriz identidad. Es importante que hagamos esto paso por paso, sin realizar mas de una operación por fila a la vez.
		$$\begin{bmatrix}
		1 & 0 & 6\\
		-1 & 2 & 0\\
		0 & 5 & -1
		\end{bmatrix} \widesim{F_2+F_1}
		\begin{bmatrix}
		1 & 0 & 6\\
		0 & 2 & 6\\
		0 & 5 & -1
		\end{bmatrix} \widesim{F_3-\frac{5}{2}F_2}
		\begin{bmatrix}
		1 & 0 & 6\\
		0 & 2 & 6\\
		0 & 0 & -16
		\end{bmatrix} \widesim{F_3 \leftarrow -\frac{1}{16}F_3}
		\begin{bmatrix}
		1 & 0 & 6\\
		0 & 2 & 6\\
		0 & 0 & 1
		\end{bmatrix}$$
		$$\widesim{F_2 \leftarrow \frac{1}{2}F_2}
		\begin{bmatrix}
		1 & 0 & 6\\
		0 & 1 & 3\\
		0 & 0 & 1
		\end{bmatrix} \widesim{F_2 - 3F_3}
		\begin{bmatrix}
		1 & 0 & 6\\
		0 & 1 & 0\\
		0 & 0 & 1
		\end{bmatrix} \widesim{F_1 - 6F_3}
		\begin{bmatrix}
		1 & 0 & 0\\
		0 & 1 & 0\\
		0 & 0 & 1
		\end{bmatrix}$$
		Notemos que todas estas operaciones las podemos escribir como matrices elementales, aplicando la operación a la matriz identidad Sean
		$$E_1 = \begin{bmatrix}
		1 & 0 & 0\\
		1 & 1 & 0\\
		0 & 0 & 1
		\end{bmatrix}, \quad E_2 = \begin{bmatrix}
		1 & 0 & 0\\
		0 & 1 & 0\\
		0 & -\frac{5}{2} & 1
		\end{bmatrix}, E_3 = \begin{bmatrix}
		1 & 0 & 0\\
		0 & 1 & 0\\
		0 & 0 & -\frac{1}{16}
		\end{bmatrix}$$
		$$E_4 = \begin{bmatrix}
		1 & 0 & 0\\
		0 & \frac{1}{2} & 0\\
		0 & 0 & 1
		\end{bmatrix}, \quad E_5 = \begin{bmatrix}
		1 & 0 & 0\\
		0 & 1 & -3\\
		0 & 0 & 1
		\end{bmatrix}, E_6 = \begin{bmatrix}
		1 & 0 & -6\\
		0 & 1 & 0\\
		0 & 0 & 1
		\end{bmatrix}$$
		tenemos que
		$$E_6E_5E_4E_3E_2E_1A = I$$
		por lo que
		$$A = E_1^{-1}E_2^{-1}E_3^{-1}E_4^{-1}E_5^{-1}E_6^{-1}$$
		donde
		$$E_1^{-1} = \begin{bmatrix}
		1 & 0 & 0\\
		-1 & 1 & 0\\
		0 & 0 & 1
		\end{bmatrix}, \quad E_2^{-1} = \begin{bmatrix}
		1 & 0 & 0\\
		0 & 1 & 0\\
		0 & \frac{5}{2} & 1
		\end{bmatrix}, E_3^{-1} = \begin{bmatrix}
		1 & 0 & 0\\
		0 & 1 & 0\\
		0 & 0 & -16
		\end{bmatrix}$$
		$$E_4^{-1} = \begin{bmatrix}
		1 & 0 & 0\\
		0 & 2 & 0\\
		0 & 0 & 1
		\end{bmatrix}, \quad E_5^{-1} = \begin{bmatrix}
		1 & 0 & 0\\
		0 & 1 & 3\\
		0 & 0 & 1
		\end{bmatrix}, E_6^{-1} = \begin{bmatrix}
		1 & 0 & 6\\
		0 & 1 & 0\\
		0 & 0 & 1
		\end{bmatrix}$$
		Finalmente,
		$$A = \begin{bmatrix}
		1 & 0 & 0\\
		-1 & 1 & 0\\
		0 & 0 & 1
		\end{bmatrix}\begin{bmatrix}
		1 & 0 & 0\\
		0 & 1 & 0\\
		0 & \frac{5}{2} & 1
		\end{bmatrix}\begin{bmatrix}
		1 & 0 & 0\\
		0 & 1 & 0\\
		0 & 0 & \frac{1}{16}
		\end{bmatrix}\begin{bmatrix}
		1 & 0 & 0\\
		0 & -\frac{1}{2} & 0\\
		0 & 0 & 1
		\end{bmatrix}\begin{bmatrix}
		1 & 0 & 0\\
		0 & 1 & 3\\
		0 & 0 & 1
		\end{bmatrix}\begin{bmatrix}
		1 & 0 & 6\\
		0 & 1 & 0\\
		0 & 0 & 1
		\end{bmatrix}$$
\end{solucion}
\item Determinar la matriz inversa de
$$A = \begin{bmatrix}
1 & -1 & 0 \\
0 & 1 & 0 \\
2 & 0 & 1
\end{bmatrix}$$
\begin{solucion}

\end{solucion}
\item Sea $A$ una matriz de $n\times m$ con columnas LI. Demuestre que si $P=A(A^tA)^{-1}A^{t}$, entonces $P^2=P$ y $P=P^t$.
\begin{solucion}

\end{solucion}
\item Sea $A$ una matriz de $3\times 3$ tal que
      $$A\left(\begin{array}{r}
  1\\0\\1
\end{array}\right)= \left(\begin{array}{r}
  1\\0\\0
\end{array}\right),\quad\quad A \left(\begin{array}{r}
  1\\-2\\4
\end{array}\right)= \left(\begin{array}{r}
  1\\0\\1
\end{array}\right), \qquad  A\left(\begin{array}{r}
  -1\\1\\1
\end{array}\right)= \left(\begin{array}{r}
  0\\1\\1
\end{array}\right).$$ Calcule $A^{-1}$.
\begin{enumerate}[a)]
\item 
\end{enumerate}
\begin{solucion}

\begin{enumerate}[a)]
\item 
\end{enumerate}
\end{solucion}
\end{preguntas}
\end{document}