\documentclass[12pt]{article}

\usepackage{fullpage}
\usepackage{graphicx}
\usepackage{amssymb}
\usepackage{amsmath}
\usepackage[none]{hyphenat}
\usepackage{parskip}
\usepackage[spanish]{babel}
\usepackage[utf8]{inputenc}
\usepackage{hyperref}
\usepackage{fancyhdr}
\usepackage{tasks}
\usepackage{mdframed}
\usepackage{xcolor}
\usepackage{pgfplots}
\usepackage[makeroom]{cancel}
\usepackage{multicol}
\usepackage[shortlabels]{enumitem}
\usepackage{stackrel}

\setlength{\headheight}{10pt}
\setlength{\headsep}{10pt}
\pagestyle{fancy}
\rhead{\ayudantia \ - \alumno}

\newcommand*{\mybox}[2]{\colorbox{#1!30}{\parbox{.98\linewidth}{#2}}}

\newenvironment{solucion}
{\begin{mdframed}[backgroundcolor=black!10]
		{\bf Solución:}\\
	}
	{
	\end{mdframed}
}

\newenvironment{alternativas}[1]
{\begin{multicols}{#1}
		\begin{enumerate}[a)]
		}
		{
		\end{enumerate}
	\end{multicols}
}

\newenvironment{preguntas}
{\begin{enumerate}\itemsep12pt
	}
	{
	\end{enumerate}
}

\newcommand{\ayudantia}{{\sc Ayudantía 4}}
\newcommand{\tituloayu}{Matrices inversas}
\newcommand{\fecha}{4 de abril de 2019}
\newcommand{\sigla}{MAT1203}
\newcommand{\nombre}{Álgebra Lineal}
\newcommand{\profesor}{Camilo Perez}
\newcommand{\ano}{2019}
\newcommand{\semestre}{1}
\newcommand{\mail}{mat1203@ifcastaneda.cl}
\newcommand{\alumno}{Ignacio Castañeda - \mail}

\newcommand{\ev}{\Big|}
\newcommand{\ra}{\rightarrow}
\newcommand{\lra}{\leftrightarrow}
\newcommand{\N}{\mathbb{N}}
\newcommand{\R}{\mathbb{R}}
\newcommand{\Exp}[1]{\mathcal{E}_{#1}}
\newcommand{\List}[1]{\mathcal{L}_{#1}}
\newcommand{\EN}{\Exp{\N}}
\newcommand{\LN}{\List{\N}}
\newcommand{\comment}[1]{}
\newcommand{\lb}{\\~\\}
\newcommand{\eop}{_{\square}}
\newcommand{\hsig}{\hat{\sigma}}
\newcommand{\widesim}[2][1.5]{
	\mathrel{\overset{#2}{\scalebox{#1}[1]{$\sim$}}}
}
\newcommand{\wsim}{\widesim{}}

\begin{document}
\thispagestyle{empty}

\begin{minipage}{2cm}
	\includegraphics[width=2cm]{../../../../img/logo.pdf}
	\vspace{0.5cm}
\end{minipage}
\begin{minipage}{\linewidth}
	\begin{tabular}{lrl}
		{\scriptsize\sc Pontificia Universidad Catolica de Chile} & \hspace*{0.7in}Curso: &
		\sigla  - \nombre\\
		{\sc Facultad de Matemáticas}&
		Profesor: & \profesor \\
		{\sc Semestre \ano-\semestre} & Ayudante: & {Ignacio Castañeda}\\
		& {Mail:} & \texttt{\mail}
	\end{tabular}
\end{minipage}

\vspace{-10mm}
\begin{center}
	{\LARGE\bf \ayudantia}\\
	\vspace{0.1cm}
	{\tituloayu}\\
	\vspace{0.1cm}
	\fecha\\
	\vspace{0.4cm}
\end{center}

\begin{preguntas}
\item Sea
	$$A = \begin{bmatrix}
	1 & 0 & 6\\
	-1 & 2 & 0\\
	0 & 5 & -1
	\end{bmatrix}$$
	Escriba $A$ como un producto de matrices elementales
\begin{solucion}
Para esto, escalonamos la matriz $A$ hasta reducirla a la matriz identidad. Es importante que hagamos esto paso por paso, sin realizar mas de una operación por fila a la vez.
		$$\begin{bmatrix}
		1 & 0 & 6\\
		-1 & 2 & 0\\
		0 & 5 & -1
		\end{bmatrix} \widesim{F_2+F_1}
		\begin{bmatrix}
		1 & 0 & 6\\
		0 & 2 & 6\\
		0 & 5 & -1
		\end{bmatrix} \widesim{F_3-\frac{5}{2}F_2}
		\begin{bmatrix}
		1 & 0 & 6\\
		0 & 2 & 6\\
		0 & 0 & -16
		\end{bmatrix} \widesim{F_3 \leftarrow -\frac{1}{16}F_3}
		\begin{bmatrix}
		1 & 0 & 6\\
		0 & 2 & 6\\
		0 & 0 & 1
		\end{bmatrix}$$
		$$\widesim{F_2 \leftarrow \frac{1}{2}F_2}
		\begin{bmatrix}
		1 & 0 & 6\\
		0 & 1 & 3\\
		0 & 0 & 1
		\end{bmatrix} \widesim{F_2 - 3F_3}
		\begin{bmatrix}
		1 & 0 & 6\\
		0 & 1 & 0\\
		0 & 0 & 1
		\end{bmatrix} \widesim{F_1 - 6F_3}
		\begin{bmatrix}
		1 & 0 & 0\\
		0 & 1 & 0\\
		0 & 0 & 1
		\end{bmatrix}$$
		Notemos que todas estas operaciones las podemos escribir como matrices elementales, aplicando la operación a la matriz identidad Sean
		$$E_1 = \begin{bmatrix}
		1 & 0 & 0\\
		1 & 1 & 0\\
		0 & 0 & 1
		\end{bmatrix}, \quad E_2 = \begin{bmatrix}
		1 & 0 & 0\\
		0 & 1 & 0\\
		0 & -\frac{5}{2} & 1
		\end{bmatrix}, E_3 = \begin{bmatrix}
		1 & 0 & 0\\
		0 & 1 & 0\\
		0 & 0 & -\frac{1}{16}
		\end{bmatrix}$$
		$$E_4 = \begin{bmatrix}
		1 & 0 & 0\\
		0 & \frac{1}{2} & 0\\
		0 & 0 & 1
		\end{bmatrix}, \quad E_5 = \begin{bmatrix}
		1 & 0 & 0\\
		0 & 1 & -3\\
		0 & 0 & 1
		\end{bmatrix}, E_6 = \begin{bmatrix}
		1 & 0 & -6\\
		0 & 1 & 0\\
		0 & 0 & 1
		\end{bmatrix}$$
		tenemos que
		$$E_6E_5E_4E_3E_2E_1A = I$$
		por lo que
		$$A = E_1^{-1}E_2^{-1}E_3^{-1}E_4^{-1}E_5^{-1}E_6^{-1}$$
		donde
		$$E_1^{-1} = \begin{bmatrix}
		1 & 0 & 0\\
		-1 & 1 & 0\\
		0 & 0 & 1
		\end{bmatrix}, \quad E_2^{-1} = \begin{bmatrix}
		1 & 0 & 0\\
		0 & 1 & 0\\
		0 & \frac{5}{2} & 1
		\end{bmatrix}, E_3^{-1} = \begin{bmatrix}
		1 & 0 & 0\\
		0 & 1 & 0\\
		0 & 0 & -16
		\end{bmatrix}$$
		$$E_4^{-1} = \begin{bmatrix}
		1 & 0 & 0\\
		0 & 2 & 0\\
		0 & 0 & 1
		\end{bmatrix}, \quad E_5^{-1} = \begin{bmatrix}
		1 & 0 & 0\\
		0 & 1 & 3\\
		0 & 0 & 1
		\end{bmatrix}, E_6^{-1} = \begin{bmatrix}
		1 & 0 & 6\\
		0 & 1 & 0\\
		0 & 0 & 1
		\end{bmatrix}$$
		Finalmente,
		$$A = \begin{bmatrix}
		1 & 0 & 0\\
		-1 & 1 & 0\\
		0 & 0 & 1
		\end{bmatrix}\begin{bmatrix}
		1 & 0 & 0\\
		0 & 1 & 0\\
		0 & \frac{5}{2} & 1
		\end{bmatrix}\begin{bmatrix}
		1 & 0 & 0\\
		0 & 1 & 0\\
		0 & 0 & \frac{1}{16}
		\end{bmatrix}\begin{bmatrix}
		1 & 0 & 0\\
		0 & -\frac{1}{2} & 0\\
		0 & 0 & 1
		\end{bmatrix}\begin{bmatrix}
		1 & 0 & 0\\
		0 & 1 & 3\\
		0 & 0 & 1
		\end{bmatrix}\begin{bmatrix}
		1 & 0 & 6\\
		0 & 1 & 0\\
		0 & 0 & 1
		\end{bmatrix}$$
\end{solucion}
\item Determinar la matriz inversa de
$$A = \begin{bmatrix}
1 & -1 & 0 \\
0 & 1 & 0 \\
2 & 0 & 1
\end{bmatrix}$$
\begin{solucion}

\end{solucion}
\item Sea $A$ una matriz de $n\times m$ con columnas LI. Demuestre que si $P=A(A^tA)^{-1}A^{t}$, entonces $P^2=P$ y $P=P^t$.
\begin{solucion}
	Recordemos que
	$$(AB)^t = B^tA^t \qquad y \qquad (AB)^{-1} = B^{-1}A^{-1}$$
	En primer lugar, demostremos que 
	$$P^2 = P \ra P = P^2$$
	Reemplazamos con $P=A(A^tA)^{-1}A^{t}$, es decir
	$$\begin{array}{rcl}
	P & = & P^2 \\
	  & = & (A(A^tA)^{-1}A^{t})(A(A^tA)^{-1}A^{t}) \\
	  & = & A(A^tA)^{-1}A^{t}A(A^tA)^{-1}A^{t} \\
	  & = & A(A^tA)^{-1}(A^{t}A)(A^tA)^{-1}A^{t} \\
	  & = & A(A^tA)^{-1}A^{t} \\
	P & = & P
	\end{array}$$
	$$\blacksquare$$
	Ahora, demostremos que $P = P^t$, esto es,
	$$\begin{array}{rcl}
	P & = & P^t \\
	  & = & (A(A^tA)^{-1}A^{t})^t \\
	  & = & ((A(A^tA)^{-1})A^{t})^t \\
	  & = & A(A(A^tA)^{-1})^t \\
	  & = & A(A^tA)^{-t}A^t \\
	  & = & A((A^tA)^t)^{-1}A^t \\
	  & = & A(A^tA)^{-1}A^t \\
	P & = & P
	\end{array}$$
	$$\blacksquare$$
\end{solucion}
\item Sea $A$ una matriz de $3\times 3$ tal que
      $$A\left(\begin{array}{r}
  1\\0\\1
\end{array}\right)= \left(\begin{array}{r}
  1\\0\\0
\end{array}\right),\quad\quad A \left(\begin{array}{r}
  1\\-2\\4
\end{array}\right)= \left(\begin{array}{r}
  1\\0\\1
\end{array}\right), \qquad  A\left(\begin{array}{r}
  -1\\1\\1
\end{array}\right)= \left(\begin{array}{r}
  0\\1\\1
\end{array}\right).$$ Calcule $A^{-1}$.
\begin{solucion}
Recordemos que para $A_{m \times n}$ y $v_1, v_2 \in \R^{n}$, siempre se cumple que
$$Av1 + Av2 = A(v_1 + v_2)$$
y recordemos también que al multiplicar dos matrices, se esta multiplicando la primera matriz con las columnas de la segunda para generar la matriz resultante, es decir, siendo $B = [v_1\ v_2\ \dots\ v_n]$,
$$AB = A[v_1\ v_2\ \dots\ v_n] = [Av_1\ Av_2\ \dots\ Av_n]$$
Procedamos con el ejercicio.\\
\\
Llamemos $v_1, v_2, v_3$ a todos los vectores que son multiplicados por $A$ en el enunciado, respectivamente y $u_1, u_2, u_3$ a sus respectivos resultados.\\
\\
Para obtener la inversa de $A$, debemos encontrar una matriz que al multiplicarla por $A$ nos de la matriz identidad.\\
\\
Busquemos entonces una forma de construir las columnas de la matriz identidad (vectores canónicos) con una combinación lineal de los vectores $u_i$. Si tuvieramos vectores más complejos, tendríamos que resolver el sistema $\alpha u_1 + \beta u_2 + \gamma u_3 = e_i$ para $i = \{1, 2, 3\}$. Sin embargo, dado que estos vectores son bastante simples, podemos hacerlo de manera analitica, esto es,
$$e_1 = u_1, \quad e_2 = , u_3-u_2+u_1, \quad e_3 = u_2 - u_1$$
Luego,
$$I = [u_1\quad u_3-u_2+u_1\quad u_2-u_1]$$
Utilizando las propiedades mencionadas al principio,
$$I = [Av_1\quad A(v_3-v_2+v_1)\quad A(v_2-v_1)]$$
$$I = A[v_1\quad v_3-v_2+v_1\quad v_2-v_1]$$
$$A^{-1} = [v_1\quad v_3-v_2+v_1\quad v_2-v_1]$$
Por lo tanto, la inversa de $A$ corresponde a
$$A^{-1} = 
\begin{bmatrix}
1 &-1 & 0 \\
0 & 3 & 0 \\
1 &-2 &-1 
\end{bmatrix}$$
\end{solucion}
\item Demuestre que $T(x_1,x_2, x_3)=(x_1,x_1+x_2, x_1+x_2+x_3)$ es invertible y encuentre una f\'ormula para $T^{-1}$.
\begin{solucion}
En primer lugar determinemos la matriz de la transformación lineal $T$.\\
Sea $Ax = T(x)$,
$$A = \begin{bmatrix}
1 & 0 & 0 \\
1 & 1 & 0 \\
1 & 1 & 1
\end{bmatrix}$$
Notemos que la matriz esta escalonada "al revés". En otras palabras, $A^t$ estaría escalonada y es evidente que tiene 3 pivotes, por lo que $A$ igual tiene 3 pivotes. Al tener igual cantidad de pivotes que de columnas, la matriz $A$ es invertible.\\
\\
Ahora, busquemos $A^-1$, esto es
$$\left[
\begin{array}{cccc|ccc}
1 & 0 & 0 & 1 & 0 & 0 \\
1 & 1 & 0 & 0 & 1 & 0 \\
1 & 1 & 1 & 0 & 0 & 1
\end{array}
\right] \sim 
\left[
\begin{array}{ccc|ccc}
1 & 0 & 0 & 1 & 0 & 0 \\
0 & 1 & 0 & -1 & 1 & 0 \\
0 & 1 & 1 & -1 & 0 & 1
\end{array}
\right] \sim 
\left[
\begin{array}{ccc|ccc}
1 & 0 & 0 & 1 & 0 & 0 \\
0 & 1 & 0 & -1 & 1 & 0 \\
0 & 0 & 1 & -1 & -1 & 1
\end{array}
\right]
$$
Es decir,
$$A^{-1} = 
\begin{bmatrix}
 1 & 0 & 0 \\
-1 & 1 & 0 \\
-1 & -1 & 1
\end{bmatrix}
$$
Finalmente, podemos escribir $T^{-1}(x)$ como
$$T^{-1}(x) = (x_1, -x_1 + x_2, -x_1-x_2+x_3)$$
\end{solucion}
\end{preguntas}
\end{document}