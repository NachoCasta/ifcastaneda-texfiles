\documentclass[12pt]{article}

\usepackage{fullpage}
\usepackage{graphicx}
\usepackage{amssymb}
\usepackage{amsmath}
\usepackage[none]{hyphenat}
\usepackage{parskip}
\usepackage[spanish]{babel}
\usepackage[utf8]{inputenc}
\usepackage{hyperref}
\usepackage{fancyhdr}
\usepackage{tasks}
\usepackage{mdframed}
\usepackage{xcolor}
\usepackage{pgfplots}
\usepackage[makeroom]{cancel}
\usepackage{multicol}
\usepackage[shortlabels]{enumitem}
\usepackage{stackrel}
\usepackage{tkz-tab}
\usepackage{xpatch}
\usepackage{tkz-euclide}
\usetkzobj{all}
\xpatchcmd{\tkzTabLine}{$0$}{$\bullet$}{}{}

\setlength{\headheight}{10pt}
\setlength{\headsep}{10pt}
\pagestyle{fancy}
\rhead{\ayudantia \ - \alumno}
\tikzset{t style/.style={style=solid}}

\newcommand*{\mybox}[2]{\colorbox{#1!30}{\parbox{.98\linewidth}{#2}}}

\newenvironment{solucion}
{\begin{mdframed}[backgroundcolor=black!10]
		{\bf Solución:}\\
	}
	{
	\end{mdframed}
}

\newenvironment{alternativas}[1]
{\begin{multicols}{#1}
		\begin{enumerate}[a)]
		}
		{
		\end{enumerate}
	\end{multicols}
}

\newenvironment{preguntas}
{\begin{enumerate}\itemsep12pt
	}
	{
	\end{enumerate}
}

\newcommand{\ayudantia}{{\sc Ayudantía 12.5}}
\newcommand{\tituloayu}{Compilado I3}
\newcommand{\fecha}{7 de junio de 2019}
\newcommand{\sigla}{MAT1203}
\newcommand{\nombre}{Álgebra Lineal}
\newcommand{\profesor}{Camilo Perez}
\newcommand{\ano}{2019}
\newcommand{\semestre}{1}
\newcommand{\mail}{mat1203@ifcastaneda.cl}
\newcommand{\alumno}{Ignacio Castañeda - \mail}

\newcommand{\ev}{\Big|}
\newcommand{\ra}{\rightarrow}
\newcommand{\lra}{\leftrightarrow}
\newcommand{\N}{\mathbb{N}}
\newcommand{\R}{\mathbb{R}}
\newcommand{\Exp}[1]{\mathcal{E}_{#1}}
\newcommand{\List}[1]{\mathcal{L}_{#1}}
\newcommand{\EN}{\Exp{\N}}
\newcommand{\LN}{\List{\N}}
\newcommand{\comment}[1]{}
\newcommand{\lb}{\\~\\}
\newcommand{\eop}{_{\square}}
\newcommand{\hsig}{\hat{\sigma}}
\newcommand{\widesim}[2][1.5]{
	\mathrel{\overset{#2}{\scalebox{#1}[1]{$\sim$}}}
}
\newcommand{\wsim}{\widesim{}}
\newcommand{\lh}{\stackrel{L'H}{=}}

\begin{document}
\thispagestyle{empty}

\begin{minipage}{2cm}
	\includegraphics[width=2cm]{../../../../img/logo.pdf}
	\vspace{0.5cm}
\end{minipage}
\begin{minipage}{\linewidth}
	\begin{tabular}{lrl}
		{\scriptsize\sc Pontificia Universidad Catolica de Chile} & \hspace*{0.7in}Curso: &
		\sigla  - \nombre\\
		{\sc Facultad de Matemáticas}&
		Profesor: & \profesor \\
		{\sc Semestre \ano-\semestre} & Ayudante: & {Ignacio Castañeda}\\
		& {Mail:} & \texttt{\mail}
	\end{tabular}
\end{minipage}

\vspace{-10mm}
\begin{center}
	{\LARGE\bf \ayudantia}\\
	\vspace{0.1cm}
	{\tituloayu}\\
	\vspace{0.1cm}
	\fecha\\
	\vspace{0.4cm}
\end{center}

\begin{preguntas}
\item Sea 
	$$U=\{p(x) \in \mathbb{P}_2 : p(1) + p(0) = p(-1)\}$$
	Determinar una base de $U$.
\item Sean $B = \begin{bmatrix} 1 & -1 \\ 0 & 0 \end{bmatrix}$ y $C = \{X \in M_2(\R)|BX=0\}$. Determine una base y la dimensión de $C$.
\item Use vectores de coordenadas para probar la independencia lineal del conjunto de polinomios $\{1-2t^2-t^3,t+2t^3,1+t-2t^2\}$.
\item Sean $p_1(t) = 1-t^2, p_2(t) = 1+t, p_3(t) = 1+t+t^2$. Se sabe que $\{p_1(t), p_2(t), p_3(t)\}$ es una base de $\mathbb{P}_2$.
\begin{enumerate}[a)]
\item Exprese los polinomios $f(t) = 3-5t + 2t^2$ y $g(t) = 1-3t$ como combinaciones lineales de $\{p_1(t), p_2(t), p_3(t)\}$.
\item Use los vectores de coordenadas encontrados en la parte anterior para determinar si el conjunto $\{p_1(t), f(t), g(t)\}$ es L.I. o L.D.
\end{enumerate}
\item Sea
$$T\left(\left[ \begin{array}{c}
a\\ b\\ c \end{array} \right] \right) = (a+b+c) + (a-b+2c)x +
(3b-c)x^2
$$
Determine una base para $Nul(T) $ y para $Im (T) $ y las dimensiones de estos subespacios.
\item Sea $T: \mathbb{P}_2 \ra \mathbb{P}_2$ una transformación lineal cuya matriz de transformación respecto a la base $B = \{1, t, t^2\}$ está dada por
$$[T]_B = \begin{bmatrix}
1 & -1 & 1\\
0 & 2 & -1\\
-1 & 3 & -2
\end{bmatrix}$$
Determine $T(t^2+1)$ y el espacio Nulo de $T$.
\item Sean $B_1 = \{1+x,1-x,1+x^2\}$ y $B_2$ bases de $P_2(\R)$ tales que para todo $p \in P_2(\R)$
	$$ \begin{bmatrix}
	1 & 0 & -1\\
	1 & 1 & 0 \\
	0 & 1 & 2
	\end{bmatrix} [p]_{B_1} = [p]_{B_2}$$
	Determine los polinomios que forman la base $B_2$.
\item Sean $B$ y $C$ bases de un espacio vectorial $V$ y $P = \begin{bmatrix}4 &-1 \\ 6 & -1\end{bmatrix}$ la matriz de cambio de coordenadas tal que $[v]_C = P[v]_B\ \forall v \in V$
\begin{enumerate}[a)]
\item Demuestre que el conjunto $W = \{v \in V: [v]_C = 2[v]_B\}$ es un subespacio de $V$.
\item Si $B=\{v_1,v_2\}$ determine una base para $W$ en términos de la base de $B$.
\end{enumerate}
\item Calcular los valores y vectores propios de
	$$ A = \begin{bmatrix} 1 & 2 & -1\\ 1 & 0 & 1\\ 4 & -4 & 5\end{bmatrix}$$
\item La matriz $\begin{bmatrix} 1 & -2 \\ 1 & 3 \end{bmatrix}$ actúa sobre $\mathbb{C}^2$. Determine los valores propios y una base para cada espacio propio de $\mathbb{C}^2$.
\item Sea $A$ una matriz invertible, y sea $\lambda$ un valor propio de $A$. Demuestre que $\lambda \neq 0$ y que $\dfrac{1}{\lambda}$ es un valor propio de $A^{-1}$.
\item Sea $A$ una matriz de $5 \times 5$ tal que $A^t = -A$ y $q$ el polinomio dado por $q(x) = 2-x^2+4x^3$. Demuestre que $2$ es valor propio de la matriz $q(A)$.
\item Sea $A$ una matriz de $2\times 2$ de rango 1 tal que $A\begin{bmatrix} 1 \\ 2\end{bmatrix} = \begin{bmatrix} 2 \\ 4 \end{bmatrix}$. ¿Es $A$ diagonalizable? Justifique.
\item Diagonalice la matriz
	$$M = \begin{bmatrix}
	1 & 0 & 0\\
	1 & 1 & 2 \\
	1 & 0 & 3
	\end{bmatrix}$$
	y encuentre una matriz $N$ tal que $N^3 = M$.
\item Diagonalice $A =
	\begin{bmatrix} 
	-1 & -2 & 2 \\
	0 & -1 & 0 \\
	0 & -2 & 1
	\end{bmatrix}$ y diagonalice $B = A^{10} + A - I$ 
\item Diagonalice ortogonalmente
	$$M = \begin{bmatrix}
	1 & 0 & 1 \\
	0 & 2 & 0 \\
	1 & 0 & 1
	\end{bmatrix}$$
\item Sean $u, v$ dos vectores ortogonales en $\R^n$ tales que $||u|| = 1, ||v|| = \sqrt[]{3/2}$. Demuestre que el conjunto 
	$$B = \{u-v, 3u+2v\}$$
	es ortogonal y encuentre las coordenadas del vector $4u - 9v$ respecto al conjunto $B$.
\item Demuestre que si $P$ es una matriz ortogonal de  $n \times n$, entonces para todo $x, y \in \R^n$ se tiene que $Px \cdot Py = x \cdot y$
\item Determine si las siguientes afirmaciones son Verdaderas o Falsas.
\begin{enumerate}[a)]
\item El espacio fila de $AB$ es subespacio del espacio fila de $B$.
\item Dada una base en un espacio vectorial $V$, el vector coordenado de un vector de $V$ con respecto a esa base, es único.
\item Si $A$ y $B$ son similares y $\lambda$ es valor propio de $A$ entonces $\lambda$ es también valor propio de $B$.
\end{enumerate}
\end{preguntas}
\end{document}