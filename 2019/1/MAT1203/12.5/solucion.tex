\documentclass[12pt]{article}

\usepackage{fullpage}
\usepackage{graphicx}
\usepackage{amssymb}
\usepackage{amsmath}
\usepackage[none]{hyphenat}
\usepackage{parskip}
\usepackage[spanish]{babel}
\usepackage[utf8]{inputenc}
\usepackage{hyperref}
\usepackage{fancyhdr}
\usepackage{tasks}
\usepackage{mdframed}
\usepackage{xcolor}
\usepackage{pgfplots}
\usepackage[makeroom]{cancel}
\usepackage{multicol}
\usepackage[shortlabels]{enumitem}
\usepackage{stackrel}
\usepackage{tkz-tab}
\usepackage{xpatch}
\xpatchcmd{\tkzTabLine}{$0$}{$\bullet$}{}{}

\setlength{\headheight}{10pt}
\setlength{\headsep}{10pt}
\pagestyle{fancy}
\rhead{\ayudantia \ - \alumno}
\tikzset{t style/.style={style=solid}}

\newcommand*{\mybox}[2]{\colorbox{#1!30}{\parbox{.98\linewidth}{#2}}}

\newenvironment{solucion}
{\begin{mdframed}[backgroundcolor=black!10]
		{\bf Solución:}\\
	}
	{
	\end{mdframed}
}

\newenvironment{alternativas}[1]
{\begin{multicols}{#1}
		\begin{enumerate}[a)]
		}
		{
		\end{enumerate}
	\end{multicols}
}

\newenvironment{preguntas}
{\begin{enumerate}\itemsep12pt
	}
	{
	\end{enumerate}
}

\newcommand{\ayudantia}{{\sc Ayudantía 12.5}}
\newcommand{\tituloayu}{Compilado I3}
\newcommand{\fecha}{7 de junio de 2019}
\newcommand{\sigla}{MAT1203}
\newcommand{\nombre}{Álgebra Lineal}
\newcommand{\profesor}{Camilo Perez}
\newcommand{\ano}{2019}
\newcommand{\semestre}{1}
\newcommand{\mail}{mat1203@ifcastaneda.cl}
\newcommand{\alumno}{Ignacio Castañeda - \mail}

\newcommand{\ev}{\Big|}
\newcommand{\ra}{\rightarrow}
\newcommand{\lra}{\leftrightarrow}
\newcommand{\N}{\mathbb{N}}
\newcommand{\R}{\mathbb{R}}
\newcommand{\Exp}[1]{\mathcal{E}_{#1}}
\newcommand{\List}[1]{\mathcal{L}_{#1}}
\newcommand{\EN}{\Exp{\N}}
\newcommand{\LN}{\List{\N}}
\newcommand{\comment}[1]{}
\newcommand{\lb}{\\~\\}
\newcommand{\eop}{_{\square}}
\newcommand{\hsig}{\hat{\sigma}}
\newcommand{\widesim}[2][1.5]{
	\mathrel{\overset{#2}{\scalebox{#1}[1]{$\sim$}}}
}
\newcommand{\wsim}{\widesim{}}
\newcommand{\lh}{\stackrel{L'H}{=}}

\begin{document}
\thispagestyle{empty}

\begin{minipage}{2cm}
	\includegraphics[width=2cm]{../../../../img/logo.pdf}
	\vspace{0.5cm}
\end{minipage}
\begin{minipage}{\linewidth}
	\begin{tabular}{lrl}
		{\scriptsize\sc Pontificia Universidad Catolica de Chile} & \hspace*{0.7in}Curso: &
		\sigla  - \nombre\\
		{\sc Facultad de Matemáticas}&
		Profesor: & \profesor \\
		{\sc Semestre \ano-\semestre} & Ayudante: & {Ignacio Castañeda}\\
		& {Mail:} & \texttt{\mail}
	\end{tabular}
\end{minipage}

\vspace{-10mm}
\begin{center}
	{\LARGE\bf \ayudantia}\\
	\vspace{0.1cm}
	{\tituloayu}\\
	\vspace{0.1cm}
	\fecha\\
	\vspace{0.4cm}
\end{center}

\begin{preguntas}
\item Sea 
	$$U=\{p(x) \in \mathbb{P}_2 : p(1) + p(0) = p(-1)\}$$
	Determinar una base de $U$.
\begin{solucion}
			Sea $p(x) \in \mathbb{P}_2$ un polinomio de la forma $p(x) = a + bx + cx^2$\\
			\\
			Para que pertenezca a $U$, debe cumplirse que
			$$p(1) + p(0) = p(-1)$$
			$$a+b+c+a=a-b+c$$
			$$a = -2b$$
			Luego,
			$$\begin{array}{rcl}
			a & = & -2b\\
			b & = & b\\
			c & = & c
			\end{array}$$
			Entonces, podemos reescribir el polinomio como
			$$p(x) = -2b + bx + cx^2$$
			$$p(x) = b(x-2) + c(x^2)$$
			Finalmente,
			$$U = Gen\{x-2, x^2\}$$
\end{solucion}
\item Use vectores de coordenadas para probar la independencia lineal del conjunto de polinomios $\{1-2t^2-t^3,t+2t^3,1+t-2t^2\}$.
\begin{solucion}
La base que utilizaremos será
		$$B = \{1,t,t^2,t^3\}$$
		Luego, podemos reescribir cada elemento del conjunto de la siguiente forma
		$$\begin{array}{lcl}
		1-2t-t^3 & \ra & (1,\ \ 0,-2,-1)\\
		t+2t^3 & \ra & (0,\ \ 1,\ \ 0,\ \ 2)\\
		1+t-2t^3 & \ra & (1,\ \ 1,-2,\ \ 0)
		\end{array}$$
		Ahora, para probar que los elementos del conjunto son $L.I.$, basta con probar que los vectores de coordenadas lo son. Para esto, ponemos los vectores en una matriz y pivoteamos:
		$$\begin{bmatrix}
		1 & 0 & 1\\
		0 & 1 & 1\\
		-2 & 0 & -2\\
		-1 & 2 & 0
		\end{bmatrix} \sim
		\begin{bmatrix}
		1 & 0 & 1\\
		0 & 1 & 1\\
		0 & 0 & 1\\
		0 & 0 & 0
		\end{bmatrix}$$
		Con lo que concluimos que son $L.I.$
		$$\blacksquare$$
\end{solucion}
\item Sean $p_1(t) = 1-t^2, p_2(t) = 1+t, p_3(t) = 1+t+t^2$. Se sabe que $\{p_1(t), p_2(t), p_3(t)\}$ es una base de $\mathbb{P}_2$.
\begin{enumerate}[a)]
\item Exprese los polinomios $f(t) = 3-5t + 2t^2$ y $g(t) = 1-3t$ como combinaciones lineales de $\{p_1(t), p_2(t), p_3(t)\}$.
\item Use los vectores de coordenadas encontrados en la parte anterior para determinar si el conjunto $\{p_1(t), f(t), g(t)\}$ es L.I. o L.D.
\end{enumerate}
\begin{solucion}

\begin{enumerate}[a)]
\item Exprese los polinomios $f(t) = 3-5t + 2t^2$ y $g(t) = 1-3t$ como combinaciones lineales de $\{p_1(t), p_2(t), p_3(t)\}$.\\
			\\
			Sea una base 
			$$B = \{p_1(t), p_2(t), p_3(t)\} = $$, lo que estamos buscando son las coordenadas de $f$ y $g$ en la base $B$.\\
			\\
			En primer lugar, veamos $f(t)$. Debemos buscar $a$, $b$, $c$, tal que
			$$f(t) = 3-5t + 2t^2 = ap_1(t) + bp_2(t) + cp_3(t)$$
			Reemplazando y reagrupando,
			$$3-5t + 2t^2 = a(1-t^2) + b(1+t) + c(1+t+t^2)$$
			$$3-5t + 2t^2 = (a+b+c) + (b+c)t + (c-a)t^2$$
			Por lo que
			$$\begin{array}{rl}
			a + b +c & = 3\\
			b + c & =-5\\
			-a+c & =2
			\end{array} \Longrightarrow
			\begin{array}{rl}
			a & = 8\\
			b & =-15\\
			c & = 10
			\end{array}$$
			Luego, las coordenadas de $f(t)$ en $B$ son $(8,-15,10)$.\\
			\\
			Veamos ahora que ocurre con $g(t)$. Nuevamente, buscamos $a$, $b$, $c$, tal que
			$$g(t) = 1-3t =  ap_1(t) + bp_2(t) + cp_3(t)$$
			Esto es,
			$$1-3t = a(1-t^2) + b(1+t) + c(1+t+t^2)$$
			$$1-3t = (a+b+c) + (b+c)t + (c-a)t^2$$
			Lo que se nos lleva al sistema
			$$\begin{array}{rl}
			a + b +c & = 1\\
			b + c & =-3\\
			-a+c & =0
			\end{array} \Longrightarrow
			\begin{array}{rl}
			a & = 4\\
			b & =-7\\
			c & = 4
			\end{array}$$
			Con lo que las coordenadas de $g(t)$ en $B$ son $(4,-7,4)$.
\item Use los vectores de coordenadas encontrados en la parte anterior para determinar si el conjunto $\{p_1(t), f(t), g(t)\}$ es L.I. o L.D.\\
			\\
			Para determinar esto, podemos usar las coordenadas de estos elementos en cualquier base. Por conveniencia, usaremos la base $B$. Aquí, los vectores de coordenadas son
			$$p_1(t) = (1,0,0)$$
			$$f(t) = (8,-15,10)$$
			$$g(t) = (4,-7,4)$$
			Armamos una matriz con ellos y pivoteamos,
			$$\begin{bmatrix}
			1 & 0 & 0 \\
			8 & -15 & 10 \\
			4 & -7 & 4
			\end{bmatrix}
			\sim
			\begin{bmatrix}
			1 & 0 & 0 \\
			0 & 1 & 0 \\
			0 & 0 & 1
			\end{bmatrix}$$
			Por lo que sus el conjunto $\{p_1(t), f(t), g(t)\}$ es $L.I.$
\end{enumerate}
\end{solucion}
\item Sea
$$T\left(\left[ \begin{array}{c}
a\\ b\\ c \end{array} \right] \right) = (a+b+c) + (a-b+2c)x +
(3b-c)x^2
$$
Determine una base para $Nul(T) $ y para $Im (T) $ y las dimensiones de estos subespacios.
\begin{solucion}
Notemos que lo que estamos generando son polinomios de grado menor o igual a 2. La base canónica para estos polinomios es $B=\{1, x, x^2\}$
En primer lugar, obtengamos la matriz asociada a la transformación lineal respecto a las bases canónicas, esto es,
$$A = \begin{bmatrix}
1 & 1 & 1 \\
1 & -1 & 2 \\
0 & 3 & -1
\end{bmatrix}$$
Recordemos que $Nul(T)$ corresponde a la solución del sistema homogeneo, es decir $Ax = 0$, esto es,
$$\begin{bmatrix}
1 & 1 & 1 \\
1 & -1 & 2 \\
0 & 3 & -1
\end{bmatrix} \sim 
\begin{bmatrix}
1 & 0 & 0 \\
0 & 1 & 0 \\
0 & 0 & 1
\end{bmatrix}$$
Por lo que la solución del sistema homogeneo es $Gen\left\{\begin{pmatrix}0 \\ 0 \\ 0 \end{pmatrix}\right\}$. Recordemos que este vector es un vector de coordenadas, sin embargo, al ser el vector nulo, este generado equivale a $Gen\{0\}$.\\

Veamos ahora el $Im (T)$, que corresponde a las bases $LI$. Esto es equivalente a ver las columnas $LI$ de $A$, que ya sabemos que son todas. Por lo tanto, 
$$Im(A) = Gen\left\{\begin{pmatrix}1 \\ 0 \\ 0 \end{pmatrix},\begin{pmatrix}0 \\ 1 \\ 0 \end{pmatrix}, \begin{pmatrix}1 \\ 0 \\ 0 \end{pmatrix}\right\}$$
Luego, traduciendo estos vectores coordenadas, concluimos que
$$Im(T) = Gen\{1, x, x^2\}$$
Finalmente, las dimensiones de $Nul(T)$ y $Im(T)$ son 0 y 3, respectivamente.
\end{solucion}
\item Sean $B_1 = \{1+x,1-x,1+x^2\}$ y $B_2$ bases de $P_2(\R)$ tales que para todo $p \in P_2(\R)$
	$$ \begin{bmatrix}
	1 & 0 & -1\\
	1 & 1 & 0 \\
	0 & 1 & 2
	\end{bmatrix} [p]_{B_1} = [p]_{B_2}$$
	Determine los polinomios que forman la base $B_2$.
\begin{solucion}
Notemos que la matriz que nos dan corresponde a la matriz cambio de base de $B_1$ a $B_2$. Recordemos que una matriz cambio de base, de $B_1$ a $B_2$, corresponde a la matriz formada por las coordenadas de los elementos de $B_1$ en la base $B_2$.
		
		En base a lo anterior, dado que tenemos los vectores de $B_1$, lo que necesitamos es justamente lo contrario, es decir, la matriz cambio de base de $B_2$ a $B_1$. De esta manera, tendriamos las coordenadas de los elementos de $B_2$ en la base $B_1$, por lo que podriamos calcularlos con facilidad, ya que tenemos estos últimos.
		
		Para encontrar esta matriz, basta con obtener la inversa de la matriz cambio de base dada, esto es
		$$ \left[\begin{array}{ccc|ccc}
		1 & 0 & -1 & 1 & 0 & 0\\
		1 & 1 & 0 & 0 & 1 & 0\\
		0 & 1 & 2 & 0 & 0 & 1
		\end{array}\right] \sim 
		\left[\begin{array}{ccc|ccc}
		1 & 0 & 0 & 2 & -1 & 1\\
		0 & 1 & 0 & -2 & 2 & -1\\
		0 & 0 & 1 & 1 & -1 & 1
		\end{array}\right]$$
		con lo que
		$$A_{B_2}^{B_1} = \begin{bmatrix}
		2 & -1 & 1\\
		-2 & 2 & -1\\
		1 & -1 & 1
		\end{bmatrix}$$
		Sea 
		$$B_2 = \{q_1(x), q_2(x), q_3(x)\}$$
		Las columnas de $A_{B_2}^{B_1}$ corresponden a los vectores de coordenada de $q_1$, $q_2$ y $q_3$, respectivamente, en la base $B_1$.
		
		Luego,
		$$q_1(x) = 2(1+x) - 2(1-x) + (1+x^2) = 1+4x+x^2$$
		$$q_2(x) = -(1+x) + 2(1-x) - (1+x^2) = -3x-x^2$$
		$$q_3(x) = (1+x) - (1-x) + (1+x^2) = 1+2x+x^2$$
		Finalmente,
		$$B_2 = \{1+4x+x^2,-3x-x^2,1+2x+x^2\}$$
\end{solucion}
\item Sean $B$ y $C$ bases de un espacio vectorial $V$ y $P = \begin{bmatrix}4 &-1 \\ 6 & -1\end{bmatrix}$ la matriz de cambio de coordenadas tal que $[v]_C = P[v]_B\ \forall v \in V$
\begin{enumerate}[a)]
\item Demuestre que el conjunto $W = \{v \in V: [v]_C = 2[v]_B\}$ es un subespacio de $V$.
\item Si $B=\{v_1,v_2\}$ determine una base para $W$ en términos de la base de $B$.
\end{enumerate}
\begin{solucion}

\begin{enumerate}[a)]
\item Demuestre que el conjunto $W = \{v \in V: [v]_C = 2[v]_B\}$ es subespacio de $V$.
			
			\begin{enumerate}
				\item No vacio $(0\in W)$
				$$\begin{array}{cc}
				[\vec{0}]_C = \vec{0}\\
				2 \cdot [\vec{0}]_B = 2 \cdot \vec{0} = 0
				\end{array} \Longrightarrow \vec{0} = \vec{0} \ra \vec{0} \in W$$	
				\item Suma y multiplicación (a la vez)	
				
				
				Sea $u, v \in W$ y $\alpha, \beta \in \R$, debemos demostrar que $\alpha u + \beta v \in W$
				$$\begin{array}{rcl}
				[\alpha u + \beta v]_C & = & 2 [\alpha u + \beta v]_B \\
				& = & 2 [\alpha u + \beta v]_B \\
				& = & 2 [\alpha u]_B + 2[\beta v]_B \\
				& = & \alpha 2[u]_B + \beta 2[v]_B \\
				& = & \alpha [u]_C + \beta [v]_C \\
				& = & [\alpha u]_C + [\beta v]_C \\
				{[}\alpha u + \beta v]_C & = & [\alpha u +\beta v]_C
				\end{array}
				$$
			\end{enumerate}
			$$\blacksquare$$
\item Si $B=\{v_1,v_2\}$ determine una base para $W$ en términos de la base de $B$.
			
			Sea $B = \{v_1, v_2\}$ la base dada, busquemos una base $B_W$ de $W$\\
			\\
			Sea $v \in W$,
			$$v = x_1v_1 + x_2v_2 \ra x = \begin{pmatrix} x_1 \\ x_2 \end{pmatrix} = [v]_B$$
			Además,
			$$\begin{array}{cclcl}
			[v]_C & = & 2[v]_B & = & 2x \\
			{[}v]_C & = & P[v]_B & = & Px
			\end{array} \Longrightarrow
			2x = Px$$
			Luego,
			$$2x = Px$$
			$$Px - 2x = 0$$
			$$(P - 2I)x = 0$$
			$$\left(\begin{bmatrix}4 &-1 \\ 6 & -1\end{bmatrix} - \begin{bmatrix}2 &0 \\ 0 & 2\end{bmatrix}\right)x = 0$$
			$$\begin{bmatrix}2 &-1 \\ 6 & -3\end{bmatrix}x = 0$$
			Resolviendo el sistema,
			$$\begin{bmatrix}2 &-1 \\ 6 & -3\end{bmatrix} \sim 
			\begin{bmatrix}2 &-1 \\ 0 & 0\end{bmatrix} \ra x_2 = 2x_1$$
			Luego,
			$$v = x_1v_1 + 2x_1v_2 = x_1(v_1 + 2v_2)$$
			Finalmente,
			$$B_W = \{v_1 + 2v_2\}$$
\end{enumerate}
\end{solucion}
\item Calcular los valores y vectores propios de
	$$ A = \begin{bmatrix} 1 & 2 & -1\\ 1 & 0 & 1\\ 4 & -4 & 5\end{bmatrix}$$
\begin{solucion}
Para calcular los valores propios de una matriz, debemos buscar todos los valores de $\lambda$ tales que
		$$A-\lambda I = 0$$
		Luego, para cada valor propio obtenido, debemos resolver el problema
		$$(A-\lambda I)v = 0$$
		Encontrando así los valores propios.\\
		\\
		Pasando ahora al ejercicio,
		$$\left|\begin{bmatrix}
		1-\lambda & 2 & -1\\
		1 & -\lambda & 1\\
		4 & -4 & 5-\lambda 
		\end{bmatrix}\right| = 0$$
		Usando cofactores, tenemos que
		$$\left|\begin{bmatrix}
		1-\lambda & 2 & -1\\
		1 & -\lambda & 1\\
		4 & -4 & 5-\lambda 
		\end{bmatrix}\right| = (1-\lambda)(-\lambda(5-\lambda) - -4) - 2((5-\lambda)-4) -(-4 - -4\lambda)$$
		Simplificando,
		$$=-\lambda^3 + 6\lambda^2 -11\lambda + 6 = -(\lambda - 1)(\lambda - 2) (\lambda - 3)$$
		Por lo que los valores propios son
		$$\lambda_1 =  1, \quad \lambda_2 = 2, \quad \lambda_3 = 3$$
		Busquemos ahora los vectores propios asociados a cada valor propio
		\begin{enumerate}[1)]
			\item $\lambda_1 = 1$\\
			\\
			Debemos resolver el sistema $(A-I)v = 0$, esto es
			\small$$\begin{bmatrix}
			0 & 2 & -1\\
			1 & -1 & 1\\
			4 & -4 & 4
			\end{bmatrix} \sim 
			\begin{bmatrix}
			1 & -1 & 1\\
			0 & 2 & -1\\
			4 & -4 & 4
			\end{bmatrix} \sim 
			\begin{bmatrix}
			1 & -1 & 1\\
			0 & 2 & -1\\
			0 & 0 & 0
			\end{bmatrix} \sim 
			\begin{bmatrix}
			1 & -1 & 1\\
			0 & 1 & -\frac{1}{2}\\
			0 & 0 & 0
			\end{bmatrix} \sim 
			\begin{bmatrix}
			1 & 0 & \frac{1}{2}\\
			0 & 1 & -\frac{1}{2}\\
			0 & 0 & 0
			\end{bmatrix}$$
			Por lo que
			$$\begin{array}{rcl}
			x_1 & = & -\dfrac{1}{2}x_3\\
			x_2 & = & \dfrac{1}{2}x_3\\
			x_3 & = & x_3
			\end{array} \Longrightarrow
			v_1 = \begin{pmatrix}
			-\frac{1}{2}\\
			\frac{1}{2}\\
			1
			\end{pmatrix} \rightarrow
			v_1 = \begin{pmatrix}
			-1\\
			1\\
			2
			\end{pmatrix}$$
			La solución del sistema es el generado de $v$. Por esto, lo simplificamos al final para obtener un vector propio sin fracciones.
			
			\item $\lambda_2 = 2$\\
			\\
			Debemos resolver el sistema $(A-2I)v = 0$, esto es
			\small$$\begin{bmatrix}
			-1 & 2 & -1\\
			1 & -2 & 1\\
			4 & -4 & 3
			\end{bmatrix} \sim 
			\begin{bmatrix}
			1 & 0 & \frac{1}{2}\\
			0 & 1 & -\frac{1}{4}\\
			0 & 0 & 0
			\end{bmatrix}$$
			Por lo que
			$$\begin{array}{rcl}
			x_1 & = & -\dfrac{1}{2}x_3\\
			x_2 & = & \dfrac{1}{4}x_3\\
			x_3 & = & x_3
			\end{array} \Longrightarrow
			v_2 = \begin{pmatrix}
			-\frac{1}{2}\\
			\frac{1}{4}\\
			1
			\end{pmatrix} \rightarrow
			v_2 = \begin{pmatrix}
			-2\\
			1\\
			4
			\end{pmatrix}$$
			
			\item $\lambda_3 = 3$\\
			\\
			Debemos resolver el sistema $(A-3I)v = 0$, esto es
			\small$$\begin{bmatrix}
			-2 & 2 & -1\\
			1 & -3 & 1\\
			4 & -4 & 2
			\end{bmatrix} \sim 
			\begin{bmatrix}
			1 & 0 & \frac{1}{4}\\
			0 & 1 & -\frac{1}{4}\\
			0 & 0 & 0
			\end{bmatrix}$$
			Por lo que
			$$\begin{array}{rcl}
			x_1 & = & -\dfrac{1}{4}x_3\\
			x_2 & = & \dfrac{1}{4}x_3\\
			x_3 & = & x_3
			\end{array} \Longrightarrow
			v_3 = \begin{pmatrix}
			-\frac{1}{4}\\
			\frac{1}{4}\\
			1
			\end{pmatrix} \rightarrow
			v_3 = \begin{pmatrix}
			-1\\
			1\\
			4
			\end{pmatrix}$$
		\end{enumerate}
		Finalmente, los vectores propios son
		$$v_1 = \begin{pmatrix}
		-1\\
		1\\
		2
		\end{pmatrix}, \quad
		v_2 = \begin{pmatrix}
		-2\\
		1\\
		4
		\end{pmatrix}, \quad
		v_3 = \begin{pmatrix}
		-1\\
		1\\
		4
		\end{pmatrix}$$
\end{solucion}
\item Determine si las siguientes afirmaciones son Verdaderas o Falsas.
\begin{enumerate}[a)]
\item El espacio fila de $AB$ es subespacio del espacio fila de $B$.
\item Dada una base en un espacio vectorial $V$, el vector coordenado de un vector de $V$ con respecto a esa base, es único.
\item Si $A$ y $B$ son similares y $\lambda$ es valor propio de $A$ entonces $\lambda$ es también valor propio de $B$.
\end{enumerate}
\begin{solucion}

\begin{enumerate}[a)]
\item El espacio fila de $AB$ es subespacio del espacio fila de $B$.
			
			$$C = AB  \ra C^T = B^TA^T \ra Col(C^T) = Col(B^TA^T)$$
			Como al multiplicar una matriz por otra podemos, en el mejor de los casos, obtener la misma dimensión de antes,
			$$Col(C^T) \subset Col(B^T) \ra Fil(C) \subset Fil(B)$$
			Luego,
			$$Fil(AB) \subset Fil(B)$$
			Con lo que concluimos que el espacio fila de $AB$ es subespacio del espacio fila de $B$. Entonces, la afirmación es {\bf Verdadera}.
\item Dada una base en un espacio vectorial $V$, el vector coordenado de un vector de $V$ con respecto a esa base, es único.
			
			Los vectores de una base son $L.I.$, por lo que existe una única combinación lineal para generar cada vector. Luego, la afirmación es {\bf Verdadera}.
\item Si $A$ y $B$ son similares y $\lambda$ es valor propio de $A$ entonces $\lambda$ es también valor propio de $B$.
			
			Tomemos, las matrices
			$$A = \begin{bmatrix}
			1 & 0 \\ 0 & 1
			\end{bmatrix}, \quad 
			B = \begin{bmatrix}
			2 & 0 \\ 0 & 2
			\end{bmatrix}$$
			Los valores propios de $A$ son $\lambda = 1$ (multiplicidad 2) y los de $B$ son $\lambda = 2$ (multiplicidad 2), por lo que la afirmación es {\bf Falsa}.
\end{enumerate}
\end{solucion}
\item Sea $A$ una matriz invertible, y sea $\lambda$ un valor propio de $A$. Demuestre que $\lambda \neq 0$ y que $\dfrac{1}{\lambda}$ es un valor propio de $A^{-1}$.
\begin{solucion}

		P.D. $\lambda \neq 0 \wedge \dfrac{1}{\lambda} \text{ es VP de }A^{-1}$
		
		Digamos que $\lambda = 0$. Luego, como $\lambda = 0$ es un VP, tenemos que
		$$det(A-\lambda I) = 0$$
		$$det(A-0 I) = 0$$
		$$det(A) = 0$$
		Pero $det(A) \neq 0$, ya que $A$ es invertible. Entonces, $\lambda \neq 0$.
		
		Notemos también que
		$$A^{-1}x 
		= A^{-1}\left(\dfrac{1}{\lambda}(\lambda x)\right)
		= A^{-1}\dfrac{1}{\lambda}(\lambda x)
		= \dfrac{1}{\lambda}A^{-1}(A x)
		= \dfrac{1}{\lambda} x$$
		$$A^{-1}x 
		= \dfrac{1}{\lambda} x$$
		De donde se desprende que $\dfrac{1}{\lambda}$ es valor propio de $A^{-1}$.
\end{solucion}
\item Sea $A$ una matriz de $2\times 2$ de rango 1 tal que $A\begin{bmatrix} 1 \\ 2\end{bmatrix} = \begin{bmatrix} 2 \\ 4 \end{bmatrix}$. ¿Es $A$ diagonalizable? Justifique.
\begin{solucion}
Notemos que
		$$A\begin{bmatrix} 1 \\ 2\end{bmatrix} = \begin{bmatrix} 2 \\ 4 \end{bmatrix}
		= A\begin{bmatrix} 1 \\ 2\end{bmatrix} = 2\begin{bmatrix} 1 \\ 2 \end{bmatrix}$$
		Luego, $\lambda_1 = 2$ es valor propio de la matriz $A$ y $v_1 = \begin{bmatrix} 1 \\ 2\end{bmatrix}$ es el vector propio asociado.
		
		Como $A$ es de rango 1, $\exists u \neq 0$ tal que $Au = $. Para ese $u$ se cumple que
		$$Au = 0 \ra Au = 0 u$$
		Luego, $\lambda_2 = 0$ es valor propio de la matriz $A$ y $v_2 = u$ es el vector propio asociado.
		
		Finalmente, como la multiplicidad algebraica es igual a la multiplicidad geométrica, la matriz es diagonalizable.
\end{solucion}
\item Diagonalice la matriz
	$$M = \begin{bmatrix}
	1 & 0 & 0\\
	1 & 1 & 2 \\
	1 & 0 & 3
	\end{bmatrix}$$
	y encuentre una matriz $N$ tal que $N^3 = M$.
\begin{solucion}
Buscamos los valores propios,
		$$det(M-\lambda I) = \left|\begin{bmatrix}
		1-\lambda & 0 & 0\\
		1 & 1-\lambda & 2 \\
		1 & 0 & 3-\lambda
		\end{bmatrix}\right|
		= (1-\lambda)(1-\lambda)(3-\lambda) = 0$$
		$$\lambda_1 = 1 \ra \text{multiplicidad 2}$$
		$$\lambda_2 = 3 \ra \text{multiplicidad 1}$$
		Buscamos ahora los vectores propios,
		\begin{itemize}
			\item $\lambda_1 = 1$
			
			$$(A-I)x = 0$$
			$$\begin{bmatrix}
			0 & 0 & 0\\
			1 & 0 & 2 \\
			1 & 0 & 2
			\end{bmatrix} \ra v_1 = \begin{pmatrix}
			2 \\ 0 \\ -1
			\end{pmatrix}, \quad
			v_2 = \begin{pmatrix}
			0 \\ 1 \\ 0
			\end{pmatrix}$$
			
			\item $\lambda_2 = 3$
			
			$$(A-3I)x = 0$$
			$$\begin{bmatrix}
			-2 & 0 & 0\\
			1 & -2 & 2 \\
			1 & 0 & 0
			\end{bmatrix} \ra v_3 = \begin{pmatrix}
			0 \\ 1 \\ 1
			\end{pmatrix}$$
		\end{itemize}
			Recordemos ahora que para diagonalizar una matriz debemos expresarla de la forma
			$$M = PDP^{-1} \ra P = \begin{bmatrix}
			v_1 & v_2 & v_3
			\end{bmatrix}, \quad
			D = \begin{bmatrix}
			\lambda_1 & 0 & 0 \\
			0 & \lambda_2 & 0 \\
			0 & 0 & \lambda_3
			\end{bmatrix}$$
			Es importante recordar que cada vector propio debe ir asociado con su respectivo valor propio (misma columna).
			
			En nuestro ejercicio, debemos considerar $\lambda_1 = \lambda_2 = 1$ y $\lambda_3 = 3$. Entonces,
			$$P = \begin{bmatrix}
			2 & 0 & 0 \\
			0 & 1 & 1 \\
			-1 & 0 & 1
			\end{bmatrix}, \quad
			D = \begin{bmatrix}
			1 & 0 & 0 \\
			0 & 1 & 0 \\
			0 & 0 & 3
			\end{bmatrix}, \quad
			P^{-1} = \begin{bmatrix}
			\frac{1}{2} & 0 & 0 \\
			-\frac{1}{2} & 1 & -1 \\
			\frac{1}{2} & 0 & 1
			\end{bmatrix}$$
			Sea
			$$R = \sqrt[3]{D} = \begin{bmatrix}
			1 & 0 & 0 \\
			0 & 1 & 0 \\
			0 & 0 & \sqrt[3]{3}
			\end{bmatrix}$$
			y sea
			$$N = PRP^{-1}$$
			Notemos que	
			$$N^3 = (PRP^{-1}) = PR\cancel{P^{-1}P}R\cancel{P^{-1}P}RP^{-1} = PR^3P^{-1} = PDP^{-1} = M$$
			Finalmente, $N = P\sqrt[3]{D}P^{-1}$ es la matriz buscada.
\end{solucion}
\item Diagonalice $A =
	\begin{bmatrix} 
	-1 & -2 & 2 \\
	0 & -1 & 0 \\
	0 & -2 & 1
	\end{bmatrix}$ y diagonalice $B = A^{10} + A - I$ 
\begin{solucion}
Buscamos los valores propios,
		$$det(A-\lambda I) = \left|\begin{bmatrix}
		-1-\lambda & -2 & 2 \\
		0 & -1-\lambda & 0 \\
		0 & -2 & 1-\lambda
		\end{bmatrix}\right|
		= (1+\lambda)^2(1-\lambda) = 0$$
		$$\lambda_1 = -1 \ra \text{multiplicidad 2}$$
		$$\lambda_2 = 1 \ra \text{multiplicidad 1}$$
		Buscamos ahora los vectores propios,
		\begin{itemize}
			\item $\lambda_1 = -1$
			
			$$(A+I)x = 0 \ra v_1 = \begin{pmatrix}
			1 \\ 0 \\ 1
			\end{pmatrix}$$
			
			\item $\lambda_2 = 1$
			
			$$(A-I)x = 0 \ra v_2 = \begin{pmatrix}
			1 \\ 0 \\ 0
			\end{pmatrix}, \quad
			v_2 = \begin{pmatrix}
			0 \\ 1 \\ 1
			\end{pmatrix}$$
		\end{itemize}
		Luego,
		$$A = PDP^{-1} \ra P = \begin{bmatrix}
		1 & 1 & 0 \\
		0 & 0 & 1 \\
		1 & 0 & 1
		\end{bmatrix}, \quad
		D = \begin{bmatrix}
		1 & 0 & 0 \\
		0 & -1 & 0 \\
		0 & 0 & -1
		\end{bmatrix}, \quad
		P^{-1} = \begin{bmatrix}
		0 & -1 & 1 \\
		1 & 1 & -1 \\
		0 & 1 & 0
		\end{bmatrix}$$
		Reemplacemos ahora en $B$,
		$$\begin{array}{lcl}
		B & = & A^{10} + A - I \\
		 & = & (PDP^{-1})^{10} + PDP^{-1} - I \\
		 & = & (PDP^{-1})^{10} + PDP^{-1} - PIP^{-1} \\
		B & = & P(D^10 + D - I)P^{-1}
		\end{array}$$
		Además, sabemos que
		$$D^{10} = \begin{bmatrix}
		1^{10} & 0 & 0 \\
		0 & (-1)^{10} & 0 \\
		0 & 0 & (-1)^{10}
		\end{bmatrix}
		 = \begin{bmatrix}
		1 & 0 & 0 \\
		0 & 1 & 0 \\
		0 & 0 & 1
		\end{bmatrix} = I$$
		Por lo tanto,
		$$B = P(I + D - I)P^{-1} = PDP^{-1}$$
		Con lo que concluimos que $A=B$, por lo que la diagonalización de $A$ también es diagonalización de $B$.
\end{solucion}
\item La matriz $\begin{bmatrix} 1 & -2 \\ 1 & 3 \end{bmatrix}$ actúa sobre $\mathbb{C}^2$. Determine los valores propios y una base para cada espacio propio de $\mathbb{C}^2$.
\begin{solucion}
En primer lugar, buscamos los valores propios,
		$$det(A-\lambda I) = \left|\begin{bmatrix} 
		1-\lambda & -2 \\ 
		1 & 3-\lambda \end{bmatrix}\right| = (1-\lambda)(3-\lambda) + 2 = \lambda^2 - 4\lambda + 5$$
		$$\lambda_{1,2} = 2 \pm i$$
		Como los valores propios son un par de complejos conjugados, los vectores propios asociados a cada uno también lo serán, por lo que basta con buscar un solo vector propio.
		
		Para $\lambda = 2+i$,
		$$(A-(2+i)\lambda)x = 0 \ra 
		\begin{bmatrix} 
		-1-i & -2 \\ 
		1 & 1-i 
		\end{bmatrix} \sim 
		\begin{bmatrix} 
		-1-i & -2 \\ 
		0 & 0
		\end{bmatrix}$$
		$$(-1-i)x_1 - 2x_2 = 0 \ra \begin{array}{lcl}
		x_1 & = & x_1 \\
		x_2 & = & \dfrac{-1-i}{2}x_1
		\end{array}$$
		Luego,
		$$v_1 = \begin{pmatrix}
		2 \\
		-1-i
		\end{pmatrix}, \quad
		v_2 = \begin{pmatrix}
		2 \\
		-1+i
		\end{pmatrix} $$
		Por último, las bases de los espacios propios están formadas por los vectores propios asociados a cada valor propio, es decir
		$$E_{2+i} = Gen\left\{\begin{pmatrix}
		2 \\
		-1-i
		\end{pmatrix}\right\}, \quad 
		E_{2-i} = Gen\left\{\begin{pmatrix}
		2 \\
		-1+i
		\end{pmatrix}\right\}$$
\end{solucion}
\item Sea $A$ una matriz de $5 \times 5$ tal que $A^t = -A$ y $q$ el polinomio dado por $q(x) = 2-x^2+4x^3$. Demuestre que $2$ es valor propio de la matriz $q(A)$.
\begin{solucion}
Sabemos que
		$$det(A) = det(A^T) \wedge det(A^T) = det(-A)$$
		$$det(A) = det(-A)$$
		$$det(A) = -det(A)$$
		$$det(A) = 0$$
		Luego, existe $u \neq 0 | Au = 0$
		
		Notemos que
		$$\begin{array}{lcl}
		q(A) & = & 2I - A^2 + 4A^3\\
		q(A) u & = & (2I - A^2 + 4A^3)u\\
		& = & 2Iu - A^2u + 4A^3u \\
		& = & 2Iu - A(Au) + 4A^2(Au) \\
		& = & 2Iu - A0 + 4A^20 \\
		& = & 2Iu \\
		q(A) u & = & 2u
		\end{array}$$
		Luego, $\lambda = 2 $ es $VP$ de $q(A)$
\end{solucion}
\item Sean $u, v$ dos vectores ortogonales en $\R^n$ tales que $||u|| = 1, ||v|| = \sqrt[]{3/2}$. Demuestre que el conjunto 
	$$B = \{u-v, 3u+2v\}$$
	es ortogonal y encuentre las coordenadas del vector $4u - 9v$ respecto al conjunto $B$.
\begin{solucion}
Para demostrar que $B$ es ortogonal, debemos demostrar que todos sus elementos son ortogonales entre si, es decir, debemos demoestrar que
		$$(u-v) \cdot (3u+2v) = 0$$
		Notemos que como $u$ y $v$ son ortogonales entre si, $u \cdot v = 0$. Luego,
		$$\begin{array}{rcl}
		(u-v) \cdot (3u+2v) & = & 3u \cdot u + 2 u \cdot v - 3u \cdot v - 2v \cdot v\\
		& = & 3||u||^2 + 2 (u \cdot v) - 3(u \cdot v) - 2||v||^2 \\
		& = & 3 + 2 \cdot 0 - 3 \cdot 0 - 2(\ \sqrt[]{3/2})^2 \\
		& = & 3 - 3 \\
		& = & 0
		\end{array}$$
		$$\blacksquare$$
		Ahora, para buscar las coordenadas de $4u - 9v$ respecto al conjunto $B$, debemos buscar $\alpha, \beta \in \R$, tal que
		$$\alpha(u-v) + \beta(3u+2v) = 4u-9v$$
		Reordenando,
		$$(\alpha+3\beta)u + (2\beta -\alpha)v= 4u -9v$$
		Luego, debemos resolver el sistema
		$$\begin{array}{rcl}
		\alpha+3\beta & = & 4\\
		-\alpha+2\beta & = & -9
		\end{array}$$
		Resolviendolo, obtenemos que
		$$\alpha = 7 \quad y \quad \beta = -1$$
		Finalmente,
		$$[4u-9v]_B = \begin{pmatrix}
		7 \\ -1
		\end{pmatrix}$$
\end{solucion}
\item Demuestre que si $P$ es una matriz ortogonal de  $n \times n$, entonces para todo $x, y \in \R^n$ se tiene que $Px \cdot Py = x \cdot y$
\begin{solucion}
Como $P$ es ortogonal, se cumple que $P^TP = I$. Luego,
		$$Px \cdot Py = (Px)^T(Py) = x^TP^TPy = x^TIx= x^Ty = x \cdot y$$
\end{solucion}
\item Diagonalice ortogonalmente
	$$M = \begin{bmatrix}
	1 & 0 & 1 \\
	0 & 2 & 0 \\
	1 & 0 & 1
	\end{bmatrix}$$
\begin{solucion}
Para hacer esto debemos hacer un procedimiento similar al que realizamos cuando queremos diagonalizar una matriz, es decir, buscar $P$ y $D$ tal que $M = PDP^{-1}$, con la diferencia de que en este caso $P$ debe ser ortogonal.
		
		Comenzamos buscando los valores propios, con lo que obtendremos
		$$\lambda_1 = 0 \ra \text{multiplicidad 1}$$
		$$\lambda_2 = 2 \ra \text{multiplicidad 2}$$
		Luego, buscamos los vectores propios asociados a cada uno de estos valores propios, los que son
		$$\lambda_1 = 0 \ra v_1 = \begin{pmatrix}
		1 \\ 0 \\ -1
		\end{pmatrix}$$
		$$\lambda_2 = 0 \ra v_2 = \begin{pmatrix}
		0 \\ 1 \\ 0
		\end{pmatrix}, v_3 = \begin{pmatrix}
		1 \\ 0 \\ 1
		\end{pmatrix}$$
		Notemos que todos estos vectores son ortogonales entre si, sin embargo, necesitamos que sean ortonormales. Recordemos que un vector propio será cualquier multiplo de los vectores calculados anteriormente, por lo que para obtener vectores ortonormales, basta con dividir cada uno por su modulo. De esta forma, nuestros nuevos vectores serán
		$$v_1 = \begin{pmatrix}
		1/\ \sqrt[]{2} \\ 0 \\ -1/\ \sqrt[]{2}
		\end{pmatrix}, v_2 = \begin{pmatrix}
		0 \\ 1 \\ 0
		\end{pmatrix}, v_3 = \begin{pmatrix}
		1/\ \sqrt[]{2} \\ 0 \\ 1/\ \sqrt[]{2}
		\end{pmatrix}$$
		Luego, $M$ se diagonaliza con
		$$P = \begin{bmatrix}
		1/\ \sqrt[]{2} & 0 & 1/\ \sqrt[]{2} \\
		0 & 1 & 0 \\
		-1/\ \sqrt[]{2} & 0 & 1/\ \sqrt[]{2}
		\end{bmatrix}, \quad D = \begin{bmatrix}
		0 & 0 & 0 \\
		0 & 2 & 0 \\
		0 & 0 & 2
		\end{bmatrix}$$
\end{solucion}
\item Sean $B = \begin{bmatrix} 1 & -1 \\ 0 & 0 \end{bmatrix}$ y $C = \{X \in M_2(\R)|BX=0\}$. Determine una base y la dimensión de $C$.
\begin{solucion}
Digamos que $X = \begin{bmatrix}
a & b \\c & d
\end{bmatrix}$.\\

Luego, debemos encontrar $\begin{bmatrix}
a & b \\c & d
\end{bmatrix}$ tal que
$$\begin{bmatrix} 1 & -1 \\ 0 & 0 \end{bmatrix}
\begin{bmatrix}
a & b \\c & d
\end{bmatrix}
= 0$$
$$a-c = 0 \qquad b - d = 0$$
$$a = c \qquad b = d$$
Por lo tanto,
$$X = \begin{bmatrix}
a & b \\
a & b
\end{bmatrix} =
a
\begin{bmatrix}
1 & 0 \\
1 & 0
\end{bmatrix} + 
b
\begin{bmatrix}
0 & 1 \\
0 & 1
\end{bmatrix} =
Gen\left\{
\begin{bmatrix}
1 & 0 \\
1 & 0
\end{bmatrix},
\begin{bmatrix}
0 & 1 \\
0 & 1
\end{bmatrix}
\right\}
$$
Finalmente, una base para $C$ es $\left\{
\begin{bmatrix}
1 & 0 \\
1 & 0
\end{bmatrix},
\begin{bmatrix}
0 & 1 \\
0 & 1
\end{bmatrix}
\right\}$ y su dimensión es 2.
\end{solucion}
\item Sea $T: \mathbb{P}_2 \ra \mathbb{P}_2$ una transformación lineal cuya matriz de transformación respecto a la base $B = \{1, t, t^2\}$ está dada por
$$[T]_B = \begin{bmatrix}
1 & -1 & 1\\
0 & 2 & -1\\
-1 & 3 & -2
\end{bmatrix}$$
Determine $T(t^2+1)$ y el espacio Nulo de $T$.
\begin{solucion}
Notemos que esta matriz lo que hace es modificar las coordenadas de un elemento $p$ y convertirlas en las coordenadas de $T(p)$, por lo que para determinar $T(t^2+1)$ primero debemos determinar sus coordenadas, esto es
$$[t^2+1]_B = \begin{pmatrix}
1 \\ 0 \\1
\end{pmatrix}$$
Luego, para obtener $T(t^2+1)$, calculamos
$$[T]_B[t^2+1]_B = \begin{bmatrix}
1 & -1 & 1\\
0 & 2 & -1\\
-1 & 3 & -2
\end{bmatrix}
\begin{pmatrix}
1 \\ 0 \\1
\end{pmatrix}
=
\begin{pmatrix}
2 \\ -1 \\3
\end{pmatrix}$$
Por lo tanto,
$$T(t^2+1) = 2-t-3t^2$$
Para determinar el espacio Nulo de $T$, debemos calcular el espacio Nulo de $[T]_B$ y a partir de eso obtener el espacio Nulo de $T$.
$$\begin{bmatrix}
1 & -1 & 1\\
0 & 2 & -1\\
-1 & 3 & -2
\end{bmatrix} \wsim
\begin{bmatrix}
1 & -1 & 1\\
0 & 2 & -1\\
0 & 0 & 0
\end{bmatrix}$$
De aqui tenemos que
$$Nul([T]_B) = Gen\left\{\begin{pmatrix}
-1 \\ 1 \\2
\end{pmatrix}\right\}$$
Finalmente,
$$Nul(T) = Gen\{-1+t+2t^2\}$$
\end{solucion}
\end{preguntas}
\end{document}