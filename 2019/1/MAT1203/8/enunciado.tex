\documentclass[12pt]{article}

\usepackage{fullpage}
\usepackage{graphicx}
\usepackage{amssymb}
\usepackage{amsmath}
\usepackage[none]{hyphenat}
\usepackage{parskip}
\usepackage[spanish]{babel}
\usepackage[utf8]{inputenc}
\usepackage{hyperref}
\usepackage{fancyhdr}
\usepackage{tasks}
\usepackage{mdframed}
\usepackage{xcolor}
\usepackage{pgfplots}
\usepackage[makeroom]{cancel}
\usepackage{multicol}
\usepackage[shortlabels]{enumitem}
\usepackage{stackrel}
\usepackage{tkz-tab}
\usepackage{xpatch}
\xpatchcmd{\tkzTabLine}{$0$}{$\bullet$}{}{}

\setlength{\headheight}{10pt}
\setlength{\headsep}{10pt}
\pagestyle{fancy}
\rhead{\ayudantia \ - \alumno}
\tikzset{t style/.style={style=solid}}

\newcommand*{\mybox}[2]{\colorbox{#1!30}{\parbox{.98\linewidth}{#2}}}

\newenvironment{solucion}
{\begin{mdframed}[backgroundcolor=black!10]
		{\bf Solución:}\\
	}
	{
	\end{mdframed}
}

\newenvironment{alternativas}[1]
{\begin{multicols}{#1}
		\begin{enumerate}[a)]
		}
		{
		\end{enumerate}
	\end{multicols}
}

\newenvironment{preguntas}
{\begin{enumerate}\itemsep12pt
	}
	{
	\end{enumerate}
}

\newcommand{\ayudantia}{{\sc Ayudantía 8}}
\newcommand{\tituloayu}{Repaso I2}
\newcommand{\fecha}{2 de mayo de 2019}
\newcommand{\sigla}{MAT1203}
\newcommand{\nombre}{Álgebra Lineal}
\newcommand{\profesor}{Camilo Perez}
\newcommand{\ano}{2019}
\newcommand{\semestre}{1}
\newcommand{\mail}{mat1203@ifcastaneda.cl}
\newcommand{\alumno}{Ignacio Castañeda - \mail}

\newcommand{\ev}{\Big|}
\newcommand{\ra}{\rightarrow}
\newcommand{\lra}{\leftrightarrow}
\newcommand{\N}{\mathbb{N}}
\newcommand{\R}{\mathbb{R}}
\newcommand{\Exp}[1]{\mathcal{E}_{#1}}
\newcommand{\List}[1]{\mathcal{L}_{#1}}
\newcommand{\EN}{\Exp{\N}}
\newcommand{\LN}{\List{\N}}
\newcommand{\comment}[1]{}
\newcommand{\lb}{\\~\\}
\newcommand{\eop}{_{\square}}
\newcommand{\hsig}{\hat{\sigma}}
\newcommand{\widesim}[2][1.5]{
	\mathrel{\overset{#2}{\scalebox{#1}[1]{$\sim$}}}
}
\newcommand{\wsim}{\widesim{}}
\newcommand{\lh}{\stackrel{L'H}{=}}

\begin{document}
\thispagestyle{empty}

\begin{minipage}{2cm}
	\includegraphics[width=2cm]{../../../../img/logo.pdf}
	\vspace{0.5cm}
\end{minipage}
\begin{minipage}{\linewidth}
	\begin{tabular}{lrl}
		{\scriptsize\sc Pontificia Universidad Catolica de Chile} & \hspace*{0.7in}Curso: &
		\sigla  - \nombre\\
		{\sc Facultad de Matemáticas}&
		Profesor: & \profesor \\
		{\sc Semestre \ano-\semestre} & Ayudante: & {Ignacio Castañeda}\\
		& {Mail:} & \texttt{\mail}
	\end{tabular}
\end{minipage}

\vspace{-10mm}
\begin{center}
	{\LARGE\bf \ayudantia}\\
	\vspace{0.1cm}
	{\tituloayu}\\
	\vspace{0.1cm}
	\fecha\\
	\vspace{0.4cm}
\end{center}

\begin{preguntas}
\item Sea $A$ un matriz tal que
$$A^3 = 
\begin{bmatrix}
2 & 0 & 2\\
-1 & 1 & 2\\
2 & -1 & -1
\end{bmatrix}$$
\begin{enumerate}[a)]
\item Calcule el determinante de la matriz $A^3$ mediante desarrollo por cofactores.
\item Demuestre que $A$ no es invertible.
\end{enumerate}
\item Calcular el determinante de siguiente matriz de $n\times n$
	$$ \begin{bmatrix}
	-3 & x & x & \dots & x \\
	x & -3 & x & \dots & x \\
	x & x & -3 & \dots & x \\
	\vdots & \vdots & \vdots & \ddots & \vdots \\
	x & x & x & \dots & -3
	\end{bmatrix}$$
\item La matriz $A = \begin{bmatrix}
	1 & 3 & 4 & -1 & 2\\
	2 & 6 & 6 & 0 & -3\\
	3 & 9 & 3 & 6 & -3\\
	3 & 9 & 0 & 9 & 0
	\end{bmatrix}$ es equivalente por filas a $B = \begin{bmatrix}
	1 & 3 & 4 & -1 & 2\\
	0 & 0 & 1 & -1 & 1\\
	0 & 0 & 0 & 0 & -8\\
	0 & 0 & 0 & 0 & 0
	\end{bmatrix}$. Sin calcular bases, determine las dimensiones de Nul $A$, Col $A$, Col $A^T$ y Nul $A^T$.
\item Sea $A$ una matriz de $n \times n$ tal que existe una matriz $B \neq 0$ tal que $AB = 0$.\\
Demuestre que $A$ no es invertible.
\item Sea $A$ una matriz de $5 \times 3$ tal que existe una matriz $C$ de $3 \times 5$ tal que $CA = I_3$ y sea $b \in R^5$ tal que la ecuación $Ax = b$ tiene solución.\\
Demuestra que $A$ es invertible.
\item Sea $V = \mathbb{P}_3$ y sea
$$W = \{p(x) \in V: p(1) = p(0) =0\}$$
Demostrar que $W$ es subespacio de $V$.
\newpage
\item En cada caso, determine si la afirmación es verdadera o falsa y justifique su respuesta.
\begin{enumerate}[a)]
\item El subconjunto de $\R^2$
		$$E=\{(x_1, x_2) \in \R^2:9x_1^2+2x_2^2 \leq 4\}$$
		es un subespacio vectorial de $\R^2$.
\item Si $A$ es una matriz de $n \times n$ tal que $A^2=A$, entonces $Col(A) \cap Nul(A) = \{0\}$.
\item Si $n$ es impar y $A$ es una matriz de $n\times n$ que satisface $A^T = -A$ entonces $Nul(A) \neq \{0\}$.
\end{enumerate}
\end{preguntas}
\end{document}