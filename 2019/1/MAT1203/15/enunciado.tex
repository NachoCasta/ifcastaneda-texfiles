\documentclass[12pt]{article}

\usepackage{fullpage}
\usepackage{graphicx}
\usepackage{amssymb}
\usepackage{amsmath}
\usepackage[none]{hyphenat}
\usepackage{parskip}
\usepackage[spanish]{babel}
\usepackage[utf8]{inputenc}
\usepackage{hyperref}
\usepackage{fancyhdr}
\usepackage{tasks}
\usepackage{mdframed}
\usepackage{xcolor}
\usepackage{pgfplots}
\usepackage[makeroom]{cancel}
\usepackage{multicol}
\usepackage[shortlabels]{enumitem}
\usepackage{stackrel}
\usepackage{tkz-tab}
\usepackage{xpatch}
\usepackage{tkz-euclide}
\usetkzobj{all}
\xpatchcmd{\tkzTabLine}{$0$}{$\bullet$}{}{}

\setlength{\headheight}{10pt}
\setlength{\headsep}{10pt}
\pagestyle{fancy}
\rhead{\ayudantia \ - \alumno}
\tikzset{t style/.style={style=solid}}

\newcommand*{\mybox}[2]{\colorbox{#1!30}{\parbox{.98\linewidth}{#2}}}

\newenvironment{solucion}
{\begin{mdframed}[backgroundcolor=black!10]
		{\bf Solución:}\\
	}
	{
	\end{mdframed}
}

\newenvironment{alternativas}[1]
{\begin{multicols}{#1}
		\begin{enumerate}[a)]
		}
		{
		\end{enumerate}
	\end{multicols}
}

\newenvironment{preguntas}
{\begin{enumerate}\itemsep12pt
	}
	{
	\end{enumerate}
}

\newcommand{\ayudantia}{{\sc Ayudantía 15}}
\newcommand{\tituloayu}{Compilado Álgebra Lineal}
\newcommand{\fecha}{22 de junio de 2019}
\newcommand{\sigla}{MAT1203}
\newcommand{\nombre}{Álgebra Lineal}
\newcommand{\profesor}{Camilo Perez}
\newcommand{\ano}{2019}
\newcommand{\semestre}{1}
\newcommand{\mail}{mat1203@ifcastaneda.cl}
\newcommand{\alumno}{Ignacio Castañeda - \mail}

\newcommand{\ev}{\Big|}
\newcommand{\ra}{\rightarrow}
\newcommand{\lra}{\leftrightarrow}
\newcommand{\N}{\mathbb{N}}
\newcommand{\R}{\mathbb{R}}
\newcommand{\Exp}[1]{\mathcal{E}_{#1}}
\newcommand{\List}[1]{\mathcal{L}_{#1}}
\newcommand{\EN}{\Exp{\N}}
\newcommand{\LN}{\List{\N}}
\newcommand{\comment}[1]{}
\newcommand{\lb}{\\~\\}
\newcommand{\eop}{_{\square}}
\newcommand{\hsig}{\hat{\sigma}}
\newcommand{\widesim}[2][1.5]{
	\mathrel{\overset{#2}{\scalebox{#1}[1]{$\sim$}}}
}
\newcommand{\wsim}{\widesim{}}
\newcommand{\lh}{\stackrel{L'H}{=}}

\begin{document}
\thispagestyle{empty}

\begin{minipage}{2cm}
	\includegraphics[width=2cm]{../../../../img/logo.pdf}
	\vspace{0.5cm}
\end{minipage}
\begin{minipage}{\linewidth}
	\begin{tabular}{lrl}
		{\scriptsize\sc Pontificia Universidad Catolica de Chile} & \hspace*{0.7in}Curso: &
		\sigla  - \nombre\\
		{\sc Facultad de Matemáticas}&
		Profesor: & \profesor \\
		{\sc Semestre \ano-\semestre} & Ayudante: & {Ignacio Castañeda}\\
		& {Mail:} & \texttt{\mail}
	\end{tabular}
\end{minipage}

\vspace{-10mm}
\begin{center}
	{\LARGE\bf \ayudantia}\\
	\vspace{0.1cm}
	{\tituloayu}\\
	\vspace{0.1cm}
	\fecha\\
	\vspace{0.4cm}
\end{center}

\begin{preguntas}
\item Realiza las siguientes operaciones con los vectores dados
	$$
	v_1 = \begin{pmatrix}
	1\\
	7\\
	8
\end{pmatrix};\qquad
	v_2 = \begin{pmatrix}
	-4\\
	2\\
	-3
\end{pmatrix}; \qquad
	v_3 = \begin{pmatrix}
	0\\
	1\\
	3
\end{pmatrix}; \qquad
	v_4 = \begin{pmatrix}
	13\\
	-3\\
	1
\end{pmatrix},
	 $$
\begin{tasks}(4)
\task $v_1 + v_2$
\task $v_3 - v_4$
\task $v_2 \cdot v_3$
\task $v_1 \times v_2$
\task $v_2 \times v_1$
\task $(v_1 \times v_2) \cdot v_2$
\task $3 - ((2v_1) \cdot v_2)$
\task $v_2 - v_3 \times v_1$
\end{tasks}
\item Verificar si los siguientes puntos son o no colineares entre si
\begin{tasks}(2)
\task $P(3,2,5), \ P(0, 1/2, -1), \ P(5, 3, 9)$
\task $P(1,1,4), \ P(-2,-3.-8), \ P(4,4,16)$
\end{tasks}
\item Sean los vectores
		$$
	v_1 = \begin{pmatrix}
	2\\
	1\\
	5
\end{pmatrix};\qquad
	v_2 = \begin{pmatrix}
	-1\\
	-3\\
	0
\end{pmatrix}$$
\begin{enumerate}[a)]
\item Buscar un vector $v_3$ que sea perpendicular a $v_1$ y $v_2$ y verificar que efectivamente sea perpendicular a ambos.
\item Buscar un vector paralelo a $(v_2-v_3)$ y verificar que lo sea
\item Encontrar un vector perpendicular a $(v_3 + 3v_1)$
\end{enumerate}
\item Detereminar si los siguientes vectores son paralelos, perpendiculares u oblicuos
\begin{tasks}(2)
\task $$ 
			\begin{pmatrix}
			-1\\
			2\\
			5
		\end{pmatrix};\qquad
			\begin{pmatrix}
			2\\
			-4\\
			-10
		\end{pmatrix}$$
\task  $$ 
			\begin{pmatrix}
			2\\
			5\\
			10
		\end{pmatrix};\qquad
			\begin{pmatrix}
			-3\\
			8\\
			-11
		\end{pmatrix}$$
\end{tasks}
\item Determinar el plano que pasa por los puntos $P_1(4,-1,-2)$, $P_2(0,0,1)$ y $P_3(2,-3,0)$.
\item Dado $P_1(0,2,-3)$ y $P_2(1,0,3)$, determinar la recta que pasa por $P_1$ y $P_2$.
\item Encontrar las ecuaciones de dos planos diferentes cuya intersección sea la recta que pasa por el punto $P_1(1,3,-2)$ y $P_2(2,0,4)$.
\item Encontrar un plano que sea perpendicular al plano cuya ecuación es $3x -7y +2z = 5$ y que pase por el punto $P(0,2,-1)$
\item Determinar la ecuación de un plano que pase por $P(4,2,1)$ y que sea paralelo al plano de ecuación $2x-5y+z=6$
\item 
\begin{enumerate}[a)]
\item Sean $a=\left(\begin{array}{r}
      1\\-1\\1
    \end{array}\right)$ y $b=\left(\begin{array}{r}
      0\\-1\\2
    \end{array}\right)$, calcule el vector\textbf{ $b-proy_a b$ }y verifique que este es ortogonal a $a$.
\item Demuestre que si $a$ y $b\in\R^3$ el vector $b-proy_a b$ es ortogonal a $a$.
\end{enumerate}
\item Determinar si el siguiente sistema es consistente o no
	$$
	\begin{array}{rcr}
	x_1 -6x_2& = & 5\\
	x_2-4x_3+x_4& = & 0\\
	-x_1+6x_2+x_3+5x_4& = & 3\\
	-x_2+5x_3+4x_4 & = & 0
	\end{array}
	$$
\item Lleve las siguientes matrices ampliadas a su forma escalonada reducida y determine la existencia y unicidad de las soluciones del sistema.
\begin{tasks}(2)
\task $
		\begin{bmatrix}
		1 & 2 & 4 & 8\\
		2 & 4 & 6 & 8\\
		3 & 6 & 9 & 12
		\end{bmatrix}
		$
\task $
		\begin{bmatrix}
		1 & 2 & 4 & 5\\
		2 & 4 & 5& 4\\
		4 & 5 & 4 & 2
		\end{bmatrix}
		$
\end{tasks}
\item Sea la matriz $A=
	\begin{bmatrix}
	3 & 5 & -4\\
	-3 & -2 & 4\\
	6 & 1 & -8
	\end{bmatrix}$
\begin{enumerate}[a)]
\item Determinar el conjunto solución de su sistema homogeneo.
\item Describir todas las soluciones de $Ax=b$ con $b=
		\begin{pmatrix}
		7\\
		-1\\
		-4
		\end{pmatrix}$ 
\end{enumerate}
\item Considere el siguiente sistema de ecuaciones, donde $a$ es una constante.
  $$\begin{array}{llll}
   x_1&+x_2&+x_3&=1 \\
    x_1&+x_2&+ax_3&=1 \\
     ax_1&+ax_2&+x_3&=a\\
      x_1&-ax_2&+ax_3&=0  
  \end{array}$$
\begin{enumerate}[a)]
\item Determine valores de $a$ para los cuales el sistema es inconsistente.
\item Determine valores de $a$ para los cuales el sistema es consistente, y encuentre la solución.
\end{enumerate}
\item Sea la matriz $A=
	\begin{bmatrix}
	1 & 3 & 4\\
	-4 & 2 & -6\\
	-3 & -2 & -7
	\end{bmatrix}
	$ y $b$ un vector en $R^3$. ¿La ecuación $Ax=b$ es consistente para todo $b$?
\item Sean los vectores 
	$$
	v_1 = \begin{pmatrix}
	1\\
	1\\
	2
	\end{pmatrix};\qquad
	v_2 = \begin{pmatrix}
	2\\
	2\\
	-3
	\end{pmatrix}; \qquad
	v_3 = \begin{pmatrix}
	0\\
	1\\
	3
	\end{pmatrix}$$
	determinar si $Gen\{v_1, v_2, v_3\} = R^3$.
\item Sean 
$$v_1=\left(\begin{array}{r}
  1\\0\\-2
\end{array}\right), \quad v_2=\left(\begin{array}{r}
  -2\\1\\7
\end{array}\right) \quad y \quad v_3=\left(\begin{array}{r}
  h\\0\\-2
\end{array}\right)$$
¿Para qué valor(es) de $h$ $Gen\{v_1, v_2, v_3\}=Gen\{v_1,v_2\}$ ?
\item Sean $\{u, v, w\}$ un conjunto de vectores linealmente independientes. Demuestre que el conjunto $\{u+v, u+2w, v+3u+w\}$ es linealmente independiente.
\item Demuestre que el conjunto $\{u, v, w\}$ es L.I. si y solo si el conjunto $\{u+v, u+w, v+w\}$ es L.I.
\item Sea $L: P_2(\R) \ra \R^2$ una transformación lineal tal que
	$$L(1+x) = \begin{pmatrix}
	1\\
	1
	\end{pmatrix}, \quad L(1+x+x^2) = \begin{pmatrix}
	1\\
	-1
	\end{pmatrix}\ y \ L(1+2x) = \begin{pmatrix}
	1\\
	2
	\end{pmatrix}$$
	determine $L(a+bx+cx^2)$ para todo $a,b,c \in \R$.
\item Sea $T$ una transformación lineal tal que
	$$T\begin{bmatrix}2 \\ 1\end{bmatrix} = \begin{bmatrix}1 \\ 0 \\ 1 \\ 0\end{bmatrix} 
	\quad y \quad
	T \begin{bmatrix}3 \\ 1\end{bmatrix}  =\begin{bmatrix}0 \\ 1 \\ 0 \\ 1\end{bmatrix}$$
	Determine la matriz que representa a $T$
\end{preguntas}
\end{document}