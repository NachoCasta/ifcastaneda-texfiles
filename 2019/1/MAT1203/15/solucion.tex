\documentclass[12pt]{article}

\usepackage{fullpage}
\usepackage{graphicx}
\usepackage{amssymb}
\usepackage{amsmath}
\usepackage[none]{hyphenat}
\usepackage{parskip}
\usepackage[spanish]{babel}
\usepackage[utf8]{inputenc}
\usepackage{hyperref}
\usepackage{fancyhdr}
\usepackage{tasks}
\usepackage{mdframed}
\usepackage{xcolor}
\usepackage{pgfplots}
\usepackage[makeroom]{cancel}
\usepackage{multicol}
\usepackage[shortlabels]{enumitem}
\usepackage{stackrel}
\usepackage{tkz-tab}
\usepackage{xpatch}
\usepackage{tkz-euclide}
\usetkzobj{all}
\xpatchcmd{\tkzTabLine}{$0$}{$\bullet$}{}{}

\setlength{\headheight}{10pt}
\setlength{\headsep}{10pt}
\pagestyle{fancy}
\rhead{\ayudantia \ - \alumno}
\tikzset{t style/.style={style=solid}}

\newcommand*{\mybox}[2]{\colorbox{#1!30}{\parbox{.98\linewidth}{#2}}}

\newenvironment{solucion}
{\begin{mdframed}[backgroundcolor=black!10]
		{\bf Solución:}\\
	}
	{
	\end{mdframed}
}

\newenvironment{alternativas}[1]
{\begin{multicols}{#1}
		\begin{enumerate}[a)]
		}
		{
		\end{enumerate}
	\end{multicols}
}

\newenvironment{preguntas}
{\begin{enumerate}\itemsep12pt
	}
	{
	\end{enumerate}
}

\newcommand{\ayudantia}{{\sc Ayudantía 15}}
\newcommand{\tituloayu}{Compilado Álgebra Lineal}
\newcommand{\fecha}{22 de junio de 2019}
\newcommand{\sigla}{MAT1203}
\newcommand{\nombre}{Álgebra Lineal}
\newcommand{\profesor}{Camilo Perez}
\newcommand{\ano}{2019}
\newcommand{\semestre}{1}
\newcommand{\mail}{mat1203@ifcastaneda.cl}
\newcommand{\alumno}{Ignacio Castañeda - \mail}

\newcommand{\ev}{\Big|}
\newcommand{\ra}{\rightarrow}
\newcommand{\lra}{\leftrightarrow}
\newcommand{\N}{\mathbb{N}}
\newcommand{\R}{\mathbb{R}}
\newcommand{\Exp}[1]{\mathcal{E}_{#1}}
\newcommand{\List}[1]{\mathcal{L}_{#1}}
\newcommand{\EN}{\Exp{\N}}
\newcommand{\LN}{\List{\N}}
\newcommand{\comment}[1]{}
\newcommand{\lb}{\\~\\}
\newcommand{\eop}{_{\square}}
\newcommand{\hsig}{\hat{\sigma}}
\newcommand{\widesim}[2][1.5]{
	\mathrel{\overset{#2}{\scalebox{#1}[1]{$\sim$}}}
}
\newcommand{\wsim}{\widesim{}}
\newcommand{\lh}{\stackrel{L'H}{=}}

\begin{document}
\thispagestyle{empty}

\begin{minipage}{2cm}
	\includegraphics[width=2cm]{../../../../img/logo.pdf}
	\vspace{0.5cm}
\end{minipage}
\begin{minipage}{\linewidth}
	\begin{tabular}{lrl}
		{\scriptsize\sc Pontificia Universidad Catolica de Chile} & \hspace*{0.7in}Curso: &
		\sigla  - \nombre\\
		{\sc Facultad de Matemáticas}&
		Profesor: & \profesor \\
		{\sc Semestre \ano-\semestre} & Ayudante: & {Ignacio Castañeda}\\
		& {Mail:} & \texttt{\mail}
	\end{tabular}
\end{minipage}

\vspace{-10mm}
\begin{center}
	{\LARGE\bf \ayudantia}\\
	\vspace{0.1cm}
	{\tituloayu}\\
	\vspace{0.1cm}
	\fecha\\
	\vspace{0.4cm}
\end{center}

\begin{preguntas}
\item Realiza las siguientes operaciones con los vectores dados
	$$
	v_1 = \begin{pmatrix}
	1\\
	7\\
	8
\end{pmatrix};\qquad
	v_2 = \begin{pmatrix}
	-4\\
	2\\
	-3
\end{pmatrix}; \qquad
	v_3 = \begin{pmatrix}
	0\\
	1\\
	3
\end{pmatrix}; \qquad
	v_4 = \begin{pmatrix}
	13\\
	-3\\
	1
\end{pmatrix},
	 $$
\begin{tasks}(4)
\task $v_1 + v_2$
\task $v_3 - v_4$
\task $v_2 \cdot v_3$
\task $v_1 \times v_2$
\task $v_2 \times v_1$
\task $(v_1 \times v_2) \cdot v_2$
\task $3 - ((2v_1) \cdot v_2)$
\task $v_2 - v_3 \times v_1$
\end{tasks}
\begin{solucion}

\begin{enumerate}[a)]
\item $v_1 + v_2 =\begin{pmatrix}
				1\\
				7\\
				8
				\end{pmatrix} + \begin{pmatrix}
				-4\\
				2\\
				-3
				\end{pmatrix} = \begin{pmatrix}
				-3\\
				9\\
				5
				\end{pmatrix}$
\item $v_3 - v_4 = \begin{pmatrix}
				0\\
				1\\
				3
				\end{pmatrix} - \begin{pmatrix}
				13\\
				-3\\
				1
				\end{pmatrix} =  \begin{pmatrix}
				-13\\
				4\\
				2
				\end{pmatrix}$
\item $v_2 \cdot v_3 = \begin{pmatrix}
				-4\\
				2\\
				-3
				\end{pmatrix} \cdot \begin{pmatrix}
				0\\
				1\\
				3
				\end{pmatrix} = -4 \cdot 0 + 2 \cdot 1 -3 \cdot 3 = -7$
\item $v_1 \times v_2 = \begin{pmatrix}
				1\\
				7\\
				8
			\end{pmatrix} \times \begin{pmatrix}
				-4\\
				2\\
				-3
			\end{pmatrix} = i \left| \begin{matrix} 7 & 8 \\ 2 & -3\end{matrix} \right| - j \left| \begin{matrix} 1 & 8 \\ 4 & -3\end{matrix} \right| + k \left| \begin{matrix} 1 & 7 \\ -4 & 2\end{matrix} \right|$\\
			$$= i (7 \cdot (-3) - 8 \cdot 2) - j (1 \cdot (-3) - 8 \cdot (-4)) + k (1 \cdot 2 - 7 \cdot(-4))$$
			$$= i (-21 -16)) - j(-3--32) + k(2 - - 28)$$
			$$-37 i - 29j + 30k = \begin{pmatrix}
			-37\\
			-29\\
			30
			\end{pmatrix}$$	
\item $v_2 \times v_1 =  \begin{pmatrix}
			-4\\
			2\\
			-3
			\end{pmatrix} \times \begin{pmatrix}
			1\\
			7\\
			8
			\end{pmatrix} = i \left| \begin{matrix} 2 & -3 \\ 7 & 8\end{matrix} \right| - j \left| \begin{matrix} 4 & -3 \\ 1 & 8\end{matrix} \right| + k \left| \begin{matrix} -4 & 2 \\ 1 & 7\end{matrix} \right|$\\
			$$= i (2 \cdot 8 - (-3) \cdot 7) - j ((-4)\cdot 8 - (-3)\cdot 1) + k ((-4) \cdot 7 -2 \cdot 1)$$
			$$= i (16 --21) + j(-32 --3) + k(-28-2)$$
			$$37 i + 29j - 30k = \begin{pmatrix}
			37\\
			29\\
			-30
			\end{pmatrix}$$	
\item $(v_1 \times v_2) \cdot v_2 = \begin{pmatrix}
			-37\\
			-29\\
			30
			\end{pmatrix} \cdot \begin{pmatrix}
			-4\\
			2\\
			-3
			\end{pmatrix} = (-37) \cdot (-4) + (-29) \cdot 2 + 30 \cdot (-3)$
			$$=148 +-58 - 90 = 0$$
\item $3 - ((2v_1) \cdot v_2) = 3 - \left(\left(2\begin{pmatrix}
			1\\
			7\\
			8
			\end{pmatrix}\right) \cdot \begin{pmatrix}
			-4\\
			2\\
			-3
			\end{pmatrix}\right) = 3 - \left(\begin{pmatrix}
			2\\
			14\\
			16
			\end{pmatrix} \cdot \begin{pmatrix}
			-4\\
			2\\
			-3
			\end{pmatrix}\right)$
			$$ = 3 - (2 \cdot (-4) + 14 \cdot 2 + 16 \cdot (-3))$$
			$$3 - (-8 + 28 -48)$$
\item $v_2 - v_3 \times v_1 = \begin{pmatrix}
			-4\\
			2\\
			-3
			\end{pmatrix} -  \begin{pmatrix}
			0\\
			1\\
			3
			\end{pmatrix} \times  \begin{pmatrix}
			1\\
			7\\
			8
			\end{pmatrix}$
			$$= -4i + 2j -3k - (i (8 - 21))- j(0 - 3) + k (0-1))$$
			$$ = -4i + 2j -3k +13i - 3j -k = 9i - j -4k $$
			$$ =\begin{pmatrix}
			9\\
			-4\\
			-2
			\end{pmatrix}$$
\end{enumerate}
\end{solucion}
\item Verificar si los siguientes puntos son o no colineares entre si
\begin{tasks}(2)
\task $P(3,2,5), \ P(0, 1/2, -1), \ P(5, 3, 9)$
\task $P(1,1,4), \ P(-2,-3.-8), \ P(4,4,16)$
\end{tasks}
\begin{solucion}

\begin{enumerate}[a)]
\item $ 
			\begin{pmatrix}
			3\\
			2\\
			5
			\end{pmatrix}; \quad
			\begin{pmatrix}
			0\\
			1/2\\
			-1
			\end{pmatrix}; \quad
			\begin{pmatrix}
			5\\
			3\\
			9
			\end{pmatrix}$\\
			En primer lugar, debemos encontrar dos vectores directores entre dos pares de puntos distintos
			$$d_1 = \begin{pmatrix}
			3\\
			2\\
			5
			\end{pmatrix} - 
			\begin{pmatrix}
			0\\
			1/2\\
			-1
			\end{pmatrix} = \begin{pmatrix}
			3\\
			3/2\\
			6
			\end{pmatrix}$$
			$$d_2 = \begin{pmatrix}
			3\\
			2\\
			5
			\end{pmatrix} - 
			\begin{pmatrix}
			5\\
			3\\
			9
			\end{pmatrix} = \begin{pmatrix}
			-2\\
			-1\\
			-4
			\end{pmatrix}$$
			Dado que es posible representar $d_1$ de la forma $d_1 = \lambda d_2$, con $\lambda = -1,5$, los puntos son colineales.
\item $ 
			\begin{pmatrix}
			1\\
			1\\
			4
			\end{pmatrix}; \quad
			\begin{pmatrix}
			-2\\
			-3\\
			-8
			\end{pmatrix}; \quad
			\begin{pmatrix}
			4\\
			4\\
			16
			\end{pmatrix}$\\
			Igual que antes, buscaremos dos vectores directores
			$$d_1 = \begin{pmatrix}
			1\\
			1\\
			4
			\end{pmatrix} - 
			\begin{pmatrix}
			-2\\
			-3\\
			-8
			\end{pmatrix} = \begin{pmatrix}
			3\\
			4\\
			12
			\end{pmatrix}$$
			$$d_2 = \begin{pmatrix}
			1\\
			1\\
			4
			\end{pmatrix} - 
			\begin{pmatrix}
			4\\
			4\\
			16
			\end{pmatrix} = \begin{pmatrix}
			-3\\
			-3\\
			-12
			\end{pmatrix}$$
			Observando las dos primeras componentes de cada vector, vemos que es imposible representar uno en función del otro, por lo que los puntos no son colineales
\end{enumerate}
\end{solucion}
\item Sean los vectores
		$$
	v_1 = \begin{pmatrix}
	2\\
	1\\
	5
\end{pmatrix};\qquad
	v_2 = \begin{pmatrix}
	-1\\
	-3\\
	0
\end{pmatrix}$$
\begin{enumerate}[a)]
\item Buscar un vector $v_3$ que sea perpendicular a $v_1$ y $v_2$ y verificar que efectivamente sea perpendicular a ambos.
\item Buscar un vector paralelo a $(v_2-v_3)$ y verificar que lo sea
\item Encontrar un vector perpendicular a $(v_3 + 3v_1)$
\end{enumerate}
\begin{solucion}

\begin{enumerate}[a)]
\item Para que sea perpendicular a ambos, basta definir $v_3 = v_1 \times v_2$, es decir
			$$v_3 = \begin{pmatrix}
			2\\
			1\\
			5
			\end{pmatrix} \times \begin{pmatrix}
			-1\\
			-3\\
			0
			\end{pmatrix} = i (0 + 15) - j (0 + 5) + k(-6 + 1) =\begin{pmatrix}
			15\\
			-5\\
			-5
			\end{pmatrix} $$
			Para verificar basta utilizar el producto punto, donde obtendremos que
			$$v_3 \cdot v_1 = 0$$
			$$v_3 \cdot v_2 = 0$$
\item En primer lugar, calculemos el vector $v_2 - v_3$
			$$v_4 = v_2 - v_3 = \begin{pmatrix}
			-1\\
			-3\\
			0
			\end{pmatrix} - \begin{pmatrix}
			15\\
			-5\\
			-5
			\end{pmatrix} = \begin{pmatrix}
			-16\\
			2\\
			5
			\end{pmatrix}$$
			Para obtener un vector paralelo, podemos simplemente ponderar el vector por un escalar, como por ejemplo, 2
			$$v_5 = 2 v_4 = 2 \begin{pmatrix}
			-16\\
			2\\
			5
			\end{pmatrix} = \begin{pmatrix}
			-32\\
			4\\
			10
			\end{pmatrix}$$
			Para verificar utilizaremos el producto cruz, donde obtendremos que
			$$v_5 \times v_4 = 0$$
\item Primero calculemos el vector $v_3 + 3v_1$
			$$v_6 = v_3 + 3v_1 = \begin{pmatrix}
			15\\
			-5\\
			-5
			\end{pmatrix}  + 3 \begin{pmatrix}
			2\\
			1\\
			5
			\end{pmatrix} = \begin{pmatrix}
			21\\
			-2\\
			10
			\end{pmatrix} $$
			Para obtener un vector perpendicular, podemos hacer producto cruz con cualquier otro vector arbitrario. Para hacer los calculos simples, utilizaremos
			$$v_7 = \begin{pmatrix}
			1\\
			0\\
			0
			\end{pmatrix} $$
			Luego, 
			$$v_8 = v_6 \times v_7 = \begin{pmatrix}
			21\\
			-2\\
			10
			\end{pmatrix} \times \begin{pmatrix}
			1\\
			0\\
			0
			\end{pmatrix} = 10j + 2k = \begin{pmatrix}
			0\\
			10\\
			2
			\end{pmatrix}$$
\end{enumerate}
\end{solucion}
\item Detereminar si los siguientes vectores son paralelos, perpendiculares u oblicuos
\begin{tasks}(2)
\task $$ 
			\begin{pmatrix}
			-1\\
			2\\
			5
		\end{pmatrix};\qquad
			\begin{pmatrix}
			2\\
			-4\\
			-10
		\end{pmatrix}$$
\task  $$ 
			\begin{pmatrix}
			2\\
			5\\
			10
		\end{pmatrix};\qquad
			\begin{pmatrix}
			-3\\
			8\\
			-11
		\end{pmatrix}$$
\end{tasks}
\begin{solucion}

\begin{enumerate}[a)]
\item $$ 
			\begin{pmatrix}
			-1\\
			2\\
			5
			\end{pmatrix};\qquad
			\begin{pmatrix}
			2\\
			-4\\
			-10
			\end{pmatrix}$$
			Si se fijan, es evidente que $v_2 = -2v_1$, por lo que comprobaremos de inmediato con el producto cruz para ver si son paralelos
			 $$ 
			\begin{pmatrix}
			-1\\
			2\\
			5
			\end{pmatrix} \times 
			\begin{pmatrix}
			2\\
			-4\\
			-10
			\end{pmatrix} = i (-20 + 20) -j (10 - 10) + k (4 - 4) = \vec{0}$$
			Por lo que ambos vectores son paralelos
\item $$ 
			\begin{pmatrix}
			2\\
			5\\
			10
			\end{pmatrix};\qquad
			\begin{pmatrix}
			-3\\
			8\\
			-11
			\end{pmatrix}$$
			Comenzaremo probando si son perpendiculares con el producto punto
			$$ 
			\begin{pmatrix}
			2\\
			5\\
			10
			\end{pmatrix} \cdot
			\begin{pmatrix}
			-3\\
			8\\
			-11
			\end{pmatrix} = -6 + 40 - 110 = -76$$
			Como es distinto de cero, vemos que no son perpendiculares.\\
			Ahora, utilizaremos el producto cruz para ver si son paralelos
			$$ 
			\begin{pmatrix}
			2\\
			5\\
			10
			\end{pmatrix} \times
			\begin{pmatrix}
			-3\\
			8\\
			-11
			\end{pmatrix} = i(-55 -80) - j (-22 +30) +k(16 + 15) = \begin{pmatrix}
			-135\\
			-8\\
			31
			\end{pmatrix}$$
			Como este no es el vector nulo, tampoco son paralelos. Con esto concluimos que ambos vectores son oblicuos.
\end{enumerate}
\end{solucion}
\item Determinar el plano que pasa por los puntos $P_1(4,-1,-2)$, $P_2(0,0,1)$ y $P_3(2,-3,0)$.
\begin{solucion}
Sabemos que un plano se define de la forma
		$$Ax +By + Cz = D$$
		Reemplazando esto con los tres puntos que tenemos, obtendremos el siguiente sistema de ecuaciones
		$$
		\begin{array}{rcrr}
		4A-B-2C & = & D& \vline\\
		C & = & D & \vline\\
		2A-3B & = & D &\vline\\
		\hline
		\end{array}
		$$
		$$
		\begin{array}{rcrr}
		4A-B-2C & = & C& \vline\\
		2A-3B & = & C &\vline\\
		\hline
		\end{array}
		$$
		$$
		\begin{array}{rcrr}
		4A-B-3C & = & 0& \vline\\
		2A-3B -C& = & 0 &\vline\\
		\hline
		\end{array}
		$$
		$$(1) - 2(2)$$
		$$5B - C = 0 \ra 5B = C$$
		Como hay 4 variables y teníamos 3 ecuaciones, debemos elegir el valor de una de ellas de manera arbitraria.
		$$C = 5$$
		$$\ra D = 5$$
		$$\ra B = 1$$
		$$4A -1 -15 = 0 \ra 4A=16 \ra A=4$$
		Finalmente, el plano es
		$$\Pi:4x + y + 5z = 5$$
\end{solucion}
\item Dado $P_1(0,2,-3)$ y $P_2(1,0,3)$, determinar la recta que pasa por $P_1$ y $P_2$.
\begin{solucion}
Buscamos un vector director
		$$\vec{d} = P_2 - P_1 = (1, -2, 0)$$
		Luego, la recta que pasa por $P_1$ y $P_2$ esta definida por
		$$<x,y,z> = (0, 2, -3) + \lambda (1,-2,0)$$
\end{solucion}
\item Encontrar las ecuaciones de dos planos diferentes cuya intersección sea la recta que pasa por el punto $P_1(1,3,-2)$ y $P_2(2,0,4)$.
\begin{solucion}
Para realizar esto, basta con utilizar dos trios de puntos que contengan a los dos puntos dados y buscar dos planos. La intersección entre estos planos será la recta que pasa por ambos puntos. Para esto, definiremos puntos arbitrarios.\\
		Diremos que
		$$P_3(0, 0, 0), \quad P_4(1,0,0)$$
		Partamos usando los puntos $P_1$, $P_2$ y $P_3$. Sabemos que la ecuación del plano es de la forma
		$$Ax + By + Cz = D$$
		Reemplazando con estos tres puntos, obtenemos el siguiente sistema
		$$
		\begin{array}{rcrr}
		A + 3B - 2C & = & D& \vline\\
		2A + 4C & = & D & \vline\\
		0 & = & D &\vline\\
		\hline
		\end{array}
		$$
		$$
		\begin{array}{rcrr}
		A + 3B - 2C & = & 0& \vline\\
		2A + 4C & = & 0 & \vline\\
		\hline
		\end{array}
		$$
		$$(2) + 2(1)$$
		$$4A + 6B = 0 \ra A = -\dfrac{6B}{4} \ra A = -\dfrac{3B}{2}$$
		Le pondremos un valor arbitrario a $B$,
		$$B = 4$$
		$$\ra A = -6$$
		$$\ra -12 + 4C = 0 \ra C = 3$$
		Por ende,
		$$\Pi_1: -6x + 4y + 3z = 0$$
		Ahora realizamos lo mismo con los puntos $P_1$, $P_2$, $P_4$. Sea la ecuación del plano
		$$Ax + By + Cz = D$$
		Reemplazando con estos tres puntos, obtenemos el siguiente sistema
		$$
		\begin{array}{rcrr}
		A + 3B - 2C & = & D& \vline\\
		2A + 4C & = & D & \vline\\
		A & = & D &\vline\\
		\hline
		\end{array}
		$$
		$$
		\begin{array}{rcrr}
		3B - 2C & = & 0& \vline\\
		A + 4C & = & 0 & \vline\\
		\hline
		\end{array}
		$$
		$$
		\begin{array}{rcrr}
		3B & = & 2C& \vline\\
		A & = & -4C & \vline\\
		\hline
		\end{array}
		$$
		Le daremos un valor arbitrario a $C$
		$$C = 3$$
		$$ \ra B = 2$$
		$$ \ra A = -12 $$
		$$ \ra D = -12 $$
		Con lo que obtenemos el plano
		$$\Pi_2: -12x+2y+3z = -12$$
		Finalmente, los planos buscados son:
		$$\Pi_1: -6x + 4y + 3z = 0$$		
		$$\Pi_2: -12x+2y+3z = -12$$
\end{solucion}
\item Encontrar un plano que sea perpendicular al plano cuya ecuación es $3x -7y +2z = 5$ y que pase por el punto $P(0,2,-1)$
\begin{solucion}
En primer lugar, debemos encontrar el vector normal del plano, que esta dado por los coeficientes de su ecuación, es decir
		$$\vec{n} = \begin{pmatrix}
		3\\-7\\2
		\end{pmatrix}$$
		Si los vectores normales de dos planos son perpendiculares entre si, los planos también lo serán, por lo que basta encontrar un vector perpendicular a $\vec{n}$ y utilizarlo como vector normal para nuestro nuevo plano. Para encontrar un vector perpendicular, realizaremos el producto cruz entre $\vec{n}$ y un vector arbitrario elegido por nosotros
		$$\begin{pmatrix} 3 \\ -7 \\ 2 \end{pmatrix} \times \begin{pmatrix} 1 \\ 1 \\ 1 \end{pmatrix} = \begin{pmatrix} -9 \\ -1 \\ 10 \end{pmatrix}$$
		Para encontrar el plano asociado, basta que utilicemos los coeficientes del vector como coeficientes de la ecuación del plano
		$$-9x -y +10z = D$$
		Por último, para encontrar $D$, reemplazamos el punto que nos dan en la ecuación del plano
		$$-2 -10 = D \ra D = -12$$
		Finalmente,
		$$\Pi: -9x -y +10z = -12$$
\end{solucion}
\item Determinar la ecuación de un plano que pase por $P(4,2,1)$ y que sea paralelo al plano de ecuación $2x-5y+z=6$
\begin{solucion}
De forma análoga, para que dos planos sean paralelos, basta con que sus vectores normales sean paralelos. Dicho de otra forma, lo que va a cambiar será el parametro $D$. Es decir, nuestro plano es
		$$2x - 5y + z = D$$
		Reemplazando con el punto,
		$$8-10+1 = D \ra D = -1$$
		Luego, el plano buscado es
		$$\Pi: 2x - 5y + z = -1$$
\end{solucion}
\item 
\begin{enumerate}[a)]
\item Sean $a=\left(\begin{array}{r}
      1\\-1\\1
    \end{array}\right)$ y $b=\left(\begin{array}{r}
      0\\-1\\2
    \end{array}\right)$, calcule el vector\textbf{ $b-proy_a b$ }y verifique que este es ortogonal a $a$.
\item Demuestre que si $a$ y $b\in\R^3$ el vector $b-proy_a b$ es ortogonal a $a$.
\end{enumerate}
\begin{solucion}

\begin{enumerate}[a)]
\item La proyección de $b$ sobre $a$ corresponde a
$$proy_{a}b = \dfrac{b \cdot a}{|a|^2}a =   \dfrac{\left(\begin{array}{r}
	0\\-1\\2
	\end{array}\right) \cdot\left(\begin{array}{r}
		1\\-1\\1
	\end{array}\right)}{\left(\begin{array}{r}
		1\\-1\\1
	\end{array}\right) \cdot \left(\begin{array}{r}
		1\\-1\\1
	\end{array}\right)}\left(\begin{array}{r}
		1\\-1\\1
	\end{array}\right) = \left(\begin{array}{r}
1\\-1\\1
\end{array}\right)$$
Luego, 
$$b-proy_a b = \left(\begin{array}{r}
0\\-1\\2
\end{array}\right) - \left(\begin{array}{r}
1\\-1\\1
\end{array}\right) = \left(\begin{array}{r}
-1\\0\\1
\end{array}\right)$$
Podemos verificar que el resultado es perpendicular a $a$ utilizando el producto punto, esto es
$$\left(\begin{array}{r}
1\\-1\\1
\end{array}\right) \cdot \left(\begin{array}{r}
-1\\0\\1
\end{array}\right) = 0$$
\item Para demostrar esto, debemos demostrar que
$$(b-proy_a b) \cdot a = 0 $$
Desarrollando,
$$\left(b-\dfrac{b \cdot a}{|a|^2}a\right) \cdot a = 0 $$
$$b \cdot a -\dfrac{b \cdot a}{|a|^2}a \cdot a = 0 $$
$$b \cdot a -\dfrac{b \cdot a}{|a|^2}|a|^2 = 0 $$
$$b \cdot a -b \cdot a = 0 $$
$$0 = 0$$
$$\blacksquare$$
\end{enumerate}
\end{solucion}
\item Determinar si el siguiente sistema es consistente o no
	$$
	\begin{array}{rcr}
	x_1 -6x_2& = & 5\\
	x_2-4x_3+x_4& = & 0\\
	-x_1+6x_2+x_3+5x_4& = & 3\\
	-x_2+5x_3+4x_4 & = & 0
	\end{array}
	$$
\begin{solucion}
En primer lugar, escribiremos el sistema en notación matricial
		$$
		\left[
		\begin{array}{cccc|c}
		1 & -6 & 0 & 0 & 5\\
		0 & 1 & -4 & 1 & 0\\
		-1& 6 & 1 & 5 & 3 \\
		0 & -1 & 5 & 4 &0
		\end{array}
		\right]$$
		A continuación, utilizando operaciones de fila, llevaremos esta matriz a su forma escalonada
		$$\left[
		\begin{array}{cccc|c}
			1 & -6 & 0 & 0 & 5\\
			0 & 1 & -4 & 1 & 0\\
			-1& 6 & 1 & 5 & 3 \\
			0 & -1 & 5 & 4 &0
		\end{array}
		\right] \stackbin[F_4+F_2]{F_3 +F_1}{\wsim}
		\left[
		\begin{array}{cccc|c}
		1 & -6 & 0 & 0 & 5\\
		0 & 1 & -4 & 1 & 0\\
		0 & 0 & 1 & 5 & 8 \\
		0 & 0 & 1 & 5 & 0
		\end{array}
		\right] \stackbin[]{F_4-F_3}{\wsim}
		\left[
		\begin{array}{cccc|c}
		1 & -6 & 0 & 0 & 5\\
		0 & 1 & -4 & 1 & 0\\
		0 & 0 & 1 & 5 & 8 \\
		0 & 0 & 0 & 0 & -8
		\end{array}
		\right] $$
		Aqui podemos apreciar que en el la última fila, todas las variables tienen coeficiente cero y la columna de coeficientes aumentados tiene valor, lo que no es posible, ya que esto significaría que se cumple la igualdad $0 = -8$, lo que no es cierto. Dicho esto, el sistema no es consistente.
\end{solucion}
\item Lleve las siguientes matrices ampliadas a su forma escalonada reducida y determine la existencia y unicidad de las soluciones del sistema.
\begin{tasks}(2)
\task $
		\begin{bmatrix}
		1 & 2 & 4 & 8\\
		2 & 4 & 6 & 8\\
		3 & 6 & 9 & 12
		\end{bmatrix}
		$
\task $
		\begin{bmatrix}
		1 & 2 & 4 & 5\\
		2 & 4 & 5& 4\\
		4 & 5 & 4 & 2
		\end{bmatrix}
		$
\end{tasks}
\begin{solucion}

\begin{enumerate}[a)]
\item $
			\begin{bmatrix}
			1 & 2 & 4 & 8\\
			2 & 4 & 6 & 8\\
			3 & 6 & 9 & 12
			\end{bmatrix}
			\stackbin[F_3 - 3F_1]{F_2-2F_1}{\wsim}
			\begin{bmatrix}
			1 & 2 & 4 & 8\\
			0 & 0 & -2 & -8\\
			0 & 0 & -3 & -12
			\end{bmatrix}
			\stackbin[F_3 \leftarrow -\frac{1}{3}F_3]{F_2 \leftarrow -\frac{1}{2}F_2}{\wsim}
			\begin{bmatrix}
			1 & 2 & 4 & 8\\
			0 & 0 & 1 & 4\\
			0 & 0 & 1 & 4
			\end{bmatrix}$\\\\
			$
			\stackbin[]{F_3- F_2}{\wsim}
			\begin{bmatrix}
			1 & 2 & 4 & 8\\
			0 & 0 & 1 & 4\\
			0 & 0 & 0 & 0
			\end{bmatrix}
			\stackbin[]{F_1- 4F_2}{\wsim}
			\begin{bmatrix}
			1 & 2 & 0 & -8\\
			0 & 0 & 1 & 4\\
			0 & 0 & 0 & 0
			\end{bmatrix}
			$\\\\
			Como hay una variable libre (2 pivotes y 3 filas), existen infinitas soluciones.
\item $
			\begin{bmatrix}
			1 & 2 & 4 & 5\\
			2 & 4 & 5& 4\\
			4 & 5 & 4 & 2
			\end{bmatrix}
			\stackbin[F_3 - 4F_1]{F_2 - 2F_1}{\wsim}
			\begin{bmatrix}
			1 & 2 & 4 & 5\\
			0 & 0 & -3& -6\\
			0 & -3 & -12 & -18
			\end{bmatrix}
			\stackbin[F_3 \leftarrow -\frac{1}{3} F_2]{F_2 \leftarrow -\frac{1}{3}F_3}{\wsim}
			\begin{bmatrix}
			1 & 2 & 4 & 5\\
			0 & 1 & 4& 6\\
			0 & 0 & 1 & 2
			\end{bmatrix}
			$\\\\
			$
			\stackbin[F_2 - 4F_3]{F_1 - 4F_3}{\wsim}
			\begin{bmatrix}
			1 & 2 & 0 & -3\\
			0 & 1 & 0& -2\\
			0 & 0 & 1 & 2
			\end{bmatrix}
			\stackbin[]{F_1 - 2F_2}{\wsim}
			\begin{bmatrix}
			1 & 0 & 0 & 1\\
			0 & 1 & 0& -2\\
			0 & 0 & 1 & 2
			\end{bmatrix}$\\\\
			Como tiene 3 pivotes y 3 filas, el sistema posee solución única
\end{enumerate}
\end{solucion}
\item Sea la matriz $A=
	\begin{bmatrix}
	3 & 5 & -4\\
	-3 & -2 & 4\\
	6 & 1 & -8
	\end{bmatrix}$
\begin{enumerate}[a)]
\item Determinar el conjunto solución de su sistema homogeneo.
\item Describir todas las soluciones de $Ax=b$ con $b=
		\begin{pmatrix}
		7\\
		-1\\
		-4
		\end{pmatrix}$ 
\end{enumerate}
\begin{solucion}

\begin{enumerate}[a)]
\item Determinar el conjunto solución de su sistema homogeneo.\\\\
			El sistema homogeneo corresponde a la ecuación $Ax=\vec{0}$, por lo que nuestra matriz aumentada sería
			$$\left[
			\begin{array}{ccc|c}
			3 & 5 & -4 & 0\\
			-3 & -2 & 4 & 0\\
			6 & 1 & -8 & 0
			\end{array}
			\right] \sim \left[
			\begin{array}{ccc|c}
			3 & 5 & -4 & 0\\
			0 & 3 & 0 & 0\\
			0 & -9 & 0 & 0
			\end{array}
			\right] \sim \left[
			\begin{array}{ccc|c}
			3 & 5 & -4 & 0\\
			0 & 3 & 0 & 0\\
			0 & 0 & 0 & 0
			\end{array}
			\right]$$
			$$\sim \left[
			\begin{array}{ccc|c}
			1 & \frac{5}{3} & -\frac{4}{3} & 0\\
			0 & 1 & 0 & 0\\
			0 & 0 & 0 & 0
			\end{array}
			\right] \sim \left[
			\begin{array}{ccc|c}
			1 & 0 & -\frac{4}{3} & 0\\
			0 & 1 & 0 & 0\\
			0 & 0 & 0 & 0
			\end{array}
			\right]$$
			Esto corresponde al sistema
			$$\begin{array}{rcl}
			x_1 - \frac{4}{3}x_3 & = & 0\\
			x_2 & = & 0\\
			0 & = & 0
			\end{array} \ra \begin{array}{rcl}
			x_1 & = & \frac{4}{3}x_3\\
			x_2 & = & 0\\
			x_3 & = & x_3
			\end{array} \ra x = \begin{pmatrix}
			\frac{4}{3}x_3\\
			0\\
			x_3
			\end{pmatrix} = x_3\begin{pmatrix}
			\frac{4}{3}\\
			0\\
			1
			\end{pmatrix}$$
			Luego, la solución del sistema homogeneo es
			$$S = Gen\left\{\begin{pmatrix}
			\frac{4}{3}\\
			0\\
			1
			\end{pmatrix}\right\}$$
\item Describir todas las soluciones de $Ax=b$ con $b=
			\begin{pmatrix}
			7\\
			-1\\
			-4
			\end{pmatrix}$ \\\\
			Para esto, hacemos lo mismo que en la parte a), pero aumentando la matriz por el vector $b$, es decir
			$$\left[
			\begin{array}{ccc|c}
			3 & 5 & -4 & 7\\
			-3 & -2 & 4 & -1\\
			6 & 1 & -8 & -4
			\end{array}
			\right] \sim \left[
			\begin{array}{ccc|c}
			3 & 5 & -4 & 7\\
			0 & 3 & 0 & 6\\
			0 & -9 & 0 & -18
			\end{array}
			\right] \sim \left[
			\begin{array}{ccc|c}
			3 & 5 & -4 & 7\\
			0 & 3 & 0 & 6\\
			0 & 0 & 0 & 0
			\end{array}
			\right]$$
			$$\sim \left[
			\begin{array}{ccc|c}
			1 & \frac{5}{3} & -\frac{4}{3} & \frac{7}{3}\\
			0 & 1 & 0 & 2\\
			0 & 0 & 0 & 0
			\end{array}
			\right] \sim \left[
			\begin{array}{ccc|c}
			1 & 0 & -\frac{4}{3} & -1\\
			0 & 1 & 0 & 2\\
			0 & 0 & 0 & 0
			\end{array}
			\right]$$
			Esto corresponde al sistema
			$$\begin{array}{rcl}
			x_1 - \frac{4}{3}x_3 & = & -1\\
			x_2 & = & 2\\
			0 & = & 0
			\end{array} \ra \begin{array}{rcl}
			x_1 & = & -1 + \frac{4}{3}x_3\\
			x_2 & = & 2\\
			x_3 & = & x_3
			\end{array}$$
			$$ x = \begin{pmatrix}
			-1 + \frac{4}{3}x_3\\
			2\\
			x_3
			\end{pmatrix} = \begin{pmatrix}
			-1\\2\\0
			\end{pmatrix} + x_3\begin{pmatrix}
			\frac{4}{3}\\
			0\\
			1
			\end{pmatrix}$$
			Luego, la solución del sistema $Ax = b$ es
			$$S = \begin{pmatrix}
			-1\\2\\0
			\end{pmatrix} + Gen\left\{\begin{pmatrix}
			\frac{4}{3}\\
			0\\
			1
			\end{pmatrix}\right\}$$
			Finalmente, podriamos amplificar el vector del generado por 3, para que quede más bonito, con lo que
			$$S = \begin{pmatrix}
			-1\\2\\0
			\end{pmatrix} + Gen\left\{\begin{pmatrix}
			4\\
			0\\
			3
			\end{pmatrix}\right\}$$
\end{enumerate}
\end{solucion}
\item Considere el siguiente sistema de ecuaciones, donde $a$ es una constante.
  $$\begin{array}{llll}
   x_1&+x_2&+x_3&=1 \\
    x_1&+x_2&+ax_3&=1 \\
     ax_1&+ax_2&+x_3&=a\\
      x_1&-ax_2&+ax_3&=0  
  \end{array}$$
\begin{enumerate}[a)]
\item Determine valores de $a$ para los cuales el sistema es inconsistente.
\item Determine valores de $a$ para los cuales el sistema es consistente, y encuentre la solución.
\end{enumerate}
\begin{solucion}
En primer lugar, escribamos el sistema en su forma matricial, esto es
	$$\begin{bmatrix}
	1 & 1 & 1 & 1\\
	1 & 1 & a & 1\\
	a & a & 1 & a\\
	1 &-a & a & 0
	\end{bmatrix}$$
	Ahora, lo que debemos hacer es pivotear la matriz intentando evitar dividir por $a$. En el caso de que sea estrictamente necesario hacerlo, debemos ver el caso donde $a=0$ por separado y luego continuar.
	$$\begin{bmatrix}
	1 & 1 & 1 & 1\\
	1 & 1 & a & 1\\
	a & a & 1 & a\\
	1 &-a & a & 0
	\end{bmatrix} 
	\stackbin[F_4 - F_1]{F_2 - F_1}{\stackbin[]{F_3 - aF_1}{\wsim}}
	\begin{bmatrix}
	1 & 1 & 1 & 1\\
	0 & 0 & a-1 & 0\\
	0 & 0 & 1-a & 0\\
	0 &-1-a & a-1 & -1
	\end{bmatrix} 
	\stackbin[]{F_2 \lra F_4}{\wsim}
	\begin{bmatrix}
	1 & 1 & 1 & 1\\
	0 &-1-a & a-1 & -1\\
	0 & 0 & 1-a & 0\\
	0 & 0 & a-1 & 0
	\end{bmatrix}$$
	$
	\stackbin[]{F_4 + F_3}{\wsim}
	\begin{bmatrix}
	1 & 1 & 1 & 1\\
	0 &-1-a & a-1 & -1\\
	0 & 0 & 1-a & 0\\
	0 & 0 & 0 & 0
	\end{bmatrix}
	\stackbin[]{F_2 + F_3}{\wsim}
	\begin{bmatrix}
	1 & 1 & 1 & 1\\
	0 &-1-a & 0 & -1\\
	0 & 0 & 1-a & 0\\
	0 & 0 & 0 & 0
	\end{bmatrix}$
\begin{enumerate}[a)]
\item 
En la tercera fila no existe ningún valor de $a$ para el cual el sistema es inconsistente, por lo que esta no nos restringe $a$.
Sin embargo, en la segunda fila, si podemos elegir un valor de $a$ para que el sistema sea inconsistente. Esto es,
$$-1-a = 0 \ra a = -1$$
\item Para el resto de valores de $a$, el sistema es consistente, es decir, para
$$a \neq -1$$
Luego, teniendo la matriz en su forma escalonada, obtener la solución del sistema es sencillo.\\
Veamos primero el caso donde $a=1$,
$$\begin{bmatrix}
1 & 1 & 1 & 1\\
0 &-2 & 0 & -1\\
0 & 0 & 0 & 0\\
0 & 0 & 0 & 0
\end{bmatrix} \sim
\begin{bmatrix}
1 & 1 & 1 & 1\\
0 & 1 & 0 & \frac{1}{2}\\
0 & 0 & 0 & 0\\
0 & 0 & 0 & 0
\end{bmatrix} \sim
\begin{bmatrix}
1 & 0 & 1 & \frac{1}{2}\\
0 & 1 & 0 & \frac{1}{2}\\
0 & 0 & 0 & 0\\
0 & 0 & 0 & 0
\end{bmatrix} $$
Luego,
$$\begin{array}{rcl}
x_1 & = & -x_3 + \dfrac{1}{2}\\
x_2 & = & \dfrac{1}{2}\\
x_3 & = & 0 
\end{array} \ra 
x = \begin{pmatrix}
\frac{1}{2} \\
\frac{1}{2} \\
0
\end{pmatrix} + \left<\begin{pmatrix}
-1\\0\\1
\end{pmatrix}\right>$$
Si es que $a \neq 1$, entonces
$$\begin{bmatrix}
1 & 1 & 1 & 1\\
0 &-1-a & 0 & -1\\
0 & 0 & 1-a & 0\\
0 & 0 & 0 & 0
\end{bmatrix}
\stackbin[F_3 \leftarrow \frac{1}{1-a}F_3]{F_2 \leftarrow \frac{1}{-1-a}F_2}{\wsim}
\begin{bmatrix}
1 & 1 & 1 & 1\\
0 & 1 & 0 & \frac{1}{1+a}\\
0 & 0 & 1 & 0\\
0 & 0 & 0 & 0
\end{bmatrix}
\stackbin[F_1-F_3]{F_1 - F_2}{\wsim}
\begin{bmatrix}
1 & 0 & 0 & \frac{a}{1+a}\\
0 & 1 & 0 & \frac{1}{1+a}\\
0 & 0 & 1 & 0\\
0 & 0 & 0 & 0
\end{bmatrix}$$
Luego,
$$\begin{array}{rcl}
x_1 & = & \dfrac{a}{1+a}\\
x_2 & = & \dfrac{1}{1+a}\\
x_3 & = & 0 
\end{array} \ra 
x = \begin{pmatrix}
\frac{a}{1+a} \\
\frac{1}{1+a} \\
0
\end{pmatrix}$$
Notemos que esta última corresponde siempre a una solución única pero que depende de $a$.
\end{enumerate}
\end{solucion}
\item Sea la matriz $A=
	\begin{bmatrix}
	1 & 3 & 4\\
	-4 & 2 & -6\\
	-3 & -2 & -7
	\end{bmatrix}
	$ y $b$ un vector en $R^3$. ¿La ecuación $Ax=b$ es consistente para todo $b$?
\begin{solucion}
Diremos que $b = \begin{pmatrix}
	b_1 \\ b_2 \\ b_3
	\end{pmatrix}$. Luego, el sistema $Ax = b$ se puede representar como
	$$\left[
	\begin{array}{ccc|c}
	1 & 3 & 4 &b_1\\
	-4 & 2 & -6 & b_2\\
	-3 & -2 & -7 & b_3
	\end{array}
	\right] \sim \left[
	\begin{array}{ccc|c}
	1 & 3 & 4 &b_1\\
	0 & 14 & 10 & b_2 + 4b_1\\
	0 & 7 & 5 & b_3 + 3b_1
	\end{array}
	\right]$$
	$$ \sim \left[
	\begin{array}{ccc|c}
	1 & 3 & 4 &b_1\\
	0 & 14 & 10 & b_2 + 4b_1\\
	0 & 0 & 0 & b_3 + 3b_1 - \frac{1}{2}(b_2+4b_1)
	\end{array}
	\right] \sim \left[
	\begin{array}{ccc|c}
	1 & 3 & 4 &b_1\\
	0 & 14 & 10 & b_2 + 4b_1\\
	0 & 0 & 0 & b_1 - \frac{1}{2}b_2 + 4b_3
	\end{array}
	\right] $$
	Luego, el sistema será consistente para $b_1 - \dfrac{1}{2}b_2 + b_3 = 0$. Esto se puede representar como
	$$\begin{array}{rcl}
	b_1 & = & \frac{1}{2}b_2 - b_3\\
	b_2 & = & b_2 \\
	b_3 & = & b_3
	\end{array} \ra b = \begin{pmatrix}
	\frac{1}{2}b_2 - b_3 \\
	b_2\\
	b_3 
	\end{pmatrix} = b_2\begin{pmatrix}
	\frac{1}{2} \\
	1\\
	0 
	\end{pmatrix} +  b_3\begin{pmatrix}
	-1 \\
	0\\
	1 
	\end{pmatrix} $$
	Finalmente,
	$$b = Gen \left\{ \begin{pmatrix}
	\frac{1}{2} \\
	1\\
	0 
	\end{pmatrix}, \begin{pmatrix}
	-1 \\
	0\\
	1
	\end{pmatrix}\right\}$$
\end{solucion}
\item Determinar las condiciones en $a$ para que la matriz
	$$A = \begin{bmatrix}
	a & 2a & 0 & 0 \\
	0 & 1 & 0 & 3a-1\\
	0 & 1 & a-1 & 2a-1\\
	a & 2a & 0 & a
	\end{bmatrix}$$
	sea invertible
\begin{solucion}
Recordemos que el que una matriz sea invertible es equivalente a que esta tenga solución única.\\\\En primer lugar, debemos pivotear $A$ para dejarla en su forma escalonada:
		$$\begin{bmatrix}
		a & 2a & 0 & 0 \\
		0 & 1 & 0 & 3a-1\\
		0 & 1 & a-1 & 2a-1\\
		a & 2a & 0 & a
		\end{bmatrix} \widesim{F_4-F_1}
		\begin{bmatrix}
		a & 2a & 0 & 0 \\
		0 & 1 & 0 & 3a-1\\
		0 & 1 & a-1 & 2a-1\\
		0 & 0 & 0 & a
		\end{bmatrix} \widesim{F_3-F_2}
		\begin{bmatrix}
		a & 2a & 0 & 0 \\
		0 & 1 & 0 & 3a-1\\
		0 & 0 & a-1 & -a\\
		0 & 0 & 0 & a
		\end{bmatrix}$$
		Notemos que la última fila nos puede dar problemas. Si $a=0$, el sistema va a tener solo 3 pivotes por lo que, en caso de ser consistente, tendría infinitas soluciones. Dicho esto, una restricción es $a\neq 0$
		
		En la tercera fila, debemos fijarnos que $a-1 \neq 0$, ya que de lo contrario, esta fila sería un múltiplo de la cuarta fila, por lo que habrían 3 pivotes y tendríamos soluciones infinitas. Dicho esto, tenemos que $a \neq 1$.
		
		De esta forma, el sistema siempre tendrá 4 pivotes y por lo tanto, tendrá solución única.
		
		En resumen, las condiciones son
		$$a \neq 0, \quad a \neq 1$$
\end{solucion}
\item Sean los vectores 
	$$
	v_1 = \begin{pmatrix}
	1\\
	1\\
	2
	\end{pmatrix};\qquad
	v_2 = \begin{pmatrix}
	2\\
	2\\
	-3
	\end{pmatrix}; \qquad
	v_3 = \begin{pmatrix}
	0\\
	1\\
	3
	\end{pmatrix}$$
	determinar si $Gen\{v_1, v_2, v_3\} = R^3$.
\begin{solucion}
Para ver si estos tres vectores generan $\R^3$, tiene que pasar que todos sean L.I. entre si. Una forma de ver esto, es formar una matriz con los vectores y ver la cantidad de pivotes que hay, es decir
		$$\begin{bmatrix}
			\\
			v_1 & v_2 & v_3\\
			&&
		\end{bmatrix} \sim 
		\begin{bmatrix}
		1 & 2 & 0\\
		1 & 2 & 1\\
		2 & -3 & 3
		\end{bmatrix} \sim
		\begin{bmatrix}
		1 & 2 & 0\\
		0 & 0 & 1\\
		0 & -7 & 3
		\end{bmatrix} \sim
		\begin{bmatrix}
		1 & 2 & 0\\
		0 & -7 & 3\\
		0 & 0 & 1
		\end{bmatrix}$$
		Como la matriz tiene 3 pivotes, quiere decir que hay 3 vectores L.I., es decir, todos son linealmente independientes entre si, formando asi $\R^3$
\end{solucion}
\item Sean 
$$v_1=\left(\begin{array}{r}
  1\\0\\-2
\end{array}\right), \quad v_2=\left(\begin{array}{r}
  -2\\1\\7
\end{array}\right) \quad y \quad v_3=\left(\begin{array}{r}
  h\\0\\-2
\end{array}\right)$$
¿Para qué valor(es) de $h$ $Gen\{v_1, v_2, v_3\}=Gen\{v_1,v_2\}$ ?
\begin{solucion}
Para que $Gen\{v_1, v_2, v_3\}=Gen\{v_1,v_2\}$, debe ocurrir que estos generados tengan la misma cantidad de vectores L.I. o dicho de otra forma, al poner estos vectores en una matriz y escalonarla, las matrices deben tener la misma cantidad de pivotes.\\
	\\
	Comencemos por $Gen\{v_1,v_2\}$, dado que $h$ no influye en este generado.
	$$\begin{bmatrix}
	\\
	v_1 & v_2\\
	&&
	\end{bmatrix} \sim 
	\begin{bmatrix}
	1 & -2\\
	0 & 1\\
	-2 & 7
	\end{bmatrix} \sim 
	\begin{bmatrix}
	1 & -2\\
	0 & 1\\
	0 & 3
	\end{bmatrix} \sim 
	\begin{bmatrix}
	1 & -2\\
	0 & 1\\
	0 & 0
	\end{bmatrix}$$
	Notemos que esta matriz tiene 2 pivotes, por lo que debemos elegir $h$, de tal manera que $Gen\{v_1, v_2, v_3\}$ también tenga 2 pivotes.\\
	\\
	Procedamos ahora con $Gen\{v_1, v_2, v_3\}$,
	$$\begin{bmatrix}
	\\
	v_1 & v_2 & v_3\\
	&&
	\end{bmatrix} \sim 
	\begin{bmatrix}
	1 & -2 & h\\
	0 & 1 & 0\\
	-2 & 7 & -2
	\end{bmatrix} \sim 
	\begin{bmatrix}
	1 & -2 & h\\
	0 & 1 & 0\\
	0 & 3 & 2h-2
	\end{bmatrix} \sim 
	\begin{bmatrix}
	1 & -2 & h\\
	0 & 1 & 0\\
	0 & 0 & 2h-2
	\end{bmatrix}$$
	Luego, para que la matriz tenga 2 pivotes, debe ocurrir que
	$$2h-2 = 0 \ra h = 1$$
\end{solucion}
\item Sean $\{u, v, w\}$ un conjunto de vectores linealmente independientes. Demuestre que el conjunto $\{u+v, u+2w, v+3u+w\}$ es linealmente independiente.
\begin{solucion}
$$P.D. \quad \{y, v, w\}\quad L.I. \ra \{u+v, u+2w, v+3u+w\}\quad L.I.$$
		Para que $\{u+v, u+2w, v+3u+w\}$ sea L.I., tiene que cumplirse que el sistema
		$$\alpha(u+v) + \beta(u+2w) + \gamma(v+3u+w) = 0$$
		tenga solucion única $\begin{pmatrix}
		\alpha\\ \beta \\ \gamma
		\end{pmatrix} = \begin{pmatrix} 0\\0\\0\end{pmatrix}$\\
		Trabajemos entonces con el sistema
		$$\alpha(u+v) + \beta(u+2w) + \gamma(v+3u+w) = 0$$
		$$\alpha u+ \alpha v + \beta u+2\beta w + \gamma v+3\gamma u+\gamma w = 0$$
		$$(\alpha + \beta + 3 \gamma)u + (\alpha + \gamma)v + (2\beta + \gamma)w = 0$$
		Recordemos que ${u, v, w}$ es L.I, por lo que debe cumplirse que
		$$\begin{array}{rcl}
		\alpha + \beta + 3 \gamma & = & 0\\
		\alpha + \gamma & = & 0\\
		2\beta + \gamma & = & 0
		\end{array}$$
		Este sistema lo podemos expresar de forma matricial,
		$$\begin{bmatrix}
		1 & 1 & 3\\
		1 & 0 & 1 \\
		0 & 2 & 1
		\end{bmatrix} \stackrel{F.E.}{\sim} \begin{bmatrix}
		1 & 0 & 1\\
		0 & 1 & -2\\
		0 & 0 & 1
		\end{bmatrix}$$
		Cuya solución es
		$$\begin{array}{rcl}
		\alpha & = & 0\\
		\beta & = & 0\\
		\gamma & = & 0
		\end{array}$$
		$$q.e.d$$
	
\end{solucion}
\item Demuestre que el conjunto $\{u, v, w\}$ es L.I. si y solo si el conjunto $\{u+v, u+w, v+w\}$ es L.I.
\begin{solucion}
Como nos dicen si y solo si, debemos demostrar en ambas direcciones
		$$(\Longrightarrow)$$
		Digamos que $\{u, v, w\}$ es L.I.\\
		\\
		P.D. $\{u+v, u+w, v+w\}$ es L.I.\\
		\\
		Para que $\{u+v, u+w, v+w\}$ sea L.I., el sistema
		$$\alpha (u+v) + \beta (u+w) + \gamma (v+w) = 0$$
		debe tener solución única $\alpha = 0$, $\beta = 0$, $\gamma = 0$
		Reordenando,
		$$\alpha (u+v) + \beta (u+w) + \gamma (v+w) = 0$$
		$$(\alpha + \beta) u + (\alpha + \gamma) v + (\beta + \gamma) w = 0$$
		Como sabemos que el conjunto $\{u, v, w\}$ es L.I., entonces tenemos que
		$$\begin{array}{rl}
		\alpha + \beta & = 0\\
		\alpha + \gamma & = 0\\
		\beta + \gamma & = 0
		\end{array}$$
		Este sistema tiene por solución
		$$\alpha = 0, \quad \beta = 0, \quad \gamma = 0$$
		Con lo que demostramos que $\{u+v, u+w, v+w\}$ es L.I.
		$$(\Longleftarrow)$$
		Digamos que $\{u+v, u+w, v+w\}$ es L.I.\\
		\\
		P.D. $\{u, v, w\}$ es L.I.\\
		\\
		Para que $\{u, v, w\}$ sea L.I., el sistema
		$$c_1u + c_2v + c_3w$$
		debe tener solución única $c_1 = 0$, $c_2 = 0$, $c_3 = 0$\\
		\\
		Como $\{u+v, u+w, v+w\}$ es L.I., sabemos que el sistema
		$$\alpha (u+v) + \beta (u+w) + \gamma (v+w) = 0$$
		tiene solución única $\alpha = 0$, $\beta = 0$, $\gamma = 0$
		Reordenando,
		$$\alpha (u+v) + \beta (u+w) + \gamma (v+w) = 0$$
		$$(\alpha + \beta) u + (\alpha + \gamma) v + (\beta + \gamma) w = 0$$
		Con lo que tenemos que
		$$c_1 = \alpha + \beta = 0, \quad c_2 = \alpha + \gamma = 0, \quad c_3 = \beta + \gamma = 0$$
		es la única solución del sistema. \\
		\\
		Luego, $\{u, v, w\}$ es L.I.
		$$q.e.d$$
\end{solucion}
\item Sea $A$ una matriz de $4 \times 4$ tal que $A=\begin{bmatrix}a_1 & 2a_1 & a_2 & a_1-a_2\end{bmatrix}$ con $a_1, a_2 \in \R^4$ vectores linealmente independientes, y sea $B$ una matriz inyectiva de $4 \times 2$ tal que $A \cdot B = 0$. Determine una matriz con las características de $B$.
\begin{solucion}
Del enunciado sabemos que $B$ es inyectiva, por lo que todas sus columnas deben ser $L.I.$\\

Además, sabemos que $A\cdot B= 0$. En otras palabras, esto significa que ambas columnas de $B$ deben combinar linealmente las columnas de $A$ de tal forma que se anulen entre si y a la vez ambas columnas de $B$ deben ser $L.I.$\\

Busquemos en primer lugar, un vector que anule las columnas de $A$. Sea
$$b = \begin{pmatrix}
b_1 \\ b_2 \\ b_3 \\ b_4
\end{pmatrix}$$
Debemos encontrar
$$Ab = 0$$
Esto es,
$$a_1b_1 + 2a_1 b_2 + a_2b_3 + (a_1-a_2)b_4 = 0$$
Reordenando,
$$(b_1 + 2b_2 + b_4)a_1 + (b_3 - b_4)a_2 = 0$$
Como $a_1$ y $a_2$ son $L.I.$, la única forma de solucionar esto es
$$\begin{array}{rcl}
b_1 + 2b_2 + b_4 & = & 0\\
b_3 - b_4 & = & 0
\end{array} 
\Longrightarrow
\begin{array}{rcl}
b_1 & = & -2b_2 - b_3\\
b_2 & = & b_2\\
b_3 & = & b_3\\
b_4 & = & b_3
\end{array}
\Longrightarrow
b = b_2\begin{pmatrix}
-2 \\ 1 \\ 0 \\ 0
\end{pmatrix} 
+
b_3\begin{pmatrix}
-1 \\ 0 \\ 1 \\ 1
\end{pmatrix}
$$
Luego, todos los vectores de esta forma cumplirán con $Ab = 0$. Como necesitamos dos vectores $L.I.$ que cumplan con esto, basta con tomar los dos vectores generadores de $b$ como las columnas de $B$, es decir
$$B = \begin{bmatrix}
-2 & -1\\
1 & 0\\
0 & 1 \\
0 & 1
\end{bmatrix}$$
\end{solucion}
\item Sea $A$ una matriz de $n \times n$ tal que existe una matriz $B \neq 0$ tal que $AB = 0$.\\
Demuestre que $A$ no es invertible.
\begin{solucion}
Como por lo menos alguna de las columnas de $B$ es $b_i \neq \vec{0}$, la ecuación $Ax = 0$ posee una solución no trivial $x = b_i$, por lo que no puede ser invertible, ya que sus columnas son $LD$
\end{solucion}
\item Sea $T: \R^7 \ra \R^9$ una transformación lineal. Suponga que el conjunto $\{u,v\}$ es linealmente independiente en $\R^7$ pero $\{T(u), T(v)\}$ es linealmente dependiente. Demuestre que la ecuación $T(x)=0$ admite soluciones no triviales.
\begin{solucion}
Como el conjunto $\{T(u), T(v)\}$ es $LD$, existen $\alpha, \beta \neq 0$, tales que 
$$\alpha T(u) + \beta T(v) = 0$$
Usando la propiedad de linealidad para las transformaciones lineales, también tenemos que
$$ T(\alpha u +\beta v) = 0$$
Por otro lado, como $\{u, v\}$ es $LI$ y $\alpha, \beta \neq 0$, el vector $\alpha u +\beta v$ no puede ser cero. Por ende, se concluye que $\alpha u +\beta v$ es una solución no trivial de $T(x) = 0$.
\end{solucion}
\item Sea $L: P_2(\R) \ra \R^2$ una transformación lineal tal que
	$$L(1+x) = \begin{pmatrix}
	1\\
	1
	\end{pmatrix}, \quad L(1+x+x^2) = \begin{pmatrix}
	1\\
	-1
	\end{pmatrix}\ y \ L(1+2x) = \begin{pmatrix}
	1\\
	2
	\end{pmatrix}$$
	determine $L(a+bx+cx^2)$ para todo $a,b,c \in \R$.
\begin{solucion}
Recordemos la siguientes propiedades de las transformaciones lineales
		$$L(v_1) + L(v_2) = L(v_1 + v_2), \quad \alpha L(v) = L(\alpha v)$$
		Debemos buscar 
		$$L(1), \quad L(x), \quad L(x^2)$$
		Para esto, debemos jugar con la información que nos dan hasta encontrar cada uno de ellos.
		\begin{center}\rule{14.5cm}{0.1pt}\end{center}
		$$L(1+x+x^2) - L(1+x) = \begin{pmatrix}
		1\\-1
		\end{pmatrix} - \begin{pmatrix}
		1\\1
		\end{pmatrix}$$
		$$L(x^2) = \begin{pmatrix}
		0\\-2
		\end{pmatrix}$$
		
		\begin{center}\rule{14.5cm}{0.1pt}\end{center}
		$$L(1+2x) - L(1+x) = \begin{pmatrix}
		1\\2
		\end{pmatrix} - \begin{pmatrix}
		1\\1
		\end{pmatrix}$$
		$$L(x) = \begin{pmatrix}
		0\\1
		\end{pmatrix}$$
		
		\begin{center}\rule{14.5cm}{0.1pt}\end{center}
		$$L(1+x) - L(x) = \begin{pmatrix}
		1\\1
		\end{pmatrix} - \begin{pmatrix}
		0\\1
		\end{pmatrix}$$
		$$L(x) = \begin{pmatrix}
		1\\0
		\end{pmatrix}$$
		
		Luego, 
		$$L(a+bx+cx^2) = aL(1) + bL(x) + cL(x^2)$$
		$$L(a+bx+cx^2) = aL\begin{pmatrix}1\\0\end{pmatrix} + b\begin{pmatrix}0\\1\end{pmatrix} + c\begin{pmatrix}0\\-2\end{pmatrix}$$
		$$L(a+bx+cx^2) = \begin{pmatrix}a\\b-2c\end{pmatrix}$$
\end{solucion}
\item Sea $T$ una transformación lineal tal que
	$$T\begin{bmatrix}2 \\ 1\end{bmatrix} = \begin{bmatrix}1 \\ 0 \\ 1 \\ 0\end{bmatrix} 
	\quad y \quad
	T \begin{bmatrix}3 \\ 1\end{bmatrix}  =\begin{bmatrix}0 \\ 1 \\ 0 \\ 1\end{bmatrix}$$
	Determine la matriz que representa a $T$
\begin{solucion}
Estamos buscando la matriz $\begin{bmatrix}
		T\begin{bmatrix}
		1 \\ 0
		\end{bmatrix}
		&
		
		T\begin{bmatrix}
		0 \\ 1
		\end{bmatrix}
		\end{bmatrix}$.\\
		Entonces, lo que necesitamos es buscar $T\begin{bmatrix}
		1 \\ 0
		\end{bmatrix}$ y $T\begin{bmatrix}
			0 \\ 1
		\end{bmatrix}$ y para ello, debemos escribir cada uno como una combinación lineal de $T\begin{bmatrix}
		2 \\ 1
		\end{bmatrix}$ y $T\begin{bmatrix}
		3 \\ 1
		\end{bmatrix}$, es decir
		$$\begin{bmatrix}
		1 \\ 0
		\end{bmatrix} = \alpha \begin{bmatrix}
		2 \\ 1
		\end{bmatrix} +  \beta \begin{bmatrix}
		3 \\ 1
		\end{bmatrix} \quad y \quad
		\begin{bmatrix}
		0 \\ 1
		\end{bmatrix} = \gamma \begin{bmatrix}
		2 \\ 1
		\end{bmatrix} +  \delta \begin{bmatrix}
		3 \\ 1
		\end{bmatrix}$$
		O sea, debemos resolver los sistemas
		$$\begin{array}{rcl}
		2\alpha + 3\beta & = & 1\\
		\alpha + \beta & = & 0
		\end{array} \qquad \qquad \qquad
		\begin{array}{rcl}
		2\gamma + 3\delta & = & 0\\
		\gamma + \delta & = & 1
		\end{array}$$
		La solución de estos sistemas es
		$$\alpha = -1, \quad \beta = 1, \quad \gamma = 3, \quad \delta = -2$$
		Por lo que
		$$\begin{bmatrix}
		1 \\ 0
		\end{bmatrix} = \begin{bmatrix}
		3 \\ 1
		\end{bmatrix}- \begin{bmatrix}
		2 \\ 1
		\end{bmatrix}  \quad y \quad
		\begin{bmatrix}
		0 \\ 1
		\end{bmatrix} = 3 \begin{bmatrix}
		2 \\ 1
		\end{bmatrix} - 2 \begin{bmatrix}
		3 \\ 1
		\end{bmatrix}$$
		Luego,
		$$T\begin{bmatrix}
		1 \\ 0
		\end{bmatrix} = T\begin{bmatrix}
		3 \\ 1
		\end{bmatrix}- T\begin{bmatrix}
		2 \\ 1
		\end{bmatrix} = \begin{bmatrix}0 \\ 1 \\ 0 \\ 1\end{bmatrix} - \begin{bmatrix}1 \\0 \\ 1 \\ 0\end{bmatrix} = \begin{bmatrix}-1 \\ 1 \\ -1 \\ 1\end{bmatrix}$$
		y
		$$T\begin{bmatrix}
		0 \\ 1
		\end{bmatrix} = 3 T\begin{bmatrix}
		2 \\ 1
		\end{bmatrix} - 2 T\begin{bmatrix}
		3 \\ 1
		\end{bmatrix} = 3 \begin{bmatrix}1 \\ 0 \\ 1 \\ 0\end{bmatrix} - 2 \begin{bmatrix}0 \\ 1 \\ 0 \\ 1\end{bmatrix} = \begin{bmatrix}3 \\ -2 \\ 3 \\ -2\end{bmatrix}$$
		Finalmente, la matriz que representa $T$ es
		$$\begin{bmatrix}
		T\begin{bmatrix}
		1 \\ 0
		\end{bmatrix}
		&
		
		T\begin{bmatrix}
		0 \\ 1
		\end{bmatrix}
		\end{bmatrix} =
		\begin{bmatrix}
		-1 & 3 \\
		1 & -2 \\
		-1 & 3 \\
		1 & -2
		\end{bmatrix}$$
\end{solucion}
\item Sea la transformación lineal $T$, dada por la matriz
	$$T = \begin{bmatrix}
	1 & 2 & -1\\
	2 & 4 & -2
	\end{bmatrix}$$
	Determine la imagen del plano de ecuación $x_1 + x_3 = 1$ por T.
\begin{solucion}
En primer lugar, escribimos el plano en su forma paramétrica
		$$\begin{array}{rcl}
		x_1 & = &1-x_3\\
		x_2 & = &x_2\\
		x_3 & = &x_3
		\end{array}$$
		Vemos entonces que 3 puntos arbitrarios pertenecienctes al plano son
		$$P_1(1,0,0), \quad P_2(1,1,0), \quad P_3(0,0,1) $$
		Luego, dos vectores directores que definen al plano son
		$$\vec{d_1} = \vec{P_1P_2} = (0,1,0), \quad =\vec{d_2} = \vec{P_1P_3} = (-1,0,1)$$
		Con esto, podemos expresar el plano en su forma vectorial
		$$\vec{P}= \vec{P_1} + \alpha \vec{d_1}  + \beta \vec{d_2}$$
		Para encontrar la imagen del plano por $T$, basta con hacer
		$$T\vec{P}= T\vec{P_1} + \alpha T\vec{d_1}  + \beta T\vec{d_2}$$
		Como
		$$T\vec{P_1} = \begin{bmatrix}
		1 & 2 & -1\\
		2 & 4 & -2
		\end{bmatrix} \begin{bmatrix}
		1 \\ 0 \\ 0
		\end{bmatrix} = \begin{bmatrix}
		1 \\ 2
		\end{bmatrix}$$
		$$T\vec{d_1} = \begin{bmatrix}
		1 & 2 & -1\\
		2 & 4 & -2
		\end{bmatrix} \begin{bmatrix}
		0 \\ 1 \\ 0
		\end{bmatrix} = \begin{bmatrix}
		2 \\ 4
		\end{bmatrix}$$
		$$T\vec{d_2} = \begin{bmatrix}
		1 & 2 & -1\\
		2 & 4 & -2
		\end{bmatrix} \begin{bmatrix}
		-1 \\ 0 \\ 1
		\end{bmatrix} = \begin{bmatrix}
		-2 \\ -4
		\end{bmatrix}$$
		Tenemos que
		$$TP =  \begin{bmatrix}
		1 \\ 2
		\end{bmatrix} + \alpha  \begin{bmatrix}
		2 \\ 4
		\end{bmatrix} + 
		\beta  \begin{bmatrix}
		-2 \\ -4
		\end{bmatrix} =  \begin{bmatrix}
		1 + 2\alpha  - 2\beta \\
		2 + 4\alpha - 5\beta
		\end{bmatrix} = 
		(1+2\alpha - 2\beta)  \begin{bmatrix}
		1 \\ 2
		\end{bmatrix}$$
		En otras palabras, la imagen del plano por la traslación $T$ corresponde a los ponderados de $\begin{bmatrix}
			1 \\ 2
		\end{bmatrix}$, es decir
		$$T\vec{P} = Gen\left\{ \begin{bmatrix}
		1 \\ 2
		\end{bmatrix} \right\}$$
\end{solucion}
\item Sea $\mathbb{P} \subset \R^3$ el plano de ecuación $x_1-x_2+1=0$ y sea $T:\R^3 \ra \R^4$ la transformación dada por $T(x) = Ax$, donde
$$A = \begin{bmatrix}
1 & 0 & 3\\
-1 & 1 & 0 \\
0 & 1 & 3\\
0 & -2 & -6
\end{bmatrix}$$
Demuestre que la imagen del plano $\mathbb{P}$ bajo la transformación $T$ es una recta en $\R^4$.
\begin{solucion}
Despejando $x_2$ en la ecuación del plano, obtenemos $x_2 = x_1 + 1$.\\

Reemplazando en la transformación lineal,
{\small $$T(x) = T(x_1, x_2, x_3) = T(x_1, x_1+1, x_3) = \begin{bmatrix}
1 & 0 & 3\\
-1 & 1 & 0 \\
0 & 1 & 3\\
0 & -2 & -6
\end{bmatrix}
\begin{pmatrix}
x_1 \\
x_1+1\\
x_3
\end{pmatrix} = 
\begin{pmatrix}
x_1 +3x_3\\
1\\
1+x_1+3x_3\\
-2-2x_1-6x_3
\end{pmatrix}$$}
Esto lo podemos descomponer en
$$T(x) = 
\begin{pmatrix}
x_1 +3x_3\\
1\\
1+x_1+3x_3\\
-2-2x_1-6x_3
\end{pmatrix} =
\begin{pmatrix}
0\\
1\\
1\\
-2
\end{pmatrix} +
x_1 \begin{pmatrix}
1\\
0\\
1\\
-2
\end{pmatrix} +
x_3 \begin{pmatrix}
3\\
0\\
3\\
-6
\end{pmatrix}
$$
Notemos ahora que el vector asociado a $x_3$ es el triple del vector asociado a $x_1$, por lo que podemos factorizarlo de la siguiente forma
$$T(x) = 
\begin{pmatrix}
0\\
1\\
1\\
-2
\end{pmatrix} +
x_1 \begin{pmatrix}
1\\
0\\
1\\
-2
\end{pmatrix} +
3x_3 \begin{pmatrix}
1\\
0\\
1\\
-2
\end{pmatrix} = 
\begin{pmatrix}
0\\
1\\
1\\
-2
\end{pmatrix} +
(x_1 + 3x_3) \begin{pmatrix}
1\\
0\\
1\\
-2
\end{pmatrix}
$$
{\small Esto corresponde a la ecuación de una recta con dirección $\begin{pmatrix}
1\\
0\\
1\\
-2
\end{pmatrix}$ y que pasa por $\begin{pmatrix}
0\\
1\\
1\\
-2
\end{pmatrix}$.}
\end{solucion}
\item Determine si las siguientes afirmaciones son verdaderas o falsas.
\begin{enumerate}[a)]
\item Para todo $u, v \in \R^n$ se tiene que $||u+v|| < ||u||+||v||$
\item Si $u \in Gen\{v_1, v_2\}$ entonces $Gen\{u, v_1\} \subset Gen\{u, v_2\}$
\item Sea $A$ una matriz de $n \times m$ tal que $n < m $, entonces las columnas de A son L.D.
\item Si $A$ es una matriz de $2 \times 3$ tal que $Ax = \begin{pmatrix}
		1\\
		1
		\end{pmatrix}$ y $ Ax = \begin{pmatrix}
		1\\
		2
		\end{pmatrix}$ tienen solución, entonces las filas de $A$ son L.I.
\end{enumerate}
\begin{solucion}

\begin{enumerate}[a)]
\item Para todo $u, v \in \R^n$ se tiene que $||u+v|| < ||u||+||v||$\\\\
			Tomemos $u = v = \vec{0}$. Luego,
			$$||\vec{0} + \vec{0}|| < ||\vec{0}|| + ||\vec{0}||$$
			$$0 < 0$$
			Lo que es falso. Finalmente, la afirmación es FALSA
\item Si $u \in Gen\{v_1, v_2\}$ entonces $Gen\{u, v_1\} \subset Gen\{u, v_2\}$\\\\
			Tomemos
			$$u = \begin{pmatrix}
			0 \\ 0 
			\end{pmatrix}, \quad v_1 = \begin{pmatrix}
			1 \\ 0 
			\end{pmatrix}, \quad v_2 = \begin{pmatrix}
			0 \\ 1
			\end{pmatrix}$$
			Aqui se cumple que $u \in Gen\{v_1, v_2\}$. Sin embargo,
			$$Gen\left\{\begin{pmatrix}
			0 \\ 0 
			\end{pmatrix}, \begin{pmatrix}
			1 \\ 0 
			\end{pmatrix}\right\} \subset Gen\left\{\begin{pmatrix}
			0 \\ 0 
			\end{pmatrix}, \begin{pmatrix}
			0 \\ 1
			\end{pmatrix}\right\}$$
			$$Gen\left\{\begin{pmatrix}
			1 \\ 0 
			\end{pmatrix}\right\} \subset Gen\left\{\begin{pmatrix}
			0 \\ 1
			\end{pmatrix}\right\}$$
			Lo que claramente no es cierto. Luego, la afirmación es FALSA
\item Sea $A$ una matriz de $n \times m$ tal que $n < m $, entonces las columnas de A son L.D.\\\\
			Esto significa que la matriz esta compuesta por $m$ vectores en $\R^n$. Dicho esto, notemos que pueden haber a lo mas $n$ vectores L.I. (máximo $n$ pivotes) Como $m>n$, deben haber vectores L.D., por lo que la afirmación es VERDADERA.
\item Si $A$ es una matriz de $2 \times 3$ tal que $Ax = \begin{pmatrix}
			1\\
			1
			\end{pmatrix}$ y $ Ax = \begin{pmatrix}
			1\\
			2
			\end{pmatrix}$ tienen solución, entonces las filas de $A$ son L.I.\\\\
			Asumiendo que ambos sistemas tienen solución, digamos que las filas de $A$ no serán L.I., es decir, una será múltiplo de la otra. Esto se puede expresar como
			$$A = \begin{bmatrix}
			a_{11} & a_{12} & a_{13} \\
			a_{21} & a_{22} & a_{23}
			\end{bmatrix} = \begin{bmatrix}
			a_{11} & a_{12} & a_{13} \\
			\lambda a_{11} & \lambda a_{12} & \lambda a_{13}
			\end{bmatrix}$$
			Resolvamos ahora el sistema $Ax = \begin{pmatrix}
			b_1 \\ b_2
			\end{pmatrix}$
			$$\begin{array}{lcl}
			a_{11}x_1 + a_{12}x_2 + a_{13}x_3 & = & b_1 \\
			a_{21}x_1 + a_{22}x_2 + a_{23}x_3 & = & b_2			
			\end{array} \ra \begin{array}{lcl}
			a_{11}x_1 + a_{12}x_2 + a_{13}x_3 & = & b_1 \\
			\lambda a_{11}x_1 + \lambda a_{12}x_2 + \lambda a_{13}x_3 & = & \lambda b_1			
			\end{array}$$
			Aqui podemos ver que a lo más uno de los dos sistemas puede tener solución, lo que contradice nuestra hipotesis inicial. Por esto, las filas de $A$ deben ser L.I. Con esto concluimos que la afirmación es VERDADERA
\end{enumerate}
\end{solucion}
\item Sea
	$$A = \begin{bmatrix}
	1 & 0 & 6\\
	-1 & 2 & 0\\
	0 & 5 & -1
	\end{bmatrix}$$
	Escriba $A$ como un producto de matrices elementales
\begin{solucion}
Para esto, escalonamos la matriz $A$ hasta reducirla a la matriz identidad. Es importante que hagamos esto paso por paso, sin realizar mas de una operación por fila a la vez.
		$$\begin{bmatrix}
		1 & 0 & 6\\
		-1 & 2 & 0\\
		0 & 5 & -1
		\end{bmatrix} \widesim{F_2+F_1}
		\begin{bmatrix}
		1 & 0 & 6\\
		0 & 2 & 6\\
		0 & 5 & -1
		\end{bmatrix} \widesim{F_3-\frac{5}{2}F_2}
		\begin{bmatrix}
		1 & 0 & 6\\
		0 & 2 & 6\\
		0 & 0 & -16
		\end{bmatrix} \widesim{F_3 \leftarrow -\frac{1}{16}F_3}
		\begin{bmatrix}
		1 & 0 & 6\\
		0 & 2 & 6\\
		0 & 0 & 1
		\end{bmatrix}$$
		$$\widesim{F_2 \leftarrow \frac{1}{2}F_2}
		\begin{bmatrix}
		1 & 0 & 6\\
		0 & 1 & 3\\
		0 & 0 & 1
		\end{bmatrix} \widesim{F_2 - 3F_3}
		\begin{bmatrix}
		1 & 0 & 6\\
		0 & 1 & 0\\
		0 & 0 & 1
		\end{bmatrix} \widesim{F_1 - 6F_3}
		\begin{bmatrix}
		1 & 0 & 0\\
		0 & 1 & 0\\
		0 & 0 & 1
		\end{bmatrix}$$
		Notemos que todas estas operaciones las podemos escribir como matrices elementales, aplicando la operación a la matriz identidad Sean
		$$E_1 = \begin{bmatrix}
		1 & 0 & 0\\
		1 & 1 & 0\\
		0 & 0 & 1
		\end{bmatrix}, \quad E_2 = \begin{bmatrix}
		1 & 0 & 0\\
		0 & 1 & 0\\
		0 & -\frac{5}{2} & 1
		\end{bmatrix}, E_3 = \begin{bmatrix}
		1 & 0 & 0\\
		0 & 1 & 0\\
		0 & 0 & -\frac{1}{16}
		\end{bmatrix}$$
		$$E_4 = \begin{bmatrix}
		1 & 0 & 0\\
		0 & \frac{1}{2} & 0\\
		0 & 0 & 1
		\end{bmatrix}, \quad E_5 = \begin{bmatrix}
		1 & 0 & 0\\
		0 & 1 & -3\\
		0 & 0 & 1
		\end{bmatrix}, E_6 = \begin{bmatrix}
		1 & 0 & -6\\
		0 & 1 & 0\\
		0 & 0 & 1
		\end{bmatrix}$$
		tenemos que
		$$E_6E_5E_4E_3E_2E_1A = I$$
		por lo que
		$$A = E_1^{-1}E_2^{-1}E_3^{-1}E_4^{-1}E_5^{-1}E_6^{-1}$$
		donde
		$$E_1^{-1} = \begin{bmatrix}
		1 & 0 & 0\\
		-1 & 1 & 0\\
		0 & 0 & 1
		\end{bmatrix}, \quad E_2^{-1} = \begin{bmatrix}
		1 & 0 & 0\\
		0 & 1 & 0\\
		0 & \frac{5}{2} & 1
		\end{bmatrix}, E_3^{-1} = \begin{bmatrix}
		1 & 0 & 0\\
		0 & 1 & 0\\
		0 & 0 & -16
		\end{bmatrix}$$
		$$E_4^{-1} = \begin{bmatrix}
		1 & 0 & 0\\
		0 & 2 & 0\\
		0 & 0 & 1
		\end{bmatrix}, \quad E_5^{-1} = \begin{bmatrix}
		1 & 0 & 0\\
		0 & 1 & 3\\
		0 & 0 & 1
		\end{bmatrix}, E_6^{-1} = \begin{bmatrix}
		1 & 0 & 6\\
		0 & 1 & 0\\
		0 & 0 & 1
		\end{bmatrix}$$
		Finalmente,
		$$A = \begin{bmatrix}
		1 & 0 & 0\\
		-1 & 1 & 0\\
		0 & 0 & 1
		\end{bmatrix}\begin{bmatrix}
		1 & 0 & 0\\
		0 & 1 & 0\\
		0 & \frac{5}{2} & 1
		\end{bmatrix}\begin{bmatrix}
		1 & 0 & 0\\
		0 & 1 & 0\\
		0 & 0 & \frac{1}{16}
		\end{bmatrix}\begin{bmatrix}
		1 & 0 & 0\\
		0 & -\frac{1}{2} & 0\\
		0 & 0 & 1
		\end{bmatrix}\begin{bmatrix}
		1 & 0 & 0\\
		0 & 1 & 3\\
		0 & 0 & 1
		\end{bmatrix}\begin{bmatrix}
		1 & 0 & 6\\
		0 & 1 & 0\\
		0 & 0 & 1
		\end{bmatrix}$$
\end{solucion}
\item Determinar la matriz inversa de
$$A = \begin{bmatrix}
1 & -1 & 0 \\
0 & 1 & 0 \\
2 & 0 & 1
\end{bmatrix}$$
\begin{solucion}
Para obtener $A^{-1}$, ampliamos con la matriz identidad y pivoteamos hasta obtener la matriz identidad al lado izquierdo. Luego, la matriz resultante en el lado derecho será la inversa.\\
\\
Procedamos,
$$\left[
\begin{array}{ccc|ccc}
1 &-1 & 0 & 1 & 0 & 0 \\
0 & 1 & 0 & 0 & 1 & 0 \\
2 & 0 & 1 & 0 & 0 & 1
\end{array}
\right] \sim
\left[
\begin{array}{ccc|ccc}
1 &-1 & 0 & 1 & 0 & 0 \\
0 & 1 & 0 & 0 & 1 & 0 \\
0 & 0 & 1 &-2 & 0 & 1
\end{array}
\right] \sim
\left[
\begin{array}{ccc|ccc}
1 & 0 & 0 & 1 &-1 & 0 \\
0 & 1 & 0 & 0 & 1 & 0 \\
0 & 0 & 1 &-2 & 0 & 1
\end{array}
\right]
$$
Finalmente,
$$A^{-1} = \begin{bmatrix}
 1 &-1 & 0 \\
 0 & 1 & 0 \\
-2 & 0 & 1
\end{bmatrix}$$
\end{solucion}
\item Sea $A$ una matriz de $n\times m$ con columnas LI. Demuestre que si $P=A(A^tA)^{-1}A^{t}$, entonces $P^2=P$ y $P=P^t$.
\begin{solucion}
	Recordemos que
	$$(AB)^t = B^tA^t \qquad y \qquad (AB)^{-1} = B^{-1}A^{-1}$$
	En primer lugar, demostremos que 
	$$P^2 = P \ra P = P^2$$
	Reemplazamos con $P=A(A^tA)^{-1}A^{t}$, es decir
	$$\begin{array}{rcl}
	P & = & P^2 \\
	  & = & (A(A^tA)^{-1}A^{t})(A(A^tA)^{-1}A^{t}) \\
	  & = & A(A^tA)^{-1}A^{t}A(A^tA)^{-1}A^{t} \\
	  & = & A(A^tA)^{-1}(A^{t}A)(A^tA)^{-1}A^{t} \\
	  & = & A(A^tA)^{-1}A^{t} \\
	P & = & P
	\end{array}$$
	$$\blacksquare$$
	Ahora, demostremos que $P = P^t$, esto es,
	$$\begin{array}{rcl}
	P & = & P^t \\
	  & = & (A(A^tA)^{-1}A^{t})^t \\
	  & = & ((A(A^tA)^{-1})A^{t})^t \\
	  & = & A(A(A^tA)^{-1})^t \\
	  & = & A(A^tA)^{-t}A^t \\
	  & = & A((A^tA)^t)^{-1}A^t \\
	  & = & A(A^tA)^{-1}A^t \\
	P & = & P
	\end{array}$$
	$$\blacksquare$$
\end{solucion}
\item Sea $A$ una matriz de $3\times 3$ tal que
      $$A\left(\begin{array}{r}
  1\\0\\1
\end{array}\right)= \left(\begin{array}{r}
  1\\0\\0
\end{array}\right),\quad\quad A \left(\begin{array}{r}
  1\\-2\\4
\end{array}\right)= \left(\begin{array}{r}
  1\\0\\1
\end{array}\right), \qquad  A\left(\begin{array}{r}
  -1\\1\\1
\end{array}\right)= \left(\begin{array}{r}
  0\\1\\1
\end{array}\right).$$ Calcule $A^{-1}$.
\begin{solucion}
Recordemos que para $A_{m \times n}$ y $v_1, v_2 \in \R^{n}$, siempre se cumple que
$$Av1 + Av2 = A(v_1 + v_2)$$
y recordemos también que al multiplicar dos matrices, se esta multiplicando la primera matriz con las columnas de la segunda para generar la matriz resultante, es decir, siendo $B = [v_1\ v_2\ \dots\ v_n]$,
$$AB = A[v_1\ v_2\ \dots\ v_n] = [Av_1\ Av_2\ \dots\ Av_n]$$
Procedamos con el ejercicio.\\
\\
Llamemos $v_1, v_2, v_3$ a todos los vectores que son multiplicados por $A$ en el enunciado, respectivamente y $u_1, u_2, u_3$ a sus respectivos resultados.\\
\\
Para obtener la inversa de $A$, debemos encontrar una matriz que al multiplicarla por $A$ nos de la matriz identidad.\\
\\
Busquemos entonces una forma de construir las columnas de la matriz identidad (vectores canónicos) con una combinación lineal de los vectores $u_i$. Si tuvieramos vectores más complejos, tendríamos que resolver el sistema $\alpha u_1 + \beta u_2 + \gamma u_3 = e_i$ para $i = \{1, 2, 3\}$. Sin embargo, dado que estos vectores son bastante simples, podemos hacerlo de manera analitica, esto es,
$$e_1 = u_1, \quad e_2 = , u_3-u_2+u_1, \quad e_3 = u_2 - u_1$$
Luego,
$$I = [u_1\quad u_3-u_2+u_1\quad u_2-u_1]$$
Utilizando las propiedades mencionadas al principio,
$$I = [Av_1\quad A(v_3-v_2+v_1)\quad A(v_2-v_1)]$$
$$I = A[v_1\quad v_3-v_2+v_1\quad v_2-v_1]$$
$$A^{-1} = [v_1\quad v_3-v_2+v_1\quad v_2-v_1]$$
Por lo tanto, la inversa de $A$ corresponde a
$$A^{-1} = 
\begin{bmatrix}
1 &-1 & 0 \\
0 & 3 & -2 \\
1 &-2 &3 
\end{bmatrix}$$
\end{solucion}
\item Demuestre que $T(x_1,x_2, x_3)=(x_1,x_1+x_2, x_1+x_2+x_3)$ es invertible y encuentre una f\'ormula para $T^{-1}$.
\begin{solucion}
En primer lugar determinemos la matriz de la transformación lineal $T$.\\
Sea $Ax = T(x)$,
$$A = \begin{bmatrix}
1 & 0 & 0 \\
1 & 1 & 0 \\
1 & 1 & 1
\end{bmatrix}$$
Notemos que la matriz esta escalonada "al revés". En otras palabras, $A^t$ estaría escalonada y es evidente que tiene 3 pivotes, por lo que $A$ igual tiene 3 pivotes. Al tener igual cantidad de pivotes que de columnas, la matriz $A$ es invertible.\\
\\
Ahora, busquemos $A^-1$, esto es
$$\left[
\begin{array}{ccc|ccc}
1 & 0 & 0 & 1 & 0 & 0 \\
1 & 1 & 0 & 0 & 1 & 0 \\
1 & 1 & 1 & 0 & 0 & 1
\end{array}
\right] \sim 
\left[
\begin{array}{ccc|ccc}
1 & 0 & 0 & 1 & 0 & 0 \\
0 & 1 & 0 & -1 & 1 & 0 \\
0 & 1 & 1 & -1 & 0 & 1
\end{array}
\right] \sim 
\left[
\begin{array}{ccc|ccc}
1 & 0 & 0 & 1 & 0 & 0 \\
0 & 1 & 0 & -1 & 1 & 0 \\
0 & 0 & 1 & 0 & -1 & 1
\end{array}
\right]
$$
Es decir,
$$A^{-1} = 
\begin{bmatrix}
 1 & 0 & 0 \\
-1 & 1 & 0 \\
0 & -1 & 1
\end{bmatrix}
$$
Finalmente, podemos escribir $T^{-1}(x)$ como
$$T^{-1}(x) = (x_1, -x_1 + x_2, -x_2+x_3)$$
\end{solucion}
\item Determine si las siguientes afirmaciones son verdaderas o falsas. En caso de ser verdaderas demuéstrelas y si son falsas de un contraejemplo.
\begin{enumerate}[a)]
\item Si $\{v_1, v_2, v_3\}$ es un conjunto linealmente dependiente de vectores en $\R^3$ entonces el conjunto $\{v_1, v_2\}$ también es linealmente dependiente.
\item Si $A$ es una matriz de $3 \times 3$ entonces la imagen del plano $x+y+z=0$ bajo la transformación $T(x)=Ax$ es un plano.
\item Si $A_{2\times 2} = (a_{ij}), \quad B_{2\times 3} = (b_{ij}), \quad AB = C = (c_{ij})$ tal que
$$a_{ij} = (-2)^{i+j}, \quad b_{ij} = (-3)^{i-j}$$
entonces $c_{23} = -\dfrac{56}{9}$
\end{enumerate}
\begin{solucion}

\begin{enumerate}[a)]
\item \textbf{Falso.}\\
\\
Un contraejemplo sería
$\left\{ 
\begin{pmatrix}
	1 \\ 0 \\ 0
\end{pmatrix},
\begin{pmatrix}
	0 \\ 1 \\ 0
\end{pmatrix},
\begin{pmatrix}
	1 \\ 1 \\ 0
\end{pmatrix}
\right\}$.\\
Como
$
\begin{pmatrix}
1 \\ 1 \\ 0
\end{pmatrix}
=
\begin{pmatrix}
1 \\ 0 \\ 0
\end{pmatrix}
+
\begin{pmatrix}
0 \\ 1 \\ 0
\end{pmatrix}$ el conjunto claramente es $LD$.\\
Sin embargo,
$\left\{ 
\begin{pmatrix}
1 \\ 0 \\ 0
\end{pmatrix},
\begin{pmatrix}
0 \\ 1 \\ 0
\end{pmatrix}
\right\}$ 
es $LI$, lo que contradice el enunciado.
\item  \textbf{Falso.}\\
\\
Un contraejemplo es tomar la matriz cero, ya que la imagen de cualquier conjunto por esta matriz es 0.
\item  \textbf{Verdadero.}\\
\\
$c_{23}$ corresponde al coeficiente de la matriz $C$ resultante del producto punto entre la segunda fila de la matriz $A$ y la tercera columna de la matriz $B$.\\

Luego,
$$c_{23} = \begin{pmatrix}
(-2)^{2+1}\\ (-2)^{2+2}
\end{pmatrix} \cdot
\begin{pmatrix}
(-3)^{1-3}\\ (-3)^{2-3}
\end{pmatrix} = 
\begin{pmatrix}
-8\\ 16
\end{pmatrix} \cdot
\begin{pmatrix}
\frac{1}{9}\\ -\frac{1}{3}
\end{pmatrix}=
-8 \cdot \dfrac{1}{9} + 16 \cdot -\dfrac{1}{3} = -\dfrac{56}{9}
$$
\end{enumerate}
\end{solucion}
\item En cada caso, determine si la afirmación es verdadera o falsa y justifique su respuesta.
\begin{enumerate}[a)]
\item Si el conjunto $\{v_1, v_2, v_3, v_4\}$ es L.I., entonces $\{v_1, v_2, v_3\}$ también lo es.
\item Si $A$ es una matriz de $3\times 5$ y $T$ es la transformación lineal definida por $T(x) = Ax$, entonces $T$ tiene dominio $\R^3$.
\item Un sistema con más ecuaciones que incógnitas es siempre consistente.
\end{enumerate}
\begin{solucion}

\begin{enumerate}[a)]
\item Si el conjunto $\{v_1, v_2, v_3, v_4\}$ es L.I., entonces $\{v_1, v_2, v_3\}$ también lo es.\\
			\\
			Digamos que $\{v_1, v_2, v_3, v_4\}$ es L.I. y que $\{v_1, v_2, v_3\}$ es L.D.\\
			Entonces, $\exists$ una combinación lineal no trivial tal que 
			$$\alpha_1 v_1 + \alpha_2 v_2 + \alpha_4 v_4 = 0$$
			Definamos ahora $alpha_3 = 0$ y agreguemos a la ecuación anterior $alpha_3 v_3$ (esto será $\vec{0}$ asi que no afecta).\\
			$$\alpha_1 v_1 + \alpha_2 v_2 + \alpha_3 v_3 + \alpha_4 v_4 = 0$$
			Luego, esta también será una combinación lineal no trivial de $\{v_1, v_2, v_3, v_4\}$.\\
			Pero $\{v_1, v_2, v_3, v_4\}$ es L.I., por lo que esto contradice nuestro supuesto.
			
			
			Finalmente, $\{v_1, v_2, v_4\}$ es L.I., por lo que la afirmación es {\bf Verdadera}.
\item Si $A$ es una matriz de $3\times 5$ y $T$ es la transformación lineal definida por $T(x) = Ax$, entonces $T$ tiene dominio $\R^3$.\\
			\\
			$x$ debe tener tantas filas como columnas tiene $A$, es decir 5, por lo que el dominio de $T$ es $\R^5$.
			
			Entonces, la afirmación es {\bf Falsa}.
\item Un sistema con más ecuaciones que incógnitas es siempre consistente.\\
			\\
			Intentemos buscar un caso donde no se cumpla esto.
			$$\begin{array}{rcl}
			x + y & = & 1 \\
			y & = & 1 \\
			y & = & 2
			\end{array} \Longrightarrow
			\begin{bmatrix}
			1 & 1 & 1\\
			0 & 1 & 1\\
			0 & 1 & 2
			\end{bmatrix} \wsim 
			\begin{bmatrix}
			1 & 1 & 1\\
			0 & 1 & 1\\
			0 & 0 & 1
			\end{bmatrix}$$
			Este sistema es claramente inconsistente.
			
			Entonces, la afirmación es {\bf Falsa}.
\end{enumerate}
\end{solucion}
\item Encuentre una factorziación $A = LU$ de
	$$A = \begin{bmatrix}
	2 & 4 & -1 & 5 & -2 \\
	-4 & -5 & 3 & -8 & 1 \\
	2 & -5 & -4 & 1 & 8 \\
	-6 & 0 & 7 & -3 & 1
	\end{bmatrix}$$
\begin{solucion}
Procedemos a llevar la matriz a su forma escalonada, deteniendonos luego de cada generación de un pivote y teniendo cuidado de no ponderar filas
		$$\begin{bmatrix}
		2 & 4 & -1 & 5 & -2 \\
		-4 & -5 & 3 & -8 & 1 \\
		2 & -5 & -4 & 1 & 8 \\
		-6 & 0 & 7 & -3 & 1
		\end{bmatrix}$$
		Antes de hacer nada, podemos ver nuestro primer pivote, por lo que todos los coeficientes abajo que hay desde el pivote (incluido) hacia abajo, serán la primera columna que utilizaremos luego para construir $L$, es decir $\begin{bmatrix} 2\\-4\\2\\-6\end{bmatrix}$
		
		Procedemos entonces con el pivoteo,
		$$\sim \begin{bmatrix}
		2 & 4 & -1 & 5 & -2 \\
		0 & 3 & 1 & 2 & -3 \\
		0 & -9 & -3 & -4 & 10 \\
		0 & 12 & 4 & 12 & -5
		\end{bmatrix}$$
		Vemos aquí que el nuevo pivote generado es $3$, por lo que la segunda columna que usaremos para construir $L$ será
		$\begin{pmatrix}
		0\\3\\-9\\12
		\end{pmatrix}$. El primer coeficiente es 0, ya que consideramos solo los numeros abajo del pivote (incluyendo al pivote). Seguimos pivoteando,
		$$\sim \begin{bmatrix}
		2 & 4 & -1 & 5 & -2 \\
		0 & 3 & 1 & 2 & -3 \\
		0 & 0 & 0 & 2 & 1 \\
		0 & 0 & 0 & 4 & 7
		\end{bmatrix}$$
		Luego, la tercera columna que usaremos será
		$\begin{pmatrix}
		0\\0\\2\\4
		\end{pmatrix}$. Continuamos,
		$$\sim \begin{bmatrix}
		2 & 4 & -1 & 5 & -2 \\
		0 & 3 & 1 & 2 & -3 \\
		0 & 0 & 0 & 2 & 1 \\
		0 & 0 & 0 & 0 & 5
		\end{bmatrix} = U$$
		Por lo que la última columna que necesitamos es $\begin{pmatrix}0\\0\\0\\5
		\end{pmatrix}$. Además, esta última matriz (la forma escalonada de $A$), corresponde a $U$.
		
		Procedamos a construir $L$. Lo que haremos será dividir cada vector seleccionado anteriormente por el coeficiente que está más arriba (pivote) y usar cada vector resultante como una columna de $L$. De esta manera,
		$$ L = \begin{bmatrix}
		1 & 0 & 0 & 0 \\
		-2 & 1 & 0 & 0 \\
		1 & -3 & 1 & 0 \\
		-3 & 4 & 2 & 1
		\end{bmatrix}$$
		Finalmente,
		$$A = \begin{bmatrix}
		1 & 0 & 0 & 0 \\
		-2 & 1 & 0 & 0 \\
		1 & -3 & 1 & 0 \\
		-3 & 4 & 2 & 1
		\end{bmatrix} \begin{bmatrix}
		2 & 4 & -1 & 5 & -2 \\
		0 & 3 & 1 & 2 & -3 \\
		0 & 0 & 0 & 2 & 1 \\
		0 & 0 & 0 & 0 & 5
		\end{bmatrix}$$
\end{solucion}
\item Sea $A = \begin{bmatrix}
	1 & 2 & 3\\
	2 & 8 & 4 \\
	3 & 4 & 11\end{bmatrix}$. Calcule la factorización $A=LDL^T$ de la matriz $A$ y en base a esto encuentre una matriz $R$ tal que $A = RR^T$.
\begin{solucion}
Procedemos a pivotear la matriz para llevarla a su forma escalonada, preocupandonos de no ponderar filas y deteniendonos cada vez que se forme un pivote.
		$$A = \begin{bmatrix}
		1 & 2 & 3\\
		2 & 8 & 4 \\
		3 & 4 & 11\end{bmatrix}$$
		Aquí ya tenemos un pivote, que es la primera columna, es decir $\begin{bmatrix} 1 \\ 2 \\ 3 \end{bmatrix}$.
		$$\sim \begin{bmatrix}
		1 & 2 & 3\\
		0 & 4 & -2 \\
		0 & -2 & 2\end{bmatrix}$$
		El segundo pivote está en la segunda columna. Tomamos solo los valores bajo el pivote (incluyendolo). Esto es $\begin{bmatrix}0 \\ 4 \\ -2\end{bmatrix}$.
		$$\sim \begin{bmatrix}
		1 & 2 & 3\\
		0 & 4 & -2 \\
		0 & 0 & 1\end{bmatrix} = U$$
		El último pivote está en la tercera columna, que corresponderá a $\begin{bmatrix}0 \\ 0 \\ 1\end{bmatrix}$.
		
		Ahora, procedemos a dividir cada columna extraida por el pivote correspondiente (coeficiente de más arriba) y luego formar la matriz $L$, esto es
		$$L = \begin{bmatrix}
		1 & 0 & 0\\
		2 & 1 & 0\\
		3 & -\frac{1}{2} & 1
		\end{bmatrix}$$
		La matriz $D$ corresponde a la diagonal de $U$, por lo que
		$$D =\begin{bmatrix}
		1 & 0 & 0\\
		0 & 4 & 0 \\
		0 & 0 & 1\end{bmatrix}$$
		Luego,
		$$A = LDL^T = \begin{bmatrix}
		1 & 0 & 0\\
		2 & 1 & 0\\
		3 & -\frac{1}{2} & 1
		\end{bmatrix}
		\begin{bmatrix}
		1 & 0 & 0\\
		0 & 4 & 0 \\
		0 & 0 & 1\end{bmatrix}
		\begin{bmatrix}
		1 & 2 & 3\\
		0 & 1 & -\frac{1}{2}\\
		0 & 0 & 1
		\end{bmatrix}$$
		Notemos ahora que, al ser $D$ una matriz diagonal, es facil calcular $\sqrt[]{D}$. Esto es
		$$\sqrt[]{D} = 
		\begin{bmatrix}
		1 & 0 & 0\\
		0 & 2 & 0 \\
		0 & 0 & 1
		\end{bmatrix}$$
		Ahora,
		$$A = LDL^T = (L\ \sqrt[]{D})(\ \sqrt[]{D}L^T) = (L\ \sqrt[]{D})(L\ \sqrt[]{D}^T)^T $$
		Pero la matriz transpuesta de una matriz diagonal es igual a la matriz en cuestión, por lo que
		$$A = (L\ \sqrt[]{D})(L\ \sqrt[]{D})^T \ra R = L\ \sqrt[]{D}$$
		$$R = \begin{bmatrix}
		1 & 0 & 0\\
		2 & 1 & 0\\
		3 & -\frac{1}{2} & 1
		\end{bmatrix}
		\begin{bmatrix}
		1 & 0 & 0\\
		0 & 2 & 0 \\
		0 & 0 & 1
		\end{bmatrix} = \begin{bmatrix}
		1 & 0 & 0\\
		2 & 2 & 0\\
		3 & -1 & 1
		\end{bmatrix}$$
		Finalmente,
		$$A = \begin{bmatrix}
		1 & 0 & 0\\
		2 & 2 & 0\\
		3 & -1& 1
		\end{bmatrix}
		\begin{bmatrix}
		1 & 2 & 3\\
		0 & 2 & -1\\
		0 & 0 & 1
		\end{bmatrix}$$
\end{solucion}
\item Sea una descomposción $PA = LU$ donde
	$$P = \begin{bmatrix}
	0 & 0 & 1\\
	0 & 1 & 0\\
	1 & 0 & 0
	\end{bmatrix}, \quad
	L = \begin{bmatrix}
	1 & 0 & 0\\
	-1 & 1 & 0\\
	0 & 1 & 1
	\end{bmatrix}, \quad
	U = \begin{bmatrix}
	2 & 1 & 0\\
	0 & 1 & 1\\
	0 & 0 & -1
	\end{bmatrix}$$
	determine usando directamente esta factorización la tercera fila de $A^{-1}$.
\begin{solucion}
La tercera fila de $A^{-1}$ es la tercera columna transpuesta de $(A^T)^{-1}$, es decir, corresponde a la solución de
		$$A^Tx = e_3$$
		Tenemos que
		$$PA = LU$$
		Transponiendo a ambos lados,
		$$(PA)^T = (LU)^T$$
		$$A^TP^T = U^TL^T$$
		$$A^T = U^TL^TP$$
		Entonces,
		$$A^Tx = e_3 \ra U^TL^TPx = e_3$$
		Digamos que
		$$L^TPx = y \ra U^Ty = e_3 \ra \begin{bmatrix}
		2 & 0 & 0\\
		1 & 1 & 0\\
		0 & 1 & -1
		\end{bmatrix}y = \begin{pmatrix}
		0 \\ 0 \\ 1
		\end{pmatrix} \ra y = \begin{pmatrix}
		0 \\ 0 \\ -1
		\end{pmatrix}$$
		Digamos ahora que
		$$Px = z \ra L^Tz = y \ra \begin{bmatrix}
		1 & -1 & 0\\
		0 & 1 & 1\\
		0 & 0 & 1
		\end{bmatrix}z = \begin{pmatrix}
		0 \\ 0 \\ -1
		\end{pmatrix} \ra z = \begin{pmatrix}
		1 \\ 1 \\ -1
		\end{pmatrix}$$
		Por último
		$$Px = z \ra \begin{bmatrix}
		2 & 1 & 0\\
		0 & 1 & 1\\
		0 & 0 & -1
		\end{bmatrix}x = \begin{pmatrix}
		1 \\ 1 \\ -1
		\end{pmatrix} \ra x = \begin{pmatrix}
		-1 \\ 1 \\ 1
		\end{pmatrix}$$
		En conclusión, la tercera fila de $A^{-1}$ es $\begin{pmatrix}
		-1 \\ 1 \\ 1
		\end{pmatrix}$
\end{solucion}
\item Sea $A \in M_4 (\R)$ tal que $|A|=5$, encuentre
\begin{tasks}(4)
\task $|2^3A|$
\task $|A^5|$
\task $|-A|$
\task $||A|A^{-1}|$
\end{tasks}
\begin{solucion}

\begin{enumerate}[a)]
\item $|2^3A| = (2^3)^4|A| = 2^{12}\cdot 5$
\item $|A^5| = |AAAAA| = |A||A||A||A||A| = |A|^5 = 5^5$
\item $|-A| = (-1)^4|A| = |A| = 5$
\item $||A|A^{-1}| = |5A^{-1}| = 5^4|A^{-1}| = 5^4\dfrac{1}{|A|} = \dfrac{5^4}{5} = 5^3 = 125$
\end{enumerate}
\end{solucion}
\item Sea $A$ una matriz de $4\times 4$, tal que $det(A) = \alpha \neq 0$
\begin{enumerate}[a)]
\item Calcule  $det(5A)+det(3A^{-1})$ en terminos de $\alpha$.
\item Se sabe que $det(Adj(A))=8$. Calcule $\alpha$.
\end{enumerate}
\begin{solucion}

\begin{enumerate}[a)]
\item Calcule  $det(5A)+det(3A^{-1})$ en terminos de $\alpha$.
			
			En primer lugar,
			$$det(5A) = 5^4det(a) = 5^4\alpha = 625 \alpha$$
			En segundo lugar,
			$$det(3A^{-1}) = 3^4det(A^{-1}) = 3^4 \dfrac{1}{\alpha} = \dfrac{81}{\alpha}$$
			Finalmente,
			$$det(5A)+det(3A^{-1}) = 625 \alpha + \dfrac{81}{\alpha}$$
\item Se sabe que $det(Adj(A))=8$. Calcule $\alpha$.
			
			Recordemos que
			$$A\ Adj(A) = det(A) I$$
			Apliquemos $det$ a ambos lados de la igualdad,
			$$det(A\ Adj(A)) = det(det(A) I)$$
			$$det(A)\ det(Adj(A)) = det(A)^4\ det(I)$$
			$$det(Adj(A)) = det(A)^3$$
			$$det(A) = \sqrt[3]{det(Adj(A))}$$
			$$det(A) = \sqrt[3]{8}$$
			$$det(A) = 2$$
\end{enumerate}
\end{solucion}
\item Sea $P$ el paralelepípedo con un vertice en el origen y vertices adyacentes en $(1,4,0)$, $(-2,-5,2)$, $(-1,2,-1)$
\begin{enumerate}[a)]
\item Determinar el volumen de $P$.
\item Se define $T: \R^3 \ra \R^3$ como la transformación lineal definida por $$T(x,y,z)=(x+y,y+z,x-z)$$
		Calcule el volumen de $T(P)$.
\end{enumerate}
\begin{solucion}

\begin{enumerate}[a)]
\item Determinar el volumen de $P$.\\
			\\
			Los vectores de las aristas son 
			$$v_1 = \begin{pmatrix}
			1 \\ 4 \\ 0
			\end{pmatrix}, \quad
			v_2 = \begin{pmatrix}
			-2 \\ -5 \\ 2
			\end{pmatrix}, \quad
			v_3 = \begin{pmatrix}
			-1 \\ 2 \\ -1
			\end{pmatrix}$$
			Podemos representar a $P$ en forma matricial de la siguiente forma:
			$$A_P = \begin{bmatrix}
			1 & -2 & -1 \\
			4 & -5 & 2 \\
			0 & 2 & -1
			\end{bmatrix}$$
			Luego, el volumen de $P$ se calcula como $|det(A_P)|$, esto es
			$$\left| det \left(\begin{bmatrix}
			1 & -2 & -1 \\
			4 & -5 & 2 \\
			0 & 2 & -1
			\end{bmatrix}\right)\right| =| 0 -2(2+4) - 1(-5+8)| = |-12 - 3| =| -15| = 15$$
			Por lo que el volumen de $P$ es $15$.
\item Se define $T: \R^3 \ra \R^3$ como la transformación lineal definida por $$T(x,y,z)=(x+y,y+z,x-z)$$
			Calcule el volumen de $T(P)$.\\
			\\
			En primer lugar, encontremos la matriz de la transformación lineal
			$$T(x,y,z)=(x+y,y+z,x-z) = \begin{pmatrix}
			x+y \\ 
			y+z \\ 
			x-z
			\end{pmatrix} = \begin{bmatrix}
			1 & 1 & 0 \\
			0 & 1 & 1 \\
			1 & 0 & -1 
			\end{bmatrix} \begin{pmatrix}
			x \\ y \\ z
			\end{pmatrix}$$
			Por lo que
			$$A_T = \begin{bmatrix}
			1 & 1 & 0 \\
			0 & 1 & 1 \\
			1 & 0 & -1 
			\end{bmatrix}$$
			Luego, el volumen de $T(P)$ será $|det(A_P)| \cdot |det(A_T)|$
			$$|det(A_T)| = \left| det \left(\begin{bmatrix}
			1 & 1 & 0 \\
			0 & 1 & 1 \\
			1 & 0 & -1 
			\end{bmatrix}\right)\right| = |1(-1-0) - (0 - 1)| = |1-1| = 0$$
			Finalmente,
			$$V(T(P)) = |det(A_P)| \cdot |det(A_T)| = 15 \cdot 0 = 0$$
\end{enumerate}
\end{solucion}
\item Sea $A$ un matriz tal que
$$A^3 = 
\begin{bmatrix}
2 & 0 & 2\\
-1 & 1 & 2\\
2 & -1 & -1
\end{bmatrix}$$
\begin{enumerate}[a)]
\item Calcule el determinante de la matriz $A^3$ mediante desarrollo por cofactores.
\item Demuestre que $A$ no es invertible.
\end{enumerate}
\begin{solucion}

\begin{enumerate}[a)]
\item Calcule el determinante de la matriz $A^3$ mediante desarrollo por cofactores.\\
\\
$$|A^3| = \left|\begin{bmatrix}
2 & 0 & 2\\
-1 & 1 & 2\\
2 & -1 & -1
\end{bmatrix}\right| = 2\left|\begin{bmatrix}
1 & 2\\
-1 & -1
\end{bmatrix}\right| + 2\left|\begin{bmatrix}
1 & 2\\
-1 & -1
\end{bmatrix}\right| = 2 - 2 = 0$$
\item Demuestre que $A$ no es invertible.\\
\\
$$|A^3| = 0 \ra |A|^3 = 0 \ra |A| = 0$$
Por lo que la matriz no es invertible.
\end{enumerate}
\end{solucion}
\item Sea $A$ una matriz de $n \times n$ definida por
	$$a_{i,j} = 
	\begin{cases}
	3 \quad si\ i\leq j \\
	1 \quad si\ i > j
	\end{cases}$$
	Calcule $det(A)$.
\begin{solucion}
En primer lugar, veamos como es $A$,
		$$A = \begin{bmatrix}
		3 & 3 & 3 & \dots & 3 & 3 \\ 
		1 & 3 & 3 & \dots & 3 & 3 \\ 
		1 & 1 & 3 & \dots & 3 & 3 \\ 
		\vdots & \vdots & \vdots & \ddots & \vdots & \vdots \\ 
		1 & 1 & 1 & \dots & 3 & 3 \\ 
		1 & 1 & 1 & \dots & 1 & 3
		\end{bmatrix}$$
		Ahora, busquemos su determinante,
		$$det(A) = \left|\begin{bmatrix}
		3 & 3 & 3 & \dots & 3 & 3 \\ 
		1 & 3 & 3 & \dots & 3 & 3 \\ 
		1 & 1 & 3 & \dots & 3 & 3 \\ 
		\vdots & \vdots & \vdots & \ddots & \vdots & \vdots \\ 
		1 & 1 & 1 & \dots & 3 & 3 \\ 
		1 & 1 & 1 & \dots & 1 & 3
		\end{bmatrix}\right|$$
		En primer lugar, ponderemos la primera fila por $\dfrac{1}{3}$. Al hacer esto, debemos multiplicar por 3 afuera para mantener el valor del determinante igual.
		$$det(A) = 3 \left|\begin{bmatrix}
		1 & 1 & 1 & \dots & 1 & 1 \\ 
		1 & 3 & 3 & \dots & 3 & 3 \\ 
		1 & 1 & 3 & \dots & 3 & 3 \\ 
		\vdots & \vdots & \vdots & \ddots & \vdots & \vdots \\ 
		1 & 1 & 1 & \dots & 3 & 3 \\ 
		1 & 1 & 1 & \dots & 1 & 3
		\end{bmatrix}\right|$$
		Ahora, a todas las filas (menos a la primera) le restaremos la primera fila
		$$det(A) = 3 \left|\begin{bmatrix}
		1 & 1 & 1 & \dots & 1 & 1 \\ 
		0 & 2 & 2 & \dots & 2 & 2 \\ 
		0 & 0 & 2 & \dots & 2 & 2 \\ 
		\vdots & \vdots & \vdots & \ddots & \vdots & \vdots \\ 
		0 & 0 & 0 & \dots & 2 & 2 \\ 
		0 & 0 & 0 & \dots & 0 & 2
		\end{bmatrix}\right|$$
		Notemos ahora que la diagonal está compuesta de un 1 y luego puros 2, por lo que
		$$det(A) = 3 \cdot 2^{n-1}$$
\end{solucion}
\item Calcular el determinante de siguiente matriz de $n\times n$
	$$ \begin{bmatrix}
	-3 & x & x & \dots & x \\
	x & -3 & x & \dots & x \\
	x & x & -3 & \dots & x \\
	\vdots & \vdots & \vdots & \ddots & \vdots \\
	x & x & x & \dots & -3
	\end{bmatrix}$$
\begin{solucion}
Calulemos el determinante de la matriz, esto es
		$$ \left|\begin{bmatrix}
		-3 & x & x & \dots & x \\
		x & -3 & x & \dots & x \\
		x & x & -3 & \dots & x \\
		\vdots & \vdots & \vdots & \ddots & \vdots \\
		x & x & x & \dots & -3
		\end{bmatrix}\right|$$
		Lo primero que haremos, será restarle la fila $n$ a todas las filas menos a la fila 1 y a la fila $n$, es decir realizaremos las operaciones $f_2-f_n$, $f_3-f_n$, \dots, $f_{n-1} - f_n$
		$$ \left|\begin{bmatrix}
		-3 & x & x & \dots & x \\
		0 & -3-x & 0 & \dots & x+3 \\
		0 & 0 & -3 & \dots & x+3 \\
		\vdots & \vdots & \vdots & \ddots & \vdots \\
		x & x & x & \dots & -3
		\end{bmatrix}\right|$$
		En segundo lugar, haremos $f_n-f_1$
		$$ \left|\begin{bmatrix}
		-3 & x & x & \dots & x \\
		0 & -3-x & 0 & \dots & x+3 \\
		0 & 0 & -3 & \dots & x+3 \\
		\vdots & \vdots & \vdots & \ddots & \vdots \\
		x+3 & 0 & 0 & \dots & -3-x
		\end{bmatrix}\right|$$
		Recordemos que el determinante de una matriz es el mismo que de su transpuesta, por lo que podemos también realizar operaciones de columnas al igual como lo hacemos con las filas. Dicho esto, haremos $c_1 + c_n$,
		$$ \left|\begin{bmatrix}
		-3+x & x & x & \dots & x \\
		x+3 & -3-x & 0 & \dots & x+3 \\
		x+3 & 0 & -3 & \dots & x+3 \\
		\vdots & \vdots & \vdots & \ddots & \vdots \\
		0 & 0 & 0 & \dots & -3-x
		\end{bmatrix}\right|$$
		Por último, realizaremos las operaciones $c_1 + c_2$, $c_1 +c_3$, \dots, $c_1 + c_{n-1}$
		$$ \left|\begin{bmatrix}
		-3+x(n-1) & x & x & \dots & x \\
		0 & -3-x & 0 & \dots & x+3 \\
		0 & 0 & -3 & \dots & x+3 \\
		\vdots & \vdots & \vdots & \ddots & \vdots \\
		0 & 0 & 0 & \dots & -3-x
		\end{bmatrix}\right|$$
		Aqui ya podemos multiplicar la diagonal de esta matriz, con lo que el determinante nos queda
		$$det = (-3 + x(n-1)) \cdot (-3-x)^{n-1} = (-3-x)^n + xn(-3-x)^{n-1}$$
\end{solucion}
\item Sean $v_1,v_2,v_3,v_4 \in \R^4$, encuentre el $|A|$ si\\
$A=[v_1-v_3+v_4 \;\;\;\;-v_2-v_3 \;\;\;\; v_3-v_1
\;\;\;\;v_1+v_2+2v_4 ]\in M_4 (\R)$
\begin{solucion}
En primer lugar, notemos que no sabemos nada de los vectores dados, es decir, estos podrían perfectamente ser $LD$. Por ende, existen dos posibles respuestas a esta pregunta: el determinante es 0 o no hay suficiente información para saber.\\

Para que el determinante sea 0, las columnas de $A$ deben ser $LD$, es decir, debe existir $\alpha_1, \alpha_2, \alpha_3, \alpha_4 \in \R$ no todos iguales a 0, tal que
$$\alpha_1(v_1-v_3+v_4) + \alpha_2(-v_2-v_3) +\alpha_3(v_3-v_1) + \alpha_4(v_1+v_2+2v_4) = 0$$
Una forma sencilla de demostrar que existe alguna combinación de valores no triviales que cumpla con esto, es buscar una en particular. Para esto diremos de manera arbitraria que $\alpha_4 = -1$ y buscaremos los otros. Con esto, obtenemos
$$\alpha_1(v_1-v_3+v_4) + \alpha_2(-v_2-v_3) +\alpha_3(v_3-v_1) =v_1+v_2+2v_4$$
Notemos que esto equivale a encontrar una combinación lineal de las 3 primeras columnas para formar la cuarta. Reordenando,
$$(\alpha_1 - \alpha_3)v_1 -\alpha_2 v_2 + (-\alpha_1 - \alpha_2 + \alpha_3)v_3 + \alpha_1 v_4 =v_1+v_2+2v_4$$
De aquí, obtenemos el sistema
$$\begin{array}{rcl}
\alpha_1 - \alpha_3 & = & 1\\
-\alpha_2 & = & 1\\
-\alpha_1 - \alpha_2 + \alpha_3 & = & 0\\
\alpha_1 & = & 2
\end{array} \ra
\begin{array}{rcl}
\alpha_1 & = & 2\\
\alpha_2 & = & -1\\
\alpha_3 & = & 1\\
\alpha_4 & = & -1
\end{array}
$$
Que corresponde a una solución no trivial, por lo que las columnas de $A$ son $LD$.\\

En conclusión, el determinante de $A$ es 0.
\end{solucion}
\item Determine la segunda columna de la inversa de $A = \left[ \begin{array}{rrr} 2&1&0\\ 0&1&2\\ 1&2&0 \end{array} \right]$ utilizando el metodo de cramer.
\begin{solucion}
El metodo de cramer dice que para resolver un sistema $Ax = b$, se puede hacer lo siguiente:\\

Sea $A_1$ la matriz $A$ con la columna $i$ reemplazada por el vector $b$, se cumple que
$$x_i = \dfrac{|A_i|}{|A|}$$
Como lo que queremos obtener es la segunda columna de la inversa de $A$, el sistema que debemos resolver es
$$Ax = e_2$$
Para aplicar cramer, en primer lugar calculemos el determinante de $A$, esto es
$$|A| = \left|  \left[ \begin{array}{rrr} 2&1&0\\ 0&1&2\\ 1&2&0 \end{array} \right] \right| = 2\cdot (-4) - (-2) = -6$$
Luego,
$$x_1 = 
\dfrac{\left|  \left[ \begin{array}{rrr} 0&1&0\\ 1&1&2\\ 0&2&0 \end{array} \right] \right|}{|A|} = 
\dfrac{0}{-6} = 0$$
$$x_2 = 
\dfrac{\left|  \left[ \begin{array}{rrr} 2&0&0\\ 0&1&2\\ 1&0&0 \end{array} \right] \right|}{|A|} = 
\dfrac{0}{-6} = 0$$
$$x_3 = 
\dfrac{\left|  \left[ \begin{array}{rrr} 2&1&0\\ 0&1&1\\ 1&2&0 \end{array} \right] \right|}{|A|} = 
\dfrac{2\cdot 1 - (-1)}{-6} = \dfrac{1}{2}$$
Finalmente,
$$x = \begin{pmatrix}
0 \\ 0 \\ \frac{1}{2}
\end{pmatrix}$$
Que corresponde a la segunda columna de la inversa de $A$.
\end{solucion}
\item Sea $V = \mathbb{P}_2$ el espacio vectorial de los polinomios con coeficientes reales de grado menor o igual a 2. Demuestre que el conjunto
	$$U=\{p(x) \in V : p(1) + p(0) = p(-1)\}$$
	Demuestre que $U$ es subespacio de $V$.
\begin{solucion}
Para demostrar que $U$ es un subespacio vectorial, debemos ver que sea no vacío, cerrado en la suma y cerrado en la multiplicación.
		
		No vacío:
		
		Sea el polinomio $p(x) = 0$, vemos que
		$$p(1) + p(0) =p(-1)$$
		$$0 = 0$$
		Por lo que $p(x) \in U$, asi que no es vacío.
		
		Suma:
		
		Sean $p, q \in U$,
		
		P.D.
		$$ \begin{array}{rcl}
		(p+q)(1) + (p+q)(0) &= &(p+q)(-1)\\
		p(1) + q(1)+ p(0)+q(0) &= &(p+q)(-1)\\
		p(1) + p(0)+ q(1) + q(0) &= &(p+q)(-1)\\
		p(-1)+ q(-1) &= &(p+q)(-1)\\
		(p+q)(-1) &= &(p+q)(-1)
		\end{array}$$
		Por lo que es cerrado en la suma
		
		Multiplicación:
		
		Sea $p \in U$ y $\alpha \in \R$,
		P.D.
		$$ \begin{array}{rcl}
		(\alpha p)(1) + (\alpha p)(0) = (\alpha p)(-1) \\
		\alpha p(1) + \alpha p(0) = (\alpha p)(-1) \\
		\alpha ( p(1) +  p(0)) = (\alpha p)(-1) \\
		\alpha p(-1) = (\alpha p)(-1)\\
		(\alpha  p)(-1) = (\alpha p)(-1)
		\end{array}$$
		 
		Por lo que es cerrado en la multiplicación
		
		Entonces, tenemos que $U$ es un subespacio de $V$.
		$$\blacksquare$$
\end{solucion}
\item Sea $V = \mathbb{P}_3$ y sea
$$W = \{p(x) \in V: p(1) = p(0) =0\}$$
Demostrar que $W$ es subespacio de $V$.
\begin{solucion}
Para demostrar que $U$ es un subespacio vectorial, debemos ver que contenga al 0, sea cerrado en la suma y cerrado en la multiplicación.

$0 \in W$:

Sea el polinomio $p(x) = 0$, vemos que
$$p(1) = p(0) = 0$$
$$0 = 0 = 0$$
Por lo que $p(x) \in U$, asi que no es vacío.\\

Suma:

Sean $p, q \in W$,

P.D. $p + q \in W$.\\

Notemos que $p$ y $q$ serán de la forma
$$p(x) = a + bx + cx^2 + dx^3 \qquad y \qquad q(x) = e + fx + gx^2 + hx^3$$
Con
$$p(0) = a = 0, \qquad q(0) = e = 0$$
$$p(1) = a + b + c + d = 0,\qquad q(1) = e + f + g + h = 0$$
Luego,
$$(p+q)(x) = (a + bx + cx^2 + dx^3) + (e + fx + gx^2 + hx^3)$$
Además,
$$(p+q)(0) = a + e = 0 + 0 = 0$$
$$(p+q)(1) = (a + b + c + d) + (e + f + g + h) = 0 + 0 = 0$$
Por lo que $p + q \in W$ y por ende $W$ es cerrado en la suma.\\

Multiplicación:

Sea $p \in W$ y $\alpha \in \R$,
P.D. $\alpha p \in W$.\\

Notemos que $p$ y $q$ serán de la forma
$$p(x) = a + bx + cx^2 + dx^3 \qquad y \qquad q(x) = e + fx + gx^2 + hx^3$$
Con
$$p(0) = a = 0, \qquad p(1) = a + b + c + d = 0$$
Luego,
$$(\alpha p)(x) = \alpha(a + bx + cx^2 + dx^3)$$
Además,
$$(\alpha p)(0) = \alpha a = \alpha \cdot 0 = 0$$
$$(\alpha p)(1) = \alpha (a + b + c + d) = \alpha \cdot 0 = 0$$

Por lo que $\alpha p \in W$ y $W$ es cerrado en la multiplicación\\

Entonces, tenemos que $W$ es un subespacio de $V$.
$$\blacksquare$$
\end{solucion}
\item La matriz $A = \begin{bmatrix}
	1 & 3 & 4 & -1 & 2\\
	2 & 6 & 6 & 0 & -3\\
	3 & 9 & 3 & 6 & -3\\
	3 & 9 & 0 & 9 & 0
	\end{bmatrix}$ es equivalente por filas a $B = \begin{bmatrix}
	1 & 3 & 4 & -1 & 2\\
	0 & 0 & 1 & -1 & 1\\
	0 & 0 & 0 & 0 & -8\\
	0 & 0 & 0 & 0 & 0
	\end{bmatrix}$. Sin calcular bases, determine las dimensiones de Nul $A$, Col $A$, Col $A^T$ y Nul $A^T$.
\begin{solucion}
$B$ tiene 2 filas $L.D.$, es decir, 2 variables libres, por lo que $A$ también. Por esto,
		$$dim(Nul\ A) = 2$$
		Como $B$ tiene 3 pivotes, entonces $A$ también. Luego,
		$$dim(Col\ A) = 3$$
		Además, $B$ tiene 3 filas no nulas (pivotes), por lo que $A$ también. Estas filas serán el espacio fila de $A$ ($Fil\ A$), cuya dimensión es equivalente al espacio columna de $A^T$, por lo que
		$$dim(Col\ A^T) = dim(Fil\ A) = 3$$
		Por lo anterior, como $A^T$ tendrá 3 columnas pivotes, tendrá una variable libre, lo que implica
		$$dim(Nul\ A^T) = 1$$
\end{solucion}
\item En cada caso, determine si la afirmación es verdadera o falsa y justifique su respuesta.
\begin{enumerate}[a)]
\item El subconjunto de $\R^2$
		$$E=\{(x_1, x_2) \in \R^2:9x_1^2+2x_2^2 \leq 4\}$$
		es un subespacio vectorial de $\R^2$.
\item Si $A$ es una matriz de $n \times n$ tal que $A^2=A$, entonces $Col(A) \cap Nul(A) = \{0\}$.
\item Si $n$ es impar y $A$ es una matriz de $n\times n$ que satisface $A^T = -A$ entonces $Nul(A) \neq \{0\}$.
\end{enumerate}
\begin{solucion}

\begin{enumerate}[a)]
\item El subconjunto de $\R^2$
			$$E=\{(x_1, x_2) \in \R^2:9x_1^2+2x_2^2 \leq 4\}$$
			es un subespacio vectorial de $\R^2$.\\
			\\
			Tomemos el par $(0,1)$
			$$9 \cdot 0^2 + 2 \cdot 1^2 \leq 4 \ra 2 \leq 4 \ra (0,1) \in E$$
			Veamos ahora con $2 \cdot (0,1) = (0,2)$,
			$$9 \cdot 0^2 + 2 \cdot 2^2 \leq 4 \ra 8 \leq 4 \ra (0,2) \not \in E$$
			Con lo que concluimos que $E$ no es cerrado en la multiplicación, por lo que no puede ser un subespacio vectorial. Entonces, la afirmación es {\bf FALSA}.
\item Si $A$ es una matriz de $n \times n$ tal que $A^2=A$, entonces $Col(A) \cap Nul(A) = \{0\}$.\\
			\\
			Digamos que $y \in Col(A) \cap Nul(A)$
			
			Como $y \in Col(A) \cap Nul(A) \ra \exists x \in \R^n (Ax = y)$\\
			\\
			Como $A^2 = A$, entonces
			$$Ay = A(Ax) = A^2x = Ax = y$$
			de donde concluimos que $y = Ay$\\
			\\
			Además, sabemos que $y \in Nul(A)$, por lo que
			$$y = Ay = 0 \ra y = 0$$
			Luego,
			$$Col(A) \cap Nul(A) = \{0\}$$
			Por lo que la afirmación es {\bf VERDADERA}.
\item Si $n$ es impar y $A$ es una matriz de $n\times n$ que satisface $A^T = -A$ entonces $Nul(A) \neq \{0\}$.\\
			\\
			$$det(-A) = (-1)^n det(A)$$
			Pero $n$ es impar, por lo que
			$$det(-A) = -det(A)$$
			Además, $det(A) = det(A^T)$. Luego,
			$$det(A) = det(A^T) = det(-A) = -det(A)$$
			$$det(A) = -det(A)$$
			$$det(A) = 0$$
			Luego, $A$ no es invertible, por lo que
			$$Nul(A) \neq \{0\}$$
			Dicho esto, la afirmación es {\bf VERDADERA}.
\end{enumerate}
\end{solucion}
\item Sea 
	$$U=\{p(x) \in \mathbb{P}_2 : p(1) + p(0) = p(-1)\}$$
	Determinar una base de $U$.
\begin{solucion}
			Sea $p(x) \in \mathbb{P}_2$ un polinomio de la forma $p(x) = a + bx + cx^2$\\
			\\
			Para que pertenezca a $U$, debe cumplirse que
			$$p(1) + p(0) = p(-1)$$
			$$a+b+c+a=a-b+c$$
			$$a = -2b$$
			Luego,
			$$\begin{array}{rcl}
			a & = & -2b\\
			b & = & b\\
			c & = & c
			\end{array}$$
			Entonces, podemos reescribir el polinomio como
			$$p(x) = -2b + bx + cx^2$$
			$$p(x) = b(x-2) + c(x^2)$$
			Finalmente,
			$$U = Gen\{x-2, x^2\}$$
\end{solucion}
\item Sea $A$ una matriz tal que $dim(Nul(A)) = 3$ y
	$$A^T \begin{pmatrix}
	1 \\ -2 \\ 1 \\ 1
	\end{pmatrix} = \begin{pmatrix}
	0 \\ 0 \\ 0 \\ 0 \\ 0
	\end{pmatrix}, \quad A^T \begin{pmatrix}
	2 \\ -1 \\ -1 \\ 1
	\end{pmatrix} = \begin{pmatrix}
	0 \\ 0 \\ 0 \\ 0 \\ 0
	\end{pmatrix}$$
\begin{enumerate}[a)]
\item Determine las dimensiones de $Col(A)$, de $Fila(A)$ y de $Nul(A^T)$
\item Determine, en los casos que sea posible, bases para los espacios $Nul(A^T)$ y $Fila(A)$.
\end{enumerate}
\begin{solucion}

\begin{enumerate}[a)]
\item Determine las dimensiones de $Col(A)$, de $Fila(A)$ y de $Nul(A^T)$\\
			\\
			Antes de comenzar, recordemos que para una matriz $A_{n\times m}$, se tiene que
			$$dim(Nul(A)) + dim(Col(A)) = m$$
			$$dim(Nul(A^T)) + dim(Col(A^T)) = n$$
			$$dim(Col(A)) = dim(Col(A^T))$$
			Para este ejercicio en particular, sabemos que $dim(Nul(A)) = 3$ y que la matriz $A$ tiene 4 filas y 5 columnas. Luego, usando la primera propiedad,
			$$dim(Nul(A)) + dim(Col(A)) = 5$$
			$$3 + dim(Col(A)) = 5$$
			$$dim(Col(A)) = 2$$
			Notemos también que la cantidad de pivotes será la misma en las filas que en las columnas, por lo que
			$$dim(Fila(A)) = dim(Col(A)) = 2$$
			Ahora, usando la segunda propiedad, tenemos que
			$$dim(Nul(A^T)) + dim(Col(A^T)) = 4$$
			Pero por la tercera propiedad sabemos que 
			$$dim(Col(A^T)) = dim(Col(A)) = 2$$
			Por lo que
			$$dim(Nul(A^T)) + 2 = 4$$
			$$dim(Nul(A^T)) = 2$$
			Finalmente,
			$$dim(Col(A))=2,\quad dim(Fila(A))=2, \quad dim(Nul(A^T))=2$$
\item Determine, en los casos que sea posible, bases para los espacios $Nul(A^T)$ y $Fila(A)$.\\
			\\
			Sabemos que $dim(Nul(A^T))=0$ y del enunciado tenemos dos vectores $L.I.$ que son pertenecen al $Nul(A^T)$, por lo que una base de $Nul(A^T)$ es
			$$\left\{ \begin{pmatrix}
			1 \\ -2 \\ 1 \\ 1
			\end{pmatrix},
			\begin{pmatrix}
			2 \\ -1 \\ -1 \\ 1
			\end{pmatrix}\right\}$$
			Por el otro lado, no podemos determinar una base para $Fila(A)$, ya que solo conocemos su dimensión pero no vectores que puedan generarlo.
\end{enumerate}
\end{solucion}
\item Sean $B = \begin{bmatrix} 1 & -1 \\ 0 & 0 \end{bmatrix}$ y $C = \{X \in M_2(\R)|BX=0\}$. Determine una base y la dimensión de $C$.
\begin{solucion}
Digamos que $X = \begin{bmatrix}
a & b \\c & d
\end{bmatrix}$.\\

Luego, debemos encontrar $\begin{bmatrix}
a & b \\c & d
\end{bmatrix}$ tal que
$$\begin{bmatrix} 1 & -1 \\ 0 & 0 \end{bmatrix}
\begin{bmatrix}
a & b \\c & d
\end{bmatrix}
= 0$$
$$a-c = 0 \qquad b - d = 0$$
$$a = c \qquad b = d$$
Por lo tanto,
$$X = \begin{bmatrix}
a & b \\
a & b
\end{bmatrix} =
a
\begin{bmatrix}
1 & 0 \\
1 & 0
\end{bmatrix} + 
b
\begin{bmatrix}
0 & 1 \\
0 & 1
\end{bmatrix} =
Gen\left\{
\begin{bmatrix}
1 & 0 \\
1 & 0
\end{bmatrix},
\begin{bmatrix}
0 & 1 \\
0 & 1
\end{bmatrix}
\right\}
$$
Finalmente, una base para $C$ es $\left\{
\begin{bmatrix}
1 & 0 \\
1 & 0
\end{bmatrix},
\begin{bmatrix}
0 & 1 \\
0 & 1
\end{bmatrix}
\right\}$ y su dimensión es 2.
\end{solucion}
\item Use vectores de coordenadas para probar la independencia lineal del conjunto de polinomios $\{1-2t^2-t^3,t+2t^3,1+t-2t^2\}$.
\begin{solucion}
La base que utilizaremos será
		$$B = \{1,t,t^2,t^3\}$$
		Luego, podemos reescribir cada elemento del conjunto de la siguiente forma
		$$\begin{array}{lcl}
		1-2t-t^3 & \ra & (1,\ \ 0,-2,-1)\\
		t+2t^3 & \ra & (0,\ \ 1,\ \ 0,\ \ 2)\\
		1+t-2t^2 & \ra & (1,\ \ 1,-2,\ \ 0)
		\end{array}$$
		Ahora, para probar que los elementos del conjunto son $L.I.$, basta con probar que los vectores de coordenadas lo son. Para esto, ponemos los vectores en una matriz y pivoteamos:
		$$\begin{bmatrix}
		1 & 0 & 1\\
		0 & 1 & 1\\
		-2 & 0 & -2\\
		-1 & 2 & 0
		\end{bmatrix} \sim
		\begin{bmatrix}
		1 & 0 & 1\\
		0 & 1 & 1\\
		0 & 0 & 1\\
		0 & 0 & 0
		\end{bmatrix}$$
		Con lo que concluimos que son $L.I.$
		$$\blacksquare$$
\end{solucion}
\item Sean $p_1(t) = 1-t^2, p_2(t) = 1+t, p_3(t) = 1+t+t^2$. Se sabe que $\{p_1(t), p_2(t), p_3(t)\}$ es una base de $\mathbb{P}_2$.
\begin{enumerate}[a)]
\item Exprese los polinomios $f(t) = 3-5t + 2t^2$ y $g(t) = 1-3t$ como combinaciones lineales de $\{p_1(t), p_2(t), p_3(t)\}$.
\item Use los vectores de coordenadas encontrados en la parte anterior para determinar si el conjunto $\{p_1(t), f(t), g(t)\}$ es L.I. o L.D.
\end{enumerate}
\begin{solucion}

\begin{enumerate}[a)]
\item Exprese los polinomios $f(t) = 3-5t + 2t^2$ y $g(t) = 1-3t$ como combinaciones lineales de $\{p_1(t), p_2(t), p_3(t)\}$.\\
			\\
			Sea una base 
			$$B = \{p_1(t), p_2(t), p_3(t)\} = $$, lo que estamos buscando son las coordenadas de $f$ y $g$ en la base $B$.\\
			\\
			En primer lugar, veamos $f(t)$. Debemos buscar $a$, $b$, $c$, tal que
			$$f(t) = 3-5t + 2t^2 = ap_1(t) + bp_2(t) + cp_3(t)$$
			Reemplazando y reagrupando,
			$$3-5t + 2t^2 = a(1-t^2) + b(1+t) + c(1+t+t^2)$$
			$$3-5t + 2t^2 = (a+b+c) + (b+c)t + (c-a)t^2$$
			Por lo que
			$$\begin{array}{rl}
			a + b +c & = 3\\
			b + c & =-5\\
			-a+c & =2
			\end{array} \Longrightarrow
			\begin{array}{rl}
			a & = 8\\
			b & =-15\\
			c & = 10
			\end{array}$$
			Luego, las coordenadas de $f(t)$ en $B$ son $(8,-15,10)$.\\
			\\
			Veamos ahora que ocurre con $g(t)$. Nuevamente, buscamos $a$, $b$, $c$, tal que
			$$g(t) = 1-3t =  ap_1(t) + bp_2(t) + cp_3(t)$$
			Esto es,
			$$1-3t = a(1-t^2) + b(1+t) + c(1+t+t^2)$$
			$$1-3t = (a+b+c) + (b+c)t + (c-a)t^2$$
			Lo que se nos lleva al sistema
			$$\begin{array}{rl}
			a + b +c & = 1\\
			b + c & =-3\\
			-a+c & =0
			\end{array} \Longrightarrow
			\begin{array}{rl}
			a & = 4\\
			b & =-7\\
			c & = 4
			\end{array}$$
			Con lo que las coordenadas de $g(t)$ en $B$ son $(4,-7,4)$.
\item Use los vectores de coordenadas encontrados en la parte anterior para determinar si el conjunto $\{p_1(t), f(t), g(t)\}$ es L.I. o L.D.\\
			\\
			Para determinar esto, podemos usar las coordenadas de estos elementos en cualquier base. Por conveniencia, usaremos la base $B$. Aquí, los vectores de coordenadas son
			$$p_1(t) = (1,0,0)$$
			$$f(t) = (8,-15,10)$$
			$$g(t) = (4,-7,4)$$
			Armamos una matriz con ellos y pivoteamos,
			$$\begin{bmatrix}
			1 & 0 & 0 \\
			8 & -15 & 10 \\
			4 & -7 & 4
			\end{bmatrix}
			\sim
			\begin{bmatrix}
			1 & 0 & 0 \\
			0 & 1 & 0 \\
			0 & 0 & 1
			\end{bmatrix}$$
			Por lo que sus el conjunto $\{p_1(t), f(t), g(t)\}$ es $L.I.$
\end{enumerate}
\end{solucion}
\item Sea
$$T\left(\left[ \begin{array}{c}
a\\ b\\ c \end{array} \right] \right) = (a+b+c) + (a-b+2c)x +
(3b-c)x^2
$$
Determine una base para $Nul(T) $ y para $Im (T) $ y las dimensiones de estos subespacios.
\begin{solucion}
Notemos que lo que estamos generando son polinomios de grado menor o igual a 2. La base canónica para estos polinomios es $B=\{1, x, x^2\}$
En primer lugar, obtengamos la matriz asociada a la transformación lineal respecto a las bases canónicas, esto es,
$$A = \begin{bmatrix}
1 & 1 & 1 \\
1 & -1 & 2 \\
0 & 3 & -1
\end{bmatrix}$$
Recordemos que $Nul(T)$ corresponde a la solución del sistema homogeneo, es decir $Ax = 0$, esto es,
$$\begin{bmatrix}
1 & 1 & 1 \\
1 & -1 & 2 \\
0 & 3 & -1
\end{bmatrix} \sim 
\begin{bmatrix}
1 & 0 & 0 \\
0 & 1 & 0 \\
0 & 0 & 1
\end{bmatrix}$$
Por lo que la solución del sistema homogeneo es $Gen\left\{\begin{pmatrix}0 \\ 0 \\ 0 \end{pmatrix}\right\}$, que corresponde al Nul(T).\\

Veamos ahora el $Im (T)$, que corresponde a las bases $LI$. Esto es equivalente a ver las columnas $LI$ de $A$, que ya sabemos que son todas. Por lo tanto, 
$$Im(A) = Gen\left\{\begin{pmatrix}1 \\ 0 \\ 0 \end{pmatrix},\begin{pmatrix}0 \\ 1 \\ 0 \end{pmatrix}, \begin{pmatrix}1 \\ 0 \\ 0 \end{pmatrix}\right\}$$
Luego, traduciendo estos vectores coordenadas, concluimos que
$$Im(T) = Gen\{1, x, x^2\}$$
Finalmente, las dimensiones de $Nul(T)$ y $Im(T)$ son 0 y 3, respectivamente.
\end{solucion}
\item Sea $T: \mathbb{P}_2 \ra \mathbb{P}_2$ una transformación lineal cuya matriz de transformación respecto a la base $B = \{1, t, t^2\}$ está dada por
$$[T]_B = \begin{bmatrix}
1 & -1 & 1\\
0 & 2 & -1\\
-1 & 3 & -2
\end{bmatrix}$$
Determine $T(t^2+1)$ y el espacio Nulo de $T$.
\begin{solucion}
Notemos que esta matriz lo que hace es modificar las coordenadas de un elemento $p$ y convertirlas en las coordenadas de $T(p)$, por lo que para determinar $T(t^2+1)$ primero debemos determinar sus coordenadas, esto es
$$[t^2+1]_B = \begin{pmatrix}
1 \\ 0 \\1
\end{pmatrix}$$
Luego, para obtener $T(t^2+1)$, calculamos
$$[T]_B[t^2+1]_B = \begin{bmatrix}
1 & -1 & 1\\
0 & 2 & -1\\
-1 & 3 & -2
\end{bmatrix}
\begin{pmatrix}
1 \\ 0 \\1
\end{pmatrix}
=
\begin{pmatrix}
2 \\ -1 \\ -3
\end{pmatrix}$$
Por lo tanto,
$$T(t^2+1) = 2-t-3t^2$$
Para determinar el espacio Nulo de $T$, debemos calcular el espacio Nulo de $[T]_B$ y a partir de eso obtener el espacio Nulo de $T$.
$$\begin{bmatrix}
1 & -1 & 1\\
0 & 2 & -1\\
-1 & 3 & -2
\end{bmatrix} \wsim
\begin{bmatrix}
1 & -1 & 1\\
0 & 2 & -1\\
0 & 0 & 0
\end{bmatrix}$$
De aqui tenemos que
$$Nul([T]_B) = Gen\left\{\begin{pmatrix}
-1 \\ 1 \\2
\end{pmatrix}\right\}$$
Finalmente,
$$Nul(T) = Gen\{-1+t+2t^2\}$$
\end{solucion}
\item Sean $B_1 = \{1+x,1-x,1+x^2\}$ y $B_2$ bases de $P_2(\R)$ tales que para todo $p \in P_2(\R)$
	$$ \begin{bmatrix}
	1 & 0 & -1\\
	1 & 1 & 0 \\
	0 & 1 & 2
	\end{bmatrix} [p]_{B_1} = [p]_{B_2}$$
	Determine los polinomios que forman la base $B_2$.
\begin{solucion}
Notemos que la matriz que nos dan corresponde a la matriz cambio de base de $B_1$ a $B_2$. Recordemos que una matriz cambio de base, de $B_1$ a $B_2$, corresponde a la matriz formada por las coordenadas de los elementos de $B_1$ en la base $B_2$.
		
		En base a lo anterior, dado que tenemos los vectores de $B_1$, lo que necesitamos es justamente lo contrario, es decir, la matriz cambio de base de $B_2$ a $B_1$. De esta manera, tendriamos las coordenadas de los elementos de $B_2$ en la base $B_1$, por lo que podriamos calcularlos con facilidad, ya que tenemos estos últimos.
		
		Para encontrar esta matriz, basta con obtener la inversa de la matriz cambio de base dada, esto es
		$$ \left[\begin{array}{ccc|ccc}
		1 & 0 & -1 & 1 & 0 & 0\\
		1 & 1 & 0 & 0 & 1 & 0\\
		0 & 1 & 2 & 0 & 0 & 1
		\end{array}\right] \sim 
		\left[\begin{array}{ccc|ccc}
		1 & 0 & 0 & 2 & -1 & 1\\
		0 & 1 & 0 & -2 & 2 & -1\\
		0 & 0 & 1 & 1 & -1 & 1
		\end{array}\right]$$
		con lo que
		$$A_{B_2}^{B_1} = \begin{bmatrix}
		2 & -1 & 1\\
		-2 & 2 & -1\\
		1 & -1 & 1
		\end{bmatrix}$$
		Sea 
		$$B_2 = \{q_1(x), q_2(x), q_3(x)\}$$
		Las columnas de $A_{B_2}^{B_1}$ corresponden a los vectores de coordenada de $q_1$, $q_2$ y $q_3$, respectivamente, en la base $B_1$.
		
		Luego,
		$$q_1(x) = 2(1+x) - 2(1-x) + (1+x^2) = 1+4x+x^2$$
		$$q_2(x) = -(1+x) + 2(1-x) - (1+x^2) = -3x-x^2$$
		$$q_3(x) = (1+x) - (1-x) + (1+x^2) = 1+2x+x^2$$
		Finalmente,
		$$B_2 = \{1+4x+x^2,-3x-x^2,1+2x+x^2\}$$
\end{solucion}
\item Sean $B$ y $C$ bases de un espacio vectorial $V$ y $P = \begin{bmatrix}4 &-1 \\ 6 & -1\end{bmatrix}$ la matriz de cambio de coordenadas tal que $[v]_C = P[v]_B\ \forall v \in V$
\begin{enumerate}[a)]
\item Demuestre que el conjunto $W = \{v \in V: [v]_C = 2[v]_B\}$ es un subespacio de $V$.
\item Si $B=\{v_1,v_2\}$ determine una base para $W$ en términos de la base de $B$.
\end{enumerate}
\begin{solucion}

\begin{enumerate}[a)]
\item Demuestre que el conjunto $W = \{v \in V: [v]_C = 2[v]_B\}$ es subespacio de $V$.
			
			\begin{enumerate}
				\item No vacio $(0\in W)$
				$$\begin{array}{cc}
				[\vec{0}]_C = \vec{0}\\
				2 \cdot [\vec{0}]_B = 2 \cdot \vec{0} = 0
				\end{array} \Longrightarrow \vec{0} = \vec{0} \ra \vec{0} \in W$$	
				\item Suma y multiplicación (a la vez)	
				
				
				Sea $u, v \in W$ y $\alpha, \beta \in \R$, debemos demostrar que $\alpha u + \beta v \in W$
				$$\begin{array}{rcl}
				[\alpha u + \beta v]_C & = & 2 [\alpha u + \beta v]_B \\
				& = & 2 [\alpha u + \beta v]_B \\
				& = & 2 [\alpha u]_B + 2[\beta v]_B \\
				& = & \alpha 2[u]_B + \beta 2[v]_B \\
				& = & \alpha [u]_C + \beta [v]_C \\
				& = & [\alpha u]_C + [\beta v]_C \\
				{[}\alpha u + \beta v]_C & = & [\alpha u +\beta v]_C
				\end{array}
				$$
			\end{enumerate}
			$$\blacksquare$$
\item Si $B=\{v_1,v_2\}$ determine una base para $W$ en términos de la base de $B$.
			
			Sea $B = \{v_1, v_2\}$ la base dada, busquemos una base $B_W$ de $W$\\
			\\
			Sea $v \in W$,
			$$v = x_1v_1 + x_2v_2 \ra x = \begin{pmatrix} x_1 \\ x_2 \end{pmatrix} = [v]_B$$
			Además,
			$$\begin{array}{cclcl}
			[v]_C & = & 2[v]_B & = & 2x \\
			{[}v]_C & = & P[v]_B & = & Px
			\end{array} \Longrightarrow
			2x = Px$$
			Luego,
			$$2x = Px$$
			$$Px - 2x = 0$$
			$$(P - 2I)x = 0$$
			$$\left(\begin{bmatrix}4 &-1 \\ 6 & -1\end{bmatrix} - \begin{bmatrix}2 &0 \\ 0 & 2\end{bmatrix}\right)x = 0$$
			$$\begin{bmatrix}2 &-1 \\ 6 & -3\end{bmatrix}x = 0$$
			Resolviendo el sistema,
			$$\begin{bmatrix}2 &-1 \\ 6 & -3\end{bmatrix} \sim 
			\begin{bmatrix}2 &-1 \\ 0 & 0\end{bmatrix} \ra x_2 = 2x_1$$
			Luego,
			$$v = x_1v_1 + 2x_1v_2 = x_1(v_1 + 2v_2)$$
			Finalmente,
			$$B_W = \{v_1 + 2v_2\}$$
\end{enumerate}
\end{solucion}
\item Calcular los valores y vectores propios de
	$$ A = \begin{bmatrix} 1 & 2 & -1\\ 1 & 0 & 1\\ 4 & -4 & 5\end{bmatrix}$$
\begin{solucion}
Para calcular los valores propios de una matriz, debemos buscar todos los valores de $\lambda$ tales que
		$$A-\lambda I = 0$$
		Luego, para cada valor propio obtenido, debemos resolver el problema
		$$(A-\lambda I)v = 0$$
		Encontrando así los valores propios.\\
		\\
		Pasando ahora al ejercicio,
		$$\left|\begin{bmatrix}
		1-\lambda & 2 & -1\\
		1 & -\lambda & 1\\
		4 & -4 & 5-\lambda 
		\end{bmatrix}\right| = 0$$
		Usando cofactores, tenemos que
		$$\left|\begin{bmatrix}
		1-\lambda & 2 & -1\\
		1 & -\lambda & 1\\
		4 & -4 & 5-\lambda 
		\end{bmatrix}\right| = (1-\lambda)(-\lambda(5-\lambda) - -4) - 2((5-\lambda)-4) -(-4 - -4\lambda)$$
		Simplificando,
		$$=-\lambda^3 + 6\lambda^2 -11\lambda + 6 = -(\lambda - 1)(\lambda - 2) (\lambda - 3)$$
		Por lo que los valores propios son
		$$\lambda_1 =  1, \quad \lambda_2 = 2, \quad \lambda_3 = 3$$
		Busquemos ahora los vectores propios asociados a cada valor propio
		\begin{enumerate}[1)]
			\item $\lambda_1 = 1$\\
			\\
			Debemos resolver el sistema $(A-I)v = 0$, esto es
			\small$$\begin{bmatrix}
			0 & 2 & -1\\
			1 & -1 & 1\\
			4 & -4 & 4
			\end{bmatrix} \sim 
			\begin{bmatrix}
			1 & -1 & 1\\
			0 & 2 & -1\\
			4 & -4 & 4
			\end{bmatrix} \sim 
			\begin{bmatrix}
			1 & -1 & 1\\
			0 & 2 & -1\\
			0 & 0 & 0
			\end{bmatrix} \sim 
			\begin{bmatrix}
			1 & -1 & 1\\
			0 & 1 & -\frac{1}{2}\\
			0 & 0 & 0
			\end{bmatrix} \sim 
			\begin{bmatrix}
			1 & 0 & \frac{1}{2}\\
			0 & 1 & -\frac{1}{2}\\
			0 & 0 & 0
			\end{bmatrix}$$
			Por lo que
			$$\begin{array}{rcl}
			x_1 & = & -\dfrac{1}{2}x_3\\
			x_2 & = & \dfrac{1}{2}x_3\\
			x_3 & = & x_3
			\end{array} \Longrightarrow
			v_1 = \begin{pmatrix}
			-\frac{1}{2}\\
			\frac{1}{2}\\
			1
			\end{pmatrix} \rightarrow
			v_1 = \begin{pmatrix}
			-1\\
			1\\
			2
			\end{pmatrix}$$
			La solución del sistema es el generado de $v$. Por esto, lo simplificamos al final para obtener un vector propio sin fracciones.
			
			\item $\lambda_2 = 2$\\
			\\
			Debemos resolver el sistema $(A-2I)v = 0$, esto es
			\small$$\begin{bmatrix}
			-1 & 2 & -1\\
			1 & -2 & 1\\
			4 & -4 & 3
			\end{bmatrix} \sim 
			\begin{bmatrix}
			1 & 0 & \frac{1}{2}\\
			0 & 1 & -\frac{1}{4}\\
			0 & 0 & 0
			\end{bmatrix}$$
			Por lo que
			$$\begin{array}{rcl}
			x_1 & = & -\dfrac{1}{2}x_3\\
			x_2 & = & \dfrac{1}{4}x_3\\
			x_3 & = & x_3
			\end{array} \Longrightarrow
			v_2 = \begin{pmatrix}
			-\frac{1}{2}\\
			\frac{1}{4}\\
			1
			\end{pmatrix} \rightarrow
			v_2 = \begin{pmatrix}
			-2\\
			1\\
			4
			\end{pmatrix}$$
			
			\item $\lambda_3 = 3$\\
			\\
			Debemos resolver el sistema $(A-3I)v = 0$, esto es
			$$\begin{bmatrix}
			-2 & 2 & -1\\
			1 & -3 & 1\\
			4 & -4 & 2
			\end{bmatrix} \sim 
			\begin{bmatrix}
			1 & 0 & \frac{1}{4}\\
			0 & 1 & -\frac{1}{4}\\
			0 & 0 & 0
			\end{bmatrix}$$
			Por lo que
			$$\begin{array}{rcl}
			x_1 & = & -\dfrac{1}{4}x_3\\
			x_2 & = & \dfrac{1}{4}x_3\\
			x_3 & = & x_3
			\end{array} \Longrightarrow
			v_3 = \begin{pmatrix}
			-\frac{1}{4}\\
			\frac{1}{4}\\
			1
			\end{pmatrix} \rightarrow
			v_3 = \begin{pmatrix}
			-1\\
			1\\
			4
			\end{pmatrix}$$
		\end{enumerate}
		Finalmente, los vectores propios son
		$$v_1 = \begin{pmatrix}
		-1\\
		1\\
		2
		\end{pmatrix}, \quad
		v_2 = \begin{pmatrix}
		-2\\
		1\\
		4
		\end{pmatrix}, \quad
		v_3 = \begin{pmatrix}
		-1\\
		1\\
		4
		\end{pmatrix}$$
\end{solucion}
\item La matriz $\begin{bmatrix} 1 & -2 \\ 1 & 3 \end{bmatrix}$ actúa sobre $\mathbb{C}^2$. Determine los valores propios y una base para cada espacio propio de $\mathbb{C}^2$.
\begin{solucion}
En primer lugar, buscamos los valores propios,
		$$det(A-\lambda I) = \left|\begin{bmatrix} 
		1-\lambda & -2 \\ 
		1 & 3-\lambda \end{bmatrix}\right| = (1-\lambda)(3-\lambda) + 2 = \lambda^2 - 4\lambda + 5$$
		$$\lambda_{1,2} = 2 \pm i$$
		Como los valores propios son un par de complejos conjugados, los vectores propios asociados a cada uno también lo serán, por lo que basta con buscar un solo vector propio.
		
		Para $\lambda = 2+i$,
		$$(A-(2+i)\lambda)x = 0 \ra 
		\begin{bmatrix} 
		-1-i & -2 \\ 
		1 & 1-i 
		\end{bmatrix} \sim 
		\begin{bmatrix} 
		-1-i & -2 \\ 
		0 & 0
		\end{bmatrix}$$
		$$(-1-i)x_1 - 2x_2 = 0 \ra \begin{array}{lcl}
		x_1 & = & x_1 \\
		x_2 & = & \dfrac{-1-i}{2}x_1
		\end{array}$$
		Luego,
		$$v_1 = \begin{pmatrix}
		2 \\
		-1-i
		\end{pmatrix}, \quad
		v_2 = \begin{pmatrix}
		2 \\
		-1+i
		\end{pmatrix} $$
		Por último, las bases de los espacios propios están formadas por los vectores propios asociados a cada valor propio, es decir
		$$E_{2+i} = Gen\left\{\begin{pmatrix}
		2 \\
		-1-i
		\end{pmatrix}\right\}, \quad 
		E_{2-i} = Gen\left\{\begin{pmatrix}
		2 \\
		-1+i
		\end{pmatrix}\right\}$$
\end{solucion}
\item Sea $A$ una matriz invertible, y sea $\lambda$ un valor propio de $A$. Demuestre que $\lambda \neq 0$ y que $\dfrac{1}{\lambda}$ es un valor propio de $A^{-1}$.
\begin{solucion}

		P.D. $\lambda \neq 0 \wedge \dfrac{1}{\lambda} \text{ es VP de }A^{-1}$
		
		Digamos que $\lambda = 0$. Luego, como $\lambda = 0$ es un VP, tenemos que
		$$det(A-\lambda I) = 0$$
		$$det(A-0 I) = 0$$
		$$det(A) = 0$$
		Pero $det(A) \neq 0$, ya que $A$ es invertible. Entonces, $\lambda \neq 0$.
		
		Notemos también que
		$$A^{-1}x 
		= A^{-1}\left(\dfrac{1}{\lambda}(\lambda x)\right)
		= A^{-1}\dfrac{1}{\lambda}(\lambda x)
		= \dfrac{1}{\lambda}A^{-1}(A x)
		= \dfrac{1}{\lambda} x$$
		$$A^{-1}x 
		= \dfrac{1}{\lambda} x$$
		De donde se desprende que $\dfrac{1}{\lambda}$ es valor propio de $A^{-1}$.
\end{solucion}
\item Sea $A$ una matriz de $5 \times 5$ tal que $A^t = -A$ y $q$ el polinomio dado por $q(x) = 2-x^2+4x^3$. Demuestre que $2$ es valor propio de la matriz $q(A)$.
\begin{solucion}
Sabemos que
		$$det(A) = det(A^T) \wedge det(A^T) = det(-A)$$
		$$det(A) = det(-A)$$
		$$det(A) = -det(A)$$
		$$det(A) = 0$$
		Luego, existe $u \neq 0 | Au = 0$
		
		Notemos que
		$$\begin{array}{lcl}
		q(A) & = & 2I - A^2 + 4A^3\\
		q(A) u & = & (2I - A^2 + 4A^3)u\\
		& = & 2Iu - A^2u + 4A^3u \\
		& = & 2Iu - A(Au) + 4A^2(Au) \\
		& = & 2Iu - A0 + 4A^20 \\
		& = & 2Iu \\
		q(A) u & = & 2u
		\end{array}$$
		Luego, $\lambda = 2 $ es $VP$ de $q(A)$
\end{solucion}
\item Sea $A$ una matriz de $2\times 2$ de rango 1 tal que $A\begin{bmatrix} 1 \\ 2\end{bmatrix} = \begin{bmatrix} 2 \\ 4 \end{bmatrix}$. ¿Es $A$ diagonalizable? Justifique.
\begin{solucion}
Notemos que
		$$A\begin{bmatrix} 1 \\ 2\end{bmatrix} = \begin{bmatrix} 2 \\ 4 \end{bmatrix}
		= A\begin{bmatrix} 1 \\ 2\end{bmatrix} = 2\begin{bmatrix} 1 \\ 2 \end{bmatrix}$$
		Luego, $\lambda_1 = 2$ es valor propio de la matriz $A$ y $v_1 = \begin{bmatrix} 1 \\ 2\end{bmatrix}$ es el vector propio asociado.
		
		Como $A$ es de rango 1, $\exists u \neq 0$ tal que $Au = $. Para ese $u$ se cumple que
		$$Au = 0 \ra Au = 0 u$$
		Luego, $\lambda_2 = 0$ es valor propio de la matriz $A$ y $v_2 = u$ es el vector propio asociado.
		
		Finalmente, como la multiplicidad algebraica es igual a la multiplicidad geométrica, la matriz es diagonalizable.
\end{solucion}
\item Diagonalice la matriz
	$$M = \begin{bmatrix}
	1 & 0 & 0\\
	1 & 1 & 2 \\
	1 & 0 & 3
	\end{bmatrix}$$
	y encuentre una matriz $N$ tal que $N^3 = M$.
\begin{solucion}
Buscamos los valores propios,
		$$det(M-\lambda I) = \left|\begin{bmatrix}
		1-\lambda & 0 & 0\\
		1 & 1-\lambda & 2 \\
		1 & 0 & 3-\lambda
		\end{bmatrix}\right|
		= (1-\lambda)(1-\lambda)(3-\lambda) = 0$$
		$$\lambda_1 = 1 \ra \text{multiplicidad 2}$$
		$$\lambda_2 = 3 \ra \text{multiplicidad 1}$$
		Buscamos ahora los vectores propios,
		\begin{itemize}
			\item $\lambda_1 = 1$
			
			$$(A-I)x = 0$$
			$$\begin{bmatrix}
			0 & 0 & 0\\
			1 & 0 & 2 \\
			1 & 0 & 2
			\end{bmatrix} \ra v_1 = \begin{pmatrix}
			2 \\ 0 \\ -1
			\end{pmatrix}, \quad
			v_2 = \begin{pmatrix}
			0 \\ 1 \\ 0
			\end{pmatrix}$$
			
			\item $\lambda_2 = 3$
			
			$$(A-3I)x = 0$$
			$$\begin{bmatrix}
			-2 & 0 & 0\\
			1 & -2 & 2 \\
			1 & 0 & 0
			\end{bmatrix} \ra v_3 = \begin{pmatrix}
			0 \\ 1 \\ 1
			\end{pmatrix}$$
		\end{itemize}
			Recordemos ahora que para diagonalizar una matriz debemos expresarla de la forma
			$$M = PDP^{-1} \ra P = \begin{bmatrix}
			v_1 & v_2 & v_3
			\end{bmatrix}, \quad
			D = \begin{bmatrix}
			\lambda_1 & 0 & 0 \\
			0 & \lambda_2 & 0 \\
			0 & 0 & \lambda_3
			\end{bmatrix}$$
			Es importante recordar que cada vector propio debe ir asociado con su respectivo valor propio (misma columna).
			
			En nuestro ejercicio, debemos considerar $\lambda_1 = \lambda_2 = 1$ y $\lambda_3 = 3$. Entonces,
			$$P = \begin{bmatrix}
			2 & 0 & 0 \\
			0 & 1 & 1 \\
			-1 & 0 & 1
			\end{bmatrix}, \quad
			D = \begin{bmatrix}
			1 & 0 & 0 \\
			0 & 1 & 0 \\
			0 & 0 & 3
			\end{bmatrix}, \quad
			P^{-1} = \begin{bmatrix}
			\frac{1}{2} & 0 & 0 \\
			-\frac{1}{2} & 1 & -1 \\
			\frac{1}{2} & 0 & 1
			\end{bmatrix}$$
			Sea
			$$R = \sqrt[3]{D} = \begin{bmatrix}
			1 & 0 & 0 \\
			0 & 1 & 0 \\
			0 & 0 & \sqrt[3]{3}
			\end{bmatrix}$$
			y sea
			$$N = PRP^{-1}$$
			Notemos que	
			$$N^3 = (PRP^{-1}) = PR\cancel{P^{-1}P}R\cancel{P^{-1}P}RP^{-1} = PR^3P^{-1} = PDP^{-1} = M$$
			Finalmente, $N = P\sqrt[3]{D}P^{-1}$ es la matriz buscada.
\end{solucion}
\item Diagonalice $A =
	\begin{bmatrix} 
	-1 & -2 & 2 \\
	0 & -1 & 0 \\
	0 & -2 & 1
	\end{bmatrix}$ y diagonalice $B = A^{10} + A - I$ 
\begin{solucion}
Buscamos los valores propios,
		$$det(A-\lambda I) = \left|\begin{bmatrix}
		-1-\lambda & -2 & 2 \\
		0 & -1-\lambda & 0 \\
		0 & -2 & 1-\lambda
		\end{bmatrix}\right|
		= (1+\lambda)^2(1-\lambda) = 0$$
		$$\lambda_1 = 1 \ra \text{multiplicidad 1}$$
		$$\lambda_2 = -1 \ra \text{multiplicidad 2}$$
		Buscamos ahora los vectores propios,
		\begin{itemize}
			\item $\lambda_1 = 1$
			
			$$(A-I)x = 0 \ra v_1 = \begin{pmatrix}
			1 \\ 0 \\ 1
			\end{pmatrix}$$
			
			\item $\lambda_2 = -1$
			
			$$(A+I)x = 0 \ra v_2 = \begin{pmatrix}
			1 \\ 0 \\ 0
			\end{pmatrix}, \quad
			v_2 = \begin{pmatrix}
			0 \\ 1 \\ 1
			\end{pmatrix}$$
		\end{itemize}
		Luego,
		$$A = PDP^{-1} \ra P = \begin{bmatrix}
		1 & 1 & 0 \\
		0 & 0 & 1 \\
		1 & 0 & 1
		\end{bmatrix}, \quad
		D = \begin{bmatrix}
		1 & 0 & 0 \\
		0 & -1 & 0 \\
		0 & 0 & -1
		\end{bmatrix}, \quad
		P^{-1} = \begin{bmatrix}
		0 & -1 & 1 \\
		1 & 1 & -1 \\
		0 & 1 & 0
		\end{bmatrix}$$
		Reemplacemos ahora en $B$,
		$$\begin{array}{lcl}
		B & = & A^{10} + A - I \\
		 & = & (PDP^{-1})^{10} + PDP^{-1} - I \\
		 & = & (PDP^{-1})^{10} + PDP^{-1} - PIP^{-1} \\
		B & = & P(D^{10} + D - I)P^{-1}
		\end{array}$$
		Además, sabemos que
		$$D^{10} = \begin{bmatrix}
		1^{10} & 0 & 0 \\
		0 & (-1)^{10} & 0 \\
		0 & 0 & (-1)^{10}
		\end{bmatrix}
		 = \begin{bmatrix}
		1 & 0 & 0 \\
		0 & 1 & 0 \\
		0 & 0 & 1
		\end{bmatrix} = I$$
		Por lo tanto,
		$$B = P(I + D - I)P^{-1} = PDP^{-1}$$
		Con lo que concluimos que $A=B$, por lo que la diagonalización de $A$ también es diagonalización de $B$.
\end{solucion}
\item Diagonalice ortogonalmente
	$$M = \begin{bmatrix}
	1 & 0 & 1 \\
	0 & 2 & 0 \\
	1 & 0 & 1
	\end{bmatrix}$$
\begin{solucion}
Para hacer esto debemos hacer un procedimiento similar al que realizamos cuando queremos diagonalizar una matriz, es decir, buscar $P$ y $D$ tal que $M = PDP^{-1}$, con la diferencia de que en este caso $P$ debe ser ortogonal.
		
		Comenzamos buscando los valores propios, con lo que obtendremos
		$$\lambda_1 = 0 \ra \text{multiplicidad 1}$$
		$$\lambda_2 = 2 \ra \text{multiplicidad 2}$$
		Luego, buscamos los vectores propios asociados a cada uno de estos valores propios, los que son
		$$\lambda_1 = 0 \ra v_1 = \begin{pmatrix}
		1 \\ 0 \\ -1
		\end{pmatrix}$$
		$$\lambda_2 = 2 \ra v_2 = \begin{pmatrix}
		0 \\ 1 \\ 0
		\end{pmatrix}, v_3 = \begin{pmatrix}
		1 \\ 0 \\ 1
		\end{pmatrix}$$
		Notemos que todos estos vectores son ortogonales entre si, sin embargo, necesitamos que sean ortonormales. Recordemos que un vector propio será cualquier multiplo de los vectores calculados anteriormente, por lo que para obtener vectores ortonormales, basta con dividir cada uno por su modulo. De esta forma, nuestros nuevos vectores serán
		$$v_1 = \begin{pmatrix}
		1/\ \sqrt[]{2} \\ 0 \\ -1/\ \sqrt[]{2}
		\end{pmatrix}, v_2 = \begin{pmatrix}
		0 \\ 1 \\ 0
		\end{pmatrix}, v_3 = \begin{pmatrix}
		1/\ \sqrt[]{2} \\ 0 \\ 1/\ \sqrt[]{2}
		\end{pmatrix}$$
		Luego, $M$ se diagonaliza con
		$$P = \begin{bmatrix}
		1/\ \sqrt[]{2} & 0 & 1/\ \sqrt[]{2} \\
		0 & 1 & 0 \\
		-1/\ \sqrt[]{2} & 0 & 1/\ \sqrt[]{2}
		\end{bmatrix}, \quad D = \begin{bmatrix}
		0 & 0 & 0 \\
		0 & 2 & 0 \\
		0 & 0 & 2
		\end{bmatrix}$$
\end{solucion}
\item Sean $u, v$ dos vectores ortogonales en $\R^n$ tales que $||u|| = 1, ||v|| = \sqrt[]{3/2}$. Demuestre que el conjunto 
	$$B = \{u-v, 3u+2v\}$$
	es ortogonal y encuentre las coordenadas del vector $4u - 9v$ respecto al conjunto $B$.
\begin{solucion}
Para demostrar que $B$ es ortogonal, debemos demostrar que todos sus elementos son ortogonales entre si, es decir, debemos demostrar que
		$$(u-v) \cdot (3u+2v) = 0$$
		Notemos que como $u$ y $v$ son ortogonales entre si, $u \cdot v = 0$. Luego,
		$$\begin{array}{rcl}
		(u-v) \cdot (3u+2v) & = & 3u \cdot u + 2 u \cdot v - 3u \cdot v - 2v \cdot v\\
		& = & 3||u||^2 + 2 (u \cdot v) - 3(u \cdot v) - 2||v||^2 \\
		& = & 3 + 2 \cdot 0 - 3 \cdot 0 - 2(\ \sqrt[]{3/2})^2 \\
		& = & 3 - 3 \\
		& = & 0
		\end{array}$$
		$$\blacksquare$$
		Ahora, para buscar las coordenadas de $4u - 9v$ respecto al conjunto $B$, debemos buscar $\alpha, \beta \in \R$, tal que
		$$\alpha(u-v) + \beta(3u+2v) = 4u-9v$$
		Reordenando,
		$$(\alpha+3\beta)u + (2\beta -\alpha)v= 4u -9v$$
		Luego, debemos resolver el sistema
		$$\begin{array}{rcl}
		\alpha+3\beta & = & 4\\
		-\alpha+2\beta & = & -9
		\end{array}$$
		Resolviendolo, obtenemos que
		$$\alpha = 7 \quad y \quad \beta = -1$$
		Finalmente,
		$$[4u-9v]_B = \begin{pmatrix}
		7 \\ -1
		\end{pmatrix}$$
\end{solucion}
\item Demuestre que si $P$ es una matriz ortogonal de  $n \times n$, entonces para todo $x, y \in \R^n$ se tiene que $Px \cdot Py = x \cdot y$
\begin{solucion}
Como $P$ es ortogonal, se cumple que $P^TP = I$. Luego,
		$$Px \cdot Py = (Px)^T(Py) = x^TP^TPy = x^TIx= x^Ty = x \cdot y$$
\end{solucion}
\item Determine si las siguientes afirmaciones son Verdaderas o Falsas.
\begin{enumerate}[a)]
\item El espacio fila de $AB$ es subespacio del espacio fila de $B$.
\item Dada una base en un espacio vectorial $V$, el vector coordenado de un vector de $V$ con respecto a esa base, es único.
\item Si $A$ y $B$ son similares y $\lambda$ es valor propio de $A$ entonces $\lambda$ es también valor propio de $B$.
\end{enumerate}
\begin{solucion}

\begin{enumerate}[a)]
\item El espacio fila de $AB$ es subespacio del espacio fila de $B$.
			
			$$C = AB  \ra C^T = B^TA^T \ra Col(C^T) = Col(B^TA^T)$$
			Como al multiplicar una matriz por otra podemos, en el mejor de los casos, obtener la misma dimensión de antes,
			$$Col(C^T) \subset Col(B^T) \ra Fil(C) \subset Fil(B)$$
			Luego,
			$$Fil(AB) \subset Fil(B)$$
			Con lo que concluimos que el espacio fila de $AB$ es subespacio del espacio fila de $B$. Entonces, la afirmación es {\bf Verdadera}.
\item Dada una base en un espacio vectorial $V$, el vector coordenado de un vector de $V$ con respecto a esa base, es único.
			
			Los vectores de una base son $L.I.$, por lo que existe una única combinación lineal para generar cada vector. Luego, la afirmación es {\bf Verdadera}.
\item Si $A$ y $B$ son similares y $\lambda$ es valor propio de $A$ entonces $\lambda$ es también valor propio de $B$.
			
			Tomemos, las matrices
			$$A = \begin{bmatrix}
			1 & 0 \\ 0 & 1
			\end{bmatrix}, \quad 
			B = \begin{bmatrix}
			2 & 0 \\ 0 & 2
			\end{bmatrix}$$
			Los valores propios de $A$ son $\lambda = 1$ (multiplicidad 2) y los de $B$ son $\lambda = 2$ (multiplicidad 2), por lo que la afirmación es {\bf Falsa}.
\end{enumerate}
\end{solucion}
\item Sea 
	$U = Gen\left\{
	\begin{bmatrix}1\\0\\1\\1\end{bmatrix},
	\begin{bmatrix}0\\1\\2\\1\end{bmatrix},
	\begin{bmatrix}1\\1\\3\\2\end{bmatrix}
	\right\}$.
\begin{enumerate}[a)]
\item Encuentre una base ortonormal de $U$.
\item Encuentre la distancia de $b = 
		\begin{bmatrix}1\\1\\1\\1\end{bmatrix}$ a $U$.
\end{enumerate}
\begin{solucion}

\begin{enumerate}[a)]
\item Encuentre una base ortonormal de $U$.\\
			\\
			En primer lugar, debemos buscar una base común y corriente de $U$. Para esto, tomemos los vectores L.I. del conjunto que lo genera. Notemos que
			$$\begin{bmatrix}
			1 & 0 & 1 \\
			0 & 1 & 1 \\
			1 & 2 & 3 \\
			1 & 1 & 2
			\end{bmatrix} \sim
			\begin{bmatrix}
			1 & 0 & 1 \\
			0 & 1 & 1 \\
			0 & 0 & 0 \\
			0 & 0 & 0
			\end{bmatrix}$$
			Por lo que concluimos que los dos vectores del generado son L.I. Entonces, una base de $U$ corresponde a
			$$B = \left\{ \begin{pmatrix}
			1 \\ 0 \\ 1 \\ 1
			\end{pmatrix},\begin{pmatrix}
			0 \\ 1 \\ 2 \\ 1
			\end{pmatrix} \right\}$$
			Recordemos que Gramm-Schmidt forma una base ortogonal a partir de una base cualquiera y funciona de la siguiente forma:
			
			Sea $B = \{v_1, v_2, \dots \}$ una base cualquiera de un espacio vectorial,
			
			$$\begin{array}{rcl}
			u_1 & = & v_1 \\\\
			u_2 & = & v_2 - \dfrac{v_2u_1}{u_1u_1}u_1 \\\\
			u_3 & = & v_3 - \dfrac{v_3u_1}{u_1u_1}u_1 - \dfrac{v_3u_2}{u_2u_2}u_2\\
			& \vdots & 			
			\end{array}$$
			Luego, $B^{\perp} = \{u_1, u_2, \dots \}$ es una base ortogonal de ese espacio vectorial.
			
			Hagamos esto con nuestra base:
			
			En primer lugar,
			$$u_1 = v_1 = \begin{pmatrix}
			1 \\ 0 \\ 1 \\ 1
			\end{pmatrix}$$
			
			Luego,
			$$u_2 = v_2 - \dfrac{v_2u_1}{u_1u_1}u_1
			= \begin{pmatrix}
			0 \\ 1 \\ 2 \\ 1
			\end{pmatrix} - \dfrac{\begin{pmatrix}
				0 \\ 1 \\ 2 \\ 1
				\end{pmatrix}\begin{pmatrix}
				1 \\ 0 \\ 1 \\ 1
				\end{pmatrix}}{\begin{pmatrix}
				1 \\ 0 \\ 1 \\ 1
				\end{pmatrix}\begin{pmatrix}
				1 \\ 0 \\ 1 \\ 1
				\end{pmatrix}}\begin{pmatrix}
			1 \\ 0 \\ 1 \\ 1
			\end{pmatrix} 
			= \begin{pmatrix}
			0 \\ 1 \\ 2 \\ 1
			\end{pmatrix} - \dfrac{3}{3}\begin{pmatrix}
			1 \\ 0 \\ 1 \\ 1
			\end{pmatrix} 
			= \begin{pmatrix}
			-1 \\ 1 \\ 1 \\ 0
			\end{pmatrix} $$
			De esta forma,
			$$B^{\perp} = \left\{\begin{pmatrix}
			1 \\ 0 \\ 1 \\ 1
			\end{pmatrix}, \begin{pmatrix}
			-1 \\ 1 \\ 1 \\ 0
			\end{pmatrix}\right\}$$
			Sin embargo, necesitamos una base ortonormal. Para esto, basta con dividir todos los vectores de la base por su norma, obteniendo
			$$\hat{B}^{\perp} = \left\{\begin{pmatrix}
			1/\ \sqrt[]{3} \\ 0 \\ 1/\ \sqrt[]{3} \\ 1/\ \sqrt[]{3}
			\end{pmatrix}, \begin{pmatrix}
			-1/\ \sqrt[]{3} \\ 1/\ \sqrt[]{3} \\ 1/\ \sqrt[]{3} \\ 0
			\end{pmatrix}\right\}$$
\item Encuentre la distancia de $b = 
			\begin{bmatrix}1\\1\\1\\1\end{bmatrix}$ a $U$.
			
			Sea $A$ una matriz con los vectores de una base de $U$ en sus columnas y $x$ un vector coordenada de un elemento cualquier perteneciente a $U$, entonces la distancia entre ese elemento y $b$ corresponde a $||Ax-b||$.
			
			Luego, la distancia entre $b$ y $U$ corresponde a la distancia más pequeña entre $b$ y un elemento de $U$, es decir,
			$$min||Ax-b||$$
			Resolver este problema es equivalente a resolver el sistema
			$$A^TAx = A^Tb$$
			Utilicemos
			$$A = \begin{bmatrix}
			1 & 0 \\
			0 & 1 \\
			1 & 2 \\
			1 & 1
			\end{bmatrix}$$
			Entonces,
			$$A^TA = \begin{bmatrix}
			3 & 3 \\
			3 & 6
			\end{bmatrix}, \quad A^Tb = \begin{pmatrix}
			3 \\ 4
			\end{pmatrix}$$
			Por lo que tenemos que resolver el sistema
			$$\begin{bmatrix}
			3 & 3 \\
			3 & 6
			\end{bmatrix}x = \begin{pmatrix}
			3 \\ 4
			\end{pmatrix} \ra x = \begin{pmatrix}
			\frac{2}{3}\\ \frac{1}{3}
			\end{pmatrix}$$
			Finalmente, la distancia corresponde a
			$$||Ax-b|| = \left|\left|\begin{bmatrix}
			1 & 0 \\
			0 & 1 \\
			1 & 2 \\
			1 & 1
			\end{bmatrix}\begin{pmatrix}
			\frac{2}{3}\\ \frac{1}{3}
			\end{pmatrix} - \begin{pmatrix}1\\1\\1\\1\end{pmatrix}\right|\right|
			= \left|\left|\begin{pmatrix}
			-\frac{1}{3}\\
			-\frac{2}{3}\\
			\frac{1}{3}\\
			0
			\end{pmatrix}\right|\right| = \sqrt[]{\dfrac{2}{3}}$$
\end{enumerate}
\end{solucion}
\item Un cierto experimento genera los datos $(1,3)$, $(2,5)$ y $(3,4)$. Describa el modelo que da un ajuste de mínimos cuadrados de esos puntos mediante una recta de la forma $y = \beta_0 +  \beta_1x$
\begin{solucion}
Los valores del experimento se pueden ver en las siguiente tabla:\\
		\\
		\begin{center}
		\begin{tabular}{|l|l|}
			\hline
			\textbf{y} & \textbf{x} \\ \hline
			3          & 1          \\ \hline
			5          & 2          \\ \hline
			4          & 3          \\ \hline
		\end{tabular}
		\end{center}
		Estamos buscando $\beta_0, \beta_1$ tal que
		$$y = \beta_0 +  \beta_1x$$
		sea un ajuste por mínimos cuadrados. Esto lo podemos escribir como
		$$Y = X\beta$$
		donde
		$$Y = \begin{pmatrix}
		3 \\ 5 \\ 4
		\end{pmatrix}, \quad 
		X = \begin{bmatrix}
		1 & 1 \\
		1 & 2 \\
		1 & 3
		\end{bmatrix}, \quad
		\beta = \begin{pmatrix}
		\beta_0 \\ \beta_1
		\end{pmatrix}$$
		Para encontrar $\beta$, debemos resolver el sistema
		$$X^TX\beta = X^TY \ra \beta = (X^TX)^{-1}X^TY$$
		$$(X^TX)^{-1} = \left(\begin{bmatrix}
		1 & 1 & 1\\
		1 & 2 & 3
		\end{bmatrix}
		\begin{bmatrix}
		1 & 1 \\
		1 & 2 \\
		1 & 3
		\end{bmatrix}\right)^{-1}
		 = 
		 \begin{bmatrix}
		 3 & 6 \\
		 6 & 14
		 \end{bmatrix}^{-1} = \begin{bmatrix}
		 \frac{7}{3} & -1 \\
		 -1 & \frac{1}{2}
		 \end{bmatrix}$$
		 $$ X^TY = \begin{bmatrix}
		 1 & 1 & 1\\
		 1 & 2 & 3
		 \end{bmatrix} \begin{pmatrix}
		 3 \\ 5 \\ 4
		 \end{pmatrix} = 
		 \begin{pmatrix}
		 12 \\ 25
		 \end{pmatrix}$$
		 Luego,
		 $$\beta = (X^TX)^{-1}X^TY
		 = \begin{bmatrix}
		 \frac{7}{3} & -1 \\
		 -1 & \frac{1}{2}
		 \end{bmatrix}\begin{pmatrix}
		 12 \\ 25
		 \end{pmatrix}
		 = \begin{pmatrix}
		 3 \\ \frac{1}{2}
		 \end{pmatrix}
		 $$
		 Finalmente, el ajuste por mínimos cuadrados es
		 $$y = 3 + \dfrac{1}{2}x$$
\end{solucion}
\item Sea $A$ la matriz simétrica que representa a la forma cuadrática
	$$Q(x) = 9x_1^2 + 7x_2^2 + 11x_3^2 - 8x_1x_2 + 8x_1x_3$$
\begin{enumerate}[a)]
\item Encuentre una matriz ortogonal $P$ tal que el cambio de variable $x = Py$ transforma la forma $x^TAx$ en una forma cuadrática sin productos cruzados.
\item Escriba la forma cuadrática en las nuevas variables y clasifíquela.
\end{enumerate}
\begin{solucion}

\begin{enumerate}[a)]
\item Encuentre una matriz ortogonal $P$ tal que el cambio de variable $x = Py$ transforma la forma $x^TAx$ en una forma cuadrática sin productos cruzados.\\
\\
En primer lugar, podemos ver que
$$A = \begin{bmatrix}
9 & -4 & 4\\
-4 & 7 & 0\\
4 & 0 & 11
\end{bmatrix}$$
Notemos que para que una forma cuadrática no tenga productos cruzados, su matriz asociada debe ser diagonal.\\

Luego, haciendo el cambio de variable,
$$x^TAx = (Py)^TA(Py) = y^TP^TAPy$$
Por lo tanto, para cumplir lo que nos piden, debe ocurrir que $P^TAP = D$ con $D$ una matriz diagonal.\\

Notemos que
$$P^TAP = D \ra A = PDP^T$$
Por lo que podemos encontrar $P$ diagonalizando ortogonalmente $A$.\\

Buscando los valores y vectores propios de $A$ obtendremos
$$\lambda_1 = 3 \ra v_1 = \begin{pmatrix}
2 \\ 2 \\ -1
\end{pmatrix},
\quad
\lambda_2 = 9 \ra v_2 = \begin{pmatrix}
1 \\ -2 \\ -2
\end{pmatrix}, 
\quad
\lambda_3 = 15 \ra v_3 = \begin{pmatrix}
2 \\ -1 \\ 2
\end{pmatrix}$$
Como queremos diagonalizar ortogonalmente, normalizamos cada vector propio, con lo que tenemos
$$v_1 = \begin{pmatrix}
\frac{2}{3} \\ \frac{2}{3} \\ -\frac{1}{3}
\end{pmatrix},
\quad
v_2 = \begin{pmatrix}
\frac{1}{3} \\ -\frac{2}{3} \\ -\frac{2}{3}
\end{pmatrix},
\quad
v_3 = \begin{pmatrix}
\frac{2}{3} \\ -\frac{1}{3} \\ \frac{2}{3}
\end{pmatrix} $$
Finalmente,
$$P = \begin{bmatrix}
\frac{2}{3} & \frac{1}{3} & \frac{2}{3} \\
\frac{2}{3} & -\frac{2}{3} & -\frac{2}{3} \\
-\frac{1}{3} & -\frac{1}{3} & \frac{2}{3}
\end{bmatrix}$$
\item Escriba la forma cuadrática en las nuevas variables y clasifíquela.\\
\\
De antes, tenemos que
$$x^TAx = (Py)^TA(Py) = y^TP^TAPy = y^TDy$$
Donde $D$ también era la matriz diagonal de la diagonalización ortogonal de $A$, por lo que
$$D = \begin{bmatrix}
3 & 0 & 0 \\
0 & 9 & 0 \\
0 & 0 & 15
\end{bmatrix}$$
Por lo tanto, como todos los elementos de la diagonal son positivos, la forma cuadrática es positiva definida.
\end{enumerate}
\end{solucion}
\item Sea $A = \begin{bmatrix}1 & -3 \\ -3 & 9\end{bmatrix}$
\begin{enumerate}[a)]
\item Encuentre una matriz $L$ cuadrada, triangular inferior con numeros 1 en la diagonal, y una matriz diagonal $D$ tal que $A = LDL^T$.
\item Encuentre la segunda descomposición de Cholesky de $A$
\item Realice un cambio de variable $x=Py$ que transforme la forma cuadrática
		$$Q\left(\begin{bmatrix}x_1\\x_2\end{bmatrix}\right) = \begin{bmatrix}x_1& x_2\end{bmatrix}A\begin{bmatrix}x_1\\ x_2\end{bmatrix}$$
		en una sin términos con producto cruzado.
\end{enumerate}
\begin{solucion}

\begin{enumerate}[a)]
\item Encuentre una matriz $L$ cuadrada, triangular inferior con numeros 1 en la diagonal, y una matriz diagonal $D$ tal que $A = LDL^T$.\\
\\
En primer lugar,  buscamos $A = LU$, esto es
$$A \begin{bmatrix}1 & -3 \\ -3 & 9\end{bmatrix} 
\wsim
\begin{bmatrix}1 & -3 \\ 0 & 0\end{bmatrix}
= U$$
Luego,
$$L = \begin{bmatrix}1 & 0 \\ -3 & 1\end{bmatrix}$$
Ahora, debemos escribir $U$ como $DL^T$. Para hacer esto, basta tomar $D$ como la diagonal de $U$, con lo que
$$D = \begin{bmatrix}1 & 0 \\ 0 & 0\end{bmatrix}$$
Finalmente,
$$A = 
\begin{bmatrix}1 & 0 \\ -3 & 1\end{bmatrix}
\begin{bmatrix}1 & 0 \\ 0 & 0\end{bmatrix}
\begin{bmatrix}1 & -3 \\ 0 & 1\end{bmatrix}
$$
\item Encuentre la segunda descomposición de C
holesky de $A$\\
\\
Para encontrar la segunda descomposición de Cholesky, debemos escribir $A$ de la forma $A = RR^T$.\\

Notemos que
$$A = LDL^T = (L\ \sqrt[]{D})(\sqrt[]{D}L^T) = (L\ \sqrt[]{D})(L\ \sqrt[]{D})^T$$
Por lo que podemos tomar
$$R =  L\ \sqrt[]{D} = 
\begin{bmatrix}1 & 0 \\ -3 & 1\end{bmatrix}
\sqrt[]{\begin{bmatrix}1 & 0 \\ 0 & 0\end{bmatrix}} =
\begin{bmatrix}1 & 0 \\ -3 & 1\end{bmatrix}
\begin{bmatrix}1 & 0 \\ 0 & 0\end{bmatrix} =
\begin{bmatrix}1 & 0 \\ -3 & 0\end{bmatrix}
$$
Finalmente,
$$A = 
\begin{bmatrix}1 & 0 \\ -3 & 0\end{bmatrix}
\begin{bmatrix}1 & -3 \\ 0 & 0\end{bmatrix}
$$
\item Realice un cambio de variable $x=Py$ que transforme la forma cuadrática
$$Q\left(\begin{bmatrix}x_1\\x_2\end{bmatrix}\right) = \begin{bmatrix}x_1& x_2\end{bmatrix}A\begin{bmatrix}x_1\\ x_2\end{bmatrix}$$
en una sin términos con producto cruzado.\\
\\
Como $A = LDL^T$,	
$$Q(x) = x^T(LDL^T)x = (x^TL)D(L^Tx) = (L^Tx)^TD(L^Tx)$$
Luego, si tomamos
$$y = L^Tx \ra x = (L^T)^{-1}y = 
\begin{bmatrix}
1 & 3 \\
0 & 1
\end{bmatrix}
\begin{pmatrix}
y_1 \\ y_2
\end{pmatrix}
= 
\begin{pmatrix}
y_1 + 3y_2 \\ y_1
\end{pmatrix}$$
Obtenemos la forma cuadrática
$$Q(y) = y^TDy = y_1^2$$
Que no tiene productos cruzados.
\end{enumerate}
\end{solucion}
\item Calcule la descomposición en valores singulares de la matriz $A = \begin{bmatrix}4 & 11 & 14\\ 8 & 7 & -2\end{bmatrix}$
\begin{solucion}
Sea $A_{m \times n}$ de rango $r$, debemos buscar una matriz $\Sigma_{m \times n}$, una matriz $U_{m \times m}$ y una matriz $V_{n \times n}$ tal que
		$$A = U \Sigma V^T$$
		La matriz $\Sigma$ será de la forma
		$$\Sigma = \begin{bmatrix}
		D & \dots & 0\\
		\vdots & \ddots & 0\\
		0 & \dots & 0
		\end{bmatrix}$$
		donde $D$ es una matriz diagonal con los primeros $r$ valores singulares de $A$ ordenados en de manera decreciente.
		
		$U$ y $V$ serán matrices ortogonales compuestas por los vectores singulares izquierdos y derechos de $A$, respectivamente.
		
		Los valores singulares de $A$ corresponden a las raices de los valores propios de la matriz $A^TA$. Estos los denominaremos como $\sigma_n = \sqrt[]{\lambda_n}$ y los ordenaremos de mayor a menor, por lo que siempre $\sigma_{n+1} \geq \sigma_n$.
		
		Los vectores singulares derechos de $A$ corresponderán a los vectores propios unitarios de $A^TA$, por lo que la matriz $V$ será de la forma 
		$$V = \begin{bmatrix}
		v_1 & v_2 & \dots & v_n
		\end{bmatrix}$$
		donde cada uno esta asociado a uno de valores singulares (recuerden que deben estar ordenados de mayor a menor).
		
		Los vectores singulares izquierdos corresponderán a $u_n = \dfrac{1}{\sigma_n}Av_n$ donde $v_n$ son los vectores propios unitarios de $A^TA$. Luego, la matriz $U$ es de la forma
		$$U = \begin{bmatrix}
		u_1 & u_2 & \dots & u_n
		\end{bmatrix}$$
		
		Veamos ahora que ocurre en el ejercicio.
		
		En primer lugar, debemos determinar los datos propios de $A^TA$ para obtener los valores y vectores singulares.
		$$A^TA = \begin{bmatrix}
		80 & 100 & 40 \\
		100 & 170 & 140\\
		40 & 140 & 200
		\end{bmatrix}$$
		Luego,
		$$|A^TA - \lambda I| = 0$$
		con lo que
		$$\lambda_1 = 360 \ra \sigma_1 = 6 \ \sqrt[]{10}$$
		$$\ \lambda_2 = 90  \ra \sigma_2 = 3 \ \sqrt[]{10}$$
		$$ \ \lambda_3 = 0  \ra \sigma_3 = 0$$
		Los vectores propios normalizados asociados son
		$$v_1 = \begin{pmatrix}
		\dfrac{1}{3}\\\\
		\dfrac{2}{3}\\\\
		\dfrac{2}{3}
		\end{pmatrix},
		v_2 = \begin{pmatrix}
		-\dfrac{2}{3}\\\\
		-\dfrac{1}{3}\\\\
		\dfrac{2}{3}
		\end{pmatrix},
		v_3 = \begin{pmatrix}
		\dfrac{2}{3}\\\\
		-\dfrac{2}{3}\\\\
		\dfrac{1}{3}
		\end{pmatrix}$$
		A continuación debemos armar las matrices de la descomposición.
		
		Pivoteando podemos notar que el rango de $A$ es 2, por lo que la matriz $D$ será
		$$D = \begin{bmatrix}
		\sigma_1 & 0\\
		0 & \sigma_2
		\end{bmatrix} 
		= \begin{bmatrix}
		6 \ \sqrt[]{10} & 0\\
		0 & 3 \ \sqrt[]{10}
		\end{bmatrix} $$
		Luego,
		$$\Sigma = \begin{bmatrix}
		6 \ \sqrt[]{10} & 0 & 0\\
		0 & 3 \ \sqrt[]{10} & 0
		\end{bmatrix}$$
		Con los vectores propios podemos determinar
		$$V = \begin{bmatrix}
		\dfrac{1}{3} & -\dfrac{2}{3} & \dfrac{2}{3} \\\\
		\dfrac{2}{3} & -\dfrac{1}{3} & -\dfrac{2}{3}  \\\\
		\dfrac{2}{3} & \dfrac{2}{3}  & \dfrac{1}{3} 
		\end{bmatrix}$$
		Para determinar $U$ necesitamos primero encontrar los vectores singulares izquierdos, esto es
		$$u_1 = \dfrac{1}{\sigma_1} A v_1 = \dfrac{1}{6 \ \sqrt[]{10}} \begin{pmatrix}
		18 \\ 6
		\end{pmatrix} = 
		\begin{pmatrix}
		\dfrac{3}{\sqrt[]{10}} \\\\
		\dfrac{1}{\sqrt[]{10}}
		\end{pmatrix}$$
		$$u_2 = \dfrac{1}{\sigma_2} A v_2 = \dfrac{1}{3 \ \sqrt[]{10}} \begin{pmatrix}
		3 \\ -9
		\end{pmatrix} = 
		\begin{pmatrix}
		\dfrac{1}{\sqrt[]{10}} \\\\
		\dfrac{-3}{\sqrt[]{10}}
		\end{pmatrix}$$
		Luego,
		$$U = \begin{bmatrix}
		\dfrac{3}{\sqrt[]{10}} & \dfrac{1}{\sqrt[]{10}} \\\\
		\dfrac{1}{\sqrt[]{10}} & \dfrac{-3}{\sqrt[]{10}}
		\end{bmatrix}$$
		Finalmente,
		$$A = U \Sigma V^T = \begin{bmatrix}
		\dfrac{3}{\sqrt[]{10}} & \dfrac{1}{\sqrt[]{10}} \\\\
		\dfrac{1}{\sqrt[]{10}} & \dfrac{-3}{\sqrt[]{10}}
		\end{bmatrix} 
		\begin{bmatrix}
		6 \ \sqrt[]{10} & 0 & 0\\
		0 & 3 \ \sqrt[]{10} & 0
		\end{bmatrix}
		\begin{bmatrix}
		\dfrac{1}{3} & \dfrac{2}{3} & \dfrac{2}{3} \\\\
		-\dfrac{2}{3} & -\dfrac{1}{3} & \dfrac{2}{3}  \\\\
		\dfrac{2}{3} & -\dfrac{2}{3}  & \dfrac{1}{3} 
		\end{bmatrix}
		$$
\end{solucion}
\item Sea la matriz
	$A = \begin{bmatrix}
	1 & 0 \\1 & 1 \\
	-1 & 1
	\end{bmatrix}$
	, determine su descomposición en valores singulares.
\begin{solucion}
En primer lugar, tenemos que
$$A^TA = \begin{bmatrix}
3 & 0 \\
0 & 2
\end{bmatrix}$$
Al buscar los valores y vectores propios y singulares de esta matriz, obtenemos
$$\lambda_1 = 3 \ra \sigma_1 = \sqrt[]{3}$$
$$\lambda_2 = 2, \ra \sigma_2 = \sqrt[]{2}$$
$$v_1 = \begin{pmatrix}
1 \\ 0
\end{pmatrix} \ra u_1 = \begin{pmatrix}
\frac{1}{\sqrt[]{3}} \\
\frac{1}{\sqrt[]{3}} \\
\frac{1}{\sqrt[]{3}}
\end{pmatrix} $$
$$v_1 = \begin{pmatrix}
0 \\ 1
\end{pmatrix} \ra u_2 = \begin{pmatrix}
0 \\
\frac{1}{\sqrt[]{2}} \\
-\frac{1}{\sqrt[]{2}}
\end{pmatrix}$$
Por lo tanto, tenemos que
$$V = \begin{bmatrix}
1 & 0 \\
0 & 1
\end{bmatrix}$$
$$D = \begin{bmatrix}
\sqrt[]{3} & 0 \\
0 & \sqrt[]{2}
\end{bmatrix} \ra \Sigma = \begin{bmatrix}
\sqrt[]{3} & 0 \\
0 & \sqrt[]{2}\\
0 & 0
\end{bmatrix}$$
Sin embargo, nos esta faltando un vector para armar $U$. Recordemos que $U$ debe ser ortogonal, por lo que podemos encontrer $u_3$ haciendo
$$u_3 = u_1 \times u_2 = \begin{pmatrix}
\frac{1}{\sqrt[]{3}} \\
\frac{1}{\sqrt[]{3}} \\
\frac{1}{\sqrt[]{3}}
\end{pmatrix} \times 
\begin{pmatrix}
0 \\
\frac{1}{\sqrt[]{2}} \\
-\frac{1}{\sqrt[]{2}}
\end{pmatrix} =
\begin{pmatrix}
-\frac{2}{\sqrt[]{6}} \\
\frac{1}{\sqrt[]{6}} \\
\frac{1}{\sqrt[]{6}}
\end{pmatrix}
$$
Luego,
$$U = \begin{bmatrix}
\frac{1}{\sqrt[]{3}} & 0 & -\frac{2}{\sqrt[]{6}}\\
\frac{1}{\sqrt[]{3}} & \frac{1}{\sqrt[]{2}} & \frac{1}{\sqrt[]{6}} \\
\frac{1}{\sqrt[]{3}} &-\frac{1}{\sqrt[]{2}} & \frac{1}{\sqrt[]{6}}
\end{bmatrix}$$
Finalmente,
$$A = U\Sigma V^T =
\begin{bmatrix}
\frac{1}{\sqrt[]{3}} & 0 & -\frac{2}{\sqrt[]{6}}\\
\frac{1}{\sqrt[]{3}} & \frac{1}{\sqrt[]{2}} & \frac{1}{\sqrt[]{6}} \\
\frac{1}{\sqrt[]{3}} &-\frac{1}{\sqrt[]{2}} & \frac{1}{\sqrt[]{6}}
\end{bmatrix}
\begin{bmatrix}
\sqrt[]{3} & 0 \\
0 & \sqrt[]{2}\\
0 & 0
\end{bmatrix}
\begin{bmatrix}
1 & 0 \\
0 & 1
\end{bmatrix}
$$
\end{solucion}
\item Sea $A$ una matriz de $5 \times 3$ tal que existe una matriz $C$ de $3 \times 5$ tal que $CA = I_3$ y sea $b \in R^5$ tal que la ecuación $Ax = b$ tiene solución.\\
Demuestra que $Ax = b$ tiene solución única.
\begin{solucion}
Digamos que $x_1$ y $x_2$ son dos soluciones de la ecuación $Ax = b$, es decir,
$$Ax_1 = b \qquad y \qquad Ax_2 = b$$
Por lo tanto,
$$Ax_1 = Ax_2$$
Multiplicando por C por la izquierda a ambos lados,
$$CAx_1 = CAx_2$$
$$Ix_1 = Ix_2$$
$$x_1 = x_2$$
Es decir, ambas soluciones son iguales. En otras palabras, $Ax = b$ tiene solución única.
\end{solucion}
\item Sea $A$ y $B$ dos matrices de $m\times n$ y $n \times p$, respectivamente. Demuestre que si las columnas de $B$ son linealmente dependientes, entonces también lo son las columnas de $AB$.
\begin{solucion}
P.D. que si $B$ es $L.D.$, $AB$ también lo es\\
		\\
		Si las columnas de $B$ son $L.D.$, entonces
		$$\exists u \neq \vec{0}\ ( Bu = \vec{0})$$
		Luego,
		$$(AB)u = A(Bu) = A\vec{0} = \vec{0}$$
		$$(AB)u = \vec{0}$$
		Recordemos que $u \neq \vec{0}$, por lo que hay una combinación lineal no trivial de las columnas de $AB$ que da $\vec{0}$. Entonces, las columnas de $AB$ son $L.D.$
\end{solucion}
\item Sea $T: \R^2 \ra \R^3$ la transformación lineal definida por
	$$ T\left(\begin{bmatrix}
	x_1\\
	x_2
	\end{bmatrix}\right) = \begin{bmatrix}
	5x_1 - 3x_2\\
	4x_1 -x_2\\
	2x_1 +3x_2
	\end{bmatrix}$$
	encuentre $x = \begin{bmatrix}
	x_1\\
	x_2
	\end{bmatrix}$ tal que $T(x) = \begin{bmatrix}
	13\\
	9\\
	1
	\end{bmatrix}$
\begin{solucion}
Notemos que la matriz de la transformación es
		$$ A = \begin{bmatrix}
		5 &- 3\\
		4& -1\\
		2& 3
		\end{bmatrix}$$
		Por que debemos resolver el sistema
		$$ \begin{bmatrix}
		5 &- 3\\
		4& -1\\
		2& 3
		\end{bmatrix}\begin{bmatrix}
		x_1\\
		x_2
		\end{bmatrix} = \begin{bmatrix}
		13\\
		9\\
		1
		\end{bmatrix}$$
		Viendolo en forma matricial,
		$$\left[
		\begin{array}{cc|c}
		5 &- 3 & 13\\
		4 & -1 & 9\\
		2 & 3 & 1
		\end{array}
		\right] \stackrel{F.E.R}{\wsim} \left[
		\begin{array}{cc|c}
		1 & 0 & 2\\
		0 & 1 & -1\\
		0 & 0 & 0
		\end{array}
		\right]$$
		Por lo que
		$$x_1 = 2, \quad x_2 = -1$$
		Finalmente,
		$$x = \begin{bmatrix}
		2 \\ -1
		\end{bmatrix}$$
\end{solucion}
\item Determine todas las matrices $A$ tales que
	$$ A\begin{bmatrix}
	3\\
	2
	\end{bmatrix}=\begin{bmatrix}
	2\\
	1
	\end{bmatrix}$$
\begin{solucion}
En primer lugar, notemos que dada las dimensiones de los vectores, la matriz $A$ es de $2 \times 2$. Luego,
		$$A = \begin{bmatrix}
		\alpha & \beta \\
		\gamma &\delta
		\end{bmatrix}$$
		Reemplazando,
		$$A\begin{bmatrix}
		3\\
		2
		\end{bmatrix}=\begin{bmatrix}
		2\\
		1
		\end{bmatrix} \Longrightarrow 
		\begin{bmatrix}
		\alpha & \beta \\
		\gamma &\delta
		\end{bmatrix}
		\begin{bmatrix}
			3\\
			2
		\end{bmatrix}=\begin{bmatrix}
			2\\
			1
		\end{bmatrix} \Longrightarrow
		\begin{array}{rc}
		3\alpha + 2\beta & = 2\\
		3\gamma + 2 \delta & = 1
		\end{array} \Longrightarrow
		\begin{array}{rc}
		3\alpha & = 2 - 2\beta\\
		3\gamma & = 1 - 2 \delta
		\end{array}$$
		Con lo que
		$$\alpha = \dfrac{2}{3} - \dfrac{2\beta}{3}$$
		$$\gamma = \dfrac{1}{3} - \dfrac{2\delta}{3}$$
		Finalmente,
		$$A = \begin{bmatrix}
		\frac{2}{3} - \frac{2\beta}{3} & \beta \\
		\frac{1}{3} - \frac{2\delta}{3} &\delta
		\end{bmatrix}$$
\end{solucion}
\item Determine condiciones sobre $\alpha$ y $\beta$ de modo que \\
	$\begin{pmatrix}
	\alpha-\beta\\
	\alpha + \beta\\
	2\alpha - \beta
	\end{pmatrix}$ pertenezca a $Gen\left\{\begin{pmatrix}
	-1\\
	-1\\
	1
	\end{pmatrix}\, \begin{pmatrix}
	1\\
	2\\
	1
	\end{pmatrix}, \begin{pmatrix}
	1\\
	3\\
	3
	\end{pmatrix}\right\}$
\begin{solucion}
Lo que debemos ver es para que valores de $\alpha$ y $\beta$, existe una combinación lineal de los vectores del generado que genera el vector dado. Esto equivale a ver para que valores de $\alpha$ y $\beta$, el siguiente sistema es consistente:
		$$
		\left[
		\begin{array}{ccc|c}
		-1 & 1 & 1 & \alpha - \beta\\
		-1 & 2 & 3 & \alpha + \beta\\
		1 & 1 & 3 & 2\alpha - \beta
		\end{array}
		\right] \stackbin[F_3+F_1]{F_2-F_1}{\wsim}
		\left[
		\begin{array}{ccc|c}
		-1 & 1 & 1 & \alpha - \beta\\
		0 & 1 & 2 & 2\beta\\
		0 & 0 & 0 & 3\alpha - 6\beta
		\end{array}	\right]$$
		Como la última fila se anula, el sistema será consistente cuando
		$$3\alpha - 6\beta = 0$$
		Es decir, la condición para $\alpha$ y $\beta$, tal que 
		$$\begin{pmatrix}
		\alpha-\beta\\
		\alpha + \beta\\
		2\alpha - \beta
		\end{pmatrix} \in Gen\left\{\begin{pmatrix}
		-1\\
		-1\\
		1
		\end{pmatrix}\, \begin{pmatrix}
		1\\
		2\\
		1
		\end{pmatrix}, \begin{pmatrix}
		1\\
		3\\
		3
		\end{pmatrix}\right\}$$
		es $\alpha = 2\beta$
\end{solucion}
\item Determine si el siguiente sistema de ecuaciones posee solución única, infinitas soluciones o no tiene solución
	$$
	\begin{array}{rcr}
	x+y-z& = & 1\\
	3x+2y+z& = & 1\\
	5x+3y+4z& = & 2\\
	-2x -y +5z & = & 6
	\end{array}$$
\begin{solucion}
Busquemos la forma escalonada de la matriz aumentada correspondiente al sistema
		$$
		\left[
		\begin{array}{ccc|c}
		1 & 1 & -1 & 1\\
		3 & 2 & 1 & 1\\
		5 & 3 & 4 & 2\\
		-2 & -1 & 5 & 6
		\end{array}
		\right] \wsim 
		\left[
		\begin{array}{ccc|c}
		1 & 1 & -1 & 1\\
		0 & -1 & 4 & 2\\
		0 & 0 & 1 & 1\\
		0 & 0 & 0 & -1
		\end{array}
		\right]$$
		Podemos ver que la última linea equivale a $0 = -1$, por lo que el sistema es inconsistente, es decir, no tiene solución.
\end{solucion}
\end{preguntas}
\end{document}