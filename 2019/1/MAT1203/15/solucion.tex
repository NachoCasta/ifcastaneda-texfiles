\documentclass[12pt]{article}

\usepackage{fullpage}
\usepackage{graphicx}
\usepackage{amssymb}
\usepackage{amsmath}
\usepackage[none]{hyphenat}
\usepackage{parskip}
\usepackage[spanish]{babel}
\usepackage[utf8]{inputenc}
\usepackage{hyperref}
\usepackage{fancyhdr}
\usepackage{tasks}
\usepackage{mdframed}
\usepackage{xcolor}
\usepackage{pgfplots}
\usepackage[makeroom]{cancel}
\usepackage{multicol}
\usepackage[shortlabels]{enumitem}
\usepackage{stackrel}
\usepackage{tkz-tab}
\usepackage{xpatch}
\usepackage{tkz-euclide}
\usetkzobj{all}
\xpatchcmd{\tkzTabLine}{$0$}{$\bullet$}{}{}

\setlength{\headheight}{10pt}
\setlength{\headsep}{10pt}
\pagestyle{fancy}
\rhead{\ayudantia \ - \alumno}
\tikzset{t style/.style={style=solid}}

\newcommand*{\mybox}[2]{\colorbox{#1!30}{\parbox{.98\linewidth}{#2}}}

\newenvironment{solucion}
{\begin{mdframed}[backgroundcolor=black!10]
		{\bf Solución:}\\
	}
	{
	\end{mdframed}
}

\newenvironment{alternativas}[1]
{\begin{multicols}{#1}
		\begin{enumerate}[a)]
		}
		{
		\end{enumerate}
	\end{multicols}
}

\newenvironment{preguntas}
{\begin{enumerate}\itemsep12pt
	}
	{
	\end{enumerate}
}

\newcommand{\ayudantia}{{\sc Ayudantía 15}}
\newcommand{\tituloayu}{Compilado Álgebra Lineal}
\newcommand{\fecha}{22 de junio de 2019}
\newcommand{\sigla}{MAT1203}
\newcommand{\nombre}{Álgebra Lineal}
\newcommand{\profesor}{Camilo Perez}
\newcommand{\ano}{2019}
\newcommand{\semestre}{1}
\newcommand{\mail}{mat1203@ifcastaneda.cl}
\newcommand{\alumno}{Ignacio Castañeda - \mail}

\newcommand{\ev}{\Big|}
\newcommand{\ra}{\rightarrow}
\newcommand{\lra}{\leftrightarrow}
\newcommand{\N}{\mathbb{N}}
\newcommand{\R}{\mathbb{R}}
\newcommand{\Exp}[1]{\mathcal{E}_{#1}}
\newcommand{\List}[1]{\mathcal{L}_{#1}}
\newcommand{\EN}{\Exp{\N}}
\newcommand{\LN}{\List{\N}}
\newcommand{\comment}[1]{}
\newcommand{\lb}{\\~\\}
\newcommand{\eop}{_{\square}}
\newcommand{\hsig}{\hat{\sigma}}
\newcommand{\widesim}[2][1.5]{
	\mathrel{\overset{#2}{\scalebox{#1}[1]{$\sim$}}}
}
\newcommand{\wsim}{\widesim{}}
\newcommand{\lh}{\stackrel{L'H}{=}}

\begin{document}
\thispagestyle{empty}

\begin{minipage}{2cm}
	\includegraphics[width=2cm]{../../../../img/logo.pdf}
	\vspace{0.5cm}
\end{minipage}
\begin{minipage}{\linewidth}
	\begin{tabular}{lrl}
		{\scriptsize\sc Pontificia Universidad Catolica de Chile} & \hspace*{0.7in}Curso: &
		\sigla  - \nombre\\
		{\sc Facultad de Matemáticas}&
		Profesor: & \profesor \\
		{\sc Semestre \ano-\semestre} & Ayudante: & {Ignacio Castañeda}\\
		& {Mail:} & \texttt{\mail}
	\end{tabular}
\end{minipage}

\vspace{-10mm}
\begin{center}
	{\LARGE\bf \ayudantia}\\
	\vspace{0.1cm}
	{\tituloayu}\\
	\vspace{0.1cm}
	\fecha\\
	\vspace{0.4cm}
\end{center}

\begin{preguntas}
\item Realiza las siguientes operaciones con los vectores dados
	$$
	v_1 = \begin{pmatrix}
	1\\
	7\\
	8
\end{pmatrix};\qquad
	v_2 = \begin{pmatrix}
	-4\\
	2\\
	-3
\end{pmatrix}; \qquad
	v_3 = \begin{pmatrix}
	0\\
	1\\
	3
\end{pmatrix}; \qquad
	v_4 = \begin{pmatrix}
	13\\
	-3\\
	1
\end{pmatrix},
	 $$
\begin{tasks}(4)
\task $v_1 + v_2$
\task $v_3 - v_4$
\task $v_2 \cdot v_3$
\task $v_1 \times v_2$
\task $v_2 \times v_1$
\task $(v_1 \times v_2) \cdot v_2$
\task $3 - ((2v_1) \cdot v_2)$
\task $v_2 - v_3 \times v_1$
\end{tasks}
\begin{solucion}

\begin{enumerate}[a)]
\item $v_1 + v_2 =\begin{pmatrix}
				1\\
				7\\
				8
				\end{pmatrix} + \begin{pmatrix}
				-4\\
				2\\
				-3
				\end{pmatrix} = \begin{pmatrix}
				-3\\
				9\\
				5
				\end{pmatrix}$
\item $v_3 - v_4 = \begin{pmatrix}
				0\\
				1\\
				3
				\end{pmatrix} - \begin{pmatrix}
				13\\
				-3\\
				1
				\end{pmatrix} =  \begin{pmatrix}
				-13\\
				4\\
				2
				\end{pmatrix}$
\item $v_2 \cdot v_3 = \begin{pmatrix}
				-4\\
				2\\
				-3
				\end{pmatrix} \cdot \begin{pmatrix}
				0\\
				1\\
				3
				\end{pmatrix} = -4 \cdot 0 + 2 \cdot 1 -3 \cdot 3 = -7$
\item $v_1 \times v_2 = \begin{pmatrix}
				1\\
				7\\
				8
			\end{pmatrix} \times \begin{pmatrix}
				-4\\
				2\\
				-3
			\end{pmatrix} = i \left| \begin{matrix} 7 & 8 \\ 2 & -3\end{matrix} \right| - j \left| \begin{matrix} 1 & 8 \\ 4 & -3\end{matrix} \right| + k \left| \begin{matrix} 1 & 7 \\ -4 & 2\end{matrix} \right|$\\
			$$= i (7 \cdot (-3) - 8 \cdot 2) - j (1 \cdot (-3) - 8 \cdot (-4)) + k (1 \cdot 2 - 7 \cdot(-4))$$
			$$= i (-21 -16)) - j(-3--32) + k(2 - - 28)$$
			$$-37 i - 29j + 30k = \begin{pmatrix}
			-37\\
			-29\\
			30
			\end{pmatrix}$$	
\item $v_2 \times v_1 =  \begin{pmatrix}
			-4\\
			2\\
			-3
			\end{pmatrix} \times \begin{pmatrix}
			1\\
			7\\
			8
			\end{pmatrix} = i \left| \begin{matrix} 2 & -3 \\ 7 & 8\end{matrix} \right| - j \left| \begin{matrix} 4 & -3 \\ 1 & 8\end{matrix} \right| + k \left| \begin{matrix} -4 & 2 \\ 1 & 7\end{matrix} \right|$\\
			$$= i (2 \cdot 8 - (-3) \cdot 7) - j ((-4)\cdot 8 - (-3)\cdot 1) + k ((-4) \cdot 7 -2 \cdot 1)$$
			$$= i (16 --21) + j(-32 --3) + k(-28-2)$$
			$$37 i + 29j - 30k = \begin{pmatrix}
			37\\
			29\\
			-30
			\end{pmatrix}$$	
\item $(v_1 \times v_2) \cdot v_2 = \begin{pmatrix}
			-37\\
			-29\\
			30
			\end{pmatrix} \cdot \begin{pmatrix}
			-4\\
			2\\
			-3
			\end{pmatrix} = (-37) \cdot (-4) + (-29) \cdot 2 + 30 \cdot (-3)$
			$$=148 +-58 - 90 = 0$$
\item $3 - ((2v_1) \cdot v_2) = 3 - \left(\left(2\begin{pmatrix}
			1\\
			7\\
			8
			\end{pmatrix}\right) \cdot \begin{pmatrix}
			-4\\
			2\\
			-3
			\end{pmatrix}\right) = 3 - \left(\begin{pmatrix}
			2\\
			14\\
			16
			\end{pmatrix} \cdot \begin{pmatrix}
			-4\\
			2\\
			-3
			\end{pmatrix}\right)$
			$$ = 3 - (2 \cdot (-4) + 14 \cdot 2 + 16 \cdot (-3))$$
			$$3 - (-8 + 28 -48)$$
\item $v_2 - v_3 \times v_1 = \begin{pmatrix}
			-4\\
			2\\
			-3
			\end{pmatrix} -  \begin{pmatrix}
			0\\
			1\\
			3
			\end{pmatrix} \times  \begin{pmatrix}
			1\\
			7\\
			8
			\end{pmatrix}$
			$$= -4i + 2j -3k - (i (8 - 21))- j(0 - 3) + k (0-1))$$
			$$ = -4i + 2j -3k +13i - 3j -k = 9i - j -4k $$
			$$ =\begin{pmatrix}
			9\\
			-4\\
			-2
			\end{pmatrix}$$
\end{enumerate}
\end{solucion}
\item Verificar si los siguientes puntos son o no colineares entre si
\begin{tasks}(2)
\task $P(3,2,5), \ P(0, 1/2, -1), \ P(5, 3, 9)$
\task $P(1,1,4), \ P(-2,-3.-8), \ P(4,4,16)$
\end{tasks}
\begin{solucion}

\begin{enumerate}[a)]
\item $ 
			\begin{pmatrix}
			3\\
			2\\
			5
			\end{pmatrix}; \quad
			\begin{pmatrix}
			0\\
			1/2\\
			-1
			\end{pmatrix}; \quad
			\begin{pmatrix}
			5\\
			3\\
			9
			\end{pmatrix}$\\
			En primer lugar, debemos encontrar dos vectores directores entre dos pares de puntos distintos
			$$d_1 = \begin{pmatrix}
			3\\
			2\\
			5
			\end{pmatrix} - 
			\begin{pmatrix}
			0\\
			1/2\\
			-1
			\end{pmatrix} = \begin{pmatrix}
			3\\
			3/2\\
			6
			\end{pmatrix}$$
			$$d_2 = \begin{pmatrix}
			3\\
			2\\
			5
			\end{pmatrix} - 
			\begin{pmatrix}
			5\\
			3\\
			9
			\end{pmatrix} = \begin{pmatrix}
			-2\\
			-1\\
			-4
			\end{pmatrix}$$
			Dado que es posible representar $d_1$ de la forma $d_1 = \lambda d_2$, con $\lambda = -1,5$, los puntos son colineales.
\item $ 
			\begin{pmatrix}
			1\\
			1\\
			4
			\end{pmatrix}; \quad
			\begin{pmatrix}
			-2\\
			-3\\
			-8
			\end{pmatrix}; \quad
			\begin{pmatrix}
			4\\
			4\\
			16
			\end{pmatrix}$\\
			Igual que antes, buscaremos dos vectores directores
			$$d_1 = \begin{pmatrix}
			1\\
			1\\
			4
			\end{pmatrix} - 
			\begin{pmatrix}
			-2\\
			-3\\
			-8
			\end{pmatrix} = \begin{pmatrix}
			3\\
			4\\
			12
			\end{pmatrix}$$
			$$d_2 = \begin{pmatrix}
			1\\
			1\\
			4
			\end{pmatrix} - 
			\begin{pmatrix}
			4\\
			4\\
			16
			\end{pmatrix} = \begin{pmatrix}
			-3\\
			-3\\
			-12
			\end{pmatrix}$$
			Observando las dos primeras componentes de cada vector, vemos que es imposible representar uno en función del otro, por lo que los puntos no son colineales
\end{enumerate}
\end{solucion}
\item Sean los vectores
		$$
	v_1 = \begin{pmatrix}
	2\\
	1\\
	5
\end{pmatrix};\qquad
	v_2 = \begin{pmatrix}
	-1\\
	-3\\
	0
\end{pmatrix}$$
\begin{enumerate}[a)]
\item Buscar un vector $v_3$ que sea perpendicular a $v_1$ y $v_2$ y verificar que efectivamente sea perpendicular a ambos.
\item Buscar un vector paralelo a $(v_2-v_3)$ y verificar que lo sea
\item Encontrar un vector perpendicular a $(v_3 + 3v_1)$
\end{enumerate}
\begin{solucion}

\begin{enumerate}[a)]
\item Para que sea perpendicular a ambos, basta definir $v_3 = v_1 \times v_2$, es decir
			$$v_3 = \begin{pmatrix}
			2\\
			1\\
			5
			\end{pmatrix} \times \begin{pmatrix}
			-1\\
			-3\\
			0
			\end{pmatrix} = i (0 + 15) - j (0 + 5) + k(-6 + 1) =\begin{pmatrix}
			15\\
			-5\\
			-5
			\end{pmatrix} $$
			Para verificar basta utilizar el producto punto, donde obtendremos que
			$$v_3 \cdot v_1 = 0$$
			$$v_3 \cdot v_2 = 0$$
\item En primer lugar, calculemos el vector $v_2 - v_3$
			$$v_4 = v_2 - v_3 = \begin{pmatrix}
			-1\\
			-3\\
			0
			\end{pmatrix} - \begin{pmatrix}
			15\\
			-5\\
			-5
			\end{pmatrix} = \begin{pmatrix}
			-16\\
			2\\
			5
			\end{pmatrix}$$
			Para obtener un vector paralelo, podemos simplemente ponderar el vector por un escalar, como por ejemplo, 2
			$$v_5 = 2 v_4 = 2 \begin{pmatrix}
			-16\\
			2\\
			5
			\end{pmatrix} = \begin{pmatrix}
			-32\\
			4\\
			10
			\end{pmatrix}$$
			Para verificar utilizaremos el producto cruz, donde obtendremos que
			$$v_5 \times v_4 = 0$$
\item Primero calculemos el vector $v_3 + 3v_1$
			$$v_6 = v_3 + 3v_1 = \begin{pmatrix}
			15\\
			-5\\
			-5
			\end{pmatrix}  + 3 \begin{pmatrix}
			2\\
			1\\
			5
			\end{pmatrix} = \begin{pmatrix}
			21\\
			-2\\
			10
			\end{pmatrix} $$
			Para obtener un vector perpendicular, podemos hacer producto cruz con cualquier otro vector arbitrario. Para hacer los calculos simples, utilizaremos
			$$v_7 = \begin{pmatrix}
			1\\
			0\\
			0
			\end{pmatrix} $$
			Luego, 
			$$v_8 = v_6 \times v_7 = \begin{pmatrix}
			21\\
			-2\\
			10
			\end{pmatrix} \times \begin{pmatrix}
			1\\
			0\\
			0
			\end{pmatrix} = 10j + 2k = \begin{pmatrix}
			0\\
			10\\
			2
			\end{pmatrix}$$
\end{enumerate}
\end{solucion}
\item Detereminar si los siguientes vectores son paralelos, perpendiculares u oblicuos
\begin{tasks}(2)
\task $$ 
			\begin{pmatrix}
			-1\\
			2\\
			5
		\end{pmatrix};\qquad
			\begin{pmatrix}
			2\\
			-4\\
			-10
		\end{pmatrix}$$
\task  $$ 
			\begin{pmatrix}
			2\\
			5\\
			10
		\end{pmatrix};\qquad
			\begin{pmatrix}
			-3\\
			8\\
			-11
		\end{pmatrix}$$
\end{tasks}
\begin{solucion}

\begin{enumerate}[a)]
\item $$ 
			\begin{pmatrix}
			-1\\
			2\\
			5
			\end{pmatrix};\qquad
			\begin{pmatrix}
			2\\
			-4\\
			-10
			\end{pmatrix}$$
			Si se fijan, es evidente que $v_2 = -2v_1$, por lo que comprobaremos de inmediato con el producto cruz para ver si son paralelos
			 $$ 
			\begin{pmatrix}
			-1\\
			2\\
			5
			\end{pmatrix} \times 
			\begin{pmatrix}
			2\\
			-4\\
			-10
			\end{pmatrix} = i (-20 + 20) -j (10 - 10) + k (4 - 4) = \vec{0}$$
			Por lo que ambos vectores son paralelos
\item $$ 
			\begin{pmatrix}
			2\\
			5\\
			10
			\end{pmatrix};\qquad
			\begin{pmatrix}
			-3\\
			8\\
			-11
			\end{pmatrix}$$
			Comenzaremo probando si son perpendiculares con el producto punto
			$$ 
			\begin{pmatrix}
			2\\
			5\\
			10
			\end{pmatrix} \cdot
			\begin{pmatrix}
			-3\\
			8\\
			-11
			\end{pmatrix} = -6 + 40 - 110 = -76$$
			Como es distinto de cero, vemos que no son perpendiculares.\\
			Ahora, utilizaremos el producto cruz para ver si son paralelos
			$$ 
			\begin{pmatrix}
			2\\
			5\\
			10
			\end{pmatrix} \times
			\begin{pmatrix}
			-3\\
			8\\
			-11
			\end{pmatrix} = i(-55 -80) - j (-22 +30) +k(16 + 15) = \begin{pmatrix}
			-135\\
			-8\\
			31
			\end{pmatrix}$$
			Como este no es el vector nulo, tampoco son paralelos. Con esto concluimos que ambos vectores son oblicuos.
\end{enumerate}
\end{solucion}
\item Determinar el plano que pasa por los puntos $P_1(4,-1,-2)$, $P_2(0,0,1)$ y $P_3(2,-3,0)$.
\begin{solucion}
Sabemos que un plano se define de la forma
		$$Ax +By + Cz = D$$
		Reemplazando esto con los tres puntos que tenemos, obtendremos el siguiente sistema de ecuaciones
		$$
		\begin{array}{rcrr}
		4A-B-2C & = & D& \vline\\
		C & = & D & \vline\\
		2A-3B & = & D &\vline\\
		\hline
		\end{array}
		$$
		$$
		\begin{array}{rcrr}
		4A-B-2C & = & C& \vline\\
		2A-3B & = & C &\vline\\
		\hline
		\end{array}
		$$
		$$
		\begin{array}{rcrr}
		4A-B-3C & = & 0& \vline\\
		2A-3B -C& = & 0 &\vline\\
		\hline
		\end{array}
		$$
		$$(1) - 2(2)$$
		$$5B - C = 0 \ra 5B = C$$
		Como hay 4 variables y teníamos 3 ecuaciones, debemos elegir el valor de una de ellas de manera arbitraria.
		$$C = 5$$
		$$\ra D = 5$$
		$$\ra B = 1$$
		$$4A -1 -15 = 0 \ra 4A=16 \ra A=4$$
		Finalmente, el plano es
		$$\Pi:4x + y + 5z = 5$$
\end{solucion}
\item Dado $P_1(0,2,-3)$ y $P_2(1,0,3)$, determinar la recta que pasa por $P_1$ y $P_2$.
\begin{solucion}
Buscamos un vector director
		$$\vec{d} = P_2 - P_1 = (1, -2, 0)$$
		Luego, la recta que pasa por $P_1$ y $P_2$ esta definida por
		$$<x,y,z> = (0, 2, -3) + \lambda (1,-2,0)$$
\end{solucion}
\item Encontrar las ecuaciones de dos planos diferentes cuya intersección sea la recta que pasa por el punto $P_1(1,3,-2)$ y $P_2(2,0,4)$.
\begin{solucion}
Para realizar esto, basta con utilizar dos trios de puntos que contengan a los dos puntos dados y buscar dos planos. La intersección entre estos planos será la recta que pasa por ambos puntos. Para esto, definiremos puntos arbitrarios.\\
		Diremos que
		$$P_3(0, 0, 0), \quad P_4(1,0,0)$$
		Partamos usando los puntos $P_1$, $P_2$ y $P_3$. Sabemos que la ecuación del plano es de la forma
		$$Ax + By + Cz = D$$
		Reemplazando con estos tres puntos, obtenemos el siguiente sistema
		$$
		\begin{array}{rcrr}
		A + 3B - 2C & = & D& \vline\\
		2A + 4C & = & D & \vline\\
		0 & = & D &\vline\\
		\hline
		\end{array}
		$$
		$$
		\begin{array}{rcrr}
		A + 3B - 2C & = & 0& \vline\\
		2A + 4C & = & 0 & \vline\\
		\hline
		\end{array}
		$$
		$$(2) + 2(1)$$
		$$4A + 6B = 0 \ra A = -\dfrac{6B}{4} \ra A = -\dfrac{3B}{2}$$
		Le pondremos un valor arbitrario a $B$,
		$$B = 4$$
		$$\ra A = -6$$
		$$\ra -12 + 4C = 0 \ra C = 3$$
		Por ende,
		$$\Pi_1: -6x + 4y + 3z = 0$$
		Ahora realizamos lo mismo con los puntos $P_1$, $P_2$, $P_4$. Sea la ecuación del plano
		$$Ax + By + Cz = D$$
		Reemplazando con estos tres puntos, obtenemos el siguiente sistema
		$$
		\begin{array}{rcrr}
		A + 3B - 2C & = & D& \vline\\
		2A + 4C & = & D & \vline\\
		A & = & D &\vline\\
		\hline
		\end{array}
		$$
		$$
		\begin{array}{rcrr}
		3B - 2C & = & 0& \vline\\
		A + 4C & = & 0 & \vline\\
		\hline
		\end{array}
		$$
		$$
		\begin{array}{rcrr}
		3B & = & 2C& \vline\\
		A & = & -4C & \vline\\
		\hline
		\end{array}
		$$
		Le daremos un valor arbitrario a $C$
		$$C = 3$$
		$$ \ra B = 2$$
		$$ \ra A = -12 $$
		$$ \ra D = -12 $$
		Con lo que obtenemos el plano
		$$\Pi_2: -12x+2y+3z = -12$$
		Finalmente, los planos buscados son:
		$$\Pi_1: -6x + 4y + 3z = 0$$		
		$$\Pi_2: -12x+2y+3z = -12$$
\end{solucion}
\item Encontrar un plano que sea perpendicular al plano cuya ecuación es $3x -7y +2z = 5$ y que pase por el punto $P(0,2,-1)$
\begin{solucion}
En primer lugar, debemos encontrar el vector normal del plano, que esta dado por los coeficientes de su ecuación, es decir
		$$\vec{n} = \begin{pmatrix}
		3\\-7\\2
		\end{pmatrix}$$
		Si los vectores normales de dos planos son perpendiculares entre si, los planos también lo serán, por lo que basta encontrar un vector perpendicular a $\vec{n}$ y utilizarlo como vector normal para nuestro nuevo plano. Para encontrar un vector perpendicular, realizaremos el producto cruz entre $\vec{n}$ y un vector arbitrario elegido por nosotros
		$$\begin{pmatrix} 3 \\ -7 \\ 2 \end{pmatrix} \times \begin{pmatrix} 1 \\ 1 \\ 1 \end{pmatrix} = \begin{pmatrix} -9 \\ -1 \\ 10 \end{pmatrix}$$
		Para encontrar el plano asociado, basta que utilicemos los coeficientes del vector como coeficientes de la ecuación del plano
		$$-9x -y +10z = D$$
		Por último, para encontrar $D$, reemplazamos el punto que nos dan en la ecuación del plano
		$$-2 -10 = D \ra D = -12$$
		Finalmente,
		$$\Pi: -9x -y +10z = -12$$
\end{solucion}
\item Determinar la ecuación de un plano que pase por $P(4,2,1)$ y que sea paralelo al plano de ecuación $2x-5y+z=6$
\begin{solucion}
De forma análoga, para que dos planos sean paralelos, basta con que sus vectores normales sean paralelos. Dicho de otra forma, lo que va a cambiar será el parametro $D$. Es decir, nuestro plano es
		$$2x - 5y + z = D$$
		Reemplazando con el punto,
		$$8-10+1 = D \ra D = -1$$
		Luego, el plano buscado es
		$$\Pi: 2x - 5y + z = -1$$
\end{solucion}
\item 
\begin{enumerate}[a)]
\item Sean $a=\left(\begin{array}{r}
      1\\-1\\1
    \end{array}\right)$ y $b=\left(\begin{array}{r}
      0\\-1\\2
    \end{array}\right)$, calcule el vector\textbf{ $b-proy_a b$ }y verifique que este es ortogonal a $a$.
\item Demuestre que si $a$ y $b\in\R^3$ el vector $b-proy_a b$ es ortogonal a $a$.
\end{enumerate}
\begin{solucion}

\begin{enumerate}[a)]
\item La proyección de $b$ sobre $a$ corresponde a
$$proy_{a}b = \dfrac{b \cdot a}{|a|^2}a =   \dfrac{\left(\begin{array}{r}
	0\\-1\\2
	\end{array}\right) \cdot\left(\begin{array}{r}
		1\\-1\\1
	\end{array}\right)}{\left(\begin{array}{r}
		1\\-1\\1
	\end{array}\right) \cdot \left(\begin{array}{r}
		1\\-1\\1
	\end{array}\right)}\left(\begin{array}{r}
		1\\-1\\1
	\end{array}\right) = \left(\begin{array}{r}
1\\-1\\1
\end{array}\right)$$
Luego, 
$$b-proy_a b = \left(\begin{array}{r}
0\\-1\\2
\end{array}\right) - \left(\begin{array}{r}
1\\-1\\1
\end{array}\right) = \left(\begin{array}{r}
-1\\0\\1
\end{array}\right)$$
Podemos verificar que el resultado es perpendicular a $a$ utilizando el producto punto, esto es
$$\left(\begin{array}{r}
1\\-1\\1
\end{array}\right) \cdot \left(\begin{array}{r}
-1\\0\\1
\end{array}\right) = 0$$
\item Para demostrar esto, debemos demostrar que
$$(b-proy_a b) \cdot a = 0 $$
Desarrollando,
$$\left(b-\dfrac{b \cdot a}{|a|^2}a\right) \cdot a = 0 $$
$$b \cdot a -\dfrac{b \cdot a}{|a|^2}a \cdot a = 0 $$
$$b \cdot a -\dfrac{b \cdot a}{|a|^2}|a|^2 = 0 $$
$$b \cdot a -b \cdot a = 0 $$
$$0 = 0$$
$$\blacksquare$$
\end{enumerate}
\end{solucion}
\end{preguntas}
\end{document}