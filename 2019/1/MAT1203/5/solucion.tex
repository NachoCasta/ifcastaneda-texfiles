\documentclass[12pt]{article}

\usepackage{fullpage}
\usepackage{graphicx}
\usepackage{amssymb}
\usepackage{amsmath}
\usepackage[none]{hyphenat}
\usepackage{parskip}
\usepackage[spanish]{babel}
\usepackage[utf8]{inputenc}
\usepackage{hyperref}
\usepackage{fancyhdr}
\usepackage{tasks}
\usepackage{mdframed}
\usepackage{xcolor}
\usepackage{pgfplots}
\usepackage[makeroom]{cancel}
\usepackage{multicol}
\usepackage[shortlabels]{enumitem}
\usepackage{stackrel}

\setlength{\headheight}{10pt}
\setlength{\headsep}{10pt}
\pagestyle{fancy}
\rhead{\ayudantia \ - \alumno}

\newcommand*{\mybox}[2]{\colorbox{#1!30}{\parbox{.98\linewidth}{#2}}}

\newenvironment{solucion}
{\begin{mdframed}[backgroundcolor=black!10]
		{\bf Solución:}\\
	}
	{
	\end{mdframed}
}

\newenvironment{alternativas}[1]
{\begin{multicols}{#1}
		\begin{enumerate}[a)]
		}
		{
		\end{enumerate}
	\end{multicols}
}

\newenvironment{preguntas}
{\begin{enumerate}\itemsep12pt
	}
	{
	\end{enumerate}
}

\newcommand{\ayudantia}{{\sc Ayudantía 5}}
\newcommand{\tituloayu}{Repaso I1}
\newcommand{\fecha}{11 de abril de 2019}
\newcommand{\sigla}{MAT1203}
\newcommand{\nombre}{Álgebra Lineal}
\newcommand{\profesor}{Camilo Perez}
\newcommand{\ano}{2019}
\newcommand{\semestre}{1}
\newcommand{\mail}{mat1203@ifcastaneda.cl}
\newcommand{\alumno}{Ignacio Castañeda - \mail}

\newcommand{\ev}{\Big|}
\newcommand{\ra}{\rightarrow}
\newcommand{\lra}{\leftrightarrow}
\newcommand{\N}{\mathbb{N}}
\newcommand{\R}{\mathbb{R}}
\newcommand{\Exp}[1]{\mathcal{E}_{#1}}
\newcommand{\List}[1]{\mathcal{L}_{#1}}
\newcommand{\EN}{\Exp{\N}}
\newcommand{\LN}{\List{\N}}
\newcommand{\comment}[1]{}
\newcommand{\lb}{\\~\\}
\newcommand{\eop}{_{\square}}
\newcommand{\hsig}{\hat{\sigma}}
\newcommand{\widesim}[2][1.5]{
	\mathrel{\overset{#2}{\scalebox{#1}[1]{$\sim$}}}
}
\newcommand{\wsim}{\widesim{}}

\begin{document}
\thispagestyle{empty}

\begin{minipage}{2cm}
	\includegraphics[width=2cm]{../../../../img/logo.pdf}
	\vspace{0.5cm}
\end{minipage}
\begin{minipage}{\linewidth}
	\begin{tabular}{lrl}
		{\scriptsize\sc Pontificia Universidad Catolica de Chile} & \hspace*{0.7in}Curso: &
		\sigla  - \nombre\\
		{\sc Facultad de Matemáticas}&
		Profesor: & \profesor \\
		{\sc Semestre \ano-\semestre} & Ayudante: & {Ignacio Castañeda}\\
		& {Mail:} & \texttt{\mail}
	\end{tabular}
\end{minipage}

\vspace{-10mm}
\begin{center}
	{\LARGE\bf \ayudantia}\\
	\vspace{0.1cm}
	{\tituloayu}\\
	\vspace{0.1cm}
	\fecha\\
	\vspace{0.4cm}
\end{center}

\begin{preguntas}
\item Demuestre que el conjunto $\{u, v, w\}$ es L.I. si y solo si el conjunto $\{u+v, u+w, v+w\}$ es L.I.
\begin{solucion}
Como nos dicen si y solo si, debemos demostrar en ambas direcciones
		$$(\Longrightarrow)$$
		Digamos que $\{u, v, w\}$ es L.I.\\
		\\
		P.D. $\{u+v, u+w, v+w\}$ es L.I.\\
		\\
		Para que $\{u+v, u+w, v+w\}$ sea L.I., el sistema
		$$\alpha (u+v) + \beta (u+w) + \gamma (v+w) = 0$$
		debe tener solución única $\alpha = 0$, $\beta = 0$, $\gamma = 0$
		Reordenando,
		$$\alpha (u+v) + \beta (u+w) + \gamma (v+w) = 0$$
		$$(\alpha + \beta) u + (\alpha + \gamma) v + (\beta + \gamma) w = 0$$
		Como sabemos que el conjunto $\{u, v, w\}$ es L.I., entonces tenemos que
		$$\begin{array}{rl}
		\alpha + \beta & = 0\\
		\alpha + \gamma & = 0\\
		\beta + \gamma & = 0
		\end{array}$$
		Este sistema tiene por solución
		$$\alpha = 0, \quad \beta = 0, \quad \gamma = 0$$
		Con lo que demostramos que $\{u+v, u+w, v+w\}$ es L.I.
		$$(\Longleftarrow)$$
		Digamos que $\{u+v, u+w, v+w\}$ es L.I.\\
		\\
		P.D. $\{u, v, w\}$ es L.I.\\
		\\
		Para que $\{u, v, w\}$ sea L.I., el sistema
		$$c_1u + c_2v + c_3w$$
		debe tener solución única $c_1 = 0$, $c_2 = 0$, $c_3 = 0$\\
		\\
		Como $\{u+v, u+w, v+w\}$ es L.I., sabemos que el sistema
		$$\alpha (u+v) + \beta (u+w) + \gamma (v+w) = 0$$
		tiene solución única $\alpha = 0$, $\beta = 0$, $\gamma = 0$
		Reordenando,
		$$\alpha (u+v) + \beta (u+w) + \gamma (v+w) = 0$$
		$$(\alpha + \beta) u + (\alpha + \gamma) v + (\beta + \gamma) w = 0$$
		Con lo que tenemos que
		$$c_1 = \alpha + \beta = 0, \quad c_2 = \alpha + \gamma = 0, \quad c_3 = \beta + \gamma = 0$$
		es la única solución del sistema. \\
		\\
		Luego, $\{u, v, w\}$ es L.I.
		$$q.e.d$$
\end{solucion}
\item Sea $A$ una matriz de $4 \times 4$ tal que $A=\begin{bmatrix}a_1 & 2a_1 & a_2 & a_1-a_2\end{bmatrix}$ con $a_1, a_2 \in \R^4$ vectores linealmente independientes, y sea $B$ una matriz inyectiva de $4 \times 2$ tal que $A \cdot B = 0$. Determine una matriz con las características de $B$.
\begin{solucion}
Del enunciado sabemos que $B$ es inyectiva, por lo que todas sus columnas deben ser $L.I.$\\

Además, sabemos que $A\cdot B= 0$. En otras palabras, esto significa que ambas columnas de $B$ deben combinar linealmente las columnas de $A$ de tal forma que se anulen entre si y a la vez ambas columnas de $B$ deben ser $L.I.$\\

Busquemos en primer lugar, un vector que anule las columnas de $A$. Sea
$$b = \begin{pmatrix}
b_1 \\ b_2 \\ b_3 \\ b_4
\end{pmatrix}$$
Debemos encontrar
$$Ab = 0$$
Esto es,
$$a_1b_1 + 2a_1 b_2 + a_2b_3 + (a_1-a_2)b_4 = 0$$
Reordenando,
$$(b_1 + 2b_2 + b_4)a_1 + (b_3 - b_4)a_2 = 0$$
Como $a_1$ y $a_2$ son $L.I.$, la única forma de solucionar esto es
$$\begin{array}{rcl}
b_1 + 2b_2 + b_4 & = & 0\\
b_3 - b_4 & = & 0
\end{array} 
\Longrightarrow
\begin{array}{rcl}
b_1 & = & -2b_2 - b_3\\
b_2 & = & b_2\\
b_3 & = & b_3\\
b_4 & = & b_3
\end{array}
\Longrightarrow
b = b_2\begin{pmatrix}
-2 \\ 1 \\ 0 \\ 0
\end{pmatrix} 
+
b_3\begin{pmatrix}
-1 \\ 0 \\ 1 \\ 1
\end{pmatrix}
$$
Luego, todos los vectores de esta forma cumplirán con $Ab = 0$. Como necesitamos dos vectores $L.I.$ que cumplan con esto, basta con tomar los dos vectores generadores de $b$ como las columnas de $B$, es decir
$$B = \begin{bmatrix}
-2 & -1\\
1 & 0\\
0 & 1 \\
0 & 1
\end{bmatrix}$$
\end{solucion}
\item Determinar las condiciones en $a$ para que la matriz
	$$A = \begin{bmatrix}
	a & 2a & 0 & 0 \\
	0 & 1 & 0 & 3a-1\\
	0 & 1 & a-1 & 2a-1\\
	a & 2a & 0 & a
	\end{bmatrix}$$
	sea invertible
\begin{solucion}
Recordemos que el que una matriz sea invertible es equivalente a que esta tenga solución única.\\\\En primer lugar, debemos pivotear $A$ para dejarla en su forma escalonada:
		$$\begin{bmatrix}
		a & 2a & 0 & 0 \\
		0 & 1 & 0 & 3a-1\\
		0 & 1 & a-1 & 2a-1\\
		a & 2a & 0 & a
		\end{bmatrix} \widesim{F_4-F_1}
		\begin{bmatrix}
		a & 2a & 0 & 0 \\
		0 & 1 & 0 & 3a-1\\
		0 & 1 & a-1 & 2a-1\\
		0 & 0 & 0 & a
		\end{bmatrix} \widesim{F_3-F_2}
		\begin{bmatrix}
		a & 2a & 0 & 0 \\
		0 & 1 & 0 & 3a-1\\
		0 & 0 & a-1 & -a\\
		0 & 0 & 0 & a
		\end{bmatrix}$$
		Notemos que la última fila nos puede dar problemas. Si $a=0$, el sistema va a tener solo 3 pivotes por lo que, en caso de ser consistente, tendría infinitas soluciones. Dicho esto, una restricción es $a\neq 0$
		
		En la tercera fila, debemos fijarnos que $a-1 \neq 0$, ya que de lo contrario, esta fila sería un múltiplo de la cuarta fila, por lo que habrían 3 pivotes y tendríamos soluciones infinitas. Dicho esto, tenemos que $a \neq 1$.
		
		De esta forma, el sistema siempre tendrá 4 pivotes y por lo tanto, tendrá solución única.
		
		En resumen, las condiciones son
		$$a \neq 0, \quad a \neq 1$$
\end{solucion}
\item Sea $T: \R^7 \ra \R^9$ una transformación lineal. Suponga que el conjunto $\{u,v\}$ es linealmente independiente en $\R^7$ pero $\{T(u), T(v)\}$ es linealmente dependiente. Demuestre que la ecuación $T(x)=0$ admite soluciones no triviales.
\begin{solucion}
Como el conjunto $\{T(u), T(v)\}$ es $LD$, existen $\alpha, \beta \neq 0$, tales que 
$$\alpha T(u) + \beta T(v) = 0$$
Usando la propiedad de linealidad para las transformaciones lineales, también tenemos que
$$ T(\alpha u +\beta v) = 0$$
Por otro lado, como $\{u, v\}$ es $LI$ y $\alpha, \beta \neq 0$, el vector $\alpha u +\beta v$ no puede ser cero. Por ende, se concluye que $\alpha u +\beta v$ es una solución no trivial de $T(x) = 0$.
\end{solucion}
\item Determine si las siguientes afirmaciones son verdaderas o falsas. En caso de ser verdaderas demuéstrelas y si son falsas de un contraejemplo.
\begin{enumerate}[a)]
\item Si $\{v_1, v_2, v_3\}$ es un conjunto linealmente dependiente de vectores en $\R^3$ entonces el conjunto $\{v_1, v_2\}$ también es linealmente dependiente.
\item Si $A$ es una matriz de $3 \times 3$ entonces la imagen del plano $x+y+z=0$ bajo la transformación $T(x)=Ax$ es un plano.
\item Si $A_{2\times 2} = (a_{ij}), \quad B_{2\times 3} = (b_{ij}), \quad AB = C = (c_{ij})$ tal que
$$a_{ij} = (-2)^{i+j}, \quad b_{ij} = (-3)^{i-j}$$
entonces $c_{23} = -\dfrac{56}{9}$
\end{enumerate}
\begin{solucion}

\begin{enumerate}[a)]
\item \textbf{Falso.}\\
\\
Un contraejemplo sería
$\left\{ 
\begin{pmatrix}
	1 \\ 0 \\ 0
\end{pmatrix},
\begin{pmatrix}
	0 \\ 1 \\ 0
\end{pmatrix},
\begin{pmatrix}
	1 \\ 1 \\ 0
\end{pmatrix}
\right\}$.\\
Como
$
\begin{pmatrix}
1 \\ 1 \\ 0
\end{pmatrix}
=
\begin{pmatrix}
1 \\ 0 \\ 0
\end{pmatrix}
+
\begin{pmatrix}
0 \\ 1 \\ 0
\end{pmatrix}$ el conjunto claramente es $LD$.\\
Sin embargo,
$\left\{ 
\begin{pmatrix}
1 \\ 0 \\ 0
\end{pmatrix},
\begin{pmatrix}
0 \\ 1 \\ 0
\end{pmatrix}
\right\}$ 
es $LI$, lo que contradice el enunciado.
\item  \textbf{Falso.}\\
\\
Un contraejemplo es tomar la matriz cero, ya que la imagen de cualquier conjunto por esta matriz es 0.
\item  \textbf{Verdadero.}\\
\\
$c_{23}$ corresponde al coeficiente de la matriz $C$ resultante del producto punto entre la segunda fila de la matriz $A$ y la tercera columna de la matriz $B$.\\

Luego,
$$c_{23} = \begin{pmatrix}
(-2)^{2+1}\\ (-2)^{2+2}
\end{pmatrix} \cdot
\begin{pmatrix}
(-3)^{1-3}\\ (-3)^{2-3}
\end{pmatrix} = 
\begin{pmatrix}
-8\\ 16
\end{pmatrix} \cdot
\begin{pmatrix}
\frac{1}{9}\\ -\frac{1}{3}
\end{pmatrix}=
-8 \cdot \dfrac{1}{9} + 16 \cdot -\dfrac{1}{3} = -\dfrac{56}{9}
$$
\end{enumerate}
\end{solucion}
\item Sea $\mathbb{P} \subset \R^3$ el plano de ecuación $x_1-x_2+1=0$ y sea $T:\R^3 \ra \R^4$ la transformación dada por $T(x) = Ax$, donde
$$A = \begin{bmatrix}
1 & 0 & 3\\
-1 & 1 & 0 \\
0 & 1 & 3\\
0 & -2 & -6
\end{bmatrix}$$
Demuestre que la imagen del plano $\mathbb{P}$ bajo la transformación $T$ es una recta en $\R^4$.
\begin{solucion}
Despejando $x_2$ en la ecuación del plano, obtenemos $x_2 = x_1 + 1$.\\

Reemplazando en la transformación lineal,
{\small $$T(x) = T(x_1, x_2, x_3) = T(x_1, x_1+1, x_3) = \begin{bmatrix}
1 & 0 & 3\\
-1 & 1 & 0 \\
0 & 1 & 3\\
0 & -2 & -6
\end{bmatrix}
\begin{pmatrix}
x_1 \\
x_1+1\\
x_3
\end{pmatrix} = 
\begin{pmatrix}
x_1 +3x_3\\
1\\
1+x_1+3x_3\\
-2-2x_1-6x_3
\end{pmatrix}$$}
Esto lo podemos descomponer en
$$T(x) = 
\begin{pmatrix}
x_1 +3x_3\\
1\\
1+x_1+3x_3\\
-2-2x_1-6x_3
\end{pmatrix} =
\begin{pmatrix}
0\\
1\\
1\\
-2
\end{pmatrix} +
x_1 \begin{pmatrix}
1\\
0\\
1\\
-2
\end{pmatrix} +
x_3 \begin{pmatrix}
3\\
0\\
3\\
-6
\end{pmatrix}
$$
Notemos ahora que el vector asociado a $x_3$ es el triple del vector asociado a $x_1$, por lo que podemos factorizarlo de la siguiente forma
$$T(x) = 
\begin{pmatrix}
0\\
1\\
1\\
-2
\end{pmatrix} +
x_1 \begin{pmatrix}
1\\
0\\
1\\
-2
\end{pmatrix} +
3x_3 \begin{pmatrix}
1\\
0\\
1\\
-2
\end{pmatrix} = 
\begin{pmatrix}
0\\
1\\
1\\
-2
\end{pmatrix} +
(x_1 + 3x_3) \begin{pmatrix}
1\\
0\\
1\\
-2
\end{pmatrix}
$$
{\small Esto corresponde a la ecuación de una recta con dirección $\begin{pmatrix}
1\\
0\\
1\\
-2
\end{pmatrix}$ y que pasa por $\begin{pmatrix}
0\\
1\\
1\\
-2
\end{pmatrix}$.}
\end{solucion}
\end{preguntas}
\end{document}