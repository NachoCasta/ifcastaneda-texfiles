\documentclass[12pt]{article}

\usepackage{fullpage}
\usepackage{graphicx}
\usepackage{amssymb}
\usepackage{amsmath}
\usepackage[none]{hyphenat}
\usepackage{parskip}
\usepackage[spanish]{babel}
\usepackage[utf8]{inputenc}
\usepackage{hyperref}
\usepackage{fancyhdr}
\usepackage{tasks}
\usepackage{mdframed}
\usepackage{xcolor}
\usepackage{pgfplots}
\usepackage[makeroom]{cancel}
\usepackage{multicol}
\usepackage[shortlabels]{enumitem}
\usepackage{stackrel}

\setlength{\headheight}{10pt}
\setlength{\headsep}{10pt}
\pagestyle{fancy}
\rhead{\ayudantia \ - \alumno}

\newcommand*{\mybox}[2]{\colorbox{#1!30}{\parbox{.98\linewidth}{#2}}}

\newenvironment{solucion}
{\begin{mdframed}[backgroundcolor=black!10]
		{\bf Solución:}\\
	}
	{
	\end{mdframed}
}

\newenvironment{alternativas}[1]
{\begin{multicols}{#1}
		\begin{enumerate}[a)]
		}
		{
		\end{enumerate}
	\end{multicols}
}

\newenvironment{preguntas}
{\begin{enumerate}\itemsep12pt
	}
	{
	\end{enumerate}
}

\newcommand{\ayudantia}{{\sc Ayudantía 5}}
\newcommand{\tituloayu}{Repaso I1}
\newcommand{\fecha}{11 de abril de 2019}
\newcommand{\sigla}{MAT1203}
\newcommand{\nombre}{Álgebra Lineal}
\newcommand{\profesor}{Camilo Perez}
\newcommand{\ano}{2019}
\newcommand{\semestre}{1}
\newcommand{\mail}{mat1203@ifcastaneda.cl}
\newcommand{\alumno}{Ignacio Castañeda - \mail}

\newcommand{\ev}{\Big|}
\newcommand{\ra}{\rightarrow}
\newcommand{\lra}{\leftrightarrow}
\newcommand{\N}{\mathbb{N}}
\newcommand{\R}{\mathbb{R}}
\newcommand{\Exp}[1]{\mathcal{E}_{#1}}
\newcommand{\List}[1]{\mathcal{L}_{#1}}
\newcommand{\EN}{\Exp{\N}}
\newcommand{\LN}{\List{\N}}
\newcommand{\comment}[1]{}
\newcommand{\lb}{\\~\\}
\newcommand{\eop}{_{\square}}
\newcommand{\hsig}{\hat{\sigma}}
\newcommand{\widesim}[2][1.5]{
	\mathrel{\overset{#2}{\scalebox{#1}[1]{$\sim$}}}
}
\newcommand{\wsim}{\widesim{}}

\begin{document}
\thispagestyle{empty}

\begin{minipage}{2cm}
	\includegraphics[width=2cm]{../../../../img/logo.pdf}
	\vspace{0.5cm}
\end{minipage}
\begin{minipage}{\linewidth}
	\begin{tabular}{lrl}
		{\scriptsize\sc Pontificia Universidad Catolica de Chile} & \hspace*{0.7in}Curso: &
		\sigla  - \nombre\\
		{\sc Facultad de Matemáticas}&
		Profesor: & \profesor \\
		{\sc Semestre \ano-\semestre} & Ayudante: & {Ignacio Castañeda}\\
		& {Mail:} & \texttt{\mail}
	\end{tabular}
\end{minipage}

\vspace{-10mm}
\begin{center}
	{\LARGE\bf \ayudantia}\\
	\vspace{0.1cm}
	{\tituloayu}\\
	\vspace{0.1cm}
	\fecha\\
	\vspace{0.4cm}
\end{center}

\begin{preguntas}
\item Demuestre que el conjunto $\{u, v, w\}$ es L.I. si y solo si el conjunto $\{u+v, u+w, v+w\}$ es L.I.
\item Sea $A$ una matriz de $4 \times 4$ tal que $A=\begin{bmatrix}a_1 & 2a_1 & a_2 & a_1-a_2\end{bmatrix}$ con $a_1, a_2 \in \R^4$ vectores linealmente independientes, y sea $B$ una matriz inyectiva de $4 \times 2$ tal que $A \cdot B = 0$. Determine una matriz con las características de $B$.
\item Determinar las condiciones en $a$ para que la matriz
	$$A = \begin{bmatrix}
	a & 2a & 0 & 0 \\
	0 & 1 & 0 & 3a-1\\
	0 & 1 & a-1 & 2a-1\\
	a & 2a & 0 & a
	\end{bmatrix}$$
	sea invertible
\item Sea $T: \R^7 \ra \R^9$ una transformación lineal. Suponga que el conjunto $\{u,v\}$ es linealmente independiente en $\R^7$ pero $\{T(u), T(v)\}$ es linealmente dependiente. Demuestre que la ecuación $T(x)=0$ admite soluciones no triviales.
\item Sea $\mathbb{P} \subset \R^3$ el plano de ecuación $x_1-x_2+1=0$ y sea $T:\R^3 \ra \R^4$ la transformación dada por $T(x) = Ax$, donde
$$A = \begin{bmatrix}
1 & 0 & 3\\
-1 & 1 & 0 \\
0 & 1 & 3\\
0 & -2 & -6
\end{bmatrix}$$
Demuestre que la imagen del plano $\mathbb{P}$ bajo la transformación $T$ es una recta en $\R^4$.
\begin{enumerate}[a)]
\item 
\end{enumerate}
\item Determine si las siguientes afirmaciones son verdaderas o falsas. En caso de ser verdaderas demuéstrelas y si son falsas de un contraejemplo.
\begin{enumerate}[a)]
\item Si $\{v_1, v_2, v_3\}$ es un conjunto linealmente dependiente de vectores en $\R^3$ entonces el conjunto $\{v_1, v_2\}$ también es linealmente dependiente.
\item Si $A$ es una matriz de $3 \times 3$ entonces la imagen del plano $x+y+z=0$ bajo la transformación $T(x)=Ax$ es un plano.
\item Si $A_{2\times 2} = (a_{ij}), \quad B_{2\times 3} = (b_{ij}), \quad AB = C = (c_{ij})$ tal que
$$a_{ij} = (-2)^{i+j}, \quad b_{ij} = (-3)^{i-j}$$
entonces $c_{23} = -\dfrac{56}{9}$
\end{enumerate}
\end{preguntas}
\end{document}