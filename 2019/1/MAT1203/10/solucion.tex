\documentclass[12pt]{article}

\usepackage{fullpage}
\usepackage{graphicx}
\usepackage{amssymb}
\usepackage{amsmath}
\usepackage[none]{hyphenat}
\usepackage{parskip}
\usepackage[spanish]{babel}
\usepackage[utf8]{inputenc}
\usepackage{hyperref}
\usepackage{fancyhdr}
\usepackage{tasks}
\usepackage{mdframed}
\usepackage{xcolor}
\usepackage{pgfplots}
\usepackage[makeroom]{cancel}
\usepackage{multicol}
\usepackage[shortlabels]{enumitem}
\usepackage{stackrel}
\usepackage{tkz-tab}
\usepackage{xpatch}
\xpatchcmd{\tkzTabLine}{$0$}{$\bullet$}{}{}

\setlength{\headheight}{10pt}
\setlength{\headsep}{10pt}
\pagestyle{fancy}
\rhead{\ayudantia \ - \alumno}
\tikzset{t style/.style={style=solid}}

\newcommand*{\mybox}[2]{\colorbox{#1!30}{\parbox{.98\linewidth}{#2}}}

\newenvironment{solucion}
{\begin{mdframed}[backgroundcolor=black!10]
		{\bf Solución:}\\
	}
	{
	\end{mdframed}
}

\newenvironment{alternativas}[1]
{\begin{multicols}{#1}
		\begin{enumerate}[a)]
		}
		{
		\end{enumerate}
	\end{multicols}
}

\newenvironment{preguntas}
{\begin{enumerate}\itemsep12pt
	}
	{
	\end{enumerate}
}

\newcommand{\ayudantia}{{\sc Ayudantía 10}}
\newcommand{\tituloayu}{Cambio de base y vectores propios}
\newcommand{\fecha}{16 de mayo de 2019}
\newcommand{\sigla}{MAT1203}
\newcommand{\nombre}{Álgebra Lineal}
\newcommand{\profesor}{Camilo Perez}
\newcommand{\ano}{2019}
\newcommand{\semestre}{1}
\newcommand{\mail}{mat1203@ifcastaneda.cl}
\newcommand{\alumno}{Ignacio Castañeda - \mail}

\newcommand{\ev}{\Big|}
\newcommand{\ra}{\rightarrow}
\newcommand{\lra}{\leftrightarrow}
\newcommand{\N}{\mathbb{N}}
\newcommand{\R}{\mathbb{R}}
\newcommand{\Exp}[1]{\mathcal{E}_{#1}}
\newcommand{\List}[1]{\mathcal{L}_{#1}}
\newcommand{\EN}{\Exp{\N}}
\newcommand{\LN}{\List{\N}}
\newcommand{\comment}[1]{}
\newcommand{\lb}{\\~\\}
\newcommand{\eop}{_{\square}}
\newcommand{\hsig}{\hat{\sigma}}
\newcommand{\widesim}[2][1.5]{
	\mathrel{\overset{#2}{\scalebox{#1}[1]{$\sim$}}}
}
\newcommand{\wsim}{\widesim{}}
\newcommand{\lh}{\stackrel{L'H}{=}}

\begin{document}
\thispagestyle{empty}

\begin{minipage}{2cm}
	\includegraphics[width=2cm]{../../../../img/logo.pdf}
	\vspace{0.5cm}
\end{minipage}
\begin{minipage}{\linewidth}
	\begin{tabular}{lrl}
		{\scriptsize\sc Pontificia Universidad Catolica de Chile} & \hspace*{0.7in}Curso: &
		\sigla  - \nombre\\
		{\sc Facultad de Matemáticas}&
		Profesor: & \profesor \\
		{\sc Semestre \ano-\semestre} & Ayudante: & {Ignacio Castañeda}\\
		& {Mail:} & \texttt{\mail}
	\end{tabular}
\end{minipage}

\vspace{-10mm}
\begin{center}
	{\LARGE\bf \ayudantia}\\
	\vspace{0.1cm}
	{\tituloayu}\\
	\vspace{0.1cm}
	\fecha\\
	\vspace{0.4cm}
\end{center}

\begin{preguntas}
\item Sean $B_1 = \{1+x,1-x,1+x^2\}$ y $B_2$ bases de $P_2(\R)$ tales que para todo $p \in P_2(\R)$
	$$ \begin{bmatrix}
	1 & 0 & -1\\
	1 & 1 & 0 \\
	0 & 1 & 2
	\end{bmatrix} [p]_{B_1} = [p]_{B_2}$$
	Determine los polinomios que forman la base $B_2$.
\begin{solucion}
Notemos que la matriz que nos dan corresponde a la matriz cambio de base de $B_1$ a $B_2$. Recordemos que una matriz cambio de base, de $B_1$ a $B_2$, corresponde a la matriz formada por las coordenadas de los elementos de $B_1$ en la base $B_2$.
		
		En base a lo anterior, dado que tenemos los vectores de $B_1$, lo que necesitamos es justamente lo contrario, es decir, la matriz cambio de base de $B_2$ a $B_1$. De esta manera, tendriamos las coordenadas de los elementos de $B_2$ en la base $B_1$, por lo que podriamos calcularlos con facilidad, ya que tenemos estos últimos.
		
		Para encontrar esta matriz, basta con obtener la inversa de la matriz cambio de base dada, esto es
		$$ \left[\begin{array}{ccc|ccc}
		1 & 0 & -1 & 1 & 0 & 0\\
		1 & 1 & 0 & 0 & 1 & 0\\
		0 & 1 & 2 & 0 & 0 & 1
		\end{array}\right] \sim 
		\left[\begin{array}{ccc|ccc}
		1 & 0 & 0 & 2 & -1 & 1\\
		0 & 1 & 0 & -2 & 2 & -1\\
		0 & 0 & 1 & 1 & -1 & 1
		\end{array}\right]$$
		con lo que
		$$A_{B_2}^{B_1} = \begin{bmatrix}
		2 & -1 & 1\\
		-2 & 2 & -1\\
		1 & -1 & 1
		\end{bmatrix}$$
		Sea 
		$$B_2 = \{q_1(x), q_2(x), q_3(x)\}$$
		Las columnas de $A_{B_2}^{B_1}$ corresponden a los vectores de coordenada de $q_1$, $q_2$ y $q_3$, respectivamente, en la base $B_1$.
		
		Luego,
		$$q_1(x) = 2(1+x) - 2(1-x) + (1+x^2) = 1+4x+x^2$$
		$$q_2(x) = -(1+x) + 2(1-x) - (1+x^2) = -3x-x^2$$
		$$q_3(x) = (1+x) - (1-x) + (1+x^2) = 1+2x+x^2$$
		Finalmente,
		$$B_2 = \{1+4x+x^2,-3x-x^2,1+2x+x^2\}$$
\end{solucion}
\item Sean $B$ y $C$ bases de un espacio vectorial $V$ y $P = \begin{bmatrix}4 &-1 \\ 6 & -1\end{bmatrix}$ la matriz de cambio de coordenadas tal que $[v]_C = P[v]_B\ \forall v \in V$
\begin{enumerate}[a)]
\item Demuestre que el conjunto $W = \{v \in V: [v]_C = 2[v]_B\}$ es un subespacio de $V$.
\item Si $B=\{v_1,v_2\}$ determine una base para $W$ en términos de la base de $B$.
\end{enumerate}
\begin{solucion}

\begin{enumerate}[a)]
\item Demuestre que el conjunto $W = \{v \in V: [v]_C = 2[v]_B\}$ es subespacio de $V$.
			
			\begin{enumerate}
				\item No vacio $(0\in W)$
				$$\begin{array}{cc}
				[\vec{0}]_C = \vec{0}\\
				2 \cdot [\vec{0}]_B = 2 \cdot \vec{0} = 0
				\end{array} \Longrightarrow \vec{0} = \vec{0} \ra \vec{0} \in W$$	
				\item Suma y multiplicación (a la vez)	
				
				
				Sea $u, v \in W$ y $\alpha, \beta \in \R$, debemos demostrar que $\alpha u + \beta v \in W$
				$$\begin{array}{rcl}
				[\alpha u + \beta v]_C & = & 2 [\alpha u + \beta v]_B \\
				& = & 2 [\alpha u + \beta v]_B \\
				& = & 2 [\alpha u]_B + 2[\beta v]_B \\
				& = & \alpha 2[u]_B + \beta 2[v]_B \\
				& = & \alpha [u]_C + \beta [v]_C \\
				& = & [\alpha u]_C + [\beta v]_C \\
				{[}\alpha u + \beta v]_C & = & [\alpha u +\beta v]_C
				\end{array}
				$$
			\end{enumerate}
			$$\blacksquare$$
\item Si $B=\{v_1,v_2\}$ determine una base para $W$ en términos de la base de $B$.
			
			Sea $B = \{v_1, v_2\}$ la base dada, busquemos una base $B_W$ de $W$\\
			\\
			Sea $v \in W$,
			$$v = x_1v_1 + x_2v_2 \ra x = \begin{pmatrix} x_1 \\ x_2 \end{pmatrix} = [v]_B$$
			Además,
			$$\begin{array}{cclcl}
			[v]_C & = & 2[v]_B & = & 2x \\
			{[}v]_C & = & P[v]_B & = & Px
			\end{array} \Longrightarrow
			2x = Px$$
			Luego,
			$$2x = Px$$
			$$Px - 2x = 0$$
			$$(P - 2I)x = 0$$
			$$\left(\begin{bmatrix}4 &-1 \\ 6 & -1\end{bmatrix} - \begin{bmatrix}2 &0 \\ 0 & 2\end{bmatrix}\right)x = 0$$
			$$\begin{bmatrix}2 &-1 \\ 6 & -3\end{bmatrix}x = 0$$
			Resolviendo el sistema,
			$$\begin{bmatrix}2 &-1 \\ 6 & -3\end{bmatrix} \sim 
			\begin{bmatrix}2 &-1 \\ 0 & 0\end{bmatrix} \ra x_2 = 2x_1$$
			Luego,
			$$v = x_1v_1 + 2x_1v_2 = x_1(v_1 + 2v_2)$$
			Finalmente,
			$$B_W = \{v_1 + 2v_2\}$$
\end{enumerate}
\end{solucion}
\item Calcular los valores y vectores propios de
	$$ A = \begin{bmatrix} 1 & 2 & -1\\ 1 & 0 & 1\\ 4 & -4 & 5\end{bmatrix}$$
\begin{solucion}
Para calcular los valores propios de una matriz, debemos buscar todos los valores de $\lambda$ tales que
		$$A-\lambda I = 0$$
		Luego, para cada valor propio obtenido, debemos resolver el problema
		$$(A-\lambda I)v = 0$$
		Encontrando así los valores propios.\\
		\\
		Pasando ahora al ejercicio,
		$$\left|\begin{bmatrix}
		1-\lambda & 2 & -1\\
		1 & -\lambda & 1\\
		4 & -4 & 5-\lambda 
		\end{bmatrix}\right| = 0$$
		Usando cofactores, tenemos que
		$$\left|\begin{bmatrix}
		1-\lambda & 2 & -1\\
		1 & -\lambda & 1\\
		4 & -4 & 5-\lambda 
		\end{bmatrix}\right| = (1-\lambda)(-\lambda(5-\lambda) - -4) - 2((5-\lambda)-4) -(-4 - -4\lambda)$$
		Simplificando,
		$$=-\lambda^3 + 6\lambda^2 -11\lambda + 6 = -(\lambda - 1)(\lambda - 2) (\lambda - 3)$$
		Por lo que los valores propios son
		$$\lambda_1 =  1, \quad \lambda_2 = 2, \quad \lambda_3 = 3$$
		Busquemos ahora los vectores propios asociados a cada valor propio
		\begin{enumerate}[1)]
			\item $\lambda_1 = 1$\\
			\\
			Debemos resolver el sistema $(A-I)v = 0$, esto es
			\small$$\begin{bmatrix}
			0 & 2 & -1\\
			1 & -1 & 1\\
			4 & -4 & 4
			\end{bmatrix} \sim 
			\begin{bmatrix}
			1 & -1 & 1\\
			0 & 2 & -1\\
			4 & -4 & 4
			\end{bmatrix} \sim 
			\begin{bmatrix}
			1 & -1 & 1\\
			0 & 2 & -1\\
			0 & 0 & 0
			\end{bmatrix} \sim 
			\begin{bmatrix}
			1 & -1 & 1\\
			0 & 1 & -\frac{1}{2}\\
			0 & 0 & 0
			\end{bmatrix} \sim 
			\begin{bmatrix}
			1 & 0 & \frac{1}{2}\\
			0 & 1 & -\frac{1}{2}\\
			0 & 0 & 0
			\end{bmatrix}$$
			Por lo que
			$$\begin{array}{rcl}
			x_1 & = & -\dfrac{1}{2}x_3\\
			x_2 & = & \dfrac{1}{2}x_3\\
			x_3 & = & x_3
			\end{array} \Longrightarrow
			v_1 = \begin{pmatrix}
			-\frac{1}{2}\\
			\frac{1}{2}\\
			1
			\end{pmatrix} \rightarrow
			v_1 = \begin{pmatrix}
			-1\\
			1\\
			2
			\end{pmatrix}$$
			La solución del sistema es el generado de $v$. Por esto, lo simplificamos al final para obtener un vector propio sin fracciones.
			
			\item $\lambda_2 = 2$\\
			\\
			Debemos resolver el sistema $(A-2I)v = 0$, esto es
			\small$$\begin{bmatrix}
			-1 & 2 & -1\\
			1 & -2 & 1\\
			4 & -4 & 3
			\end{bmatrix} \sim 
			\begin{bmatrix}
			1 & 0 & \frac{1}{2}\\
			0 & 1 & -\frac{1}{4}\\
			0 & 0 & 0
			\end{bmatrix}$$
			Por lo que
			$$\begin{array}{rcl}
			x_1 & = & -\dfrac{1}{2}x_3\\
			x_2 & = & \dfrac{1}{4}x_3\\
			x_3 & = & x_3
			\end{array} \Longrightarrow
			v_2 = \begin{pmatrix}
			-\frac{1}{2}\\
			\frac{1}{4}\\
			1
			\end{pmatrix} \rightarrow
			v_2 = \begin{pmatrix}
			-2\\
			1\\
			4
			\end{pmatrix}$$
			
			\item $\lambda_3 = 3$\\
			\\
			Debemos resolver el sistema $(A-2I)v = 0$, esto es
			\small$$\begin{bmatrix}
			-2 & 2 & -1\\
			1 & -3 & 1\\
			4 & -4 & 2
			\end{bmatrix} \sim 
			\begin{bmatrix}
			1 & 0 & \frac{1}{4}\\
			0 & 1 & -\frac{1}{4}\\
			0 & 0 & 0
			\end{bmatrix}$$
			Por lo que
			$$\begin{array}{rcl}
			x_1 & = & -\dfrac{1}{4}x_3\\
			x_2 & = & \dfrac{1}{4}x_3\\
			x_3 & = & x_3
			\end{array} \Longrightarrow
			v_3 = \begin{pmatrix}
			-\frac{1}{4}\\
			\frac{1}{4}\\
			1
			\end{pmatrix} \rightarrow
			v_3 = \begin{pmatrix}
			-1\\
			1\\
			4
			\end{pmatrix}$$
		\end{enumerate}
		Finalmente, los vectores propios son
		$$v_1 = \begin{pmatrix}
		-1\\
		1\\
		2
		\end{pmatrix}, \quad
		v_2 = \begin{pmatrix}
		-2\\
		1\\
		4
		\end{pmatrix}, \quad
		v_3 = \begin{pmatrix}
		-1\\
		1\\
		4
		\end{pmatrix}$$
\end{solucion}
\item Determine si las siguientes afirmaciones son Verdaderas o Falsas.
\begin{enumerate}[a)]
\item El espacio fila de $AB$ es subespacio del espacio fila de $B$.
\item Dada una base en un espacio vectorial $V$, el vector coordenado de un vector de $V$ con respecto a esa base, es único.
\item Si $A$ y $B$ son similares y $\lambda$ es valor propio de $A$ entonces $\lambda$ es también valor propio de $B$.
\end{enumerate}
\begin{solucion}

\begin{enumerate}[a)]
\item El espacio fila de $AB$ es subespacio del espacio fila de $B$.
			
			$$C = AB  \ra C^T = B^TA^T \ra Col(C^T) = Col(B^TA^T)$$
			Como al multiplicar una matriz por otra podemos, en el mejor de los casos, obtener la misma dimensión de antes,
			$$Col(C^T) \subset Col(B^T) \ra Fil(C) \subset Fil(B)$$
			Luego,
			$$Fil(AB) \subset Fil(B)$$
			Con lo que concluimos que el espacio fila de $AB$ es subespacio del espacio fila de $B$. Entonces, la afirmación es {\bf Verdadera}.
\item Dada una base en un espacio vectorial $V$, el vector coordenado de un vector de $V$ con respecto a esa base, es único.
			
			Los vectores de una base son $L.I.$, por lo que existe una única combinación lineal para generar cada vector. Luego, la afirmación es {\bf Verdadera}.
\item Si $A$ y $B$ son similares y $\lambda$ es valor propio de $A$ entonces $\lambda$ es también valor propio de $B$.
			
			Tomemos, las matrices
			$$A = \begin{bmatrix}
			1 & 0 \\ 0 & 1
			\end{bmatrix}, \quad 
			B = \begin{bmatrix}
			2 & 0 \\ 0 & 2
			\end{bmatrix}$$
			Los valores propios de $A$ son $\lambda = 1$ (multiplicidad 2) y los de $B$ son $\lambda = 2$ (multiplicidad 2), por lo que la afirmación es {\bf Falsa}.
\end{enumerate}
\end{solucion}
\end{preguntas}
\end{document}