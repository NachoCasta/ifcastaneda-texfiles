\documentclass[12pt]{article}

\usepackage{fullpage}
\usepackage{graphicx}
\usepackage{amssymb}
\usepackage{amsmath}
\usepackage[none]{hyphenat}
\usepackage{parskip}
\usepackage[spanish]{babel}
\usepackage[utf8]{inputenc}
\usepackage{hyperref}
\usepackage{fancyhdr}
\usepackage{tasks}
\usepackage{mdframed}
\usepackage{xcolor}
\usepackage{pgfplots}
\usepackage[makeroom]{cancel}
\usepackage{multicol}
\usepackage[shortlabels]{enumitem}
\usepackage{stackrel}
\usepackage{tkz-tab}
\usepackage{xpatch}
\xpatchcmd{\tkzTabLine}{$0$}{$\bullet$}{}{}

\setlength{\headheight}{10pt}
\setlength{\headsep}{10pt}
\pagestyle{fancy}
\rhead{\ayudantia \ - \alumno}
\tikzset{t style/.style={style=solid}}

\newcommand*{\mybox}[2]{\colorbox{#1!30}{\parbox{.98\linewidth}{#2}}}

\newenvironment{solucion}
{\begin{mdframed}[backgroundcolor=black!10]
		{\bf Solución:}\\
	}
	{
	\end{mdframed}
}

\newenvironment{alternativas}[1]
{\begin{multicols}{#1}
		\begin{enumerate}[a)]
		}
		{
		\end{enumerate}
	\end{multicols}
}

\newenvironment{preguntas}
{\begin{enumerate}\itemsep12pt
	}
	{
	\end{enumerate}
}

\newcommand{\ayudantia}{{\sc Ayudantía 6}}
\newcommand{\tituloayu}{Descomposiciones y determinantes}
\newcommand{\fecha}{18 de abril de 2019}
\newcommand{\sigla}{MAT1203}
\newcommand{\nombre}{Álgebra Lineal}
\newcommand{\profesor}{Camilo Perez}
\newcommand{\ano}{2019}
\newcommand{\semestre}{1}
\newcommand{\mail}{mat1203@ifcastaneda.cl}
\newcommand{\alumno}{Ignacio Castañeda - \mail}

\newcommand{\ev}{\Big|}
\newcommand{\ra}{\rightarrow}
\newcommand{\lra}{\leftrightarrow}
\newcommand{\N}{\mathbb{N}}
\newcommand{\R}{\mathbb{R}}
\newcommand{\Exp}[1]{\mathcal{E}_{#1}}
\newcommand{\List}[1]{\mathcal{L}_{#1}}
\newcommand{\EN}{\Exp{\N}}
\newcommand{\LN}{\List{\N}}
\newcommand{\comment}[1]{}
\newcommand{\lb}{\\~\\}
\newcommand{\eop}{_{\square}}
\newcommand{\hsig}{\hat{\sigma}}
\newcommand{\widesim}[2][1.5]{
	\mathrel{\overset{#2}{\scalebox{#1}[1]{$\sim$}}}
}
\newcommand{\wsim}{\widesim{}}
\newcommand{\lh}{\stackrel{L'H}{=}}

\begin{document}
\thispagestyle{empty}

\begin{minipage}{2cm}
	\includegraphics[width=2cm]{../../../../img/logo.pdf}
	\vspace{0.5cm}
\end{minipage}
\begin{minipage}{\linewidth}
	\begin{tabular}{lrl}
		{\scriptsize\sc Pontificia Universidad Catolica de Chile} & \hspace*{0.7in}Curso: &
		\sigla  - \nombre\\
		{\sc Facultad de Matemáticas}&
		Profesor: & \profesor \\
		{\sc Semestre \ano-\semestre} & Ayudante: & {Ignacio Castañeda}\\
		& {Mail:} & \texttt{\mail}
	\end{tabular}
\end{minipage}

\vspace{-10mm}
\begin{center}
	{\LARGE\bf \ayudantia}\\
	\vspace{0.1cm}
	{\tituloayu}\\
	\vspace{0.1cm}
	\fecha\\
	\vspace{0.4cm}
\end{center}

\begin{preguntas}
\item Encuentre una factorziación $A = LU$ de
	$$A = \begin{bmatrix}
	2 & 4 & -1 & 5 & -2 \\
	-4 & -5 & 3 & -8 & 1 \\
	2 & -5 & -4 & 1 & 8 \\
	-6 & 0 & 7 & -3 & 1
	\end{bmatrix}$$
\item Sea $A = \begin{bmatrix}
	1 & 2 & 3\\
	2 & 8 & 4 \\
	3 & 4 & 11\end{bmatrix}$. Calcule la factorización $A=LDL^T$ de la matriz $A$ y en base a esto encuentre una matriz $R$ tal que $A = RR^T$.
\item Sea una descomposción $PA = LU$ donde
	$$P = \begin{bmatrix}
	0 & 0 & 1\\
	0 & 1 & 0\\
	1 & 0 & 0
	\end{bmatrix}, \quad
	L = \begin{bmatrix}
	1 & 0 & 0\\
	-1 & 1 & 0\\
	0 & 1 & 1
	\end{bmatrix}, \quad
	U = \begin{bmatrix}
	2 & 1 & 0\\
	0 & 1 & 1\\
	0 & 0 & -1
	\end{bmatrix}$$
	determine usando directamente esta factorización la tercera fila de $A^{-1}$.
\item Sea $A$ una matriz de $n \times n$ definida por
	$$a_{i,j} = 
	\begin{cases}
	3 \quad si\ i\leq j \\
	1 \quad si\ i > j
	\end{cases}$$
	Calcule $det(A)$.
\item Sean $v_1,v_2,v_3,v_4 \in \R^4$, encuentre el $|A|$ si\\
$A=[v_1-v_3+v_4 \;\;\;\;-v_2-v_3 \;\;\;\; v_3-v_1
\;\;\;\;v_1+v_2+2v_4 ]\in M_4 (\R)$
\item Sea $A \in M_4 (\R)$ tal que $|A|=5$, encuentre
\begin{tasks}(4)
\task $|2^3A|$
\task $|A^5|$
\task $|-A|$
\task $||A|A^{-1}|$
\end{tasks}
\end{preguntas}
\end{document}