\documentclass[12pt]{article}

\usepackage{fullpage}
\usepackage{graphicx}
\usepackage{amssymb}
\usepackage{amsmath}
\usepackage[none]{hyphenat}
\usepackage{parskip}
\usepackage[spanish]{babel}
\usepackage[utf8]{inputenc}
\usepackage{hyperref}
\usepackage{fancyhdr}
\usepackage{tasks}
\usepackage{mdframed}
\usepackage{xcolor}
\usepackage{pgfplots}
\usepackage[makeroom]{cancel}
\usepackage{multicol}
\usepackage[shortlabels]{enumitem}
\usepackage{stackrel}
\usepackage{tkz-tab}
\usepackage{xpatch}
\xpatchcmd{\tkzTabLine}{$0$}{$\bullet$}{}{}

\setlength{\headheight}{10pt}
\setlength{\headsep}{10pt}
\pagestyle{fancy}
\rhead{\ayudantia \ - \alumno}
\tikzset{t style/.style={style=solid}}

\newcommand*{\mybox}[2]{\colorbox{#1!30}{\parbox{.98\linewidth}{#2}}}

\newenvironment{solucion}
{\begin{mdframed}[backgroundcolor=black!10]
		{\bf Solución:}\\
	}
	{
	\end{mdframed}
}

\newenvironment{alternativas}[1]
{\begin{multicols}{#1}
		\begin{enumerate}[a)]
		}
		{
		\end{enumerate}
	\end{multicols}
}

\newenvironment{preguntas}
{\begin{enumerate}\itemsep12pt
	}
	{
	\end{enumerate}
}

\newcommand{\ayudantia}{{\sc Ayudantía 14}}
\newcommand{\tituloayu}{Sustitución trigonométrica y fracciones parciales}
\newcommand{\fecha}{18 de junio de 2019}
\newcommand{\sigla}{MAT1610}
\newcommand{\nombre}{Cálculo I}
\newcommand{\profesor}{Amal Taarabt}
\newcommand{\ano}{2019}
\newcommand{\semestre}{1}
\newcommand{\mail}{mat1610@ifcastaneda.cl}
\newcommand{\alumno}{Ignacio Castañeda - \mail}

\newcommand{\ev}{\Big|}
\newcommand{\ra}{\rightarrow}
\newcommand{\lra}{\leftrightarrow}
\newcommand{\N}{\mathbb{N}}
\newcommand{\R}{\mathbb{R}}
\newcommand{\Exp}[1]{\mathcal{E}_{#1}}
\newcommand{\List}[1]{\mathcal{L}_{#1}}
\newcommand{\EN}{\Exp{\N}}
\newcommand{\LN}{\List{\N}}
\newcommand{\comment}[1]{}
\newcommand{\lb}{\\~\\}
\newcommand{\eop}{_{\square}}
\newcommand{\hsig}{\hat{\sigma}}
\newcommand{\widesim}[2][1.5]{
	\mathrel{\overset{#2}{\scalebox{#1}[1]{$\sim$}}}
}
\newcommand{\wsim}{\widesim{}}
\newcommand{\lh}{\stackrel{L'H}{=}}

\begin{document}
\thispagestyle{empty}

\begin{minipage}{2cm}
	\includegraphics[width=2cm]{../../../../img/logo.pdf}
	\vspace{0.5cm}
\end{minipage}
\begin{minipage}{\linewidth}
	\begin{tabular}{lrl}
		{\scriptsize\sc Pontificia Universidad Catolica de Chile} & \hspace*{0.7in}Curso: &
		\sigla  - \nombre\\
		{\sc Facultad de Matemáticas}&
		Profesor: & \profesor \\
		{\sc Semestre \ano-\semestre} & Ayudante: & {Ignacio Castañeda}\\
		& {Mail:} & \texttt{\mail}
	\end{tabular}
\end{minipage}

\vspace{-10mm}
\begin{center}
	{\LARGE\bf \ayudantia}\\
	\vspace{0.1cm}
	{\tituloayu}\\
	\vspace{0.1cm}
	\fecha\\
	\vspace{0.4cm}
\end{center}

\begin{preguntas}
\item Resuelva las siguientes integrales
\begin{tasks}(2)
\task $\displaystyle\int \dfrac{dx}{\sqrt[]{(4-x^2)^3}}$
\task $\displaystyle\int \dfrac{xdx}{\sqrt[]{1+x^2}}$
\task $\displaystyle\int \dfrac{dx}{x\ \sqrt[]{x^2-1}}$
\task $\displaystyle\int \dfrac{dx}{\sqrt[]{16+6x-x^2}}$
\end{tasks}
\begin{solucion}

\begin{enumerate}[a)]
\item $\displaystyle\int \dfrac{dx}{\sqrt[]{(4-x^2)^3}}$\\\\
			Como en la integral hay presente una expresión de la forma $a^2-x^2$, realizamos el siguiente cambio de variable
			$$x = asen(\theta) \ra x = 2sen(\theta) \ra dx = 2cos(\theta)d\theta$$
			Luego, la integral es
			$$\displaystyle\int \dfrac{2cos(\theta)d\theta}{\sqrt[]{(4-4sen^2(\theta))^3}}
			= \displaystyle\int \dfrac{2cos(\theta)d\theta}{\sqrt[]{(4(1-sen^2(\theta)))^3}}
			= \displaystyle\int \dfrac{2cos(\theta)d\theta}{\sqrt[]{(4cos^2(\theta))^3}}$$
			$$= \displaystyle\int \dfrac{2cos(\theta)d\theta}{(\sqrt[]{4cos^2(\theta)})^3}
			= \displaystyle\int \dfrac{2cos(\theta)d\theta}{(2cos(\theta))^3}
			= \dfrac{1}{4}\displaystyle\int \dfrac{d\theta}{(cos^2(\theta))}
			= \dfrac{1}{4}\displaystyle\int sec^2(\theta)d\theta$$
			$$= \dfrac{1}{4} tan(\theta) + c$$
			Recordemos que $x = 2sen(\theta)$, por lo que
			$$\theta = arcsen\left(\dfrac{x}{2}\right)$$
			Esto se puede representar con el siguiente triangulo auxiliar, ya que $\theta$ es el ángulo cuyo seno es $\dfrac{x}{2}$
			\begin{center}
				\begin{tikzpicture}[scale=1.25]%,cap=round,>=latex]
				
				\coordinate [label=left:$ $] (A) at (-1.5cm,-1.cm);
				\coordinate [label=right:$ $] (C) at (1.5cm,-1.0cm);				\coordinate [label=above:$ $] (B) at (1.5cm,1.0cm);
				\draw (A) -- node[above] {$2$} (B) -- node[right] {$x$} (C) -- node[below] {$\sqrt[]{4-x^2}$} (A);
				
				\draw (1.25cm,-1.0cm) rectangle (1.5cm,-0.75cm);
				
				\tkzMarkAngle[size=0.8cm,%
				opacity=.4](C,A,B)
				\tkzLabelAngle[pos = 0.6](C,A,B){$\theta$}
				
				\end{tikzpicture}
			\end{center}
			Luego, podemos ver que
			$$tan(\theta) = \dfrac{x}{\sqrt[]{4-x^2}}$$
			Finalmente,
			$$\displaystyle\int \dfrac{dx}{\sqrt[]{(4-x^2)^3}} =
			\dfrac{x}{4\ \sqrt[]{4-x^2}} + c$$
\item $\displaystyle\int \dfrac{xdx}{\sqrt[]{1+x^2}}$\\\\
			Como identificamos una expresión de la forma $a^2+x^2$, usamos la sustitución
			$$x = tan(\theta) \ra dx = sec^2(\theta)$$
			Luego tenemos
			$$\displaystyle\int \dfrac{tan(\theta)sec^2(\theta)d\theta}{\sqrt[]{1+tan^2(\theta)}}
			= \displaystyle\int \dfrac{tan(\theta)sec^2(\theta)d\theta}{\sqrt[]{sec^2(\theta)}}
			= \displaystyle\int \dfrac{tan(\theta)sec^2(\theta)d\theta}{sec(\theta)}$$
			$$= \displaystyle\int sec(\theta)tan(\theta)d\theta = sec(\theta) + c$$
			Como $x = tan(\theta) \ra \theta = arctan(x)$, nuestro triangulo auxiliar sería
			\begin{center}
				\begin{tikzpicture}[scale=1.25]%,cap=round,>=latex]
				
				\coordinate [label=left:$ $] (A) at (-1.5cm,-1.cm);
				\coordinate [label=right:$ $] (C) at (1.5cm,-1.0cm);				\coordinate [label=above:$ $] (B) at (1.5cm,1.0cm);
				\draw (A) -- node[above] {$\sqrt[]{1+x^2}$} (B) -- node[right] {$x$} (C) -- node[below] {$1$} (A);
				
				\draw (1.25cm,-1.0cm) rectangle (1.5cm,-0.75cm);
				
				\tkzMarkAngle[size=0.8cm,%
				opacity=.4](C,A,B)
				\tkzLabelAngle[pos = 0.6](C,A,B){$\theta$}
				
				\end{tikzpicture}
			\end{center}
			Finalmente, 
			$$\displaystyle\int \dfrac{xdx}{\sqrt[]{1+x^2}} = \sqrt[]{1+x^2} + c$$
\item $\displaystyle\int \dfrac{dx}{x\ \sqrt[]{x^2-1}}$\\\\
			Al identificar la expresión de la forma $x^2-a^2$, sabemos que debemos ocupar
			$$x = sec(\theta) \ra dx = sec(\theta)tan(\theta)$$
			Luego obtenemos
			$$\displaystyle\int \dfrac{sec(\theta)tan(\theta)d\theta}{sec(\theta)\ \sqrt[]{sec^2(\theta)-1}}
			= \displaystyle\int \dfrac{tan(\theta)d\theta}{\sqrt[]{tan^2(\theta)}}
			= \displaystyle\int d\theta = \theta + c$$
			Como $x = tan(\theta) \ra \theta = arctan(x)$
			Finalmente,
			$$\displaystyle\int \dfrac{dx}{x\ \sqrt[]{x^2-1}} = arctan(\theta) + c$$
\item $\displaystyle\int \dfrac{dx}{\sqrt[]{16+6x-x^2}}$\\\\
			Como no podemos ver ninguna expresión como las mencionadas anteriormente, vamos a hacer completación de cuadrados
			$$16+6x-x^2 = 25 - 9 + 6x - x^2 = 25 - (x^2-6x+9) = 25 - (x-3)^2$$
			Luego, la integral es
			$$\displaystyle\int \dfrac{dx}{\sqrt[]{25 - (x-3)^2}}$$
			Aqui si podemos identificar una expresión del tipo $a^2-x^2$, por lo que el cambio que corresponde hacer es
			$$x-3 = 5sen(\theta) \ra dx = 5cos(\theta)d\theta$$
			Entonces obtenemos la integral
			$$\displaystyle\int \dfrac{5cos(\theta)d\theta}{\sqrt[]{25 - 25sen^2(\theta)}}
			=\displaystyle\int \dfrac{5cos(\theta)d\theta}{\sqrt[]{25(1 - sen^2(\theta))}}
			=\displaystyle\int \dfrac{5cos(\theta)d\theta}{\sqrt[]{25cos^2(\theta)}}$$
			$$=\displaystyle\int \dfrac{5cos(\theta)d\theta}{5cos(\theta)} 
			= \displaystyle\int d\theta = \theta + c$$
			Como $x-3 = 5sen(\theta) \ra \theta = arcsen\left(\dfrac{x-3}{5}\right)$\\
			Finalmente,
			$$\displaystyle\int \dfrac{dx}{\sqrt[]{16+6x-x^2}} = arcsen\left(\dfrac{x-3}{5}\right) + c$$
\end{enumerate}
\end{solucion}
\item Resuelva la siguiente integral
	$$\displaystyle\int \dfrac{dx}{x(200-x)}$$
\begin{solucion}
En primer lugar, utilizamos fracciones parciales para descomponer la integral
		$$\dfrac{1}{x(2-x)} = \dfrac{A}{x} + \dfrac{B}{2-x}$$
		$$\dfrac{1}{x(2-x)} = \dfrac{A(2-x) + Bx}{x(2-x)}$$
		$$\dfrac{1}{x(2-x)} = \dfrac{2A-Ax + Bx}{x(2-x)}$$
		$$\dfrac{1}{x(2-x)} = \dfrac{2A+ (B-A)x}{x(2-x)}$$
		De esta forma, tenemos el siguiente sistema de ecuaciones
		$$2A = 1, \quad B-A = 0 \ra A = \dfrac{1}{2}, \quad B = \dfrac{1}{2}$$
		Entonces, tenemos que la descomposición es
		$$\dfrac{1}{x(2-x)} = \dfrac{1}{2x} + \dfrac{1}{2(2-x)}$$
		Luego, la integral original la podemos escribir como
		$$\displaystyle\int \dfrac{dx}{x(2-x)} = \displaystyle\int \dfrac{dx}{2x} + \displaystyle\int \dfrac{dx}{2(2-x)}$$
		$$ = \dfrac{1}{2} \left( \displaystyle\int \dfrac{dx}{x} + \displaystyle\int \dfrac{dx}{2-x} \right)$$
		$$ = \dfrac{1}{2} (ln|x| - ln|2-x| ) + c$$
\end{solucion}
\item Calcular la integral
	$$\displaystyle\int_0^1 \dfrac{x^3+6x^2+13x+10}{x+1}dx$$
\begin{solucion}
Como vemos que el grado del numerador es mayor que el del denominador, realizamos división polinómica, es decir, calculamos
		$$x^3+6x^2+13x+10 : x+1$$
		Esto lo podemos hacer utilizando división sintética
		$$
		\renewcommand\arraystretch{1.5}
		\setlength\doublerulesep{0pt}
		\begin{array}{rrrrr}
		\multicolumn{1}{r|}{1} & 1 & 6 & 13 & 10\\\cline{2-5}
		& & 1& 5 & 8 \\\cline{2-5}
		& 1 & 5& 8 & 2 
		\end{array}
		$$
		De donde obtenemos que
		$$\dfrac{x^3+6x^2+13x+10}{x+1} = x^2 +5x + 8 + \dfrac{2}{x+1}$$
		Luego,
		$$\displaystyle\int_0^1 \dfrac{x^3+6x^2+13x+10}{x+1}dx = \displaystyle\int_0^1 \left(x^2 + 5x + 8 +  \dfrac{2}{x+1}\right)dx$$
		$$= \left( \dfrac{x^3}{3} + \dfrac{5x^2}{2} + 8x + 2ln(x+1)\right) \ev_0^1 = \dfrac{1}{3} + \dfrac{5}{2} + 8 + 2ln(2) = \dfrac{65}{6} + 2 ln(2)$$
\end{solucion}
\end{preguntas}
\end{document}