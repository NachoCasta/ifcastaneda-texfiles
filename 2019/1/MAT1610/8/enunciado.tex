\documentclass[12pt]{article}

\usepackage{fullpage}
\usepackage{graphicx}
\usepackage{amssymb}
\usepackage{amsmath}
\usepackage[none]{hyphenat}
\usepackage{parskip}
\usepackage[spanish]{babel}
\usepackage[utf8]{inputenc}
\usepackage{hyperref}
\usepackage{fancyhdr}
\usepackage{tasks}
\usepackage{mdframed}
\usepackage{xcolor}
\usepackage{pgfplots}
\usepackage[makeroom]{cancel}
\usepackage{multicol}
\usepackage[shortlabels]{enumitem}
\usepackage{stackrel}
\usepackage{tkz-tab}
\usepackage{xpatch}
\xpatchcmd{\tkzTabLine}{$0$}{$\bullet$}{}{}

\setlength{\headheight}{10pt}
\setlength{\headsep}{10pt}
\pagestyle{fancy}
\rhead{\ayudantia \ - \alumno}
\tikzset{t style/.style={style=solid}}

\newcommand*{\mybox}[2]{\colorbox{#1!30}{\parbox{.98\linewidth}{#2}}}

\newenvironment{solucion}
{\begin{mdframed}[backgroundcolor=black!10]
		{\bf Solución:}\\
	}
	{
	\end{mdframed}
}

\newenvironment{alternativas}[1]
{\begin{multicols}{#1}
		\begin{enumerate}[a)]
		}
		{
		\end{enumerate}
	\end{multicols}
}

\newenvironment{preguntas}
{\begin{enumerate}\itemsep12pt
	}
	{
	\end{enumerate}
}

\newcommand{\ayudantia}{{\sc Ayudantía 8}}
\newcommand{\tituloayu}{Repaso I2}
\newcommand{\fecha}{30 de abril de 2019}
\newcommand{\sigla}{MAT1610}
\newcommand{\nombre}{Cálculo I}
\newcommand{\profesor}{Amal Taarabt}
\newcommand{\ano}{2019}
\newcommand{\semestre}{1}
\newcommand{\mail}{mat1610@ifcastaneda.cl}
\newcommand{\alumno}{Ignacio Castañeda - \mail}

\newcommand{\ev}{\Big|}
\newcommand{\ra}{\rightarrow}
\newcommand{\lra}{\leftrightarrow}
\newcommand{\N}{\mathbb{N}}
\newcommand{\R}{\mathbb{R}}
\newcommand{\Exp}[1]{\mathcal{E}_{#1}}
\newcommand{\List}[1]{\mathcal{L}_{#1}}
\newcommand{\EN}{\Exp{\N}}
\newcommand{\LN}{\List{\N}}
\newcommand{\comment}[1]{}
\newcommand{\lb}{\\~\\}
\newcommand{\eop}{_{\square}}
\newcommand{\hsig}{\hat{\sigma}}
\newcommand{\widesim}[2][1.5]{
	\mathrel{\overset{#2}{\scalebox{#1}[1]{$\sim$}}}
}
\newcommand{\wsim}{\widesim{}}
\newcommand{\lh}{\stackrel{L'H}{=}}

\begin{document}
\thispagestyle{empty}

\begin{minipage}{2cm}
	\includegraphics[width=2cm]{../../../../img/logo.pdf}
	\vspace{0.5cm}
\end{minipage}
\begin{minipage}{\linewidth}
	\begin{tabular}{lrl}
		{\scriptsize\sc Pontificia Universidad Catolica de Chile} & \hspace*{0.7in}Curso: &
		\sigla  - \nombre\\
		{\sc Facultad de Matemáticas}&
		Profesor: & \profesor \\
		{\sc Semestre \ano-\semestre} & Ayudante: & {Ignacio Castañeda}\\
		& {Mail:} & \texttt{\mail}
	\end{tabular}
\end{minipage}

\vspace{-10mm}
\begin{center}
	{\LARGE\bf \ayudantia}\\
	\vspace{0.1cm}
	{\tituloayu}\\
	\vspace{0.1cm}
	\fecha\\
	\vspace{0.4cm}
\end{center}

\begin{preguntas}
\item Sea $f(x) = ln(x^2+3^x)$. Determine $f'(0) + f''(0)$.
\item Determine todos los puntos de la curva $x^2y^2 + e^{3y} = e$ cuya tangente es horizontal.
\item Sea $f$ una función derivable en un intervalo $(a,b)$ tal que $f'(x) = \dfrac{e^{f(x)}}{1+(f(x))^2}$ para todo $x \in (a,b)$. Demuestre que $f$ es invertible y determine $(f^{-1})'(x)$.
\item Determine el máximo y mínimo absoluto de la función 
$$f(x) = 2\cos (x) + \sin (2x)$$
en el intervalo $[0, 2\pi]$.
\item Demuestre que la ecuación $\sin (x) = 2x-1$ tiene exáctamente una raíz real.
\item Determine las asíntotas de la función $f(x) = xe^{1/x}$.
\item La ley de los gases para un gas ideal a la temperatura absoluta $T$ (en Kelvin) y la presión $P$ (en atmósferas) con un volumen $V$ (en litros) es
$$PV = nRT$$
donde $n$ es constante y corresponde al número de moles del gas y $R = 0,0821$ es la constante de los gases.\\
Suponga que en el instante $t_0$ la presión $P$ es igual a 8 $atm$ y que esta aumenta a razón de 0,1 $atm/min$. Además se sabe que en ese mismo instante el volumen $V$ es de 10 litros y que este disminuye a razón de 0,15 $lt/min$.\\
Determine la razón de cambio de $T$, con respecto al tiempo, en el instante $t_0$, sabiendo que la constante $n = 10\ mol$.
\end{preguntas}
\end{document}