\documentclass[12pt]{article}

\usepackage{fullpage}
\usepackage{graphicx}
\usepackage{amssymb}
\usepackage{amsmath}
\usepackage[none]{hyphenat}
\usepackage{parskip}
\usepackage[spanish]{babel}
\usepackage[utf8]{inputenc}
\usepackage{hyperref}
\usepackage{fancyhdr}
\usepackage{tasks}
\usepackage{mdframed}
\usepackage{xcolor}
\usepackage{pgfplots}
\usepackage[makeroom]{cancel}
\usepackage{multicol}
\usepackage[shortlabels]{enumitem}
\usepackage{stackrel}
\usepackage{tkz-tab}
\usepackage{xpatch}
\xpatchcmd{\tkzTabLine}{$0$}{$\bullet$}{}{}

\setlength{\headheight}{10pt}
\setlength{\headsep}{10pt}
\pagestyle{fancy}
\rhead{\ayudantia \ - \alumno}
\tikzset{t style/.style={style=solid}}

\newcommand*{\mybox}[2]{\colorbox{#1!30}{\parbox{.98\linewidth}{#2}}}

\newenvironment{solucion}
{\begin{mdframed}[backgroundcolor=black!10]
		{\bf Solución:}\\
	}
	{
	\end{mdframed}
}

\newenvironment{alternativas}[1]
{\begin{multicols}{#1}
		\begin{enumerate}[a)]
		}
		{
		\end{enumerate}
	\end{multicols}
}

\newenvironment{preguntas}
{\begin{enumerate}\itemsep12pt
	}
	{
	\end{enumerate}
}

\newcommand{\ayudantia}{{\sc Ayudantía 12.5}}
\newcommand{\tituloayu}{Compilado I3}
\newcommand{\fecha}{5 de junio de 2019}
\newcommand{\sigla}{MAT1610}
\newcommand{\nombre}{Cálculo I}
\newcommand{\profesor}{Amal Taarabt}
\newcommand{\ano}{2019}
\newcommand{\semestre}{1}
\newcommand{\mail}{mat1610@ifcastaneda.cl}
\newcommand{\alumno}{Ignacio Castañeda - \mail}

\newcommand{\ev}{\Big|}
\newcommand{\ra}{\rightarrow}
\newcommand{\lra}{\leftrightarrow}
\newcommand{\N}{\mathbb{N}}
\newcommand{\R}{\mathbb{R}}
\newcommand{\Exp}[1]{\mathcal{E}_{#1}}
\newcommand{\List}[1]{\mathcal{L}_{#1}}
\newcommand{\EN}{\Exp{\N}}
\newcommand{\LN}{\List{\N}}
\newcommand{\comment}[1]{}
\newcommand{\lb}{\\~\\}
\newcommand{\eop}{_{\square}}
\newcommand{\hsig}{\hat{\sigma}}
\newcommand{\widesim}[2][1.5]{
	\mathrel{\overset{#2}{\scalebox{#1}[1]{$\sim$}}}
}
\newcommand{\wsim}{\widesim{}}
\newcommand{\lh}{\stackrel{L'H}{=}}

\begin{document}
\thispagestyle{empty}

\begin{minipage}{2cm}
	\includegraphics[width=2cm]{../../../../img/logo.pdf}
	\vspace{0.5cm}
\end{minipage}
\begin{minipage}{\linewidth}
	\begin{tabular}{lrl}
		{\scriptsize\sc Pontificia Universidad Catolica de Chile} & \hspace*{0.7in}Curso: &
		\sigla  - \nombre\\
		{\sc Facultad de Matemáticas}&
		Profesor: & \profesor \\
		{\sc Semestre \ano-\semestre} & Ayudante: & {Ignacio Castañeda}\\
		& {Mail:} & \texttt{\mail}
	\end{tabular}
\end{minipage}

\vspace{-10mm}
\begin{center}
	{\LARGE\bf \ayudantia}\\
	\vspace{0.1cm}
	{\tituloayu}\\
	\vspace{0.1cm}
	\fecha\\
	\vspace{0.4cm}
\end{center}

\begin{preguntas}
\item Encuentra una antiderivada para cada una de las siguientes funciones
\begin{tasks}(2)
\task $f(x) = 4x^3$
\task $f(x) = 8x\sin (2x^2)$
\task $f(x) = \dfrac{3}{2x}$
\task $f(x) = \dfrac{1}{1+4x^2}$
\end{tasks}
\item Calcule la siguiente integral, usando la definición de esta
	$$\int_2^5(4-2x)dx$$
\item Exprese los siguientes límites como una integral
	
\begin{tasks}(2)
\task $\lim\limits_{n\ra \infty} \sum\limits_{k=0}^n \dfrac{1}{n+3k}$
\task $\lim\limits_{n \ra \infty} \dfrac{n+1}{n^2+1} + \dfrac{n+2}{n^2+4} + \dots  + \dfrac{n+n}{n^2+n^2}$
\end{tasks}
\item Calcule $\lim\limits_{n \ra \infty} \sqrt[]{\dfrac{1}{n^3}} + \sqrt[]{\dfrac{2}{n^3}} + \dots  + \sqrt[]{\dfrac{1}{n^2}}$
\item Resuelva las siguientes integrales
\begin{tasks}(2)
\task $\displaystyle\int 2xdx$
\task $\displaystyle\int e^{ln(x^2)}dx$
\task $\displaystyle\int 4cos(2x)dx$
\task $\displaystyle\int 6e^{3x}dx$
\end{tasks}
\item Resolver las siguientes integrales indefinidas
\begin{tasks}(2)
\task $\displaystyle\int (3x^2 + 2x + 1)dx$
\task $\displaystyle\int x(x+1)(x+2)dx$
\task $\displaystyle\int sen^2(x)dx$
\task $\displaystyle\int (1+e)^xdx$
\end{tasks}
\item Sea
$$F(x) = \displaystyle\int_0^{x^2} \dfrac{xtan(t)}{1+t^2}dt$$
	Calule $F'(\pi /4)$
\item Calcular $F'(0)$, siendo
	$$F(x) = \displaystyle\int_0^{5x+1} \dfrac{e^{t^2}}{1+t^4}dt$$
\item Demuestre que la función
$$F(x) = \displaystyle\int_{1-x}^{1+x} \ln(t^2)dt, \quad x \in [0, \frac{1}{2}]$$
es decreciente
\item Calcule el siguiente límite
	$$\lim\limits_{x \ra \infty} \dfrac{\displaystyle\int_0^{arctan(x)} sen(ln(t^2+t+1)) dt}{x}$$
\item Resolver las siguientes integrales indefinidas
\begin{tasks}(2)
\task $\displaystyle\int \dfrac{e^{ln(x)}}{x^2+7}dx$
\task $\displaystyle\int (x+2)sen(x^2+4x-6)dx$
\task $\displaystyle\int \dfrac{3x}{\sqrt[3]{x^2+3}}dx$
\task $\displaystyle\int \dfrac{e^x-1}{e^x+1}dx$
\end{tasks}
\item Resolver las siguientes integrales definidas
\begin{tasks}(2)
\task $\displaystyle\int_0^2 \dfrac{x^2}{x^3+8} dx$
\task $\displaystyle\int_{-1}^1 xsen(1-x^2)dx$
\end{tasks}
\item Calcular
\begin{tasks}(3)
\task $\displaystyle\int_0^{\frac{1}{\sqrt[]{2}}} \dfrac{x \arcsin(x^2)}{\sqrt[]{1-x^4}}dx$
\task $\displaystyle\int_0^{\ln(2)} e^x\ \sqrt[]{e^x-1}dx$
\task $\displaystyle\int_{-3}^{4} |x-1|dx$
\end{tasks}
\item Calule el área de la región acotada por las curvas $y=x^2-4$ e $y = -x^2-2x$ en el intervalo $I =  [-3.3]$.
\item Encuentre el área acotada por las curvas por las curvas $y=x(x^2+11)$ e $y=6(x^2+1)$.
\item Determine el área limitada por la parabola $y^2=4x$ y la recta $y=x$.
\item Calcule el área de la región que se encuentra bajo la gráfica de la función $f(x) = e^xx^2$, donde $x \in [1,2]$
\item Hallar el área de la elipse
	$$\dfrac{x^2}{9} + \dfrac{y^2}{4} = 1$$
\item Usando integrales, calcule el área del siguiente triangulo
	\begin{center}
		\begin{tikzpicture}
		\begin{axis}[
		axis lines = left,
		xlabel = $x$,
		ylabel = $y$,
		]
		\addplot [
		domain=0:2,  
		color=red,
		]
		{x};
		\addplot [
		domain=0:1, 
		color=red,
		]
		{-x};
		\addplot [
		domain=1:2, 
		color=red,
		]
		{3*x-4};
		
		\end{axis}
		\end{tikzpicture}
	\end{center}
\end{preguntas}
\end{document}