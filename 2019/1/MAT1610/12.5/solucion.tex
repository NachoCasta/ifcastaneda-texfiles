\documentclass[12pt]{article}

\usepackage{fullpage}
\usepackage{graphicx}
\usepackage{amssymb}
\usepackage{amsmath}
\usepackage[none]{hyphenat}
\usepackage{parskip}
\usepackage[spanish]{babel}
\usepackage[utf8]{inputenc}
\usepackage{hyperref}
\usepackage{fancyhdr}
\usepackage{tasks}
\usepackage{mdframed}
\usepackage{xcolor}
\usepackage{pgfplots}
\usepackage[makeroom]{cancel}
\usepackage{multicol}
\usepackage[shortlabels]{enumitem}
\usepackage{stackrel}
\usepackage{tkz-tab}
\usepackage{xpatch}
\xpatchcmd{\tkzTabLine}{$0$}{$\bullet$}{}{}

\setlength{\headheight}{10pt}
\setlength{\headsep}{10pt}
\pagestyle{fancy}
\rhead{\ayudantia \ - \alumno}
\tikzset{t style/.style={style=solid}}

\newcommand*{\mybox}[2]{\colorbox{#1!30}{\parbox{.98\linewidth}{#2}}}

\newenvironment{solucion}
{\begin{mdframed}[backgroundcolor=black!10]
		{\bf Solución:}\\
	}
	{
	\end{mdframed}
}

\newenvironment{alternativas}[1]
{\begin{multicols}{#1}
		\begin{enumerate}[a)]
		}
		{
		\end{enumerate}
	\end{multicols}
}

\newenvironment{preguntas}
{\begin{enumerate}\itemsep12pt
	}
	{
	\end{enumerate}
}

\newcommand{\ayudantia}{{\sc Ayudantía 12.5}}
\newcommand{\tituloayu}{Compilado I3}
\newcommand{\fecha}{5 de junio de 2019}
\newcommand{\sigla}{MAT1610}
\newcommand{\nombre}{Cálculo I}
\newcommand{\profesor}{Amal Taarabt}
\newcommand{\ano}{2019}
\newcommand{\semestre}{1}
\newcommand{\mail}{mat1610@ifcastaneda.cl}
\newcommand{\alumno}{Ignacio Castañeda - \mail}

\newcommand{\ev}{\Big|}
\newcommand{\ra}{\rightarrow}
\newcommand{\lra}{\leftrightarrow}
\newcommand{\N}{\mathbb{N}}
\newcommand{\R}{\mathbb{R}}
\newcommand{\Exp}[1]{\mathcal{E}_{#1}}
\newcommand{\List}[1]{\mathcal{L}_{#1}}
\newcommand{\EN}{\Exp{\N}}
\newcommand{\LN}{\List{\N}}
\newcommand{\comment}[1]{}
\newcommand{\lb}{\\~\\}
\newcommand{\eop}{_{\square}}
\newcommand{\hsig}{\hat{\sigma}}
\newcommand{\widesim}[2][1.5]{
	\mathrel{\overset{#2}{\scalebox{#1}[1]{$\sim$}}}
}
\newcommand{\wsim}{\widesim{}}
\newcommand{\lh}{\stackrel{L'H}{=}}

\begin{document}
\thispagestyle{empty}

\begin{minipage}{2cm}
	\includegraphics[width=2cm]{../../../../img/logo.pdf}
	\vspace{0.5cm}
\end{minipage}
\begin{minipage}{\linewidth}
	\begin{tabular}{lrl}
		{\scriptsize\sc Pontificia Universidad Catolica de Chile} & \hspace*{0.7in}Curso: &
		\sigla  - \nombre\\
		{\sc Facultad de Matemáticas}&
		Profesor: & \profesor \\
		{\sc Semestre \ano-\semestre} & Ayudante: & {Ignacio Castañeda}\\
		& {Mail:} & \texttt{\mail}
	\end{tabular}
\end{minipage}

\vspace{-10mm}
\begin{center}
	{\LARGE\bf \ayudantia}\\
	\vspace{0.1cm}
	{\tituloayu}\\
	\vspace{0.1cm}
	\fecha\\
	\vspace{0.4cm}
\end{center}

\begin{preguntas}
\item Encuentra una antiderivada para cada una de las siguientes funciones
\begin{tasks}(2)
\task $f(x) = 4x^3$
\task $f(x) = 8x\sin (2x^2)$
\task $f(x) = \dfrac{3}{2x}$
\task $f(x) = \dfrac{1}{1+4x^2}$
\end{tasks}
\begin{solucion}
Una antiderivada corresponde a una función cuya derivada nos de la función original, por lo que debemos pensar que cosa al derivarla nos dará la función que tenemos.
\begin{enumerate}[a)]
\item $f(x) = 4x^3$\\
\\
Notemos que al derivar algo de la forma $x^n$, siempre se pierde un grado, por lo que la antiderivada debe tener un grado más que $f(x)$ y contrarestar el exponente que pasará multiplicando al derivar, esto es,
$$F(x) = x^4$$
\item $f(x) = 8x\sin (2x^2)$\\
\\
Para obtener la función $\sin(x)$ debemos derivar el $-\cos(x)$. Sin embargo, necesitamos que "aparezca" un $x$ afuera multiplicando la función. Esto lo podemos lograr con la regla de la cadena. Además, como al derivar una función con la regla de la cadena, lo que esta adentro se mantiene, debemos dejar $2x^2$ adentro del coseno. Luego,
$$F(x) = -2\cos(2x^2)$$
\item $f(x) = \dfrac{3}{2x}$\\
\\
Notemos que $f(x) = \dfrac{3}{2} \cdot \dfrac{1}{x}$. Luego,
$$F(x) = \dfrac{3}{2} \ln(x)$$
\item $f(x) = \dfrac{1}{1+4x^2}$\\
\\
Es evidente que $f(x)= \dfrac{1}{1+(2x)^2}$, por lo que
$$F(x) = \dfrac{\arctan(2x)}{2}$$
\end{enumerate}
\end{solucion}
\item Calcule la siguiente integral, usando la definición de esta
	$$\int_2^5(4-2x)dx$$
\begin{solucion}
Usando la integral por definición, podemos escribir la integral como
		$$\int_2^5(4-2x)dx = \lim\limits_{n\ra \infty} \dfrac{5-2}{n} \sum\limits_{i=1}^n f \left(2+\dfrac{(5-2)i}{n}\right) $$
		$$\int_2^5(4-2x)dx = \lim\limits_{n\ra \infty} \dfrac{5-2}{n} \sum\limits_{i=1}^n \left[4-2 \left(2+\dfrac{(5-2)i}{n}\right)\right]$$
		Luego, procedemos a resolver el límite
		$$= \lim\limits_{n\ra \infty} \dfrac{3}{n} \sum\limits_{i=1}^n \left[4-2 \left(2+\dfrac{3i}{n}\right)\right]$$
		$$= \lim\limits_{n\ra \infty} \dfrac{3}{n} \sum\limits_{i=1}^n \left[4-4-\dfrac{6i}{n}\right]$$
		$$= \lim\limits_{n\ra \infty} \dfrac{3}{n} \sum\limits_{i=1}^n \left[-\dfrac{6i}{n}\right]$$
		$$= -18\lim\limits_{n\ra \infty} \dfrac{1}{n^2} \sum\limits_{i=1}^n i$$
		$$= -18\lim\limits_{n\ra \infty} \dfrac{1}{n^2} \dfrac{n(n+1)}{2}$$
		$$= -9\lim\limits_{n\ra \infty} \dfrac{n+1}{n}$$
		$$= -9$$
\end{solucion}
\item Exprese los siguientes límites como una integral
	
\begin{tasks}(2)
\task $\lim\limits_{n\ra \infty} \sum\limits_{k=0}^n \dfrac{1}{n+3k}$
\task $\lim\limits_{n \ra \infty} \dfrac{n+1}{n^2+1} + \dfrac{n+2}{n^2+4} + \dots  + \dfrac{n+n}{n^2+n^2}$
\end{tasks}
\begin{solucion}
Para lograr esto, debemos acomodar el límite para que tenga la estructura de una Suma de Riemann, es decir,
		$$\lim\limits_{n \ra \infty} \dfrac{b-a}{n}\sum\limits_{k=0} f\left(a+\dfrac{(b-a)k}{n}\right)$$
\begin{enumerate}[a)]
\item Reorganizando el límite pedido,
		$$ = \lim\limits_{n\ra \infty} \sum\limits_{k=0}^n \dfrac{1}{n+3k}$$
		$$ = \lim\limits_{n\ra \infty} \sum\limits_{k=0}^n \dfrac{1}{n\left(1+\dfrac{3k}{n}\right)}$$
		$$ = \dfrac{1}{3} \lim\limits_{n\ra \infty} \dfrac{3}{n} \sum\limits_{k=0}^n \dfrac{1}{1+\dfrac{3k}{n}}$$
		Llegamos a una suma de Riemann, que esta asociada a la función $f(x) = \dfrac{1}{1+x}$ en el intervalo $[0, 3]$, por lo que
		$$\lim\limits_{n\ra \infty} \sum\limits_{k=0}^n \dfrac{1}{n+3k} = \dfrac{1}{3} \displaystyle\int_0^3 \dfrac{dx}{1+x}$$
\item $\lim\limits_{n \ra \infty} \dfrac{n+1}{n^2+1} + \dfrac{n+2}{n^2+4} + \dots  + \dfrac{n+n}{n^2+n^2}$\\
\\
Notemos que esto se puede esribir como,
$$\lim\limits_{n \ra \infty} \sum\limits_{k=1}^n \dfrac{n+k}{n^2+k^2}$$
Amplificando por $\dfrac{1}{n^2}$,
$$= \lim\limits_{n \ra \infty} \sum\limits_{k=1}^n \dfrac{n+k}{n^2+k^2}\cdot \dfrac{\frac{1}{n^2}}{\frac{1}{n^2}}$$
$$= \lim\limits_{n \ra \infty} \sum\limits_{k=1}^n \dfrac{\frac{1}{n}+\frac{k}{n^2}}{1+\left(\frac{k}{n}\right)^2}$$
Luego, factorizamos por $\dfrac{1}{n}$,
$$= \lim\limits_{n \ra \infty} \sum\limits_{k=1}^n \dfrac{1}{n} \cdot \dfrac{1+\frac{k}{n}}{1+\left(\frac{k}{n}\right)^2}$$
$$= \lim\limits_{n \ra \infty} \dfrac{1}{n}  \sum\limits_{k=1}^n \dfrac{1+\frac{k}{n}}{1+\left(\frac{k}{n}\right)^2}$$
Ahora, tenemos una Suma de Riemann que esta asociada al intervalo $[0, 1]$ y la función $f(x) = \dfrac{1+x}{1+x^2}$, por lo que
$$\lim\limits_{n \ra \infty} \dfrac{n+1}{n^2+1} + \dfrac{n+2}{n^2+4} + \dots  + \dfrac{n+n}{n^2+n^2} = \int_0^1 \dfrac{1+x}{1+x^2} dx$$
\end{enumerate}
\end{solucion}
\item Calcule $\lim\limits_{n \ra \infty} \sqrt[]{\dfrac{1}{n^3}} + \sqrt[]{\dfrac{2}{n^3}} + \dots  + \sqrt[]{\dfrac{1}{n^2}}$
\begin{solucion}
Notemos que podemos escribir el límite como
$$\lim\limits_{n \ra \infty} \sum\limits_{k=1}^{n}\sqrt[]{\dfrac{k}{n^3}} =
\lim\limits_{n \ra \infty} \dfrac{1}{n} \sum\limits_{k=1}^{n}\sqrt[]{\dfrac{k}{n}}$$
Luego, podemos ver que $a=0$, $b=1$ y $f(x) = \sqrt[]{x}$, por lo que
$$
\lim\limits_{n \ra \infty} \dfrac{1}{n} \sum\limits_{k=1}^{n}\sqrt[]{\dfrac{k}{n}} = \displaystyle\int_0^1\sqrt[]{x}dx$$
Resolvemos,
$$\displaystyle\int_0^1\sqrt[]{x}dx = 
\displaystyle\int_0^1 x^{1/2}dx =
\dfrac{2}{3}x^{3/2} \ev_0^1 =
\dfrac{2}{3}
$$
Por lo que
$$\lim\limits_{n \ra \infty} \sqrt[]{\dfrac{1}{n^3}} + \sqrt[]{\dfrac{2}{n^3}} + \dots  + \sqrt[]{\dfrac{1}{n^2}} = \dfrac{2}{3}$$
\end{solucion}
\item Resuelva las siguientes integrales
\begin{tasks}(2)
\task $\displaystyle\int 2xdx$
\task $\displaystyle\int e^{ln(x^2)}dx$
\task $\displaystyle\int 4cos(2x)dx$
\task $\displaystyle\int 6e^{3x}dx$
\end{tasks}
\begin{solucion}

\begin{enumerate}[a)]
\item $\displaystyle\int 2xdx = x^2 + c$
\item $\displaystyle\int e^{ln(x^2)}dx = \displaystyle\int x^2dx = \dfrac{x^3}{3} + c$
\item $\displaystyle\int 4cos(2x)dx = 2sen(2x) + c$
\item $\displaystyle\int 6e^{3x}dx = 2e^{3x} + c$
\end{enumerate}
\end{solucion}
\item Resolver las siguientes integrales indefinidas
\begin{tasks}(2)
\task $\displaystyle\int (3x^2 + 2x + 1)dx$
\task $\displaystyle\int x(x+1)(x+2)dx$
\task $\displaystyle\int sen^2(x)dx$
\task $\displaystyle\int (1+e)^xdx$
\end{tasks}
\begin{solucion}

\begin{enumerate}[a)]
\item $\displaystyle\int (3x^2 + 2x + 1)dx = \displaystyle\int 3x^2dx + \displaystyle\int 2xdx + \displaystyle\int dx  = 3\displaystyle\int x^2dx + 2\displaystyle\int xdx + \displaystyle\int dx $
			$$= 3\dfrac{x^3}{3} + 2\dfrac{x^2}{2} + x + c = x^3 + x^2 + x + c$$
\item $\displaystyle\int x(x+1)(x+2)dx = \displaystyle\int (x^3+3x^2+2x)dx = \dfrac{x^4}{4}+x^3+x+c$
\item $\displaystyle\int sen^2(x)dx = \displaystyle\int \dfrac{1-cos(2x)}{2}dx = \displaystyle\int \dfrac{1}{2}dx - \displaystyle\int \dfrac{cos(2x)}{2} = \dfrac{x}{2} -\dfrac{sen(2x)}{4} + c$
\item $\displaystyle\int (1+e)^xdx = \dfrac{(1+e)^x}{ln(1+e)}$
\end{enumerate}
\end{solucion}
\item Sea
$$F(x) = \displaystyle\int_0^{x^2} \dfrac{xtan(t)}{1+t^2}dt$$
	Calule $F'(\pi /4)$
\begin{solucion}
En primer lugar, saquemos la variable $x$ de la integral
		$$F(x) = x\displaystyle\int_0^{x^2} \dfrac{tan(t)}{1+t^2}dt$$
		Fijemonos que estamos frente a dos funciones, por lo que debemos usar la regla de la multiplicación y luego el TFC.
		$$F'(x) = \displaystyle\int_0^{x^2} \dfrac{tan(t)}{1+t^2}dt + 2x^2\left(\dfrac{tan(t)}{1+t^2}\ev_0^{x^2}\right)$$
		$$= \displaystyle\int_0^{x^2} \dfrac{tan(t)}{1+t^2}dt + 2x^2\left(\dfrac{tan(x^2)}{1+x^4}\right)$$
		Luego, evaluamos en $x = \pi /4, obteniendo$
		$$F'(\pi /4) = \displaystyle\int_0^{(\pi /4)^2} \dfrac{tan(t)}{1+t^2}dt + 2(\pi /4)^2\left(\dfrac{tan((\pi /4)^2)}{1+(\pi /4)^4}\right)$$
\end{solucion}
\item Calcular $F'(0)$, siendo
	$$F(x) = \displaystyle\int_0^{5x+1} \dfrac{e^{t^2}}{1+t^4}dt$$
\begin{solucion}
Usando el TFC, tenemos que
		$$F'(x) = \dfrac{e^{(5x+1)^2}}{1+(5x+1)^4} \cdot 5$$
		Luego, evaluamos en $x=0$
		$$F'(0) = \dfrac{5e^{(1)^2}}{1+(1)^4} = \dfrac{5e}{2}$$
\end{solucion}
\item Demuestre que la función
$$F(x) = \displaystyle\int_{1-x}^{1+x} \ln(t^2)dt, \quad x \in [0, \frac{1}{2}]$$
es decreciente
\begin{solucion}
Recordemos que para un función cualquiera
$$F(x) = \displaystyle\int_a^{g(x)}f(t)dt \ra F'(x) = f(g(x))g'(x)$$
Vamos ahora con el ejercicio.\\

Notemos que podemos escribir $F(x)$ de la siguiente forma
$$F(x) =
\displaystyle\int_{1-x}^{1} \ln(t^2)dt + \displaystyle\int_{1}^{1+x} \ln(t^2)dt =
-\displaystyle\int_{1}^{1-x} \ln(t^2)dt + \displaystyle\int_{1}^{1+x} \ln(t^2)dt
$$
Luego, al derivar,
$$F'(x) = -\ln((1-x)^2)\cdot (-1) + \ln((1+x)^2)$$
Simplificando,
$$F'(x) = \ln((1-x)^2) + \ln((1+x)^2)
= \ln(((1-x)(1+x))^2)
= 2\ln(1-x^2)$$
Luego, como 
$$0 \leq x \leq \dfrac{1}{2} \ra \dfrac{3}{4} \leq 1-x^2 \leq 1$$
Concluimos que 
$$F'(x) = 2\ln(1-x^2) \leq 0$$
Por lo que la función es decreciente.
\end{solucion}
\item Resolver las siguientes integrales indefinidas
\begin{tasks}(2)
\task $\displaystyle\int \dfrac{e^{ln(x)}}{x^2+7}dx$
\task $\displaystyle\int (x+2)sen(x^2+4x-6)dx$
\task $\displaystyle\int \dfrac{3x}{\sqrt[3]{x^2+3}}dx$
\task $\displaystyle\int \dfrac{e^x-1}{e^x+1}dx$
\end{tasks}
\begin{solucion}

\begin{enumerate}[a)]
\item $\displaystyle\int \dfrac{e^{ln(x)}}{x^2+7}dx = \displaystyle\int \dfrac{x}{x^2+7}dx$\\
			Usamos la sustitución $u = x^2+7 \ra du = 2xdx$
			$$= \dfrac{1}{2} \displaystyle\int \dfrac{2}{u}dx = \dfrac{1}{2} ln(u) + c = \dfrac{1}{2}ln(x^2+7) + c$$
\item $\displaystyle\int (x+2)sen(x^2+4x-6)dx$\\
			Utilizando la sustitución $u = x^2+4x-6 \ra du = 2x + 4 \ra du = 2(x+2)$
			$$=\dfrac{1}{2}\displaystyle\int sen(u)du = -\dfrac{1}{2}cos(u) = -\dfrac{1}{2}cos(x^2+4x-6)+c$$
\item $\displaystyle\int \dfrac{3x}{\sqrt[3]{x^2+3}}dx$\\
			La sustitución a usar es $u=x^2+3 \ra du=2xdx$
			$$= \dfrac{3}{2} \displaystyle\int \dfrac{1}{\sqrt[3]{u}}du = \dfrac{3}{2} \displaystyle\int \dfrac{1}{u^{1/3}}du = \dfrac{3}{2} \displaystyle\int u^{-1/3}du = \dfrac{3}{2}\dfrac{u^{2/3}}{\frac{2}{3}} + c = \dfrac{9}{4}(t^2+3)^{2/3}+c$$
\item $\displaystyle\int \dfrac{e^x-1}{e^x+1}dx = \displaystyle\int \dfrac{e^x}{e^x+1}dx - \displaystyle\int \dfrac{1}{e^x+1}dx$\\
			Resolvamos ambas por separado. En primer lugar,
			$$ \displaystyle\int \dfrac{e^x}{e^x+1}dx$$
			Utilizamos la sustitución $u=e^x+1 \ra du=e^x$
			$$= \displaystyle\int \dfrac{du}{u} = ln(u) + c = ln(e^x+1)+c$$
			Ahora, resolveremos
			$$-\displaystyle\int \dfrac{1}{e^x+1}dx = -\displaystyle\int \dfrac{e^xe^{-x}}{e^x+1}dx= -\displaystyle\int \dfrac{e^{-x}}{e^{-x}(e^x+1)}dx = -\displaystyle\int \dfrac{e^{-x}}{1+e^{-x}}dx$$
			Usando $u=1+e^{-x} \ra du = -e^{-x}$
			$$= \displaystyle\int \dfrac{du}{u}dx = ln(u) +c = ln(1+e^{-x})+c$$
			Por último, al sumar ambas obtenemos
			$$\displaystyle\int \dfrac{e^x-1}{e^x+1}dx = ln(1+e^x) + ln(1+e^{-x}) + c = ln((1+e^x)(1+e^{-x}))+c$$
\end{enumerate}
\end{solucion}
\item Resolver las siguientes integrales definidas
\begin{tasks}(2)
\task $\displaystyle\int_0^2 \dfrac{x^2}{x^3+8} dx$
\task $\displaystyle\int_{-1}^1 xsen(1-x^2)dx$
\end{tasks}
\begin{solucion}

\begin{enumerate}[a)]
\item $\displaystyle\int_0^2 \dfrac{x^2}{x^3+8} dx$\\
			Usando la sustitución
			$$u=x^3+8 \ra du =  3x^2dx$$
			$$x \in (0, 2) \ra u \in (8, 16)$$
			$$\displaystyle\int_0^2 \dfrac{x^2}{x^3+8}dx = \dfrac{1}{3}\displaystyle\int_8^{16} \dfrac{du}{u} = \dfrac{1}{3}ln(u) \ev_8^{16} = \dfrac{1}{3}(ln(16)-ln(8)) = \dfrac{1}{3}ln\left(\dfrac{16}{8}\right) = \dfrac{1}{3}ln(2)$$
\item $\displaystyle\int_{-1}^1 xsen(1-x^2)dx$\\
			Usamos el cambio de variable
			$$u = 1-x^2 \ra du = -2xdx$$
			$$x \in (-1, 1) \ra u \in (0, 0)$$
			Como la variable $u$ va de 0 a 0, entonces la integral es igual a 0. Luego,
			$$\displaystyle\int_{-1}^1 xsen(1-x^2)dx = 0$$
\end{enumerate}
\end{solucion}
\item Calcular
\begin{tasks}(3)
\task $\displaystyle\int_0^{\frac{1}{\sqrt[]{2}}} \dfrac{x \arcsin(x^2)}{\sqrt[]{1-x^4}}dx$
\task $\displaystyle\int_0^{\ln(2)} e^x\ \sqrt[]{e^x-1}dx$
\task $\displaystyle\int_{-3}^{4} |x-1|dx$
\end{tasks}
\begin{solucion}

\begin{enumerate}[a)]
\item $\displaystyle\int_0^{\frac{1}{\sqrt[]{2}}} \dfrac{x \arcsin(x^2)}{\sqrt[]{1-x^4}}dx$\\
\\
Usando la sustitución
$$u = \arcsin(x^2) \ra du = \dfrac{2x}{\sqrt[]{1-x^4}}dx$$
$$x \in (0, \frac{1}{\sqrt[]{2}}) \ra u \in (0, \frac{\pi}{6})$$
$$\displaystyle\int_0^{\frac{1}{\sqrt[]{2}}} \dfrac{x \arcsin(x^2)}{\sqrt[]{1-x^4}}dx = 
\dfrac{1}{2} \displaystyle\int_0^{\frac{\pi}{6}} udu =
\dfrac{1}{4}u^2 \ev_0^{\frac{\pi}{6}} =
\dfrac{\pi^2}{144}
$$
\item $\displaystyle\int_0^{\ln(2)} e^x\ \sqrt[]{e^x-1}dx$\\
\\
Usando la sustitución
$$u = e^x-1 \ra du = e^xdx$$
$$x \in (0, \ln(2)) \ra u \in (0, 1)$$
$$\displaystyle\int_0^{\ln(2)} e^x\ \sqrt[]{e^x-1}dx=
\displaystyle\int_0^{1} \sqrt[]{u}du = \dfrac{2}{3}u^{3/2} \ev_0^1 = \dfrac{2}{3}
$$
\item $\displaystyle\int_{-3}^{4} |x-1|dx$\\
\\
Al haber un valor absoluto, debemos separar la integral, de manera de ver ambos casos (positivo y negativo) por separado. El cambio de signo del valor absoluto ocurre en $x=1$, por lo que ahí es donde hay que separar la integral, esto es,
$$\displaystyle\int_{-3}^{4} |x-1|dx = 
\displaystyle\int_{-3}^{1} |x-1|dx + 
\displaystyle\int_{1}^{4} |x-1|dx$$
Aplicamos valor absoluto
$$ = 
\displaystyle\int_{-3}^{1} -(x-1)dx + 
\displaystyle\int_{1}^{4} (x-1)dx$$
Resolvemos,
$$= \left(x-\dfrac{x^2}{2}\right)\ev_{-3}^1 +
\left(x-\dfrac{x^2}{2}\right)\ev_{1}^4 = 8 + \dfrac{9}{2} = \dfrac{25}{2}$$
\end{enumerate}
\end{solucion}
\item Calule el área de la región acotada por las curvas $y=x^2-4$ e $y = -x^2-2x$ en el intervalo $I =  [-3.3]$.
\begin{solucion}
Con ayuda del gráfico
		\begin{center}
			\begin{tikzpicture}
			\begin{axis}[
			axis lines = left,
			xlabel = $x$,
			ylabel = $y$,
			]
			\addplot [
			domain=-3:3,  
			color=red,
			]
			{x^2-4};
			\addplot [
			domain=-3:3, 
			color=blue,
			]
			{-x^2-2*x};
			
			\end{axis}
			\end{tikzpicture}
		\end{center}
		Podemos ver que el área acotada por las curvas la podemos separar en 3, quedandonos
		$$A = \displaystyle\int_{-3}^{-2}[x^2-4-(-x^2-2x)]dx 
		+ \displaystyle\int_{-2}^{1}[-x^2-2x-(x^2-4)]dx 
		+ \displaystyle\int_{1}^{3}[x^2-4-(-x^2-2x)]dx$$
		$$A = \displaystyle\int_{-3}^{-2}(2x^2+2x-4)dx 
		+ \displaystyle\int_{-2}^{1}(4-2x-2x^2)dx 
		+ \displaystyle\int_{1}^{3}(2x^2+2x-4)dx$$
		$$A = \dfrac{11}{3} + 9 + \dfrac{52}{3} = \dfrac{90}{3} = 30$$
\end{solucion}
\item Encuentre el área acotada por las curvas por las curvas $y=x(x^2+11)$ e $y=6(x^2+1)$.
\begin{solucion}
Buscamos los puntos de intersección de ambas curvas, igualandolas
		$$x(x^2+11) = 6(x^2+1)$$
		$$x(x^2+11) - 6(x^2+1) = 0$$
		$$x^3-6x^2+11x-6 = 0$$
		$$(x-1)(x-2)(x-3)=0$$
		Tenemos entonces que las curvas se intersectan en $x=1, x=2$ y $x=3$.\\
		Necesitamos ver ahora cual curva va arriba y cual abajo. Notemos que
		$$x^3-6x^2+11x-6 \geq 0\quad si \qquad 1 \leq x \leq 2$$
		$$x^3-6x^2+11x-6 \leq 0\quad si \qquad 2 \leq x \leq 3$$
		Con lo que concluimos que en $[1,2]$ la curva $y=x(x^2+11)$ va sobre la curva $y=6(x^2+1)$ y que en $[2,3]$ pasa lo contrario.\\
		Finalmente,
		$$A = \displaystyle\int_{1}^{2}[x(x^2+11) - 6(x^2+1)]dx 
		+ \displaystyle\int_{2}^{3}[6(x^2+1) - x(x^2+11)]dx$$ 
		$$A = \displaystyle\int_{1}^{2}(x^3-6x^2+11x-6)dx 
		+ \displaystyle\int_{2}^{3}(6-11x+6x^2-x^3)dx$$ 
		$$A = \dfrac{1}{4} + \dfrac{1}{4} = \dfrac{1}{2}$$
\end{solucion}
\item Determine el área limitada por la parabola $y^2=4x$ y la recta $y=x$.
\begin{solucion}
\begin{center}
			\begin{tikzpicture}
			\begin{axis}[
			axis lines = left,
			xlabel = $x$,
			ylabel = $y$,
			]
			\addplot [
			domain=0:4,  
			color=red,
			]
			{2*x^(1/2)};
			\addplot [
			domain=0:4, 
			color=blue,
			]
			{x};
			
			\end{axis}
			\end{tikzpicture}
		\end{center}
		Integramos en el eje $y$, en $[0,4]$. Para esto, usaremos las ecuaciones $x=y$ y $x=\dfrac{y^2}{4}$, por lo que el área buscada será
		$$A = \displaystyle\int_0^4 y-\dfrac{y^2}{4} dy$$
		$$A = \dfrac{16}{2} - \dfrac{64}{12} = \dfrac{8}{3}$$
\end{solucion}
\item Hallar el área de la elipse
	$$\dfrac{x^2}{9} + \dfrac{y^2}{4} = 1$$
\begin{solucion}
En primer lugar, despejemos $y$, es decir
		$$y = \pm\dfrac{2\ \sqrt[]{9 - x^2}}{3}$$
		Para encontrar el área, podemos buscar el área de un cuarto de la elipse y multiplicarla por 4, es decir
		$$A = 4\displaystyle\int_0^3 \dfrac{2\ \sqrt[]{9 - x^2}}{3}dx$$
		Utilizamos el cambio de variable $x =3sen(\theta) \ra dx = 3cos(\theta)d\theta$
		$$A = 4\displaystyle\int_0^{\pi/2} \dfrac{2\ \sqrt[]{9 - 9sen^2(\theta)}}{3} 3cos(\theta)d\theta$$
		$$A = 4\displaystyle\int_0^{\pi/2} \dfrac{2 \cdot 3cos(\theta)}{3} 3cos(\theta)d\theta$$
		$$A = 24\displaystyle\int_0^{\pi/2} cos^2(\theta)d\theta$$
		$$A = 24\displaystyle\int_0^{\pi/2} \dfrac{1+cos(2\theta)}{2}d\theta$$
		$$A = 24 \cdot \dfrac{\pi}{4}$$
		$$A = 6 \pi$$
\end{solucion}
\item Usando integrales, calcule el área del siguiente triangulo
	\begin{center}
		\begin{tikzpicture}
		\begin{axis}[
		axis lines = left,
		xlabel = $x$,
		ylabel = $y$,
		]
		\addplot [
		domain=0:2,  
		color=red,
		]
		{x};
		\addplot [
		domain=0:1, 
		color=red,
		]
		{-x};
		\addplot [
		domain=1:2, 
		color=red,
		]
		{3*x-4};
		
		\end{axis}
		\end{tikzpicture}
	\end{center}
\begin{solucion}
Notemos que los lados del triangulo se pueden representar como rectas. Esto es,
		$$L_1:\quad y = x$$
		$$L_2:\quad y = -x$$
		$$L_3:\quad y = 3x-4$$
		Luego, al área de triangulo será
		$$A = \displaystyle\int_0^1 (x - (-x))dx + \displaystyle\int_1^2 (x - (3x-4))dx$$
		$$A = \displaystyle\int_0^1 2xdx + \displaystyle\int_1^2 (4-2x)dx$$
		$$A = 2$$
\end{solucion}
\end{preguntas}
\end{document}