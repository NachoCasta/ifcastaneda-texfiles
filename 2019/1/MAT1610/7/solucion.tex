\documentclass[12pt]{article}

\usepackage{fullpage}
\usepackage{graphicx}
\usepackage{amssymb}
\usepackage{amsmath}
\usepackage[none]{hyphenat}
\usepackage{parskip}
\usepackage[spanish]{babel}
\usepackage[utf8]{inputenc}
\usepackage{hyperref}
\usepackage{fancyhdr}
\usepackage{tasks}
\usepackage{mdframed}
\usepackage{xcolor}
\usepackage{pgfplots}
\usepackage[makeroom]{cancel}
\usepackage{multicol}
\usepackage[shortlabels]{enumitem}
\usepackage{stackrel}

\setlength{\headheight}{10pt}
\setlength{\headsep}{10pt}
\pagestyle{fancy}
\rhead{\ayudantia \ - \alumno}

\newcommand*{\mybox}[2]{\colorbox{#1!30}{\parbox{.98\linewidth}{#2}}}

\newenvironment{solucion}
{\begin{mdframed}[backgroundcolor=black!10]
		{\bf Solución:}\\
	}
	{
	\end{mdframed}
}

\newenvironment{alternativas}[1]
{\begin{multicols}{#1}
		\begin{enumerate}[a)]
		}
		{
		\end{enumerate}
	\end{multicols}
}

\newenvironment{preguntas}
{\begin{enumerate}\itemsep12pt
	}
	{
	\end{enumerate}
}

\newcommand{\ayudantia}{{\sc Ayudantía 7}}
\newcommand{\tituloayu}{L'Hôpital y trazo de curvas}
\newcommand{\fecha}{20 de abril de 2019}
\newcommand{\sigla}{MAT1610}
\newcommand{\nombre}{Cálculo I}
\newcommand{\profesor}{Amal Taarabt}
\newcommand{\ano}{2019}
\newcommand{\semestre}{1}
\newcommand{\mail}{mat1610@ifcastaneda.cl}
\newcommand{\alumno}{Ignacio Castañeda - \mail}

\newcommand{\ev}{\Big|}
\newcommand{\ra}{\rightarrow}
\newcommand{\lra}{\leftrightarrow}
\newcommand{\N}{\mathbb{N}}
\newcommand{\R}{\mathbb{R}}
\newcommand{\Exp}[1]{\mathcal{E}_{#1}}
\newcommand{\List}[1]{\mathcal{L}_{#1}}
\newcommand{\EN}{\Exp{\N}}
\newcommand{\LN}{\List{\N}}
\newcommand{\comment}[1]{}
\newcommand{\lb}{\\~\\}
\newcommand{\eop}{_{\square}}
\newcommand{\hsig}{\hat{\sigma}}
\newcommand{\widesim}[2][1.5]{
	\mathrel{\overset{#2}{\scalebox{#1}[1]{$\sim$}}}
}
\newcommand{\wsim}{\widesim{}}

\begin{document}
\thispagestyle{empty}

\begin{minipage}{2cm}
	\includegraphics[width=2cm]{../../../../img/logo.pdf}
	\vspace{0.5cm}
\end{minipage}
\begin{minipage}{\linewidth}
	\begin{tabular}{lrl}
		{\scriptsize\sc Pontificia Universidad Catolica de Chile} & \hspace*{0.7in}Curso: &
		\sigla  - \nombre\\
		{\sc Facultad de Matemáticas}&
		Profesor: & \profesor \\
		{\sc Semestre \ano-\semestre} & Ayudante: & {Ignacio Castañeda}\\
		& {Mail:} & \texttt{\mail}
	\end{tabular}
\end{minipage}

\vspace{-10mm}
\begin{center}
	{\LARGE\bf \ayudantia}\\
	\vspace{0.1cm}
	{\tituloayu}\\
	\vspace{0.1cm}
	\fecha\\
	\vspace{0.4cm}
\end{center}

\begin{preguntas}
\item Determine los siguientes límites, en caso de que existan
\begin{tasks}(3)
\task $\lim\limits_{x\ra 0}\dfrac{(\arcsin x)^2}{1-cos(3x)}$
\task $\lim\limits_{x\ra 0} x^{\frac{1}{\ln (e^x-1)}}$
\task $\lim\limits_{x\ra \infty} \left(1+\dfrac{a}{x}\right)^{bx}$
\end{tasks}
\begin{solucion}

\begin{enumerate}[a)]
\item $\lim\limits_{x\ra 0}\dfrac{(\arcsin x)^2}{1-cos(3x)}$\\
\\
Notemos que al evaluar, este límite es de la forma $\dfrac{0}{0}$, por lo que podemos aplicar L'Hopital, esto es
{\footnotesize$$\lim\limits_{x\ra 0}\dfrac{(\arcsin x)^2}{1-cos(3x)} 
\stackrel{L'H}{=} \lim\limits_{x\ra 0}\dfrac{((\arcsin x)^2)'}{(1-cos(3x))'} 
= \lim\limits_{x\ra 0}\dfrac{2(\arcsin x)\dfrac{1}{\sqrt[]{1-x^2}}}{3\sin(3x)}
= \lim\limits_{x\ra 0}\dfrac{2(\arcsin x)}{3\ \sqrt[]{1-x^2}\sin(3x)} $$}\\
Notemos que al evaluar este último, seguimos obteniendo $\dfrac{0}{0}$, por lo que debemos aplicar L'Hopital nuevamente,
{\scriptsize$$\lim\limits_{x\ra 0}\dfrac{2(\arcsin x)}{3\ \sqrt[]{1-x^2}\sin(3x)}
\stackrel{L'H}{=} \lim\limits_{x\ra 0}\dfrac{(2(\arcsin x))'}{(3\ \sqrt[]{1-x^2}\sin(3x))'}
=\lim\limits_{x\ra 0}\dfrac{\dfrac{2}{\sqrt[]{1-x^2}}}{3\dfrac{1}{2\ \sqrt[]{1-x^2}}\cdot(-2x)\sin(3x) + 3\ \sqrt[]{1-x^2}\cos(3x)} $$}\\
Evaluando esto, tenemos que
$$\lim\limits_{x\ra 0}\dfrac{(\arcsin x)^2}{1-cos(3x)} = \dfrac{2}{9}$$
\item $\lim\limits_{x\ra 0} x^{\frac{1}{\ln (e^x-1)}}$\\
\\
En primer lugar, notemos que
$$\lim\limits_{x\ra 0} x^{\frac{1}{\ln (e^x-1)}} = 
\lim\limits_{x\ra 0} e^{\ln(x^{\frac{1}{\ln (e^x-1)}})} = 
\lim\limits_{x\ra 0} e^{\frac{ln(x)}{\ln (e^x-1)}} = 
e^{\lim\limits_{x\ra 0}\frac{ln(x)}{\ln (e^x-1)}}$$
Por lo tanto, lo que haremos será resolver el límite del exponente y luego elevar $e$ con el resultado obtenido para finalmente obtener el límite pedido.\\

Es decir, debemos resolver
$$\lim\limits_{x\ra 0}\frac{ln(x)}{\ln (e^x-1)}$$
Notemos que al evaluar, obtenemos $\dfrac{\infty}{\infty}$, por lo que podemos aplicar L'Hopital, esto es
$$\lim\limits_{x\ra 0}\frac{ln(x)}{\ln (e^x-1)} \stackrel{L'H}{=}
\lim\limits_{x\ra 0}\frac{\dfrac{1}{x}}{\dfrac{e^x}{e^x-1}} =
\lim\limits_{x\ra 0}\frac{e^x-1}{xe^x}  \stackrel{L'H}{=} 
\lim\limits_{x\ra 0}\frac{e^x}{e^x + xe^x} = 1$$
No olvidemos que esto corresponde al exponente del límite pedido, por lo que 
$$\lim\limits_{x\ra 0} x^{\frac{1}{\ln (e^x-1)}} = e^1 = e$$
\item $\lim\limits_{x\ra \infty} \left(1+\dfrac{a}{x}\right)^{bx}$\\
\\
De forma similar, al ejercicio anterior, tenemos que
$$\lim\limits_{x\ra \infty} \left(1+\dfrac{a}{x}\right)^{bx} =
\lim\limits_{x\ra \infty} e^{\ln\left(\left(1+\frac{a}{x}\right)^{bx}\right)} =
\lim\limits_{x\ra \infty} e^{bx\ln\left(1+\frac{a}{x}\right)} =
e^{\lim\limits_{x\ra \infty} bx\ln\left(1+\frac{a}{x}\right)}$$
Luego, debemos resolver
$$\lim\limits_{x\ra \infty} bx\ln\left(1+\dfrac{a}{x}\right)$$
Notemos que este límite es de la forma $\infty \cdot 0$.\\

Reordenando,
$$\lim\limits_{x\ra \infty} bx\ln\left(1+\dfrac{a}{x}\right) =
\lim\limits_{x\ra \infty} \dfrac{\ln\left(1+\dfrac{a}{x}\right)}{\dfrac{1}{bx}} = \dfrac{0}{0} $$
Luego,
{\scriptsize$$\stackrel{L'H}{=} \lim\limits_{x\ra \infty}  \dfrac{\dfrac{-\frac{a}{x^2}}{1+\frac{a}{x}}}{-\dfrac{b}{(bx)^2}} =
\lim\limits_{x\ra \infty} \dfrac{\dfrac{\frac{a}{x^2}}{\frac{a+x}{x}}}{\dfrac{1}{bx^2}} =
\lim\limits_{x\ra \infty} \dfrac{\dfrac{ax}{x^2(a+x)}}{\dfrac{1}{bx^2}} =
\lim\limits_{x\ra \infty} \dfrac{abx^3}{x^2(a+x)} =
\lim\limits_{x\ra \infty} \dfrac{abx}{a+x} \stackrel{L'H}{=} 
\lim\limits_{x\ra \infty} ab = ab$$}\\
Finalmente,
$$\lim\limits_{x\ra \infty} \left(1+\dfrac{a}{x}\right)^{bx} = e^{ab}$$
\end{enumerate}
\end{solucion}
\item Considere la función $f(x) = x + 1 - \dfrac{2}{x} - 3\dfrac{ln(x)}{x}$ definida en $(0, \infty)$.
\begin{enumerate}[a)]
\item Determine, en caso que existan, las asíntotas verticales y horizontales de $f$.
\item Determine los intervalos de crecimiento, decrecimiento y la concavidad de $f$.
\end{enumerate}
\begin{solucion}

\begin{enumerate}[a)]
\item 
\item 
\end{enumerate}
\end{solucion}
\item Grafique la curva
$$y = 4\arctan x - \dfrac{x^3}{3} - x$$
indicando su dominio, asíntotas horizontales y verticales, intervalos de crecimiento y decrecimiento, máximos y mínimos, sentido de la concavidad y puntos de inflexión.
\begin{solucion}

\end{solucion}
\end{preguntas}
\end{document}