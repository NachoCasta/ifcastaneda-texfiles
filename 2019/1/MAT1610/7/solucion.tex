\documentclass[12pt]{article}

\usepackage{fullpage}
\usepackage{graphicx}
\usepackage{amssymb}
\usepackage{amsmath}
\usepackage[none]{hyphenat}
\usepackage{parskip}
\usepackage[spanish]{babel}
\usepackage[utf8]{inputenc}
\usepackage{hyperref}
\usepackage{fancyhdr}
\usepackage{tasks}
\usepackage{mdframed}
\usepackage{xcolor}
\usepackage{pgfplots}
\usepackage[makeroom]{cancel}
\usepackage{multicol}
\usepackage[shortlabels]{enumitem}
\usepackage{stackrel}
\usepackage{tkz-tab}
\usepackage{xpatch}
\xpatchcmd{\tkzTabLine}{$0$}{$\bullet$}{}{}

\setlength{\headheight}{10pt}
\setlength{\headsep}{10pt}
\pagestyle{fancy}
\rhead{\ayudantia \ - \alumno}
\tikzset{t style/.style={style=solid}}

\newcommand*{\mybox}[2]{\colorbox{#1!30}{\parbox{.98\linewidth}{#2}}}

\newenvironment{solucion}
{\begin{mdframed}[backgroundcolor=black!10]
		{\bf Solución:}\\
	}
	{
	\end{mdframed}
}

\newenvironment{alternativas}[1]
{\begin{multicols}{#1}
		\begin{enumerate}[a)]
		}
		{
		\end{enumerate}
	\end{multicols}
}

\newenvironment{preguntas}
{\begin{enumerate}\itemsep12pt
	}
	{
	\end{enumerate}
}

\newcommand{\ayudantia}{{\sc Ayudantía 7}}
\newcommand{\tituloayu}{L'Hôpital y trazo de curvas}
\newcommand{\fecha}{20 de abril de 2019}
\newcommand{\sigla}{MAT1610}
\newcommand{\nombre}{Cálculo I}
\newcommand{\profesor}{Amal Taarabt}
\newcommand{\ano}{2019}
\newcommand{\semestre}{1}
\newcommand{\mail}{mat1610@ifcastaneda.cl}
\newcommand{\alumno}{Ignacio Castañeda - \mail}

\newcommand{\ev}{\Big|}
\newcommand{\ra}{\rightarrow}
\newcommand{\lra}{\leftrightarrow}
\newcommand{\N}{\mathbb{N}}
\newcommand{\R}{\mathbb{R}}
\newcommand{\Exp}[1]{\mathcal{E}_{#1}}
\newcommand{\List}[1]{\mathcal{L}_{#1}}
\newcommand{\EN}{\Exp{\N}}
\newcommand{\LN}{\List{\N}}
\newcommand{\comment}[1]{}
\newcommand{\lb}{\\~\\}
\newcommand{\eop}{_{\square}}
\newcommand{\hsig}{\hat{\sigma}}
\newcommand{\widesim}[2][1.5]{
	\mathrel{\overset{#2}{\scalebox{#1}[1]{$\sim$}}}
}
\newcommand{\wsim}{\widesim{}}
\newcommand{\lh}{\stackrel{L'H}{=}}

\begin{document}
\thispagestyle{empty}

\begin{minipage}{2cm}
	\includegraphics[width=2cm]{../../../../img/logo.pdf}
	\vspace{0.5cm}
\end{minipage}
\begin{minipage}{\linewidth}
	\begin{tabular}{lrl}
		{\scriptsize\sc Pontificia Universidad Catolica de Chile} & \hspace*{0.7in}Curso: &
		\sigla  - \nombre\\
		{\sc Facultad de Matemáticas}&
		Profesor: & \profesor \\
		{\sc Semestre \ano-\semestre} & Ayudante: & {Ignacio Castañeda}\\
		& {Mail:} & \texttt{\mail}
	\end{tabular}
\end{minipage}

\vspace{-10mm}
\begin{center}
	{\LARGE\bf \ayudantia}\\
	\vspace{0.1cm}
	{\tituloayu}\\
	\vspace{0.1cm}
	\fecha\\
	\vspace{0.4cm}
\end{center}

\begin{preguntas}
\item Determine los siguientes límites, en caso de que existan
\begin{tasks}(3)
\task $\lim\limits_{x\ra 0}\dfrac{(\arcsin x)^2}{1-cos(3x)}$
\task $\lim\limits_{x\ra 0} x^{\frac{1}{\ln (e^x-1)}}$
\task $\lim\limits_{x\ra \infty} \left(1+\dfrac{a}{x}\right)^{bx}$
\end{tasks}
\begin{solucion}

\begin{enumerate}[a)]
\item $\lim\limits_{x\ra 0}\dfrac{(\arcsin x)^2}{1-cos(3x)}$\\
\\
Notemos que al evaluar, este límite es de la forma $\dfrac{0}{0}$, por lo que podemos aplicar L'Hopital, esto es
{\footnotesize$$\lim\limits_{x\ra 0}\dfrac{(\arcsin x)^2}{1-cos(3x)} 
\stackrel{L'H}{=} \lim\limits_{x\ra 0}\dfrac{((\arcsin x)^2)'}{(1-cos(3x))'} 
= \lim\limits_{x\ra 0}\dfrac{2(\arcsin x)\dfrac{1}{\sqrt[]{1-x^2}}}{3\sin(3x)}
= \lim\limits_{x\ra 0}\dfrac{2(\arcsin x)}{3\ \sqrt[]{1-x^2}\sin(3x)} $$}\\
Notemos que al evaluar este último, seguimos obteniendo $\dfrac{0}{0}$, por lo que debemos aplicar L'Hopital nuevamente,
{\scriptsize$$\lim\limits_{x\ra 0}\dfrac{2(\arcsin x)}{3\ \sqrt[]{1-x^2}\sin(3x)}
\stackrel{L'H}{=} \lim\limits_{x\ra 0}\dfrac{(2(\arcsin x))'}{(3\ \sqrt[]{1-x^2}\sin(3x))'}
=\lim\limits_{x\ra 0}\dfrac{\dfrac{2}{\sqrt[]{1-x^2}}}{3\dfrac{1}{2\ \sqrt[]{1-x^2}}\cdot(-2x)\sin(3x) + 3\ \sqrt[]{1-x^2}\cos(3x)3} $$}\\
Evaluando esto, tenemos que
$$\lim\limits_{x\ra 0}\dfrac{(\arcsin x)^2}{1-cos(3x)} = \dfrac{2}{9}$$
\item $\lim\limits_{x\ra 0} x^{\frac{1}{\ln (e^x-1)}}$\\
\\
En primer lugar, notemos que
$$\lim\limits_{x\ra 0} x^{\frac{1}{\ln (e^x-1)}} = 
\lim\limits_{x\ra 0} e^{\ln(x^{\frac{1}{\ln (e^x-1)}})} = 
\lim\limits_{x\ra 0} e^{\frac{ln(x)}{\ln (e^x-1)}} = 
e^{\lim\limits_{x\ra 0}\frac{ln(x)}{\ln (e^x-1)}}$$
Por lo tanto, lo que haremos será resolver el límite del exponente y luego elevar $e$ con el resultado obtenido para finalmente obtener el límite pedido.\\

Es decir, debemos resolver
$$\lim\limits_{x\ra 0}\frac{ln(x)}{\ln (e^x-1)}$$
Notemos que al evaluar, obtenemos $\dfrac{\infty}{\infty}$, por lo que podemos aplicar L'Hopital, esto es
$$\lim\limits_{x\ra 0}\frac{ln(x)}{\ln (e^x-1)} \stackrel{L'H}{=}
\lim\limits_{x\ra 0}\frac{\dfrac{1}{x}}{\dfrac{e^x}{e^x-1}} =
\lim\limits_{x\ra 0}\frac{e^x-1}{xe^x}  \stackrel{L'H}{=} 
\lim\limits_{x\ra 0}\frac{e^x}{e^x + xe^x} = 1$$
No olvidemos que esto corresponde al exponente del límite pedido, por lo que 
$$\lim\limits_{x\ra 0} x^{\frac{1}{\ln (e^x-1)}} = e^1 = e$$
\item $\lim\limits_{x\ra \infty} \left(1+\dfrac{a}{x}\right)^{bx}$\\
\\
De forma similar, al ejercicio anterior, tenemos que
$$\lim\limits_{x\ra \infty} \left(1+\dfrac{a}{x}\right)^{bx} =
\lim\limits_{x\ra \infty} e^{\ln\left(\left(1+\frac{a}{x}\right)^{bx}\right)} =
\lim\limits_{x\ra \infty} e^{bx\ln\left(1+\frac{a}{x}\right)} =
e^{\lim\limits_{x\ra \infty} bx\ln\left(1+\frac{a}{x}\right)}$$
Luego, debemos resolver
$$\lim\limits_{x\ra \infty} bx\ln\left(1+\dfrac{a}{x}\right)$$
Notemos que este límite es de la forma $\infty \cdot 0$.\\

Reordenando,
$$\lim\limits_{x\ra \infty} bx\ln\left(1+\dfrac{a}{x}\right) =
\lim\limits_{x\ra \infty} \dfrac{\ln\left(1+\dfrac{a}{x}\right)}{\dfrac{1}{bx}} = \dfrac{0}{0} $$
Luego,
{\scriptsize$$\stackrel{L'H}{=} \lim\limits_{x\ra \infty}  \dfrac{\dfrac{-\frac{a}{x^2}}{1+\frac{a}{x}}}{-\dfrac{b}{(bx)^2}} =
\lim\limits_{x\ra \infty} \dfrac{\dfrac{\frac{a}{x^2}}{\frac{a+x}{x}}}{\dfrac{1}{bx^2}} =
\lim\limits_{x\ra \infty} \dfrac{\dfrac{ax}{x^2(a+x)}}{\dfrac{1}{bx^2}} =
\lim\limits_{x\ra \infty} \dfrac{abx^3}{x^2(a+x)} =
\lim\limits_{x\ra \infty} \dfrac{abx}{a+x} \stackrel{L'H}{=} 
\lim\limits_{x\ra \infty} ab = ab$$}\\
Finalmente,
$$\lim\limits_{x\ra \infty} \left(1+\dfrac{a}{x}\right)^{bx} = e^{ab}$$
\end{enumerate}
\end{solucion}
\item Considere la función $f(x) = x + 1 - \dfrac{2}{x} - 3\dfrac{ln(x)}{x}$ definida en $(0, \infty)$.
\begin{enumerate}[a)]
\item Determine, en caso que existan, las asíntotas verticales y horizontales de $f$.
\item Determine los intervalos de crecimiento, decrecimiento y la concavidad de $f$.
\end{enumerate}
\begin{solucion}

\begin{enumerate}[a)]
\item Los candidatos a asíntotas verticales son los puntos donde la función se indefine. En este caso, solo tenemos $x=0$. Para comprobar, buscamos
{\small$$\lim\limits_{x \ra 0+} f(x) =
x + 1 - \dfrac{2}{x} - 3\dfrac{ln(x)}{x} = 
\lim\limits_{x \ra 0+} f(x) = 
\dfrac{x^2+x - 2 - 3ln(x)}{x} = 
\dfrac{\infty}{0} = 
\infty
$$}\\
Es decir, la recta $x=0$ es una asíntota vertical de $f$.\\

Para ver la existencia de asíntotas horizontales, buscamos
$$
\lim\limits_{x \ra \infty} f(x) =
\lim\limits_{x \ra \infty} x + 1 - \dfrac{2}{x} - 3\dfrac{ln(x)}{x}$$
Notemos que
$$\lim\limits_{x \ra \infty} 3\dfrac{ln(x)}{x} \stackrel{L'H}{=}
\dfrac{3}{x} = 0
$$
Por lo que
$$
\lim\limits_{x \ra \infty} f(x) =
\lim\limits_{x \ra \infty} x + 1 - \dfrac{2}{x} - 3\dfrac{ln(x)}{x} = \infty$$
Por ende, no existen asíntotas horizontales.
\item Para determinar los intervalos de crecimiento, debemos analizar la derivada de la función, esta es
$$f'(x) = 
1 + \dfrac{2}{x^2} - \dfrac{3\dfrac{x}{x} - 3ln(x)}{x^2} = 
1 + \dfrac{-1 + 3ln(x)}{x^2} = \dfrac{x^2-1+3ln(x)}{x^2}$$
De manera analítica, notamos que $f'(1) = 0$, siendo $x=1$ el único 0 de la función, por lo que debemos analizar que ocurre a ambos lados de $x=1$.\\

Por un lado, si $x < 1$, notemos que $f'(x) < 0$. Por el contrario, si $x > 1$, $f'(x) > 0$.\\

Por ende, los intervalos de crecimiento y decrecimiento corresponden a $(0, 1)$ y $(1, \infty)$, respectivamente.\\

Para ver que ocurre con la concavidad, derivamos denuevo, esto es
$$f''(x) = 
\dfrac{\frac{3x^2}{x} - 6x(-1 + 3ln(x))}{x^4} =
\dfrac{5-6ln(x)}{x^3}$$
Para buscar donde cambia de signo, hacemos
$$f''(x) = 0 \ra \dfrac{5-6ln(x)}{x^3} = 0 \ra 5 - 6ln(x) = 0 \ra ln(x) = \dfrac{5}{6} \ra x = e^{5/6}$$
Finalmente, notamos que $x < e^{5/6} \ra f''(x) >0$ y $x > e^{5/6} \ra f''(x) < 0$, por lo que la función tiene concavidad positiva (convexa) en $x\in (0, e^{5/6})$ y concavidad negativa (concava) en $x \in (e^{5/6}, \infty)$.
\end{enumerate}
\end{solucion}
\item Grafique la curva
$$y = 4\arctan x - \dfrac{x^3}{3} - x$$
indicando su dominio, asíntotas horizontales y verticales, intervalos de crecimiento y decrecimiento, máximos y mínimos, sentido de la concavidad y puntos de inflexión.
\begin{solucion}
Para graficar la curva, es necesario encontrar primero todo lo pedido en el enunciado, por lo que haremos eso primero.\\

En primer lugar, notemos que la función no se indefine para ningún valor de $x$ y ninguna de las funciones que la componen tiene restricciones de dominio, por lo que el dominio de la función es $x \in \R$.\\

Al no indefinirse para ningún $x$, tampoco tendrá asíntotas verticales.\\

Para ver si tiene asíntotas horizontales, haremos
$$\lim\limits_{x\ra \infty}  4\arctan x - \dfrac{x^3}{3} - x = -\infty, \qquad \lim\limits_{x\ra -\infty}  4\arctan x - \dfrac{x^3}{3} - x = \infty$$
De aquí concluimos que tampoco posee asíntotas horizontales, sin embargo esto igual nos da información acerca de como se ve la función.\\

Ahora, debemos encontrar los intervalos de crecimiento y decrecimiento. Para esto, calculemos la derivada,
$$f'(x) = \dfrac{4}{1+x^2} - x^2 - 1$$
Buscamos los puntos críticos,
$$f'(x) = 0$$
$$\dfrac{4}{1+x^2} - x^2 - 1 = 0$$
$$\dfrac{4-x^2(1+x^2) -(1+x^2)}{1+x^2} = 0$$
$$4-x^2(1+x^2) -(1+x^2) = 0$$
$$-x^4 - 2x^2 +3 = 0$$
$$-(x^4 + 2x^2 -3) = 0$$
$$-(x^2 + 3)(x^2 -1) = 0$$
$$-(x^2 + 3)(x+1)(x-1) = 0$$
Por lo que los puntos críticos son
$$x=-1, \qquad x=1$$
Para ver los signos de la derivada, podemos hacer una tabla de signos:
$$
\begin{tikzpicture}
\tkzTabInit[lgt=2,espcl=2,deltacl=0]
{ /.8, $-(x^2+3)$ /.8, $x+1$ /.8, $x-1$ /.8, /.8}
{,$-1$,$1$,} % four main references
\tkzTabLine {,-,t,-,t,-,} % seven denotations
\tkzTabLine {,-,z,+,t,+,}
\tkzTabLine {,-,t,-,z,+,}
\tkzTabLine {,-,z,+,z,-,}
\end{tikzpicture}
$$
Por lo tanto tenemos:
\begin{itemize}
	\item Intervalo de crecimiento: $(-1, 1)$
	\item Intervalo de decrecimiento: $(-\infty, -1) \cup (1, \infty)$
\end{itemize}
Analicemos ahora la segunda derivada, esta es
$$f''(x) = 
\dfrac{-8x}{(1+x^2)^2} - 2x = 
\dfrac{-8x - 2x(1+x^2)^2}{(1+x^2)^2} = 
\dfrac{-2x(x^4+2x^2+5)}{(1+x^2)^2}$$
Luego, 
$$f''(x) = \dfrac{-2x(x^4+2x^2+5)}{(1+x^2)^2} = 0 \ra x = 0$$
Notemos también que para $x < 0 \ra f''(x) > 0$ y $x > 0 \ra f''(x) < 0$, por lo que:
\begin{itemize}
	\item Intervalo de concavidad positiva: $(-\infty, 0)$
	\item Intervalo de concavidad negativa: $(0, \infty)$
\end{itemize}
Con esto, podemos tambier clasificar los puntos criticos, dandonos cuenta que
$$f(-1) = -\pi + \dfrac{4}{3} \ra \text{mínimo local}, \qquad f(-1) = \pi - \dfrac{4}{3} \ra \text{maximo local}$$
Además, en el punto $x=0$ tenemos un punto de inflexión.\\

Finalmente, teniendo toda esta información, podemos trazar la gráfica de la función, que viene dada por
\begin{center}
\begin{tikzpicture}
\begin{axis}[ 
width=320pt,compat=1.5.1,grid style={ultra thin},every axis plot post/.append style={thick},
x tick label style={font=\tiny},y tick label style={font=\tiny},
scale only axis,grid=major,axis lines=middle,
xlabel={$x$},
ylabel={$y$},
xmin=-3,
xmax=3,
domain=-5:5,
ymin=-3,
ymax=3,
xtick={-5, -4,...,5},
ytick={-5, -4,...,5},
legend style={at={(0.5,-0.05)},anchor=north,nodes={right}},
] 
\addplot[mark=none,color=blue, samples=500]{4*rad(atan(x)) - x^3/3 - x}; 
\end{axis}
\end{tikzpicture}
\end{center}
\end{solucion}
\end{preguntas}
\end{document}