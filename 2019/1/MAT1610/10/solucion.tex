\documentclass[12pt]{article}

\usepackage{fullpage}
\usepackage{graphicx}
\usepackage{amssymb}
\usepackage{amsmath}
\usepackage[none]{hyphenat}
\usepackage{parskip}
\usepackage[spanish]{babel}
\usepackage[utf8]{inputenc}
\usepackage{hyperref}
\usepackage{fancyhdr}
\usepackage{tasks}
\usepackage{mdframed}
\usepackage{xcolor}
\usepackage{pgfplots}
\usepackage[makeroom]{cancel}
\usepackage{multicol}
\usepackage[shortlabels]{enumitem}
\usepackage{stackrel}
\usepackage{tkz-tab}
\usepackage{xpatch}
\xpatchcmd{\tkzTabLine}{$0$}{$\bullet$}{}{}

\setlength{\headheight}{10pt}
\setlength{\headsep}{10pt}
\pagestyle{fancy}
\rhead{\ayudantia \ - \alumno}
\tikzset{t style/.style={style=solid}}

\newcommand*{\mybox}[2]{\colorbox{#1!30}{\parbox{.98\linewidth}{#2}}}

\newenvironment{solucion}
{\begin{mdframed}[backgroundcolor=black!10]
		{\bf Solución:}\\
	}
	{
	\end{mdframed}
}

\newenvironment{alternativas}[1]
{\begin{multicols}{#1}
		\begin{enumerate}[a)]
		}
		{
		\end{enumerate}
	\end{multicols}
}

\newenvironment{preguntas}
{\begin{enumerate}\itemsep12pt
	}
	{
	\end{enumerate}
}

\newcommand{\ayudantia}{{\sc Ayudantía 10}}
\newcommand{\tituloayu}{Sumas de Riemann e Integrales}
\newcommand{\fecha}{14 de mayo de 2019}
\newcommand{\sigla}{MAT1610}
\newcommand{\nombre}{Cálculo I}
\newcommand{\profesor}{Amal Taarabt}
\newcommand{\ano}{2019}
\newcommand{\semestre}{1}
\newcommand{\mail}{mat1610@ifcastaneda.cl}
\newcommand{\alumno}{Ignacio Castañeda - \mail}

\newcommand{\ev}{\Big|}
\newcommand{\ra}{\rightarrow}
\newcommand{\lra}{\leftrightarrow}
\newcommand{\N}{\mathbb{N}}
\newcommand{\R}{\mathbb{R}}
\newcommand{\Exp}[1]{\mathcal{E}_{#1}}
\newcommand{\List}[1]{\mathcal{L}_{#1}}
\newcommand{\EN}{\Exp{\N}}
\newcommand{\LN}{\List{\N}}
\newcommand{\comment}[1]{}
\newcommand{\lb}{\\~\\}
\newcommand{\eop}{_{\square}}
\newcommand{\hsig}{\hat{\sigma}}
\newcommand{\widesim}[2][1.5]{
	\mathrel{\overset{#2}{\scalebox{#1}[1]{$\sim$}}}
}
\newcommand{\wsim}{\widesim{}}
\newcommand{\lh}{\stackrel{L'H}{=}}

\begin{document}
\thispagestyle{empty}

\begin{minipage}{2cm}
	\includegraphics[width=2cm]{../../../../img/logo.pdf}
	\vspace{0.5cm}
\end{minipage}
\begin{minipage}{\linewidth}
	\begin{tabular}{lrl}
		{\scriptsize\sc Pontificia Universidad Catolica de Chile} & \hspace*{0.7in}Curso: &
		\sigla  - \nombre\\
		{\sc Facultad de Matemáticas}&
		Profesor: & \profesor \\
		{\sc Semestre \ano-\semestre} & Ayudante: & {Ignacio Castañeda}\\
		& {Mail:} & \texttt{\mail}
	\end{tabular}
\end{minipage}

\vspace{-10mm}
\begin{center}
	{\LARGE\bf \ayudantia}\\
	\vspace{0.1cm}
	{\tituloayu}\\
	\vspace{0.1cm}
	\fecha\\
	\vspace{0.4cm}
\end{center}

\begin{preguntas}
\item Calcule la siguiente integral, usando la definición de esta
	$$\int_2^5(4-2x)dx$$
\begin{solucion}
Usando la integral por definición, podemos escribir la integral como
		$$\int_2^5(4-2x)dx = \lim\limits_{n\ra \infty} \dfrac{5-2}{n} \sum\limits_{i=1}^n f \left(2+\dfrac{(5-2)i}{n}\right) $$
		$$\int_2^5(4-2x)dx = \lim\limits_{n\ra \infty} \dfrac{5-2}{n} \sum\limits_{i=1}^n \left[4-2 \left(2+\dfrac{(5-2)i}{n}\right)\right]$$
		Luego, procedemos a resolver el límite
		$$= \lim\limits_{n\ra \infty} \dfrac{3}{n} \sum\limits_{i=1}^n \left[4-2 \left(2+\dfrac{3i}{n}\right)\right]$$
		$$= \lim\limits_{n\ra \infty} \dfrac{3}{n} \sum\limits_{i=1}^n \left[4-4-\dfrac{6i}{n}\right]$$
		$$= \lim\limits_{n\ra \infty} \dfrac{3}{n} \sum\limits_{i=1}^n \left[-\dfrac{6i}{n}\right]$$
		$$= -18\lim\limits_{n\ra \infty} \dfrac{1}{n^2} \sum\limits_{i=1}^n i$$
		$$= -18\lim\limits_{n\ra \infty} \dfrac{1}{n^2} \dfrac{n(n+1)}{2}$$
		$$= -9\lim\limits_{n\ra \infty} \dfrac{n+1}{n}$$
		$$= -9$$
\end{solucion}
\item Exprese los siguientes límites como una integral
	
\begin{enumerate}[a)]
\item $\lim\limits_{n\ra \infty} \sum\limits_{k=0}^n \dfrac{1}{n+3k}$
\item $\lim\limits_{n\ra \infty} \sum\limits_{k=1}^n \dfrac{1}{(n+k)^2}$
\end{enumerate}
\begin{solucion}
Para lograr esto, debemos acomodar el límite para que tenga la estructura de una Suma de Riemann, es decir,
		$$\lim\limits_{n \ra \infty} \dfrac{b-a}{n}\sum\limits_{k=0} f\left(a+\dfrac{(b-a)k}{n}\right)$$
\begin{enumerate}[a)]
\item Reorganizando el límite pedido,
		$$ = \lim\limits_{n\ra \infty} \sum\limits_{k=0}^n \dfrac{1}{n+3k}$$
		$$ = \lim\limits_{n\ra \infty} \sum\limits_{k=0}^n \dfrac{1}{n\left(1+\dfrac{3k}{n}\right)}$$
		$$ = \dfrac{1}{3} \lim\limits_{n\ra \infty} \dfrac{3}{n} \sum\limits_{k=0}^n \dfrac{1}{1+\dfrac{3k}{n}}$$
		Llegamos a una suma de Riemann, que esta asociada a la función $f(x) = \dfrac{1}{1+x}$ en el intervalo $[0, 3]$, por lo que
		$$\lim\limits_{n\ra \infty} \sum\limits_{k=0}^n \dfrac{1}{n+3k} = \dfrac{1}{3} \displaystyle\int_0^3 \dfrac{dx}{1+x}$$
\item 
\end{enumerate}
\end{solucion}
\item Convierta el siguiente límite en una integral, expresándolo como una Suma de Riemann
$$\lim\limits_{n \ra \infty} \dfrac{n+1}{n^2+1} + \dfrac{n+2}{n^2+4} + \dots  + \dfrac{n+n}{n^2+n^2}$$
\begin{solucion}

\end{solucion}
\item Resuelva las siguientes integrales
\begin{tasks}(2)
\task $\displaystyle\int 2xdx$
\task $\displaystyle\int e^{ln(x^2)}dx$
\task $\displaystyle\int 4cos(2x)dx$
\task $\displaystyle\int 6e^{3x}dx$
\end{tasks}
\begin{solucion}

\begin{enumerate}[a)]
\item $\displaystyle\int 2xdx = x^2 + c$
\item $\displaystyle\int e^{ln(x^2)}dx = \displaystyle\int x^2dx = \dfrac{x^3}{3} + c$
\item $\displaystyle\int 4cos(2x)dx = 2sen(2x) + c$
\item $\displaystyle\int 6e^{3x}dx = 2e^{3x} + c$
\end{enumerate}
\end{solucion}
\item $$F(x) = \displaystyle\int_0^{x^2} \dfrac{xtan(t)}{1+t^2}dt$$
	Calule $F'(\pi /4)$
\begin{solucion}
En primer lugar, saquemos la variable $x$ de la integral
		$$F(x) = x\displaystyle\int_0^{x^2} \dfrac{tan(t)}{1+t^2}dt$$
		Fijemonos que estamos frente a dos funciones, por lo que debemos usar la regla de la multiplicación y luego el TFC.
		$$F'(x) = \displaystyle\int_0^{x^2} \dfrac{tan(t)}{1+t^2}dt + 2x^2\left(\dfrac{tan(t)}{1+t^2}\ev_0^{x^2}\right)$$
		$$= \displaystyle\int_0^{x^2} \dfrac{tan(t)}{1+t^2}dt + 2x^2\left(\dfrac{tan(x^2)}{1+x^4}\right)$$
		Luego, evaluamos en $x = \pi /4, obteniendo$
		$$F'(\pi /4) = \displaystyle\int_0^{(\pi /4)^2} \dfrac{tan(t)}{1+t^2}dt + 2(\pi /4)^2\left(\dfrac{tan((\pi /4)^2)}{1+(\pi /4)^4}\right)$$
\end{solucion}
\end{preguntas}
\end{document}