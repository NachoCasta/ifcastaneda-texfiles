\documentclass[12pt]{article}

\usepackage{fullpage}
\usepackage{graphicx}
\usepackage{amssymb}
\usepackage{amsmath}
\usepackage[none]{hyphenat}
\usepackage{parskip}
\usepackage[spanish]{babel}
\usepackage[utf8]{inputenc}
\usepackage{hyperref}
\usepackage{fancyhdr}
\usepackage{tasks}
\usepackage{mdframed}
\usepackage{xcolor}
\usepackage{pgfplots}
\usepackage[makeroom]{cancel}
\usepackage{multicol}
\usepackage[shortlabels]{enumitem}
\usepackage{tabto}

\setlength{\headheight}{10pt}
\setlength{\headsep}{10pt}
\pagestyle{fancy}
\rhead{\ayudantia \ - \alumno}

\newcommand*{\mybox}[2]{\colorbox{#1!30}{\parbox{.98\linewidth}{#2}}}

\newenvironment{solucion}
{\begin{mdframed}[backgroundcolor=black!10]
		{\bf Solución:}\\
	}
	{
	\end{mdframed}
}

\newenvironment{alternativas}[1]
{\begin{multicols}{#1}
		\begin{enumerate}[a)]
		}
		{
		\end{enumerate}
	\end{multicols}
}

\newenvironment{preguntas}
{\begin{enumerate}\itemsep12pt
	}
	{
	\end{enumerate}
}

\newcommand{\ayudantia}{{\sc Ayudantía 2}}
\newcommand{\tituloayu}{Límites trigonométricos y continuidad}
\newcommand{\fecha}{19 de marzo de 2019}
\newcommand{\sigla}{MAT1610}
\newcommand{\nombre}{Cálculo I}
\newcommand{\profesor}{Amal Taarabt}
\newcommand{\ano}{2019}
\newcommand{\semestre}{1}
\newcommand{\mail}{mat1610@ifcastaneda.cl}
\newcommand{\alumno}{Ignacio Castañeda - \mail}

\newcommand{\ev}{\Big|}
\newcommand{\ra}{\rightarrow}
\newcommand{\lra}{\leftrightarrow}
\newcommand{\N}{\mathbb{N}}
\newcommand{\R}{\mathbb{R}}
\newcommand{\Exp}[1]{\mathcal{E}_{#1}}
\newcommand{\List}[1]{\mathcal{L}_{#1}}
\newcommand{\EN}{\Exp{\N}}
\newcommand{\LN}{\List{\N}}
\newcommand{\comment}[1]{}
\newcommand{\lb}{\\~\\}
\newcommand{\eop}{_{\square}}
\newcommand{\hsig}{\hat{\sigma}}

\begin{document}
\thispagestyle{empty}

\begin{minipage}{2cm}
	\includegraphics[width=2cm]{../../../../img/logo.pdf}
	\vspace{0.5cm}
\end{minipage}
\begin{minipage}{\linewidth}
	\begin{tabular}{lrl}
		{\scriptsize\sc Pontificia Universidad Catolica de Chile} & \hspace*{0.7in}Curso: &
		\sigla  - \nombre\\
		{\sc Facultad de Matemáticas}&
		Profesor: & \profesor \\
		{\sc Semestre \ano-\semestre} & Ayudante: & {Ignacio Castañeda}\\
		& {Mail:} & \texttt{\mail}
	\end{tabular}
\end{minipage}

\vspace{-10mm}
\begin{center}
	{\LARGE\bf \ayudantia}\\
	\vspace{0.1cm}
	{\tituloayu}\\
	\vspace{0.1cm}
	\fecha\\
	\vspace{0.4cm}
\end{center}

\begin{preguntas}
\item Calcule los siguientes límites, en caso de que existan
\begin{tasks}(2)
\task $\lim\limits_{x \ra 0} \dfrac{\tan x}{x}$ 
\task $\lim\limits_{x \ra 0} \dfrac{\sin 3x}{5x}$ 
\task $\lim\limits_{x \ra 0} \dfrac{\tan x}{\sin 4x}$ 
\task $\lim\limits_{x \ra 0} \dfrac{1 - \sqrt[]{\cos x}}{x^2}$ 
\task $\lim\limits_{x \ra \frac{\pi}{2}} \left(\dfrac{\pi}{2} - x \right) \tan x$ 
\task $\lim\limits_{x \ra \pi} \dfrac{(x - \pi)^2}{\sin ^2 x}$ 
\end{tasks}
\begin{solucion}

\begin{enumerate}[a)]
\item 
\item 
\item 
\item 
\item 
\item 
\end{enumerate}
\end{solucion}
\item Analice las discontinuidades de la función
$$ f(x) = \dfrac{\sin(\pi(x+1))}{x-x^2} $$
determinando si son removibles
\begin{solucion}

\end{solucion}
\item Sea 
$$f(x) = \begin{cases}
3x - 2 & x \leq 2\\
ax+b & 2 < x < 3 \\
2b-3ax & x \geq 3
\end{cases}$$
Determina el valor de las constantes $a$ y $b$ para que la función $f(x)$ sea continua $\forall x \in \R$
\begin{solucion}

\end{solucion}
\end{preguntas}
\end{document}