\documentclass[12pt]{article}

\usepackage{fullpage}
\usepackage{graphicx}
\usepackage{amssymb}
\usepackage{amsmath}
\usepackage[none]{hyphenat}
\usepackage{parskip}
\usepackage[spanish]{babel}
\usepackage[utf8]{inputenc}
\usepackage{hyperref}
\usepackage{fancyhdr}
\usepackage{tasks}
\usepackage{mdframed}
\usepackage{xcolor}
\usepackage{pgfplots}
\usepackage[makeroom]{cancel}
\usepackage{multicol}
\usepackage[shortlabels]{enumitem}
\usepackage{stackrel}

\setlength{\headheight}{10pt}
\setlength{\headsep}{10pt}
\pagestyle{fancy}
\rhead{\ayudantia \ - \alumno}

\newcommand*{\mybox}[2]{\colorbox{#1!30}{\parbox{.98\linewidth}{#2}}}

\newenvironment{solucion}
{\begin{mdframed}[backgroundcolor=black!10]
		{\bf Solución:}\\
	}
	{
	\end{mdframed}
}

\newenvironment{alternativas}[1]
{\begin{multicols}{#1}
		\begin{enumerate}[a)]
		}
		{
		\end{enumerate}
	\end{multicols}
}

\newenvironment{preguntas}
{\begin{enumerate}\itemsep12pt
	}
	{
	\end{enumerate}
}

\newcommand{\ayudantia}{{\sc Ayudantía 2}}
\newcommand{\tituloayu}{Límites trigonométricos, continuidad y TVI}
\newcommand{\fecha}{19 de marzo de 2019}
\newcommand{\sigla}{MAT1610}
\newcommand{\nombre}{Cálculo I}
\newcommand{\profesor}{Amal Taarabt}
\newcommand{\ano}{2019}
\newcommand{\semestre}{1}
\newcommand{\mail}{mat1610@ifcastaneda.cl}
\newcommand{\alumno}{Ignacio Castañeda - \mail}

\newcommand{\ev}{\Big|}
\newcommand{\ra}{\rightarrow}
\newcommand{\lra}{\leftrightarrow}
\newcommand{\N}{\mathbb{N}}
\newcommand{\R}{\mathbb{R}}
\newcommand{\Exp}[1]{\mathcal{E}_{#1}}
\newcommand{\List}[1]{\mathcal{L}_{#1}}
\newcommand{\EN}{\Exp{\N}}
\newcommand{\LN}{\List{\N}}
\newcommand{\comment}[1]{}
\newcommand{\lb}{\\~\\}
\newcommand{\eop}{_{\square}}
\newcommand{\hsig}{\hat{\sigma}}
\newcommand{\widesim}[2][1.5]{
	\mathrel{\overset{#2}{\scalebox{#1}[1]{$\sim$}}}
}
\newcommand{\wsim}{\widesim{}}

\begin{document}
\thispagestyle{empty}

\begin{minipage}{2cm}
	\includegraphics[width=2cm]{../../../../img/logo.pdf}
	\vspace{0.5cm}
\end{minipage}
\begin{minipage}{\linewidth}
	\begin{tabular}{lrl}
		{\scriptsize\sc Pontificia Universidad Catolica de Chile} & \hspace*{0.7in}Curso: &
		\sigla  - \nombre\\
		{\sc Facultad de Matemáticas}&
		Profesor: & \profesor \\
		{\sc Semestre \ano-\semestre} & Ayudante: & {Ignacio Castañeda}\\
		& {Mail:} & \texttt{\mail}
	\end{tabular}
\end{minipage}

\vspace{-10mm}
\begin{center}
	{\LARGE\bf \ayudantia}\\
	\vspace{0.1cm}
	{\tituloayu}\\
	\vspace{0.1cm}
	\fecha\\
	\vspace{0.4cm}
\end{center}

\begin{preguntas}
\item Calcule los siguientes límites, en caso de que existan
\begin{tasks}(2)
\task $\lim\limits_{x \ra 0} \dfrac{\tan x}{x}$ 
\task $\lim\limits_{x \ra 0} \dfrac{\sin 3x}{5x}$ 
\task $\lim\limits_{x \ra 0} \dfrac{\tan x}{\sin 4x}$ 
\task $\lim\limits_{x \ra 0} \dfrac{1 - \sqrt[]{\cos x}}{x^2}$ 
\task $\lim\limits_{x \ra \frac{\pi}{2}} \left(\dfrac{\pi}{2} - x \right) \tan x$ 
\task $\lim\limits_{x \ra \pi} \dfrac{(x - \pi)^2}{\sin ^2 x}$ 
\end{tasks}
\begin{solucion}
Por lo general, para resolver límites que involucran funciones trigonométricas, debemos simplificar a modo de llegar a uno de los siguientes límites notables y luego reemplazar:
	$$\lim\limits_{x \ra 0} \dfrac{\sin x}{x} = 1, \quad
	\lim\limits_{x \ra 0} \dfrac{1-\cos x}{x} = 0, \quad
	\lim\limits_{x \ra 0} \dfrac{1 - \cos x}{x^2} = \dfrac{1}{2}$$
\begin{enumerate}[a)]
\item $\lim\limits_{x \ra 0} \dfrac{\tan x}{x} = 
\lim\limits_{x \ra 0} \dfrac{\sin x}{x \cos x} = 
\lim\limits_{x \ra 0} \dfrac{1}{\cos x} = 1$ 
\item $\lim\limits_{x \ra 0} \dfrac{\sin 3x}{5x} = 
\lim\limits_{x \ra 0} \dfrac{\sin 3x}{5x} \cdot \dfrac{3}{3} =
\lim\limits_{x \ra 0} \dfrac{\sin 3x}{3x} \cdot \dfrac{3}{5} = 
\lim\limits_{x \ra 0} \dfrac{3}{5}$ 
\item $\lim\limits_{x \ra 0} \dfrac{\tan x}{\sin 4x} =
\lim\limits_{x \ra 0} \dfrac{\sin x}{\cos x} \cdot \dfrac{1}{\sin 4x} =
\lim\limits_{x \ra 0} \dfrac{\sin x}{\cos x} \cdot \dfrac{1}{\sin 4x} \cdot \dfrac{4x}{4x}  =
\lim\limits_{x \ra 0} \dfrac{\sin x}{x} \cdot \dfrac{4x}{\sin 4x} \cdot \dfrac{4}{\cos x} $ \\
$=\lim\limits_{x \ra 0} \dfrac{4}{\cos x} = 4$
\item $\lim\limits_{x \ra 0} \dfrac{1 - \sqrt[]{\cos x}}{x^2} =
\lim\limits_{x \ra 0} \dfrac{1 - \sqrt[]{\cos x}}{x^2} \cdot \dfrac{1 + \sqrt[]{\cos x}}{1 + \sqrt[]{\cos x}} =
\lim\limits_{x \ra 0} \dfrac{1 - \cos x}{x^2} \cdot \dfrac{1}{1 + \sqrt[]{\cos x}} = \dfrac{1}{4}$ 
\item $\lim\limits_{x \ra \frac{\pi}{2}} \left(\dfrac{\pi}{2} - x \right) \tan x$\\
\\
En este caso utilizaremos el cambio de variable $u = \dfrac{\pi}{2} - x$, con lo que
$$\lim\limits_{x \ra \frac{\pi}{2}} \left(\dfrac{\pi}{2} - x \right) \tan x =
\lim\limits_{u \ra 0} u \tan \left(\dfrac{\pi}{2} - u \right) =
\lim\limits_{u \ra 0} u \tan u =
0$$
\item $\lim\limits_{x \ra \pi} \dfrac{(x - \pi)^2}{\sin ^2 x}$ \\
\\
De manera análoga, usaremos el cambio de variable $u = \pi - x$, con lo que
$$\lim\limits_{x \ra \pi} \dfrac{(x - \pi)^2}{\sin ^2 x} = 
\lim\limits_{u \ra 0} \dfrac{u^2}{(\sin (\pi + u))^2} = 
\lim\limits_{u \ra 0} \dfrac{u^2}{(-\sin u)^2} =  
\lim\limits_{u \ra 0} \dfrac{1}{\left(\dfrac{\sin u}{u}\right)^2} = 1$$
\end{enumerate}
\end{solucion}
\item Analice las discontinuidades de la función
$$ f(x) = \dfrac{\sin(\pi(x+1))}{x-x^2} $$
determinando si son removibles
\begin{solucion}
Factorizando, podemos ver que
$$ f(x) = \dfrac{\sin(\pi(x+1))}{x(1-x)} $$
es claro que las discontinuidades se producen en $x=0$ y $x=1$.\\
Para ver si estas son removibles, debemos ver si el límite de la función existe en las discontinuidades.\\\\
Comenzando por $x=0$,
$$\lim\limits_{x \ra 0} \dfrac{\sin(\pi x+ \pi)}{x(1-x)}$$
Usando la propiedad $\sin (\alpha + \beta) = \sin \alpha \cos \beta + \cos \alpha \sin \beta$,
$$=\lim\limits_{x \ra 0} -\dfrac{\sin(\pi x)}{x(1-x)}$$
Luego,
$$=\lim\limits_{x \ra 0} -\dfrac{\sin(\pi x)}{\pi x} \cdot \dfrac{\pi}{1-x} = 
-\pi$$
Por lo tanto en $x=0$ hay una discontinuidad removible.\\
\\
Para $x=1$, usaremos el cambio de variable $u=1-x$, con lo que
$$\lim\limits_{x \ra 1} \dfrac{\sin(\pi x+ \pi)}{x(1-x)} =
\lim\limits_{u \ra 0} \dfrac{\sin(\pi (2-u))}{u(1-u)} =
\lim\limits_{u \ra 0} \dfrac{\sin(2\pi - \pi u)}{u(1-u)}$$
Utilizando la propiedad $\sin (\alpha - \beta) = \sin \alpha \cos \beta - \cos \alpha \sin \beta$,
$$=\lim\limits_{x \ra 0} -\dfrac{\sin(\pi x)}{x(1-x)}$$
Luego,
$$=\lim\limits_{x \ra 0} -\dfrac{\sin(\pi x)}{\pi x} \cdot \dfrac{\pi}{1-x} = 
-\pi$$
\end{solucion}
\item Sea 
$$f(x) = \begin{cases}
3x - 2 & x \leq 2\\
ax+b & 2 < x < 3 \\
2b-3ax & x \geq 3
\end{cases}$$
Determina el valor de las constantes $a$ y $b$ para que la función $f(x)$ sea continua $\forall x \in \R$
\begin{solucion}

Para $x \neq 2$ y $x \neq 3$, la función es continua, ya que corresponde a una composición de funciones continuas.\\
\\
Para $x=2$, los límites laterales deben coincidir, por lo que
$$\lim\limits_{x \ra 2} 3x-2 = \lim\limits_{x \ra 2} ax + b$$
$$4 = 2a + b$$
De manera análoga, en $x=3$,
$$\lim\limits_{x \ra 3} ax + b = \lim\limits_{x \ra 3} 2b - 3ax$$
$$3a + b = 2b - 9a \ra 12a - b = 0$$
Resolviendo el sistema de ecuaciones
$$
\begin{array}{rcrr}
2a+b & = & 4& \vline\\
12a-b & = & 0& \vline\\
\hline
\end{array}
$$
Obtenemos
$$a = \dfrac{2}{7}, \qquad b = \dfrac{24}{7}$$
\end{solucion}
\item Sean $f(x)$ y $g(x)$ funciones continuas en el intervalo $[0, 1]$, tal que $f(0)=0$, $f(1)=1$, $g(0)=1$ y $g(1)=0$, demuestre que existe algun $c \in [0,1]$ tal que $f(c) = g(c)$.
\begin{solucion}
Definamos una función auxiliar. Sea
$$h(x) = f(x) - g(x)$$
Notemos que, dado que $h(x)$ es una composición de funciones continuas, esta también es continua.\\
Por lo tanto, demostrar que $f(c) = g(c)$ es equivalente a demostrar que $h(c)=0$.\\
\\
De la información que nos dan, podemos deducir que
$$h(0) = f(0) - g(0) = -1, \qquad h(1) = f(1) - g(1) = 1$$
Luego, por Teorema del Valor Intermedio (TVI), como $h(x)$ es una función continua en el intervalo $[0,1]$ y $-1 < 0 < 1$, $\exists c \in [0, 1]$ tal que $h(c) = 0$. En consecuencia, también existe $c \in [0, 1]$ tal que $f(c) = g(c)$.
\end{solucion}
\end{preguntas}
\end{document}