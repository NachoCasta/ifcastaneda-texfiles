\documentclass[12pt]{article}

\usepackage{fullpage}
\usepackage{graphicx}
\usepackage{amssymb}
\usepackage{amsmath}
\usepackage[none]{hyphenat}
\usepackage{parskip}
\usepackage[spanish]{babel}
\usepackage[utf8]{inputenc}
\usepackage{hyperref}
\usepackage{fancyhdr}
\usepackage{tasks}
\usepackage{mdframed}
\usepackage{xcolor}
\usepackage{pgfplots}
\usepackage[makeroom]{cancel}
\usepackage{multicol}
\usepackage[shortlabels]{enumitem}
\usepackage{stackrel}

\setlength{\headheight}{10pt}
\setlength{\headsep}{10pt}
\pagestyle{fancy}
\rhead{\ayudantia \ - \alumno}

\newcommand*{\mybox}[2]{\colorbox{#1!30}{\parbox{.98\linewidth}{#2}}}

\newenvironment{solucion}
{\begin{mdframed}[backgroundcolor=black!10]
		{\bf Solución:}\\
	}
	{
	\end{mdframed}
}

\newenvironment{alternativas}[1]
{\begin{multicols}{#1}
		\begin{enumerate}[a)]
		}
		{
		\end{enumerate}
	\end{multicols}
}

\newenvironment{preguntas}
{\begin{enumerate}\itemsep12pt
	}
	{
	\end{enumerate}
}

\newcommand{\ayudantia}{{\sc Ayudantía 3}}
\newcommand{\tituloayu}{Asíntotas y derivadas}
\newcommand{\fecha}{26 de marzo de 2019}
\newcommand{\sigla}{MAT1610}
\newcommand{\nombre}{Cálculo I}
\newcommand{\profesor}{Amal Taarabt}
\newcommand{\ano}{2019}
\newcommand{\semestre}{1}
\newcommand{\mail}{mat1610@ifcastaneda.cl}
\newcommand{\alumno}{Ignacio Castañeda - \mail}

\newcommand{\ev}{\Big|}
\newcommand{\ra}{\rightarrow}
\newcommand{\lra}{\leftrightarrow}
\newcommand{\N}{\mathbb{N}}
\newcommand{\R}{\mathbb{R}}
\newcommand{\Exp}[1]{\mathcal{E}_{#1}}
\newcommand{\List}[1]{\mathcal{L}_{#1}}
\newcommand{\EN}{\Exp{\N}}
\newcommand{\LN}{\List{\N}}
\newcommand{\comment}[1]{}
\newcommand{\lb}{\\~\\}
\newcommand{\eop}{_{\square}}
\newcommand{\hsig}{\hat{\sigma}}
\newcommand{\widesim}[2][1.5]{
	\mathrel{\overset{#2}{\scalebox{#1}[1]{$\sim$}}}
}
\newcommand{\wsim}{\widesim{}}

\begin{document}
\thispagestyle{empty}

\begin{minipage}{2cm}
	\includegraphics[width=2cm]{../../../../img/logo.pdf}
	\vspace{0.5cm}
\end{minipage}
\begin{minipage}{\linewidth}
	\begin{tabular}{lrl}
		{\scriptsize\sc Pontificia Universidad Catolica de Chile} & \hspace*{0.7in}Curso: &
		\sigla  - \nombre\\
		{\sc Facultad de Matemáticas}&
		Profesor: & \profesor \\
		{\sc Semestre \ano-\semestre} & Ayudante: & {Ignacio Castañeda}\\
		& {Mail:} & \texttt{\mail}
	\end{tabular}
\end{minipage}

\vspace{-10mm}
\begin{center}
	{\LARGE\bf \ayudantia}\\
	\vspace{0.1cm}
	{\tituloayu}\\
	\vspace{0.1cm}
	\fecha\\
	\vspace{0.4cm}
\end{center}

\begin{preguntas}
\item Determine las asíntotas horizontales, verticales y oblicuas de las siguientes funciones, en caso de que existan
\begin{tasks}(2)
\task $f(x) = \dfrac{9-x^2}{x+2}$
\task $f(x) = \dfrac{x^2-5x+6}{x^2+2x-8}$
\end{tasks}
\item Encuentra la función $g(x)$ que es tangente a la función $f(x) = x^2 + 1$ en el punto $x = 3$.
\item Calcular la derivada por definición de las siguientes funciones
\begin{tasks}(2)
\task $f(x) = x+5$
\task $f(x) = 2x^2-3$
\task $f(x) = e^x$
\task $f(x) = x^n$
\end{tasks}
\item Demuestra que la función $f(x) = e^{-x} + ln(x)$ posee al menos una solución.
\item Sea $f:[a,b] \ra \R$ una función derivable en $(a,b)$ y sea $x_0 \in (a,b)$ fijo, se define
$$ g(h) = \dfrac{f(x_0 + h) - f(x_0-h)}{2h}$$
Calcular $\lim\limits_{h \ra 0} g(h)$
\item Dada la función
$$f(x) = \begin{cases}
\dfrac{x-p}{x+1} & x > 0\\
x^2+qx & x \leq 0
\end{cases}$$
Determine los valores de $p$ y $q$ de manera que $f$ sea derivable en $x=0$
\end{preguntas}
\end{document}