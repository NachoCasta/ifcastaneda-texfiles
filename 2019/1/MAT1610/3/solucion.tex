\documentclass[12pt]{article}

\usepackage{fullpage}
\usepackage{graphicx}
\usepackage{amssymb}
\usepackage{amsmath}
\usepackage[none]{hyphenat}
\usepackage{parskip}
\usepackage[spanish]{babel}
\usepackage[utf8]{inputenc}
\usepackage{hyperref}
\usepackage{fancyhdr}
\usepackage{tasks}
\usepackage{mdframed}
\usepackage{xcolor}
\usepackage{pgfplots}
\usepackage[makeroom]{cancel}
\usepackage{multicol}
\usepackage[shortlabels]{enumitem}
\usepackage{stackrel}

\setlength{\headheight}{10pt}
\setlength{\headsep}{10pt}
\pagestyle{fancy}
\rhead{\ayudantia \ - \alumno}

\newcommand*{\mybox}[2]{\colorbox{#1!30}{\parbox{.98\linewidth}{#2}}}

\newenvironment{solucion}
{\begin{mdframed}[backgroundcolor=black!10]
		{\bf Solución:}\\
	}
	{
	\end{mdframed}
}

\newenvironment{alternativas}[1]
{\begin{multicols}{#1}
		\begin{enumerate}[a)]
		}
		{
		\end{enumerate}
	\end{multicols}
}

\newenvironment{preguntas}
{\begin{enumerate}\itemsep12pt
	}
	{
	\end{enumerate}
}

\newcommand{\ayudantia}{{\sc Ayudantía 3}}
\newcommand{\tituloayu}{Asíntotas y derivadas}
\newcommand{\fecha}{26 de marzo de 2019}
\newcommand{\sigla}{MAT1610}
\newcommand{\nombre}{Cálculo I}
\newcommand{\profesor}{Amal Taarabt}
\newcommand{\ano}{2019}
\newcommand{\semestre}{1}
\newcommand{\mail}{mat1610@ifcastaneda.cl}
\newcommand{\alumno}{Ignacio Castañeda - \mail}

\newcommand{\ev}{\Big|}
\newcommand{\ra}{\rightarrow}
\newcommand{\lra}{\leftrightarrow}
\newcommand{\N}{\mathbb{N}}
\newcommand{\R}{\mathbb{R}}
\newcommand{\Exp}[1]{\mathcal{E}_{#1}}
\newcommand{\List}[1]{\mathcal{L}_{#1}}
\newcommand{\EN}{\Exp{\N}}
\newcommand{\LN}{\List{\N}}
\newcommand{\comment}[1]{}
\newcommand{\lb}{\\~\\}
\newcommand{\eop}{_{\square}}
\newcommand{\hsig}{\hat{\sigma}}
\newcommand{\widesim}[2][1.5]{
	\mathrel{\overset{#2}{\scalebox{#1}[1]{$\sim$}}}
}
\newcommand{\wsim}{\widesim{}}

\begin{document}
\thispagestyle{empty}

\begin{minipage}{2cm}
	\includegraphics[width=2cm]{../../../../img/logo.pdf}
	\vspace{0.5cm}
\end{minipage}
\begin{minipage}{\linewidth}
	\begin{tabular}{lrl}
		{\scriptsize\sc Pontificia Universidad Catolica de Chile} & \hspace*{0.7in}Curso: &
		\sigla  - \nombre\\
		{\sc Facultad de Matemáticas}&
		Profesor: & \profesor \\
		{\sc Semestre \ano-\semestre} & Ayudante: & {Ignacio Castañeda}\\
		& {Mail:} & \texttt{\mail}
	\end{tabular}
\end{minipage}

\vspace{-10mm}
\begin{center}
	{\LARGE\bf \ayudantia}\\
	\vspace{0.1cm}
	{\tituloayu}\\
	\vspace{0.1cm}
	\fecha\\
	\vspace{0.4cm}
\end{center}

\begin{preguntas}
\item Demuestra que la función $f(x) = e^{-x} + ln(x)$ posee al menos una solución.
\begin{solucion}
Evaluemos la función en puntos que podamos identificar claramente el signo.\\
\\
$$f(1) = e^{-1} + ln(1) = \dfrac{1}{e} > 0$$
$$f(0.001) = e^{-0.001} + ln(0.001) < 0$$
Luego, por Teorema del Valor Intermedio, como $f(x)$ es un función continua en $[0.001, 1]$ y sabemos que $f(0.001) < 0$ y $f(1) > 0$, $\exists c \in [0.001, 1]$ tal que $f(c) = 0$, por lo que la función tiene al menos una solución.
\end{solucion}
\item Determine las asíntotas horizontales, verticales y oblicuas de las siguientes funciones, en caso de que existan
\begin{tasks}(2)
\task $f(x) = \dfrac{9-x^2}{x+2}$
\task $f(x) = \dfrac{x^2-5x+6}{x^2+2x-8}$
\end{tasks}
\begin{solucion}

\begin{enumerate}[a)]
\item En primer lugar, veamos las asíntotas verticales. Todas las discontinuidades de la función serán los candidatos a asíntotas verticales.\\
\\
En este caso, esto corresponde solo a $x=-2$. Veamos si el límite existe en $x=-2$
$$\lim\limits_{x\ra -2} \dfrac{9-x^2}{x+2} = \dfrac{5}{0} = \infty = \not \exists$$
Como el límite no existe, entonces la función no posee asíntota vertical.\\
\\
Ahora, veamos las asíntotas horizontales
\item En primer lugar, veamos las asíntotas verticales. Notemos que
$$f(x) = \dfrac{x^2-5x+6}{x^2+2x-8} = \dfrac{(x-2)(x-3)}{(x+4)(x-2)}$$
Luego, nuestros candidatos son $x=2$ y $x=-4$
$$\lim\limits_{x\ra 2} \dfrac{(x-2)(x-3)}{(x+4)(x-2)} = \lim\limits_{x\ra 2} \dfrac{x-3}{x+4} = -\dfrac{1}{6}$$ 
$$\lim\limits_{x\ra -4} \dfrac{(x-2)(x-3)}{(x+4)(x-2)} = \lim\limits_{x\ra -4} \dfrac{x-3}{x+4} = -\dfrac{1}{0} = -\infty = \not\exists$$ 
Por lo tanto, la función posee una asíntota vertical en $x=-4$
\end{enumerate}
\end{solucion}
\item Encuentra la función $g(x)$ que es tangente a la función $f(x) = x^2 + 1$ en el punto $x = 3$.
\begin{solucion}
En primer lugar, debemos obtener la derivada (pendiente) de la función $f(x)$ en el punto $x=3$. Esto es
$$\lim\limits_{h\ra h} \dfrac{f(3+h) - f(3)}{h} =
\lim\limits_{h\ra h} \dfrac{(3+h)^2+1 - 10}{h} =
\lim\limits_{h\ra h} \dfrac{9+6h+h^2+1 - 10}{h}$$
$$=
\lim\limits_{h\ra h} \dfrac{6h+h^2}{h} =
\lim\limits_{h\ra h} \dfrac{h(6+h)}{h} = 
6
$$
Luego, la pendiente en $x=3$ es $6$. Por lo tanto, la función que buscamos será de la forma
$$g(x) = 6x + n$$
Además, sabemos que $g(3) = f(3)$, por lo que evaluando,
$$f(3) = g(3) \ra 10 = 18 + n \ra n = -8$$
Finalmente,
$$g(x) = 6x - 8$$
\end{solucion}
\item Calcular la derivada por definición de las siguientes funciones
\begin{tasks}(2)
\task $f(x) = x+5$
\task $f(x) = 2x^2-3$
\end{tasks}
\begin{solucion}
Recordemos que 
	$$f'(x) = \lim\limits_{h \ra 0} \dfrac{f(x+h) - f(x)}{h}$$
\begin{enumerate}[a)]
\item $$f'(x) = \lim\limits_{h \ra 0} \dfrac{(x+h+5) - (x+5)}{h} =
 \lim\limits_{h \ra 0} \dfrac{x+h+5 -x-5}{h} =
 \lim\limits_{h \ra 0} \dfrac{h}{h} = 1$$
\item $$f'(x) = \lim\limits_{h \ra 0} \dfrac{(2(x+h)^2-3) - (2x^2-3)}{h} =
\lim\limits_{h \ra 0} \dfrac{2(x^2+2xh+h^2)-3-2x^2+3}{h}$$
$$=
\lim\limits_{h \ra 0} \dfrac{2x^2+4xh+2h^2 - 3 -2x^2+3}{h} =
\lim\limits_{h \ra 0} \dfrac{h(4x+2h)}{h} = 
\lim\limits_{h \ra 0} 4x+2h =
4x$$
\end{enumerate}
\end{solucion}
\item Sea $f:[a,b] \ra \R$ una función derivable en $(a,b)$ y sea $x_0 \in (a,b)$ fijo, se define
$$ g(h) = \dfrac{f(x_0 + h) - f(x_0-h)}{2h}$$
Calcular $\lim\limits_{h \ra 0} g(h)$
\begin{solucion}
El límite a calcular es
$$ \lim\limits_{h \ra 0} \dfrac{f(x_0 + h) - f(x_0-h)}{2h}$$
Es evidente que esto se parece a la definición del límite, sin embargo faltan algunos terminos para completarlo.\\
Vamos a restar y sumar el termino $f(x_0)$ al numerador, con lo que obtenemos
$$= \lim\limits_{h \ra 0} \dfrac{f(x_0 + h) - f(x_0) + f(x_0) - f(x_0-h)}{2h}$$
$$= \lim\limits_{h \ra 0} \dfrac{f(x_0 + h) - f(x_0)}{2h} + \lim\limits_{h \ra 0} \dfrac{f(x_0) - f(x_0-h)}{2h}$$
$$= \dfrac{1}{2}\lim\limits_{h \ra 0} \dfrac{f(x_0 + h) - f(x_0)}{h} + \dfrac{1}{2}\lim\limits_{h \ra 0} \dfrac{f(x_0-h) - f(x_0)}{-h}$$
Es evidente que el lado izquierdo es la definición de límite en el punto $x=x_0$. Sin embargo, en el lado izquierdo, debemos realizar el cambio de variable $u = -x$. Luego,
$$= \dfrac{1}{2}f'(x_0) + \dfrac{1}{2}\lim\limits_{u \ra 0} \dfrac{f(x_0+u) - f(x_0)}{u}$$
$$= \dfrac{1}{2}f'(x_0) + \dfrac{1}{2}f'(x_0)$$
Finalmente,
$$\lim\limits_{h \ra 0} g(h) = f'(x_0)$$
\end{solucion}
\item Dada la función
$$f(x) = \begin{cases}
\dfrac{1-x}{1-\sqrt[]{x}} & x > 1\\
\alpha x + \beta & x \leq 1
\end{cases}$$
Determine los valores de $\alpha$ y $\beta$ de manera que $f$ sea derivable en $x=1$
\begin{solucion}
Observemos que para que $f$ sea derivable en $x=1$, $f$ debe ser continua en $x=1$, por lo que se debe cumplir
$$\lim\limits_{x\ra 1}f(x) = f(1) = \alpha  + \beta$$
Para esto, debemos igualar también los límites laterales.
$$\lim\limits_{x\ra 1-} \alpha x + \beta = \lim\limits_{x\ra 1+} \dfrac{1-x}{1-\sqrt[]{x}} $$
$$\alpha + \beta = \lim\limits_{x\ra 1+} \dfrac{1-x}{1-\sqrt[]{x}} \cdot \dfrac{1+\sqrt[]{x}}{1+\sqrt[]{x}} $$
$$\alpha + \beta = \lim\limits_{x\ra 1+} \dfrac{(1-x)(1+\sqrt[]{x})}{1-x}$$
$$\alpha + \beta = \lim\limits_{x\ra 1+} 1+\sqrt[]{x}$$
$$\alpha + \beta = 2$$
Juntando todo, nuestra primera condición es
$$\alpha + \beta = 2$$
Notemos que $f(1) = \alpha + \beta = 2$\\
\\
La otra condición para que sea derivable, es que exista derivada en $x=1$, es decir, debe existir
$$\lim\limits_{h\ra 0} \dfrac{f(1+h) - f(1)}{h}$$
Observe que
\small$$\lim\limits_{h\ra 0+} \dfrac{f(1+h) - f(1)}{h} = 
\lim\limits_{h\ra 0+} \dfrac{\frac{1-(1+h)}{1-\sqrt[]{1+h}}- 2}{h} = 
\lim\limits_{h\ra 0+} \dfrac{\frac{-h(1+\sqrt[]{1+h})}{1-(1+h)}- 2}{h} = 
\lim\limits_{h\ra 0+} \dfrac{\frac{-h(1+\sqrt[]{1-h})}{-h}- 2}{h}  $$
$$= 
\lim\limits_{h\ra 0+} \dfrac{\sqrt[]{1-h} - 1}{h} = 
\lim\limits_{h\ra 0+} \dfrac{1-h - 1}{h(1+\sqrt[]{1-h})} = 
\lim\limits_{h\ra 0+} \dfrac{-h}{h(1+\sqrt[]{1-h})} =
\dfrac{1}{2}$$
Por otra parte
$$\lim\limits_{h\ra 0-} \dfrac{f(1+h) - f(1)}{h} = 
\lim\limits_{h\ra 0-} \dfrac{\alpha(1+h) +\beta - \alpha - \beta}{h} = 
\lim\limits_{h\ra 0-} \dfrac{\alpha h }{h} =
\alpha
$$
Luego se debe cumplir que
$$\alpha = \dfrac{1}{2}$$
Con esto, concluimos que
$$\alpha = \dfrac{1}{2}, \qquad \beta = \dfrac{3}{2}$$
\end{solucion}
\end{preguntas}
\end{document}