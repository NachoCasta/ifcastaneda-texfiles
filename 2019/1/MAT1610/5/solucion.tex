\documentclass[12pt]{article}

\usepackage{fullpage}
\usepackage{graphicx}
\usepackage{amssymb}
\usepackage{amsmath}
\usepackage[none]{hyphenat}
\usepackage{parskip}
\usepackage[spanish]{babel}
\usepackage[utf8]{inputenc}
\usepackage{hyperref}
\usepackage{fancyhdr}
\usepackage{tasks}
\usepackage{mdframed}
\usepackage{xcolor}
\usepackage{pgfplots}
\usepackage[makeroom]{cancel}
\usepackage{multicol}
\usepackage[shortlabels]{enumitem}
\usepackage{stackrel}
\usepackage{tkz-tab}
\usepackage{xpatch}
\xpatchcmd{\tkzTabLine}{$0$}{$\bullet$}{}{}

\setlength{\headheight}{10pt}
\setlength{\headsep}{10pt}
\pagestyle{fancy}
\rhead{\ayudantia \ - \alumno}
\tikzset{t style/.style={style=solid}}

\newcommand*{\mybox}[2]{\colorbox{#1!30}{\parbox{.98\linewidth}{#2}}}

\newenvironment{solucion}
{\begin{mdframed}[backgroundcolor=black!10]
		{\bf Solución:}\\
	}
	{
	\end{mdframed}
}

\newenvironment{alternativas}[1]
{\begin{multicols}{#1}
		\begin{enumerate}[a)]
		}
		{
		\end{enumerate}
	\end{multicols}
}

\newenvironment{preguntas}
{\begin{enumerate}\itemsep12pt
	}
	{
	\end{enumerate}
}

\newcommand{\ayudantia}{{\sc Ayudantía 5}}
\newcommand{\tituloayu}{Derivadas implícitas, derivación logarítmica, derivada de la inversa\\y razón de cambio relacionada}
\newcommand{\fecha}{9 de abril de 2019}
\newcommand{\sigla}{MAT1610}
\newcommand{\nombre}{Cálculo I}
\newcommand{\profesor}{Amal Taarabt}
\newcommand{\ano}{2019}
\newcommand{\semestre}{1}
\newcommand{\mail}{mat1610@ifcastaneda.cl}
\newcommand{\alumno}{Ignacio Castañeda - \mail}

\newcommand{\ev}{\Big|}
\newcommand{\ra}{\rightarrow}
\newcommand{\lra}{\leftrightarrow}
\newcommand{\N}{\mathbb{N}}
\newcommand{\R}{\mathbb{R}}
\newcommand{\Exp}[1]{\mathcal{E}_{#1}}
\newcommand{\List}[1]{\mathcal{L}_{#1}}
\newcommand{\EN}{\Exp{\N}}
\newcommand{\LN}{\List{\N}}
\newcommand{\comment}[1]{}
\newcommand{\lb}{\\~\\}
\newcommand{\eop}{_{\square}}
\newcommand{\hsig}{\hat{\sigma}}
\newcommand{\widesim}[2][1.5]{
	\mathrel{\overset{#2}{\scalebox{#1}[1]{$\sim$}}}
}
\newcommand{\wsim}{\widesim{}}
\newcommand{\lh}{\stackrel{L'H}{=}}

\begin{document}
\thispagestyle{empty}

\begin{minipage}{2cm}
	\includegraphics[width=2cm]{../../../../img/logo.pdf}
	\vspace{0.5cm}
\end{minipage}
\begin{minipage}{\linewidth}
	\begin{tabular}{lrl}
		{\scriptsize\sc Pontificia Universidad Catolica de Chile} & \hspace*{0.7in}Curso: &
		\sigla  - \nombre\\
		{\sc Facultad de Matemáticas}&
		Profesor: & \profesor \\
		{\sc Semestre \ano-\semestre} & Ayudante: & {Ignacio Castañeda}\\
		& {Mail:} & \texttt{\mail}
	\end{tabular}
\end{minipage}

\vspace{-10mm}
\begin{center}
	{\LARGE\bf \ayudantia}\\
	\vspace{0.1cm}
	{\tituloayu}\\
	\vspace{0.1cm}
	\fecha\\
	\vspace{0.4cm}
\end{center}

\begin{preguntas}
\item Calcular $y'$
\begin{tasks}(2)
\task $x^2+y^2-7=0$
\task $x^2y-xy^2+y^2=4$
\end{tasks}
\begin{solucion}
Para responder estas preguntas, debemos mirar la derivación de la siguiente forma. Si tenemos por ejemplo 
$$f(x) = x^2$$
y queremos derivar esta función, debemos aplicar la regla de la cadena, por lo que
$$f'(x) = 2x \cdot x' = 2x \cdot 1 = 2x$$
Esto ocurre porque estamos derivando en función de $x$, por lo que $$\dfrac{\delta}{\delta x} x = 1$$
Sin embargo, si estuvieramos derivando otra variable que no sea $x$ en función de $x$, la derivada no es necesariamente $1$, por lo que habría que dejarlo expresa, por ejemplo
$$\dfrac{\delta}{\delta x} y = y'$$
En resumen, cuando estemos derivando algo que tenga un $y$ (o algo que no se este derivando en función de su misma variable), cuando hayamos terminado de derivar aplicamos el último paso de la regla de la cadena (que usualmente omitimos porque la derivada de $x$ es 1) y lo dejamos expresado como $y'$
\begin{enumerate}[a)]
\item $x^2+y^2-7=0$\\
\\
Derivamos toda la igualdad en función de $x$, esto es
$$2x + 2yy' = 0 $$
Y luego, despejamos $y'$
$$y' = -\dfrac{x}{y}$$
\item $x^2y-xy^2+y^2=4$\\
\\
Derivamos en función de $x$,
$$(x^2y)'-(xy^2)'+(y^2)'=0$$
$$2xy + x^2y'-y^2 - 2yy'x+2yy'=0$$
Dejamos todo lo que tenga $y'$ a un lado,
$$x^2y' - 2yy'x+2yy'= -2xy +y^2$$
$$y'(x^2 - 2yx+2y)= -2xy +y^2$$
Finalmente,
$$y'= \dfrac{-2xy +y^2}{x^2 - 2yx+2y}$$
\end{enumerate}
\end{solucion}
\item Sea
$$-x^2+xy+y^2=1$$
Encuentre el valor de $y''$ cuando $(x,y)=(1,1)$
\begin{solucion}
Derivando implícitamente la ecuación, obtenemos
$$-2x + y + xy' + 2yy' = 0$$
Luego,
$$xy' + 2yy' = 2x-y$$
$$y' = \dfrac{2x-y}{x+2y}$$
Esto solo será válido cuando $x+2y\neq 0$.\\

Ahora debemos derivar nuevamente para obtener la segunda derivada implícita. Es recomendable hacer esto en la igualdad antes de despejar $y'$, ya que de esa manera no tendrémos que aplicar la regla de la división. \\

Entonces, tomamos
$$xy' + 2yy' = 2x-y$$
Y derivamos implícitamente,
$$y' + xy'' + 2y'^2+2yy'' = 2 - y'$$
Luego, despejamos $y''$,
$$xy''+2yy'' = 2 - y' - y' - 2y'^2$$
$$y''(x+2y) = 2 - 2y' - 2y'^2$$
$$y'' = \dfrac{2 - 2y' - 2y'^2}{x+2y}$$
Ahora, notemos que
$$y'(1,1) = \dfrac{2-1}{1+2} = \dfrac{1}{3}$$
Luego,
$$y''(1,1) = \dfrac{2-2\dfrac{1}{3} - 2\left(\dfrac{1}{3}\right)^2}{1+2} 
= \dfrac{2-\dfrac{2}{3}-\dfrac{2}{9}}{3}
= \dfrac{2-\dfrac{2}{3}-\dfrac{2}{9}}{3}
= \dfrac{10}{27}
$$
\end{solucion}
\item Encuentre las derivadas de las siguientes funciones
\begin{tasks}(2)
\task $f(x) = x^{\sin x}$
\task $f(x) = (x^2+3)^{5x-1}$
\end{tasks}
\begin{solucion}
Para calcular estas derivadas, usaremos la técnica de derivación logarítmica. Esta consiste en aplicar logaritmo natural a ambos lados de la función, derivar implícitamente y despejar $y'$, obteniendo así la derivada de la función.
\begin{enumerate}[a)]
\item $f(x) = x^{\sin x}$\\
\\
Podemos expresar esta función como
$$y = x^{\sin x}$$
Aplicamos logaritmo,
$$\ln y = \ln(x^{\sin x})$$
$$\ln y = \sin x\ \ln x$$
Derivamos implícitamente,
$$\dfrac{y'}{y} = \cos x \ln x + \dfrac{\sin x}{\ln x}$$
$$y' = y(\cos x \ln x + \dfrac{\sin x}{\ln x})$$
Finalmente, reemplazamos con $y = x^{\sin x}$,
$$y' = x^{\sin x}(\cos x \ln x + \dfrac{\sin x}{\ln x})$$
\item $f(x) = (x^2+3)^{5x-1}$\\
\\
Igual que antes, escribimos la función como
$$y = (x^2+3)^{5x-1}$$
Aplicamos logaritmo
$$\ln y = \ln((x^2+3)^{5x-1})$$
$$\ln y = (5x-1)\ln(x^2+3)$$
Derivamos implícitamente,
$$\dfrac{y'}{y} = 5\ln(x^2+3) + \dfrac{(5x-1)2x}{x^2+3}$$
$$y' = y(5\ln(x^2+3) + \dfrac{(5x-1)2x}{x^2+3})$$
Finalmente,
$$y' = (x^2+3)^{5x-1}(5\ln(x^2+3) + \dfrac{(5x-1)2x}{x^2+3})$$
\end{enumerate}
\end{solucion}
\item Dada la función invertible $f(x) = x^3 + 3x + 6$, calcular $(f^{-1})'(6)$
\begin{solucion}
Recordemos que 
$$(f^{-1})'(x) = \dfrac{1}{f'(f^{-1}(x))}$$
Luego,
$$(f^{-1})'(6) = \dfrac{1}{f'(f^{-1}(6))}$$
Para obtener $f^{-1}(6)$ debemos preguntarnos: ¿Que $x$ hace que $f(x) = 6$? Analíticamente podemos ver que $f(0) = 6$, por lo que $f^{-1}(6) = 0$.\\

Entonces, 
$$(f^{-1})'(6) = \dfrac{1}{f'(0)}$$
Derivando,
$$f'(x) = 3x^2+3 \ra f'(0) = 3$$
Finalmente,
$$(f^{-1})'(6) = \dfrac{1}{3}$$
\end{solucion}
\item Sean $f$ y $g$ funciones derivables e invertibles. La tabla adjunta muestra los valores de $f$, $g$ y sus derivadas sobre algunos valores. Calcule $(f \circ f)'(3)$ y $(g^{-1} \circ f)'(2)$
$$
\begin{tabular}{|l|l|l|l|l|}
\hline
  & f & g & f' & g' \\ \hline
1 & 3 & 2 & 4  & 1  \\ \hline
2 & 1 & 3 & 3  & 2  \\ \hline
3 & 2 & 4 & 2  & 4  \\ \hline
4 & 4 & 1 & 1  & 2  \\ \hline
\end{tabular}
$$
\begin{solucion}
En primer lugar,
$$(f \circ f)'(3) = (f(f(3)))' = f'(f(3))\cdot f'(3) = f'(2) \cdot f'(3) = 3 \cdot 2 = 6$$
En segundo lugar,	
$$(g^{-1} \circ f)'(2) = (g^{-1}(f(2)))' = (g^{-1})'(f(2)) \cdot f'(2) = (g^{-1})'(1)\cdot 3 = 3\dfrac{1}{g'(g^{-1}(1))}$$
Para obtener $g^{-1}(1)$ miramos la tabla y vemos que valor de $x$ hace que $g(x) = 1$. Esto corresponde a $x=4$, por lo que
$$(g^{-1} \circ f)'(2) = 3\dfrac{1}{g'(4)} = \dfrac{3}{2}$$
\end{solucion}
\item En un estanque de forma cónica, con radio basal 5 $m$ y altura 10 $m$, con el vértice hacia abajo, se hace entrar agua a razón de 9 $m^3/min$. ¿Cuán rápido varía el nivel del agua cuando esta tiene una profundidad de 6 metros?
\begin{solucion}
Acorde al enunciado, lo que tenemos es
\begin{center}
\includegraphics[]{cono.PNG}
\end{center}
Los datos que tenemos son los siguientes:\\
En primer lugar, sabemos que el agua varía 9 $m^3/min$, es decir $\dfrac{dV}{dt} = 9$\\
Lo que debemos determinar es $\dfrac{dh}{dt}$ cuando $h=6$\\
\\
Recordemos que el volumen de un cono se define como
$$V = \dfrac{\pi r^2 h}{3}$$
Además, sabemos por las dimensiones del cono, que
$$\dfrac{r}{h} = \dfrac{5}{10}$$
Por lo que 
$$r = \dfrac{h}{2}$$
Luego,
$$V = \dfrac{\pi h^3}{12}$$
Derivando esto implícitamente en función de $t$ (notar que ambas variables habrá que derivarlas en otra variable), tenemos
$$\dfrac{dV}{dt} = \dfrac{\pi}{12}3h^2\dfrac{dh}{dt}$$
Despejando $\dfrac{dh}{dt}$,
$$\dfrac{dh}{dt} = \dfrac{4}{\pi h^2} \cdot \dfrac{dV}{dt}$$
Finalmente, reemplazamos con $\dfrac{dV}{dt} = 9$ y $h = 6$, obteniendo
$$\dfrac{dh}{dt} = \dfrac{4}{36\pi}\cdot 9 = \dfrac{1}{\pi} m/min$$
\end{solucion}
\end{preguntas}
\end{document}