\documentclass[12pt]{article}

\usepackage{fullpage}
\usepackage{graphicx}
\usepackage{amssymb}
\usepackage{amsmath}
\usepackage[none]{hyphenat}
\usepackage{parskip}
\usepackage[spanish]{babel}
\usepackage[utf8]{inputenc}
\usepackage{hyperref}
\usepackage{fancyhdr}
\usepackage{tasks}
\usepackage{mdframed}
\usepackage{xcolor}
\usepackage{pgfplots}
\usepackage[makeroom]{cancel}
\usepackage{multicol}
\usepackage[shortlabels]{enumitem}
\usepackage{stackrel}
\usepackage{tkz-tab}
\usepackage{xpatch}
\xpatchcmd{\tkzTabLine}{$0$}{$\bullet$}{}{}

\setlength{\headheight}{10pt}
\setlength{\headsep}{10pt}
\pagestyle{fancy}
\rhead{\ayudantia \ - \alumno}
\tikzset{t style/.style={style=solid}}

\newcommand*{\mybox}[2]{\colorbox{#1!30}{\parbox{.98\linewidth}{#2}}}

\newenvironment{solucion}
{\begin{mdframed}[backgroundcolor=black!10]
		{\bf Solución:}\\
	}
	{
	\end{mdframed}
}

\newenvironment{alternativas}[1]
{\begin{multicols}{#1}
		\begin{enumerate}[a)]
		}
		{
		\end{enumerate}
	\end{multicols}
}

\newenvironment{preguntas}
{\begin{enumerate}\itemsep12pt
	}
	{
	\end{enumerate}
}

\newcommand{\ayudantia}{{\sc Ayudantía 11}}
\newcommand{\tituloayu}{Sustitución y área entre curvas}
\newcommand{\fecha}{28 de mayo de 2019}
\newcommand{\sigla}{MAT1610}
\newcommand{\nombre}{Cálculo I}
\newcommand{\profesor}{Amal Taarabt}
\newcommand{\ano}{2019}
\newcommand{\semestre}{1}
\newcommand{\mail}{mat1610@ifcastaneda.cl}
\newcommand{\alumno}{Ignacio Castañeda - \mail}

\newcommand{\ev}{\Big|}
\newcommand{\ra}{\rightarrow}
\newcommand{\lra}{\leftrightarrow}
\newcommand{\N}{\mathbb{N}}
\newcommand{\R}{\mathbb{R}}
\newcommand{\Exp}[1]{\mathcal{E}_{#1}}
\newcommand{\List}[1]{\mathcal{L}_{#1}}
\newcommand{\EN}{\Exp{\N}}
\newcommand{\LN}{\List{\N}}
\newcommand{\comment}[1]{}
\newcommand{\lb}{\\~\\}
\newcommand{\eop}{_{\square}}
\newcommand{\hsig}{\hat{\sigma}}
\newcommand{\widesim}[2][1.5]{
	\mathrel{\overset{#2}{\scalebox{#1}[1]{$\sim$}}}
}
\newcommand{\wsim}{\widesim{}}
\newcommand{\lh}{\stackrel{L'H}{=}}

\begin{document}
\thispagestyle{empty}

\begin{minipage}{2cm}
	\includegraphics[width=2cm]{../../../../img/logo.pdf}
	\vspace{0.5cm}
\end{minipage}
\begin{minipage}{\linewidth}
	\begin{tabular}{lrl}
		{\scriptsize\sc Pontificia Universidad Catolica de Chile} & \hspace*{0.7in}Curso: &
		\sigla  - \nombre\\
		{\sc Facultad de Matemáticas}&
		Profesor: & \profesor \\
		{\sc Semestre \ano-\semestre} & Ayudante: & {Ignacio Castañeda}\\
		& {Mail:} & \texttt{\mail}
	\end{tabular}
\end{minipage}

\vspace{-10mm}
\begin{center}
	{\LARGE\bf \ayudantia}\\
	\vspace{0.1cm}
	{\tituloayu}\\
	\vspace{0.1cm}
	\fecha\\
	\vspace{0.4cm}
\end{center}

\begin{preguntas}
\item Resolver las siguientes integrales indefinidas
\begin{tasks}(2)
\task $\displaystyle\int \dfrac{e^{ln(x)}}{x^2+7}dx$
\task $\displaystyle\int (x+2)sen(x^2+4x-6)dx$
\task $\displaystyle\int \dfrac{3x}{\sqrt[3]{x^2+3}}dx$
\task $\displaystyle\int \dfrac{e^x-1}{e^x+1}dx$
\end{tasks}
\begin{solucion}

\begin{enumerate}[a)]
\item $\displaystyle\int \dfrac{e^{ln(x)}}{x^2+7}dx = \displaystyle\int \dfrac{x}{x^2+7}dx$\\
			Usamos la sustitución $u = x^2+7 \ra du = 2xdx$
			$$= \dfrac{1}{2} \displaystyle\int \dfrac{2}{u}dx = \dfrac{1}{2} ln(u) + c = \dfrac{1}{2}ln(x^2+7) + c$$
\item $\displaystyle\int (x+2)sen(x^2+4x-6)dx$\\
			Utilizando la sustitución $u = x^2+4x-6 \ra du = 2x + 4 \ra du = 2(x+2)$
			$$=\dfrac{1}{2}\displaystyle\int sen(u)du = -\dfrac{1}{2}cos(u) = -\dfrac{1}{2}cos(x^2+4x-6)+c$$
\item $\displaystyle\int \dfrac{3x}{\sqrt[3]{x^2+3}}dx$\\
			La sustitución a usar es $u=x^2+3 \ra du=2xdx$
			$$= \dfrac{3}{2} \displaystyle\int \dfrac{1}{\sqrt[3]{u}}du = \dfrac{3}{2} \displaystyle\int \dfrac{1}{u^{1/3}}du = \dfrac{3}{2} \displaystyle\int u^{-1/3}du = \dfrac{3}{2}\dfrac{u^{2/3}}{\frac{2}{3}} + c = \dfrac{9}{4}(t^2+3)^{2/3}+c$$
\item $\displaystyle\int \dfrac{e^x-1}{e^x+1}dx = \displaystyle\int \dfrac{e^x}{e^x+1}dx - \displaystyle\int \dfrac{1}{e^x+1}dx$\\
			Resolvamos ambas por separado. En primer lugar,
			$$ \displaystyle\int \dfrac{e^x}{e^x+1}dx$$
			Utilizamos la sustitución $u=e^x+1 \ra du=e^x$
			$$= \displaystyle\int \dfrac{du}{u} = ln(u) + c = ln(e^x+1)+c$$
			Ahora, resolveremos
			$$-\displaystyle\int \dfrac{1}{e^x+1}dx = -\displaystyle\int \dfrac{e^xe^{-x}}{e^x+1}dx= -\displaystyle\int \dfrac{e^{-x}}{e^{-x}(e^x+1)}dx = -\displaystyle\int \dfrac{e^{-x}}{1+e^{-x}}dx$$
			Usando $u=1+e^{-x} \ra du = -e^{-x}$
			$$= \displaystyle\int \dfrac{du}{u}dx = ln(u) +c = ln(1+e^{-x})+c$$
			Por último, al sumar ambas obtenemos
			$$\displaystyle\int \dfrac{e^x-1}{e^x+1}dx = ln(1+e^x) + ln(1+e^{-x}) + c = ln((1+e^x)(1+e^{-x}))+c$$
\end{enumerate}
\end{solucion}
\item Resolver las siguientes integrales definidas
\begin{tasks}(2)
\task $\displaystyle\int_0^2 \dfrac{x^2}{x^3+8} dx$
\task $\displaystyle\int_{-1}^1 xsen(1-x^2)dx$
\end{tasks}
\begin{solucion}

\begin{enumerate}[a)]
\item $\displaystyle\int_0^2 \dfrac{x^2}{x^3+8} dx$\\
			Usando la sustitución
			$$u=x^3+8 \ra du =  3x^2dx$$
			$$x \in (0, 2) \ra u \in (8, 16)$$
			$$\displaystyle\int_0^2 \dfrac{x^2}{x^3+8}dx = \dfrac{1}{3}\displaystyle\int_8^{16} \dfrac{du}{u} = \dfrac{1}{3}ln(u) \ev_8^{16} = \dfrac{1}{3}(ln(16)-ln(8)) = \dfrac{1}{3}ln\left(\dfrac{16}{8}\right) = \dfrac{1}{3}ln(2)$$
\item $\displaystyle\int_{-1}^1 xsen(1-x^2)dx$\\
			Usamos el cambio de variable
			$$u = 1-x^2 \ra du = -2xdx$$
			$$x \in (-1, 1) \ra u \in (0, 0)$$
			Como la variable $u$ va de 0 a 0, entonces la integral es igual a 0. Luego,
			$$\displaystyle\int_{-1}^1 xsen(1-x^2)dx = 0$$
\end{enumerate}
\end{solucion}
\item Calule el área de la región acotada por las curvas $y=x^2-4$ e $y = -x^2-2x$ en el intervalo $I =  [-3.3]$.
\begin{solucion}
Con ayuda del gráfico
		\begin{center}
			\begin{tikzpicture}
			\begin{axis}[
			axis lines = left,
			xlabel = $x$,
			ylabel = $y$,
			]
			\addplot [
			domain=-3:3,  
			color=red,
			]
			{x^2-4};
			\addplot [
			domain=-3:3, 
			color=blue,
			]
			{-x^2-2*x};
			
			\end{axis}
			\end{tikzpicture}
		\end{center}
		Podemos ver que el área acotada por las curvas la podemos separar en 3, quedandonos
		$$A = \displaystyle\int_{-3}^{-2}[x^2-4-(-x^2-2x)]dx 
		+ \displaystyle\int_{-2}^{1}[-x^2-2x-(x^2-4)]dx 
		+ \displaystyle\int_{1}^{3}[x^2-4-(-x^2-2x)]dx$$
		$$A = \displaystyle\int_{-3}^{-2}(2x^2+2x-4)dx 
		+ \displaystyle\int_{-2}^{1}(4-2x-2x^2)dx 
		+ \displaystyle\int_{1}^{3}(2x^2+2x-4)dx$$
		$$A = \dfrac{11}{3} + 9 + \dfrac{52}{3} = \dfrac{90}{3} = 30$$
\end{solucion}
\item Encuentre el área acotada por las curvas por las curvas $y=x(x^2+11)$ e $y=6(x^2+1)$.
\begin{solucion}
Buscamos los puntos de intersección de ambas curvas, igualandolas
		$$x(x^2+11) = 6(x^2+1)$$
		$$x(x^2+11) - 6(x^2+1) = 0$$
		$$x^3-6x^2+11x-6 = 0$$
		$$(x-1)(x-2)(x-3)=0$$
		Tenemos entonces que las curvas se intersectan en $x=1, x=2$ y $x=3$.\\
		Necesitamos ver ahora cual curva va arriba y cual abajo. Notemos que
		$$x^3-6x^2+11x-6 \geq 0\quad si \qquad 1 \leq x \leq 2$$
		$$x^3-6x^2+11x-6 \leq 0\quad si \qquad 2 \leq x \leq 3$$
		Con lo que concluimos que en $[1,2]$ la curva $y=x(x^2+11)$ va sobre la curva $y=6(x^2+1)$ y que en $[2,3]$ pasa lo contrario.\\
		Finalmente,
		$$A = \displaystyle\int_{1}^{2}[x(x^2+11) - 6(x^2+1)]dx 
		+ \displaystyle\int_{2}^{3}[6(x^2+1) - x(x^2+11)]dx$$ 
		$$A = \displaystyle\int_{1}^{2}(x^3-6x^2+11x-6)dx 
		+ \displaystyle\int_{2}^{3}(6-11x+6x^2-x^3)dx$$ 
		$$A = \dfrac{1}{4} + \dfrac{1}{4} = \dfrac{1}{2}$$
\end{solucion}
\item Hallar el área de la elipse
	$$\dfrac{x^2}{9} + \dfrac{y^2}{4} = 1$$
\begin{solucion}
En primer lugar, despejemos $y$, es decir
		$$y = \pm\dfrac{2\ \sqrt[]{9 - x^2}}{3}$$
		Para encontrar el área, podemos buscar el área de un cuarto de la elipse y multiplicarla por 4, es decir
		$$A = 4\displaystyle\int_0^3 \dfrac{2\ \sqrt[]{9 - x^2}}{3}dx$$
		Utilizamos el cambio de variable $x =3sen(\theta) \ra dx = 3cos(\theta)d\theta$
		$$A = 4\displaystyle\int_0^{\pi/2} \dfrac{2\ \sqrt[]{9 - 9sen^2(\theta)}}{3} 3cos(\theta)d\theta$$
		$$A = 4\displaystyle\int_0^{\pi/2} \dfrac{2 \cdot 3cos(\theta)}{3} 3cos(\theta)d\theta$$
		$$A = 24\displaystyle\int_0^{\pi/2} cos^2(\theta)d\theta$$
		$$A = 24\displaystyle\int_0^{\pi/2} \dfrac{1+cos(2\theta)}{2}d\theta$$
		$$A = 24 \cdot \dfrac{\pi}{4}$$
		$$A = 6 \pi$$
\end{solucion}
\end{preguntas}
\end{document}