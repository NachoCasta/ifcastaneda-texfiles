\documentclass[12pt]{article}

\usepackage{fullpage}
\usepackage{graphicx}
\usepackage{amssymb}
\usepackage{amsmath}
\usepackage[none]{hyphenat}
\usepackage{parskip}
\usepackage[spanish]{babel}
\usepackage[utf8]{inputenc}
\usepackage{hyperref}
\usepackage{fancyhdr}
\usepackage{tasks}
\usepackage{mdframed}
\usepackage{xcolor}
\usepackage{pgfplots}
\usepackage[makeroom]{cancel}
\usepackage{multicol}
\usepackage[shortlabels]{enumitem}
\usepackage{stackrel}
\usepackage{tkz-tab}
\usepackage{xpatch}
\xpatchcmd{\tkzTabLine}{$0$}{$\bullet$}{}{}

\setlength{\headheight}{10pt}
\setlength{\headsep}{10pt}
\pagestyle{fancy}
\rhead{\ayudantia \ - \alumno}
\tikzset{t style/.style={style=solid}}

\newcommand*{\mybox}[2]{\colorbox{#1!30}{\parbox{.98\linewidth}{#2}}}

\newenvironment{solucion}
{\begin{mdframed}[backgroundcolor=black!10]
		{\bf Solución:}\\
	}
	{
	\end{mdframed}
}

\newenvironment{alternativas}[1]
{\begin{multicols}{#1}
		\begin{enumerate}[a)]
		}
		{
		\end{enumerate}
	\end{multicols}
}

\newenvironment{preguntas}
{\begin{enumerate}\itemsep12pt
	}
	{
	\end{enumerate}
}

\newcommand{\ayudantia}{{\sc Ayudantía 6}}
\newcommand{\tituloayu}{Máximos y mínimos. TVM}
\newcommand{\fecha}{16 de abril de 2019}
\newcommand{\sigla}{MAT1610}
\newcommand{\nombre}{Cálculo I}
\newcommand{\profesor}{Amal Taarabt}
\newcommand{\ano}{2019}
\newcommand{\semestre}{1}
\newcommand{\mail}{mat1610@ifcastaneda.cl}
\newcommand{\alumno}{Ignacio Castañeda - \mail}

\newcommand{\ev}{\Big|}
\newcommand{\ra}{\rightarrow}
\newcommand{\lra}{\leftrightarrow}
\newcommand{\N}{\mathbb{N}}
\newcommand{\R}{\mathbb{R}}
\newcommand{\Exp}[1]{\mathcal{E}_{#1}}
\newcommand{\List}[1]{\mathcal{L}_{#1}}
\newcommand{\EN}{\Exp{\N}}
\newcommand{\LN}{\List{\N}}
\newcommand{\comment}[1]{}
\newcommand{\lb}{\\~\\}
\newcommand{\eop}{_{\square}}
\newcommand{\hsig}{\hat{\sigma}}
\newcommand{\widesim}[2][1.5]{
	\mathrel{\overset{#2}{\scalebox{#1}[1]{$\sim$}}}
}
\newcommand{\wsim}{\widesim{}}
\newcommand{\lh}{\stackrel{L'H}{=}}

\begin{document}
\thispagestyle{empty}

\begin{minipage}{2cm}
	\includegraphics[width=2cm]{../../../../img/logo.pdf}
	\vspace{0.5cm}
\end{minipage}
\begin{minipage}{\linewidth}
	\begin{tabular}{lrl}
		{\scriptsize\sc Pontificia Universidad Catolica de Chile} & \hspace*{0.7in}Curso: &
		\sigla  - \nombre\\
		{\sc Facultad de Matemáticas}&
		Profesor: & \profesor \\
		{\sc Semestre \ano-\semestre} & Ayudante: & {Ignacio Castañeda}\\
		& {Mail:} & \texttt{\mail}
	\end{tabular}
\end{minipage}

\vspace{-10mm}
\begin{center}
	{\LARGE\bf \ayudantia}\\
	\vspace{0.1cm}
	{\tituloayu}\\
	\vspace{0.1cm}
	\fecha\\
	\vspace{0.4cm}
\end{center}

\begin{preguntas}
\item Para cada una de las siguientes funciones, encontrar sus máximos y mínimos locales, en caso de haber
\begin{tasks}(2)
\task $f(x) = x^4 - 8x^2 + 3$
\task $f(x)=x+ln(x^2-1)$
\end{tasks}
\begin{solucion}

\begin{enumerate}[a)]
\item $f(x) = x^4 - 8x^2 + 3$\\
\\
En primer lugar, buscamos los puntos críticos, esto es
$$f'(x) = 4x^3 - 16x = 0$$
$$4x(x^2 - 4) = 0$$
$$4x(x+2)(x-2) = 0$$
$$x = -2, \qquad x = 0, \qquad x = 2$$
Luego, obtenemos la segunda derivada,
$$f''(x) = 12x^2 - 16$$
Y evaluamos en los puntos críticos para clasificarlos
$$f''(-2) = 48 - 16 > 0 \ra \text{mínimo local}$$
$$f''(0) = -16 < 0 \ra \text{máximo local}$$
$$f''(2) = 48 - 16 > 0 \ra \text{mínimo local}$$
Finalmente, evaluamos estos puntos en la función original para obtener los máximos y mínimos locales
$$f(-2) = -13 \ra \text{mínimo local}$$
$$f(0) = 3 \ra \text{máximo local}$$
$$f(2) = -13 \ra \text{mínimo local}$$
\item $f(x)=x+ln(x^2-1)$\\
\\
En primer lugar, notemos que el dominio de $f$ corresponde a 
$$x^2-1 > 0 \ra x \in (-\infty, -1) \cup (1, \infty)$$
Luego, derivamos para encontrar los puntos críticos,
$$f'(x) = 1 + \dfrac{2x}{x^2-1} = \dfrac{x^2 +2x -1}{x^2-1} = 0$$
Utilizando la formula de la ecuación de segundo grado,
$$x = -1 + \sqrt[]{2}, \qquad x = -1 - \sqrt[]{2}$$
Sin embargo, notemos que $-1 < -1 + \sqrt[]{2} < 1$, por lo que no es un punto crítico, dado que no esta en el dominio de la función.

Ahora, derivamos nuevamente para clasíficar nuestro punto crítico,
$$f''(x) = 
\dfrac{2(x^2-1) - 4x^2}{(x^2-1)^2} = 
\dfrac{-2 - 2x^2}{(x^2-1)^2} < 0$$
$f''(x)$ es negativa para todo $x$, por lo que $x = -1 - \sqrt[]{2}$ es un máximo local.
\end{enumerate}
\end{solucion}
\item Encuentre los máximos y mínimos globales de la función
$$f(x) = \dfrac{|x|}{1+x^2}$$
en el intervalo $x \in [-3,2]$
\begin{solucion}
Notemos que podemos escribir $f(x)$ como
$$f(x)= \begin{cases}
\dfrac{x}{1+x^2} & x \geq 0\\\\
\dfrac{-x}{1+x^2} & x < 0
\end{cases}$$
Esta función no es derivable en $x=0$, sin embargo, podemos derivar a ambos lados y luego ver que ocurre. Entonces, derivando ambos lados por separado, tenemos que
$$x > 0 \ra f'(x) = \dfrac{(1+x^2) - 2x^2}{(1+x^2)^2} = \dfrac{1-x^2}{(1+x^2)^2} = \dfrac{(1+x)(1-x)}{(1+x^2)^2}$$
$$x < 0 \ra f'(x) = \dfrac{-(1+x^2) + 2x^2}{(1+x^2)^2} = \dfrac{x^2-1}{(1+x^2)^2} = \dfrac{(x+1)(x-1)}{(1+x^2)^2}$$
De aquí, obtenemos los puntos $x=1$ y $x=-1$ como puntos críticos. Además, debemos agregar a los puntos críticos, los bordes de nuestro intervalo y el punto donde la función no es derivable, es decir, $x=-3$, $x=2$ y $x=0$.\\

Ahora, evaluamos en cada uno de ellos, obteniendo
$$f(-3) = \dfrac{3}{10},\ f(-1) = \dfrac{1}{2}, \ f(0) = 0, \ f(1) = \dfrac{1}{2}, \ f(2) = \dfrac{2}{5}$$
Finalmente, el máximo es $\dfrac{1}{2}$ y el mínimo es $0$.
\end{solucion}
\item Estudiar, según el valor de la constante $k$, los puntos críticos y los puntos de inflexión de la función
$$f(x) = x^3-3kx^2+12x$$
\begin{solucion}
En primer lugar, calculemos las derivadas de $f$, esto es
$$f'(x) = 3x^2 - 6kx + 12$$
$$f''(x) = 6x - 6k$$
Para buscar los puntos críticos, igualamos la primera derivada a 0, con lo que
$$3x^2 - 6kx + 12 = 0 \ra x^2 - 2kx + 4= 0 \ra x = \dfrac{2k \pm \sqrt[]{4k^2-16}}{2} \ra x = k \pm \sqrt[]{k^2-4}$$
Con respecto a los puntos de inflexión, igualando la segunda derivada a 0 podemos ver que
$$6x- 6k = 0 \ra x = k,$$
por lo que siempre habrá un punto de inflexión en $x=k$\\

Luego, tendremos los siguientes escenarios.\\

Para $k \in (-2,2)$, no habrá ningún punto crítico, ya que la ecuación anterior no tiene solución.\\

Para $k = 2 y k = -2$, habrá un punto crítico en $x=k$. Además, este punto crítico será un punto de inflexión.\\

Para $k\in (-\infty, -2) \cup (2, \infty)$, habrán dos puntos críticos en 
$$x = k + \sqrt[]{k^2-4} \qquad y \qquad x = k - \sqrt[]{k^2-4}$$
Notemos además que para $x < k \ra f''(x) < 0$ y para $x > k \ra f''(x) > 0$, por lo que el punto crítico $x = k + \sqrt[]{k^2-4} > k$ corresponderá a un mínimo y $x = k - \sqrt[]{k^2-4} < k$ corresponderá a un máximo.
\end{solucion}
\item Probar que la ecuación $1+2x+3x^2+4x^3=0$ tiene solución única.
\begin{solucion}
Para demostrar esto debemos hacer dos cosas. En primer lugar, debemos demostrar que la ecuación tiene alguna solución (TVI) y en segundo lugar, debemos demostrar que esta no tiene más de una solución (TVM).\\

Para demostrar que la ecuación tiene alguna solución, definimos la función auxiliar
$$f(x) = 1+2x+3x^2+4x^3$$
Al evaluar, podemos ver que
$$f(0) = 1, \qquad f(-1) = -2$$
Notemos que $f$ es una función continua en todos los reales. Luego, por TVI, 
$$\exists c \in (-1,0) \text{ tal que } f(c) = 0$$
Para demostrar que la ecuación no tiene más de una solución, lo haremos por contradicción. Digamos que la ecuación tiene 2 soluciones, $x_1$ y $x_2$ con $x_1 < x_2$.\\

Luego, $f(x_1) = 0$ y $f(x_2) = 0$. Entonces, por TVM
$$\exists c \in (x_1, x_2) \text{ tal que } f'(c) = \dfrac{f(x_2) - f(x_1)}{x_2-x_1} = 0$$
Sin embargo, notemos que
$$f'(x) = 12x^2 + 6x + 2$$
Corresponde a una ecuación de segundo grado con determinante 
$$\Delta = 6^2-4\cdot 12 \cdot 2 = -60 < 0,$$
por lo que no tiene solución real. Esto es una contradicción con nuestra suposición anterior, por lo que la ecuación no puede tener dos soluciones.\\

En conclusión, la ecuación tiene solución única. 
$$\blacksquare$$
\end{solucion}
\item Sea $f$ una función derivable en $[0, \infty)$, tal que $\forall x \in [0, \infty)(f(2x)=2f(x))$.\\
Demuestre que
$$\forall x \in (0, \infty) \exists c > 0 \left(f'(c) = \dfrac{f(x)}{x}\right)$$
\begin{solucion}
La función $f$ es derivable en el intervalo $[x, 2x]$ para $x \geq 0$. Luego, por TVM,
$$\exists c \in (x,2x) \text{ tal que } f'(c) = \dfrac{f(2x)-f(x)}{2x-x} = \dfrac{2f(x)-f(x)}{x} = \dfrac{f(x)}{x}$$
Por lo tanto,
$$\forall x \in (0, \infty) \exists c > 0 \left(f'(c) = \dfrac{f(x)}{x}\right)$$
\end{solucion}
\item Si $f$ es una función dos veces derivable en $[a,b]$ tal que $f(a)=f(b)=0$ y $f(c) > 0$ con $a < c < b$, demuestre que
$$\exists \alpha \in (a,b) (f''(\alpha) < 0)$$
\begin{solucion}
Como $f$ es dos veces derivable en $[a,b]$ y $a < c < b$, entonces $f$ cumple con las condiciones del TVM en los intervalos $[a,c]$ y $[c,b]$, Aplicando TVM en ambos, tenemos que
$$\exists \alpha_1 \in (a,c) \text{ tal que } f'(\alpha_1) = \dfrac{f(c)-f(a)}{c-a} = \dfrac{f(c)}{c-a} > 0$$
$$\exists \alpha_2 \in (c,b) \text{ tal que } f'(\alpha_2) = \dfrac{f(b)-f(c)}{b-c} = -\dfrac{f(c)}{b-c} < 0$$
Notemos ahora que 
$$a < \alpha_1 < c < \alpha_2 < b$$
Además, como $f''$ es derivable en $[a, b]$, las condiciones del TVM también se cumplen para $f''$ en el intervalo $[\alpha_1, \alpha_2]$. Luego, por TVM,
$$\exists \alpha \in (\alpha_1, \alpha_2) \text{ tal que } f''(\alpha) = \dfrac{f'(\alpha_2) - f'(\alpha_1)}{\alpha_2 - \alpha_1} < 0$$
Esto último lo sabemos por los signos de $f'(\alpha_1)$ y $f'(\alpha_2)$ que fueron concluidos anteriormente.
$$\blacksquare$$
\end{solucion}
\end{preguntas}
\end{document}