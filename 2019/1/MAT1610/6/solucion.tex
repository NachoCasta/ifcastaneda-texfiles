\documentclass[12pt]{article}

\usepackage{fullpage}
\usepackage{graphicx}
\usepackage{amssymb}
\usepackage{amsmath}
\usepackage[none]{hyphenat}
\usepackage{parskip}
\usepackage[spanish]{babel}
\usepackage[utf8]{inputenc}
\usepackage{hyperref}
\usepackage{fancyhdr}
\usepackage{tasks}
\usepackage{mdframed}
\usepackage{xcolor}
\usepackage{pgfplots}
\usepackage[makeroom]{cancel}
\usepackage{multicol}
\usepackage[shortlabels]{enumitem}
\usepackage{stackrel}

\setlength{\headheight}{10pt}
\setlength{\headsep}{10pt}
\pagestyle{fancy}
\rhead{\ayudantia \ - \alumno}

\newcommand*{\mybox}[2]{\colorbox{#1!30}{\parbox{.98\linewidth}{#2}}}

\newenvironment{solucion}
{\begin{mdframed}[backgroundcolor=black!10]
		{\bf Solución:}\\
	}
	{
	\end{mdframed}
}

\newenvironment{alternativas}[1]
{\begin{multicols}{#1}
		\begin{enumerate}[a)]
		}
		{
		\end{enumerate}
	\end{multicols}
}

\newenvironment{preguntas}
{\begin{enumerate}\itemsep12pt
	}
	{
	\end{enumerate}
}

\newcommand{\ayudantia}{{\sc Ayudantía 6}}
\newcommand{\tituloayu}{Máximos y mínimos. TVM}
\newcommand{\fecha}{16 de abril de 2019}
\newcommand{\sigla}{MAT1610}
\newcommand{\nombre}{Cálculo I}
\newcommand{\profesor}{Amal Taarabt}
\newcommand{\ano}{2019}
\newcommand{\semestre}{1}
\newcommand{\mail}{mat1610@ifcastaneda.cl}
\newcommand{\alumno}{Ignacio Castañeda - \mail}

\newcommand{\ev}{\Big|}
\newcommand{\ra}{\rightarrow}
\newcommand{\lra}{\leftrightarrow}
\newcommand{\N}{\mathbb{N}}
\newcommand{\R}{\mathbb{R}}
\newcommand{\Exp}[1]{\mathcal{E}_{#1}}
\newcommand{\List}[1]{\mathcal{L}_{#1}}
\newcommand{\EN}{\Exp{\N}}
\newcommand{\LN}{\List{\N}}
\newcommand{\comment}[1]{}
\newcommand{\lb}{\\~\\}
\newcommand{\eop}{_{\square}}
\newcommand{\hsig}{\hat{\sigma}}
\newcommand{\widesim}[2][1.5]{
	\mathrel{\overset{#2}{\scalebox{#1}[1]{$\sim$}}}
}
\newcommand{\wsim}{\widesim{}}

\begin{document}
\thispagestyle{empty}

\begin{minipage}{2cm}
	\includegraphics[width=2cm]{../../../../img/logo.pdf}
	\vspace{0.5cm}
\end{minipage}
\begin{minipage}{\linewidth}
	\begin{tabular}{lrl}
		{\scriptsize\sc Pontificia Universidad Catolica de Chile} & \hspace*{0.7in}Curso: &
		\sigla  - \nombre\\
		{\sc Facultad de Matemáticas}&
		Profesor: & \profesor \\
		{\sc Semestre \ano-\semestre} & Ayudante: & {Ignacio Castañeda}\\
		& {Mail:} & \texttt{\mail}
	\end{tabular}
\end{minipage}

\vspace{-10mm}
\begin{center}
	{\LARGE\bf \ayudantia}\\
	\vspace{0.1cm}
	{\tituloayu}\\
	\vspace{0.1cm}
	\fecha\\
	\vspace{0.4cm}
\end{center}

\begin{preguntas}
\item Para cada una de las siguientes funciones, encontrar sus máximos y mínimos locales, en caso de haber
\begin{tasks}(2)
\task $f(x) = x^4 - 8x^2 + 3$
\task $f(x) = \dfrac{x^2-x-2}{x^2-6x+9}$
\task $f(x) = e^x(2x^2+x-8)$
\task $f(x)=x+ln(x^2-1)$
\end{tasks}
\begin{solucion}

\begin{enumerate}[a)]
\item 
\item 
\item 
\item 
\end{enumerate}
\end{solucion}
\item Encuentre los máximos y mínimos de la función
$$f(x) = \dfrac{|x|}{1+x^2}$$
en el intervalo $x \in [-3,2]$
\begin{solucion}

\end{solucion}
\item Estudiar, según el valor de la constante $k$, los puntos críticos y los puntos de inflexión de la función
$$f(x) = x^3-3kx^2+12x$$
\begin{solucion}

\end{solucion}
\item Probar que la ecuación $1+2x+3x^2+4x^3=0$ tiene solución única.
\begin{solucion}

\end{solucion}
\item Sea $f$ una función derivable en $[0, \infty)$, tal que $\forall x \in [0, \infty)(f(2x)=2f(x))$.\\
Demuestre que
$$\forall x \in (0, \infty) \exists c > 0 \left(f'(x) = \dfrac{f(x)}{x}\right)$$
\begin{solucion}

\end{solucion}
\item Si $f$ es una función dos veces derivable en $[a,b]$ tal que $f(a)=f(b)=0$ y $f(c) > 0$ con $a < c < b$, demuestre que
$$\exists \alpha \in (a,b) (f''(\alpha) < 0)$$
\begin{solucion}

\end{solucion}
\end{preguntas}
\end{document}