\documentclass[12pt]{article}

\usepackage{fullpage}
\usepackage{graphicx}
\usepackage{amssymb}
\usepackage{amsmath}
\usepackage[none]{hyphenat}
\usepackage{parskip}
\usepackage[spanish]{babel}
\usepackage[utf8]{inputenc}
\usepackage{hyperref}
\usepackage{fancyhdr}
\usepackage{tasks}
\usepackage{mdframed}
\usepackage{xcolor}
\usepackage{pgfplots}
\usepackage[makeroom]{cancel}
\usepackage{multicol}
\usepackage[shortlabels]{enumitem}
\usepackage{stackrel}
\usepackage{tkz-tab}
\usepackage{xpatch}
\usepackage{tkz-euclide}
\usetkzobj{all}
\xpatchcmd{\tkzTabLine}{$0$}{$\bullet$}{}{}

\setlength{\headheight}{10pt}
\setlength{\headsep}{10pt}
\pagestyle{fancy}
\rhead{\ayudantia \ - \alumno}
\tikzset{t style/.style={style=solid}}

\newcommand*{\mybox}[2]{\colorbox{#1!30}{\parbox{.98\linewidth}{#2}}}

\newenvironment{solucion}
{\begin{mdframed}[backgroundcolor=black!10]
		{\bf Solución:}\\
	}
	{
	\end{mdframed}
}

\newenvironment{alternativas}[1]
{\begin{multicols}{#1}
		\begin{enumerate}[a)]
		}
		{
		\end{enumerate}
	\end{multicols}
}

\newenvironment{preguntas}
{\begin{enumerate}\itemsep12pt
	}
	{
	\end{enumerate}
}

\newcommand{\ayudantia}{{\sc Ayudantía 14.5}}
\newcommand{\tituloayu}{Compilado Examen}
\newcommand{\fecha}{19 de junio de 2019}
\newcommand{\sigla}{MAT1610}
\newcommand{\nombre}{Cálculo I}
\newcommand{\profesor}{Amal Taarabt}
\newcommand{\ano}{2019}
\newcommand{\semestre}{1}
\newcommand{\mail}{mat1610@ifcastaneda.cl}
\newcommand{\alumno}{Ignacio Castañeda - \mail}

\newcommand{\ev}{\Big|}
\newcommand{\ra}{\rightarrow}
\newcommand{\lra}{\leftrightarrow}
\newcommand{\N}{\mathbb{N}}
\newcommand{\R}{\mathbb{R}}
\newcommand{\Exp}[1]{\mathcal{E}_{#1}}
\newcommand{\List}[1]{\mathcal{L}_{#1}}
\newcommand{\EN}{\Exp{\N}}
\newcommand{\LN}{\List{\N}}
\newcommand{\comment}[1]{}
\newcommand{\lb}{\\~\\}
\newcommand{\eop}{_{\square}}
\newcommand{\hsig}{\hat{\sigma}}
\newcommand{\widesim}[2][1.5]{
	\mathrel{\overset{#2}{\scalebox{#1}[1]{$\sim$}}}
}
\newcommand{\wsim}{\widesim{}}
\newcommand{\lh}{\stackrel{L'H}{=}}

\begin{document}
\thispagestyle{empty}

\begin{minipage}{2cm}
	\includegraphics[width=2cm]{../../../../img/logo.pdf}
	\vspace{0.5cm}
\end{minipage}
\begin{minipage}{\linewidth}
	\begin{tabular}{lrl}
		{\scriptsize\sc Pontificia Universidad Catolica de Chile} & \hspace*{0.7in}Curso: &
		\sigla  - \nombre\\
		{\sc Facultad de Matemáticas}&
		Profesor: & \profesor \\
		{\sc Semestre \ano-\semestre} & Ayudante: & {Ignacio Castañeda}\\
		& {Mail:} & \texttt{\mail}
	\end{tabular}
\end{minipage}

\vspace{-10mm}
\begin{center}
	{\LARGE\bf \ayudantia}\\
	\vspace{0.1cm}
	{\tituloayu}\\
	\vspace{0.1cm}
	\fecha\\
	\vspace{0.4cm}
\end{center}

\begin{preguntas}
\item Calcule el volumen del sólido generado al rotar la curva $y=x^2$ en torno al eje $y$ con $0 \leq y \leq 4$
\begin{solucion}
En primer lugar, grafiquemos la curva:
		\begin{center}
			\begin{tikzpicture}
			\begin{axis}[
			axis lines = left,
			xlabel = $x$,
			ylabel = $y$,
			]
			\addplot [
			domain=-3:3,  
			color=red,
			]
			{x^2};
			
			\end{axis}
			\end{tikzpicture}
		\end{center}
	
		Notemos que cada sección transversal tendrá un radio $x$. Sin embargo, como el objeto se rota en torno al eje $y$, debemos integrar en la variable y, por lo que debemos expresar todo en función de $y$. Luego, el radio de cada sección transversal es 
		$$y = x^2 \ra x = \sqrt[]{y}$$
		De esta manera, al área de cada sección transversal será
		$$A = \pi (\ \sqrt[]{y})^2 = \pi y$$
		Si decimos además que cada sección transversal tiene un ancho $dy$,
		$$dV =  \pi y dy$$
	 	Por lo que el volumen buscado es
	 	$$V = \displaystyle \int_0^4 \pi y dy = \pi \dfrac{y^2}{2} \ev_0^4 = \pi \dfrac{16}{2} = 8 \pi$$
\end{solucion}
\item Encuentre el volumen resultante de girar la curva $y=\dfrac{1}{\sqrt[]{x}}$ en torno al eje $x$ con $x \in [1,e]$
\begin{solucion}
Graficando:
		\begin{center}
			\begin{tikzpicture}
			\begin{axis}[
			axis lines = left,
			xlabel = $x$,
			ylabel = $y$,
			xmin = 0,
			ymax = 3,
			ymin = 0,
			]
			\addplot [
			domain=0:e,  
			color=red,
			]
			{1/x};
			
			\end{axis}
			\end{tikzpicture}
		\end{center}
		
		Vemos que nuestras secciones transversales tienen radio $y$. Sin embargo, como tendremos que integrar en $x$, dado que es el eje en torno al cual se gira, debemos escribir el radio en función de $x$. Por lo tanto, el radio es $\dfrac{1}{\sqrt[]{x}}$.
		
		Luego, el área de cada sección será
		$$A = \pi \dfrac{1}{x}$$
		Asi, el volumen de cada sección será
		$$dV = \pi \dfrac{1}{x} dx$$
		Finalmente, el volumen que buscamos corresponde a integrar en el dominio, es decir
		$$V = \displaystyle \int_1^e \pi \dfrac{1}{x} dx = \pi ln(x) \ev_1^e = \pi - 0 = \pi$$
\end{solucion}
\item La base de un sólido es un círculo de radio $a$ y todas sus secciones perpendiculares a un diametro fijo de éste son triángulos rectángulos isósceles con la hipotenusa sobre la base del sólido. Hallar el volúmen del sólido.
\begin{solucion}
En primer lugar, recordemos que la ecuación de un círculo de radio $a$ es
		$$x^2 + y^2 = a^2$$
		Podemos usar muchas interpretaciones del enunciado. En este caso nos imaginaremos que cada triangulo corresponde a fijar una $x$ en el círculo (quedando una linea vertical) y poniendo ahí nuestro triángulo rectángulo isóceles. Notemos que lo que queda sobre un cuarto del círculo tambien es un triángulo rectángulo isóceles, que tiene por catetos el alto de la linea que trazamos antes (que corresponde a $y$) y la altura del triángulo original. Nuestro objetivo será calcular el volumen sobre este cuarto de círculo y multiplicarlo por 4 para obtener el volúmen total.
		
		Como integraremos en el eje $x$, necesitamos el cateto del triángulo en función de $y$. Es decir, despejando en la ecuación del círculo,
		$$y = \sqrt[]{a^2-x^2}$$
		Como este es el catéto y nuestro triángulo es rectángulo isóceles, sabemos que su área será
		$$A =  \dfrac{(\sqrt[]{a^2-x^2})^2}{2} = \dfrac{a^2-x^2}{2}$$
		Luego,
		$$dV = \dfrac{a^2-x^2}{2} dx$$
		Finalmente, el volumen es
		$$V = 4 \displaystyle \int_0^a \dfrac{a^2-x^2}{2} dx = 4\left(\dfrac{a^2x}{2} - \dfrac{x^3}{6}\right) \ev_0^a = 4\left(\dfrac{a^3}{2} - \dfrac{a^3}{6}\right) = 4\left(\dfrac{2a^3}{6}\right) = \dfrac{4}{3}a^3$$
\end{solucion}
\item La base de un sólido es la región acotada por las parabolas
	$$y=x^2 \quad e \quad y = 3-2x^2$$
	y sus secciones perpendiculares al eje $y$ son triángulos equiláteros. Hallar el volúmen del sólido.
\begin{solucion}
Graficando:
		\begin{center}
			\begin{tikzpicture}
			\begin{axis}[
			axis lines = left,
			xlabel = $x$,
			ylabel = $y$,
			xmin = -1,
			xmax = 1,
			ymax = 3,
			ymin = 0,
			]
			\addplot [
			domain=-1:1,  
			color=red,
			]
			{3-2*x^2};
			\addplot [
			domain=-1:1,  
			color=red,
			]
			{x^2};
			
			
			\end{axis}
			\end{tikzpicture}
		\end{center}
		
		Recordemos que el área de un triángulo equilatero de lado $a$ es $\dfrac{\sqrt[]{3}}{4}a^2$
		
		Notemos que el lado de nuestro triángulo será siempre igual a $2x$. Sin embargo, debemos integrar en el eje $y$, por lo que tenemos que ver que ocurre en cada caso:
		
		Cuando $0 \leq y \leq 1$,
		$$y = x^2 \ra x = \sqrt[]{y} \ra a = 2\ \sqrt[]{y} \ra A = \dfrac{\sqrt[]{3}}{4} (2\ \sqrt[]{y})^2= \sqrt[]{3}y$$
		Mientras que si $1 \leq y \leq 3$,
		$$y = 3-2x^2 \ra x = \sqrt[]{\dfrac{3-y}{2}} \ra a = 2\ \sqrt[]{\dfrac{3-y}{2}} \ra A =\dfrac{\sqrt[]{3}}{4} \left(2\ \sqrt[]{\dfrac{3-y}{2}}\right)^2 = \dfrac{ \sqrt[]{3}}{2} (3-y)$$
		
		Por lo tanto, el volumen pedido es
		$$V = \sqrt[]{3} \displaystyle\int_0 ^1 y dy  + \dfrac{ \sqrt[]{3}}{2} \displaystyle\int_1^3(3-y)dy = \dfrac{ \sqrt[]{3}}{2}(1+2) = \dfrac{3\ \sqrt[]{3}}{2}$$
\end{solucion}
\item Determinar el volumen del sólido generado al rotar las curvas $y=x$ e $y= x^2$ en torno al eje $x$, para $x \in [0,1]$
\begin{solucion}
Graficando:
		\begin{center}
			\begin{tikzpicture}
			\begin{axis}[
			axis lines = left,
			xlabel = $x$,
			ylabel = $y$,
			xmin = 0,
			xmax = 1,
			ymax = 1,
			ymin = 0,
			]
			\addplot [
			domain=0:1,  
			color=red,
			]
			{x};
			\addplot [
			domain=0:1,  
			color=red,
			]
			{x^2};
			
			
			\end{axis}
			\end{tikzpicture}
		\end{center}
		
		Notemos que al rotar esto en torno al eje $x$, las secciones transversales serán circulos con un hoyo en el centro. Esto lo podemos ver como el área de un círculo grande menos el área de un circulo más chico.
		
		El círculo grande tendrá radio $r_1 = x$, mientras que el círculo pequeño tendrá área $r_2 = x^2$. 
		
		Luego, el área de cada sección será
		$$A = \pi r_1^2 - \pi r_2^2 = \pi(x^2-x^4) $$
		Asi,
		$$dV = \pi(x^2-x^4)dx$$
		Finalmente, el volumen del sólido es
		$$V = \int_0^1 \pi(x^2-x^4)dx = \pi\left(\dfrac{x^3}{3} - \dfrac{x^5}{5}\right)\ev_0^1 = \pi \left(\dfrac{1}{3}-\dfrac{1}{5}\right) = \dfrac{2\pi}{15}$$
\end{solucion}
\item Resuelva las siguientes integrales
\begin{tasks}(2)
\task $\displaystyle\int xcos(x)dx$
\task $\displaystyle\int arctan(x)dx$
\task $\displaystyle\int ln(x)dx$
\task $\displaystyle\int ln^2(x)dx$
\end{tasks}
\begin{solucion}
Recordemos primero que
		$$\displaystyle\int Udv = UV - \displaystyle\int Vdu$$
		Además, para todos estos ejercicios debemos recordar lo siguiente: ILATE,
		donde\\\\
		I: Inversa trigonométrica\\
		L: Logarítmica\\
		A: Algebraica\\
		T: Trigonométrica\\
		E: Exponencial\\\\
		La función que esté primero en la lista, la usaremos como $U$. La que este despues, será $dv$.
\begin{enumerate}[a)]
\item $\displaystyle\int xcos(x)dx$\\\\
			Aquí podemos ver dos funciones: $x$ que es algebraica y $cos(x)$ que es trigonométrica. Como $x^2$ esta primero en la lista, entonces
			$$U = x, \quad dv = cos(x)dx$$
			Para obtener $V$ y $du$, derivamos e integramos según corresponda
			$$du = dx, \quad V = sen(x)$$
			Luego, aplicando la regla de la vaca, tenemos que
			$$\displaystyle\int xcos(x)dx = xsen(x) - \displaystyle\int sen(x)dx = xsen(x) +cos(x) + c $$
\item $\displaystyle\int arctan(x)dx$\\\\
			Es evidente que en este caso debemos tomar
			$$U = arctan(x), \quad dv = dx \ra du = \dfrac{dx}{1+x^2}, \quad V = x$$
			Luego,
			$$\displaystyle\int arctan(x)dx 
			= x\ arctan(x) - \displaystyle\int \dfrac{xdx}{1+x^2} $$
			Utilizamos $w = 1+x^2 \ra dw = 2xdx$, 
			$$= x\ arctan(x) - \dfrac{1}{2}\displaystyle\int \dfrac{dw}{w} + c
			= x\ arctan(x) - \dfrac{1}{2}ln(w)$$
			$$= \ arctan(x) - \dfrac{1}{2}ln(1+x^2) + c$$
\item $\displaystyle\int ln(x)dx$\\\\
			En este caso,
			$$U = ln(x), \quad dv = dx \ra du = \dfrac{dx}{x}, \quad V = x$$
			Luego, 
			$$\displaystyle\int ln(x)dx
			= xln(x) - \displaystyle\int \dfrac{xdx}{x}
			= xln(x) - \displaystyle\int dx
			= xln(x) - x + c
			$$
\item $\displaystyle\int ln^2(x)dx$\\\\
			De la misma forma que antes
			$$U = ln^2(x), \quad dv = dx \ra du = 2ln(x)\dfrac{1}{x}dx, \quad V = x$$
			Con esto,
			$$\displaystyle\int ln^2(x)dx = xln^2(x) - 2\displaystyle\int ln(x)dx$$
			Por el ejercicio anterior sabemos que
			$$\displaystyle\int ln(x)dx = xln(x) -x$$
			Luego,
			$$=xln^2(x) - 2(xln(x) - x) + c$$
			Finalmente,
			$$\displaystyle\int ln^2(x)dx = xln^2(x) - 2xln(x) + x + c$$
\end{enumerate}
\end{solucion}
\item Calcular
	$$\displaystyle\int x^2\ arctan(x)dx$$
\begin{solucion}
Usamos integración por partes, con
		$$U = arctan(x) \ra du = \dfrac{dx}{1+x^2}$$
		$$dv = x^2 \ra V = \dfrac{x^3}{3}dx$$
		Luego,
		$$\displaystyle\int x^2\ arctan(x)dx = \dfrac{x^3}{3}arctan(x) - \dfrac{1}{3} \displaystyle\int \dfrac{x^3}{x^2+1}dx$$
		Para la integral del lado derecho, usamos sustitución con $m = 1+x^2 \ra dm = 2xdx$, con lo que
		$$\displaystyle\int \dfrac{x^3}{x^2+1}dx
		= \dfrac{1}{2}\displaystyle\int \dfrac{m-1}{m}dm
		= \dfrac{1}{2}\displaystyle\int \left(1-\dfrac{1}{m}\right)dm = \dfrac{1}{2}(m-ln|m|)$$
		Volvemos a la variable original
		$$= \dfrac{1}{2}(1+x^2-ln(1+x^2))$$
		Finalmente, reemplazando en el inicio,
		$$\displaystyle\int x^2\ arctan(x)dx = \dfrac{x^3}{3}arctan(x) - \dfrac{1}{6} (1+x^2-ln(1+x^2))$$
\end{solucion}
\item Resuelva las siguientes integrales
\begin{tasks}(2)
\task $\displaystyle\int cos^2(x)dx$
\task $\displaystyle\int cos^3(x)sen^2(x)dx$
\task $\displaystyle\int sec^4(x)tan^2(x)dx$
\task $\displaystyle\int sec^3(x)tan^3(x)dx$
\end{tasks}
\begin{solucion}
En este tipo de integrales, encontraremos 2 grandes casos, con múltiples casos dentro de ellos.\\

En primer lugar,
$$\displaystyle\int sen^n(x)cos^m(x)dx$$
Caso 1: $n$ ó $m$ impar\\

En este caso, se toma el que tenga el exponente impar (si ambos son impares se puede elegir cualquiera) y se separa un factor de él, el cual se pone al final. El resto de los términos se transforman a la otra función, es decir, si separamos el $seno$, todo se transforma a $coseno$ y viceversa. Luego se hace la sustitución con la función trigonométrica que \textbf{no} separamos.

Por ejemplo, si el $m$ fuese impar, esto se vería así:
$$= \displaystyle\int sen^n(x)(1-sen^2(x))^{\frac{m-1}{2}}cos(x)dx$$
Luego, $u = sen(x) \ra du = cos(x)dx$, quedando
$$= \displaystyle\int u^n(1-u^2)^{\frac{m-1}{2}}du$$
Caso 2: $n$ y $m$ par\\

En este caso, usamos las propiedades
$$sen^2(x) = \dfrac{1-cos(2x)}{2} \qquad y \qquad cos^2(x) = \dfrac{1+cos(2x)}{2}$$
Para reducir el grado de los $senos$ y $cosenos$ y separamos en multiples integrales.\\

En segundo lugar,
$$\displaystyle\int sec^n(x)tan^m(x)dx$$
Caso 1: $n$ par\\

En este caso, separamos un $sec^2(x)$ de la integral, dejandolo al final y convertimos todas las $secantes$ a $tangentes$. Luego, hacemos la sustitución $u = tan(x) \ra du = sec^2(x)dx$. Esto es,
$$=\displaystyle\int (1+tan^2(x))^{\frac{n-2}{2}}tan^m(x)sec^2(x)dx$$
Luego, $u = tan(x) \ra du = sec^2(x)dx$, quedando
$$=\displaystyle\int (1+u^2)^{\frac{n-2}{2}}u^m du$$
Caso 2: $m$ impar\\

En este caso, separamos un $sec(x)tan(x)$ de la integral, dejandolo al final y convertimos todas las $tangentes$ a $secantes$. Luego, hacemos la sustitución $u = sec(x) \ra du = sec(x)tan(x)dx$. Esto es,
$$=\displaystyle\int sec^{n-1}(x)(sec^2(x) - 1)^{\frac{m-1}{2}}sec(x)tan(x)dx$$
Luego, $u = sec(x) \ra du = sec(x)tan(x)dx$, quedando
$$=\displaystyle\int u^{n-1}(u^2 - 1)^{\frac{m-1}{2}}du$$

Caso 3: $n$ impar y $m$ par\\

\begin{center}
\includegraphics[width=8cm]{../../../../img/lie_down}
\end{center}
Procedamos ahora con los ejercicios:
\begin{enumerate}[a)]
\item $\displaystyle\int cos^2(x)dx$\\\\
			Recordemos que $cos^2(x) = \dfrac{1+cos(2x)}{2}$
			Luego,
			$$\displaystyle\int cos^2(x)dx 
			= \displaystyle\int \dfrac{1+cos(2x)}{2}dx
			= \displaystyle\int \dfrac{dx}{2} + \displaystyle\int \dfrac{cos(2x)dx}{2}
			= \dfrac{x}{2} + \dfrac{sen(2x)}{4} + c
			$$
\item $\displaystyle\int cos^3(x)sen^2(x)dx$\\\\
			Cuando estamos ante senos y cosenos y alguno de los exponentes es impar, debemos separar la función que tiene dicho exponente para dejarlo par y proceder a convertir todo a la otra función, es decir
			$$\displaystyle\int cos^3(x)sen^2(x)dx
			= \displaystyle\int cos^2(x)sen^2(x)cos(x)dx$$
			$$= \displaystyle\int (1-sen^2(x))sen^2(x)cos(x)dx$$
			Luego, utilizamos el cambio
			$$u = sen(x) \ra du = cos(x)dx$$
			De esta forma,
			$$= \displaystyle\int (1-u^2)u^2du
			= \displaystyle\int (u^2-u^4)du
			= \dfrac{u^3}{3} - \dfrac{u^5}{5} + c
			$$
			Volviendo a la variable original,
			$$\displaystyle\int cos^3(x)sen^2(x)dx 
			= \dfrac{sen^3(x)}{3} - \dfrac{sen^5(x)}{5} + c$$
			
\item $\displaystyle\int sec^4(x)tan^2(x)dx$\\\\
			En el caso de las secantes y tangentes, mientras el exponente de la secante y la tangente no sean impar y pan, respectivamente de manera simultanea, siempre podremos acuidar a pasar todo a tangente o todo a secante, de la siguiente forma
			$$\displaystyle\int sec^4(x)tan^2(x)dx
			=\displaystyle\int sec^2(x)tan^2(x)sec^2(x)dx$$
			$$=\displaystyle\int (1+tan^2(x))tan^2(x)sec^2(x)dx$$
			Luego,
			$$u = tan(x) \ra du = sec^2(x)dx$$
			Con lo que tenemos
			$$=\displaystyle\int (1+u^2)u^2du
			= \displaystyle\int (u^2+u^4)du
			= \dfrac{u^3}{3} + \dfrac{u^5}{5} + c$$
			Finalmente, volvemos a la variable original, con lo que
			$$\displaystyle\int sec^4(x)tan^2(x)dx = \dfrac{tan^3(x)}{3} + \dfrac{tan^5(x)}{5} + c$$
\item $\displaystyle\int sec^3(x)tan^3(x)dx$\\\\
			Como dije antes, también podemos convertir todo en secantes. Esto se hace de la siguiente forma
			$$\displaystyle\int sec^3(x)tan^3(x)dx
			= \displaystyle\int sec^2(x)tan^2(x)sec(x)tan(x)dx$$
			$$ = \displaystyle\int sec^2(x)(sec^2(x)-1)sec(x)tan(x)dx$$
			En este caso, hacemos el cambio
			$$u = sec(x) \ra du = sec(x)tan(x)$$
			Con lo que obtenemos
			$$ = \displaystyle\int u^2(u^2-1)du
			= \displaystyle\int (u^4-u^2)du
			= \dfrac{u^5}{5} - \dfrac{u^3}{3} + c$$
			Finalmente, volvemos a la variable original, con lo que
			$$= \dfrac{sec^5(x)}{5} - \dfrac{sec^3(x)}{3} + c$$	
\end{enumerate}
\end{solucion}
\item Calcule la siguiente integral
	$$\displaystyle\int sen^3(x)cos^4(x)dx$$
\begin{solucion}
Como el exponente del seno es impar, lo separamos y dejamos un seno aislado
		$$\displaystyle\int sen^2(x)cos^4(x) sen(x)dx$$
		Luego, pasamos todo a coseno
		$$\displaystyle\int (1-cos^2(x))cos^4(x) sen(x)dx$$
		A continuación, utilizamos la sustitución $u = cos(x) \ra du = -sen(x)dx$
		$$\displaystyle\int (1-u^2)u^4(- du) = -\displaystyle\int (u^4-u^6)du = -(\dfrac{u^5}{5} - \dfrac{u^7}{7}) +c= \dfrac{u^7}{7}-\dfrac{u^5}{5} + c$$
		Por último, volvemos a la variable original, con lo que
		$$\displaystyle\int sen^3(x)cos^4(x)dx = \dfrac{cos^7(x)}{7}-\dfrac{cos^5(x)}{5} + c$$
\end{solucion}
\item Resuelva las siguientes integrales
\begin{tasks}(2)
\task $\displaystyle\int \dfrac{dx}{\sqrt[]{(4-x^2)^3}}$
\task $\displaystyle\int \dfrac{xdx}{\sqrt[]{1+x^2}}$
\task $\displaystyle\int \dfrac{dx}{x\ \sqrt[]{x^2-1}}$
\task $\displaystyle\int \dfrac{dx}{\sqrt[]{16+6x-x^2}}$
\end{tasks}
\begin{solucion}

\begin{enumerate}[a)]
\item $\displaystyle\int \dfrac{dx}{\sqrt[]{(4-x^2)^3}}$\\\\
			Como en la integral hay presente una expresión de la forma $a^2-x^2$, realizamos el siguiente cambio de variable
			$$x = asen(\theta) \ra x = 2sen(\theta) \ra dx = 2cos(\theta)d\theta$$
			Luego, la integral es
			$$\displaystyle\int \dfrac{2cos(\theta)d\theta}{\sqrt[]{(4-4sen^2(\theta))^3}}
			= \displaystyle\int \dfrac{2cos(\theta)d\theta}{\sqrt[]{(4(1-sen^2(\theta)))^3}}
			= \displaystyle\int \dfrac{2cos(\theta)d\theta}{\sqrt[]{(4cos^2(\theta))^3}}$$
			$$= \displaystyle\int \dfrac{2cos(\theta)d\theta}{(\sqrt[]{4cos^2(\theta)})^3}
			= \displaystyle\int \dfrac{2cos(\theta)d\theta}{(2cos(\theta))^3}
			= \dfrac{1}{4}\displaystyle\int \dfrac{d\theta}{(cos^2(\theta))}
			= \dfrac{1}{4}\displaystyle\int sec^2(\theta)d\theta$$
			$$= \dfrac{1}{4} tan(\theta) + c$$
			Recordemos que $x = 2sen(\theta)$, por lo que
			$$\theta = arcsen\left(\dfrac{x}{2}\right)$$
			Esto se puede representar con el siguiente triangulo auxiliar, ya que $\theta$ es el ángulo cuyo seno es $\dfrac{x}{2}$
			\begin{center}
				\begin{tikzpicture}[scale=1.25]%,cap=round,>=latex]
				
				\coordinate [label=left:$ $] (A) at (-1.5cm,-1.cm);
				\coordinate [label=right:$ $] (C) at (1.5cm,-1.0cm);				\coordinate [label=above:$ $] (B) at (1.5cm,1.0cm);
				\draw (A) -- node[above] {$2$} (B) -- node[right] {$x$} (C) -- node[below] {$\sqrt[]{4-x^2}$} (A);
				
				\draw (1.25cm,-1.0cm) rectangle (1.5cm,-0.75cm);
				
				\tkzMarkAngle[size=0.8cm,%
				opacity=.4](C,A,B)
				\tkzLabelAngle[pos = 0.6](C,A,B){$\theta$}
				
				\end{tikzpicture}
			\end{center}
			Luego, podemos ver que
			$$tan(\theta) = \dfrac{x}{\sqrt[]{4-x^2}}$$
			Finalmente,
			$$\displaystyle\int \dfrac{dx}{\sqrt[]{(4-x^2)^3}} =
			\dfrac{x}{4\ \sqrt[]{4-x^2}} + c$$
\item $\displaystyle\int \dfrac{xdx}{\sqrt[]{1+x^2}}$\\\\
			Como identificamos una expresión de la forma $a^2+x^2$, usamos la sustitución
			$$x = tan(\theta) \ra dx = sec^2(\theta)$$
			Luego tenemos
			$$\displaystyle\int \dfrac{tan(\theta)sec^2(\theta)d\theta}{\sqrt[]{1+tan^2(\theta)}}
			= \displaystyle\int \dfrac{tan(\theta)sec^2(\theta)d\theta}{\sqrt[]{sec^2(\theta)}}
			= \displaystyle\int \dfrac{tan(\theta)sec^2(\theta)d\theta}{sec(\theta)}$$
			$$= \displaystyle\int sec(\theta)tan(\theta)d\theta = sec(\theta) + c$$
			Como $x = tan(\theta) \ra \theta = arctan(x)$, nuestro triangulo auxiliar sería
			\begin{center}
				\begin{tikzpicture}[scale=1.25]%,cap=round,>=latex]
				
				\coordinate [label=left:$ $] (A) at (-1.5cm,-1.cm);
				\coordinate [label=right:$ $] (C) at (1.5cm,-1.0cm);				\coordinate [label=above:$ $] (B) at (1.5cm,1.0cm);
				\draw (A) -- node[above] {$\sqrt[]{1+x^2}$} (B) -- node[right] {$x$} (C) -- node[below] {$1$} (A);
				
				\draw (1.25cm,-1.0cm) rectangle (1.5cm,-0.75cm);
				
				\tkzMarkAngle[size=0.8cm,%
				opacity=.4](C,A,B)
				\tkzLabelAngle[pos = 0.6](C,A,B){$\theta$}
				
				\end{tikzpicture}
			\end{center}
			Finalmente, 
			$$\displaystyle\int \dfrac{xdx}{\sqrt[]{1+x^2}} = \sqrt[]{1+x^2} + c$$
\item $\displaystyle\int \dfrac{dx}{x\ \sqrt[]{x^2-1}}$\\\\
			Al identificar la expresión de la forma $x^2-a^2$, sabemos que debemos ocupar
			$$x = sec(\theta) \ra dx = sec(\theta)tan(\theta)$$
			Luego obtenemos
			$$\displaystyle\int \dfrac{sec(\theta)tan(\theta)d\theta}{sec(\theta)\ \sqrt[]{sec^2(\theta)-1}}
			= \displaystyle\int \dfrac{tan(\theta)d\theta}{\sqrt[]{tan^2(\theta)}}
			= \displaystyle\int d\theta = \theta + c$$
			Como $x = sec(\theta) \ra \theta = arcsec(x)$
			Finalmente,
			$$\displaystyle\int \dfrac{dx}{x\ \sqrt[]{x^2-1}} = arcsec(x) + c$$
\item $\displaystyle\int \dfrac{dx}{\sqrt[]{16+6x-x^2}}$\\\\
			Como no podemos ver ninguna expresión como las mencionadas anteriormente, vamos a hacer completación de cuadrados
			$$16+6x-x^2 = 25 - 9 + 6x - x^2 = 25 - (x^2-6x+9) = 25 - (x-3)^2$$
			Luego, la integral es
			$$\displaystyle\int \dfrac{dx}{\sqrt[]{25 - (x-3)^2}}$$
			Aqui si podemos identificar una expresión del tipo $a^2-x^2$, por lo que el cambio que corresponde hacer es
			$$x-3 = 5sen(\theta) \ra dx = 5cos(\theta)d\theta$$
			Entonces obtenemos la integral
			$$\displaystyle\int \dfrac{5cos(\theta)d\theta}{\sqrt[]{25 - 25sen^2(\theta)}}
			=\displaystyle\int \dfrac{5cos(\theta)d\theta}{\sqrt[]{25(1 - sen^2(\theta))}}
			=\displaystyle\int \dfrac{5cos(\theta)d\theta}{\sqrt[]{25cos^2(\theta)}}$$
			$$=\displaystyle\int \dfrac{5cos(\theta)d\theta}{5cos(\theta)} 
			= \displaystyle\int d\theta = \theta + c$$
			Como $x-3 = 5sen(\theta) \ra \theta = arcsen\left(\dfrac{x-3}{5}\right)$\\
			Finalmente,
			$$\displaystyle\int \dfrac{dx}{\sqrt[]{16+6x-x^2}} = arcsen\left(\dfrac{x-3}{5}\right) + c$$
\end{enumerate}
\end{solucion}
\item Resuelva la siguiente integral
	$$\displaystyle\int \dfrac{dx}{x(2-x)}$$
\begin{solucion}
En primer lugar, utilizamos fracciones parciales para descomponer la integral
		$$\dfrac{1}{x(2-x)} = \dfrac{A}{x} + \dfrac{B}{2-x}$$
		$$\dfrac{1}{x(2-x)} = \dfrac{A(2-x) + Bx}{x(2-x)}$$
		$$\dfrac{1}{x(2-x)} = \dfrac{2A-Ax + Bx}{x(2-x)}$$
		$$\dfrac{1}{x(2-x)} = \dfrac{2A+ (B-A)x}{x(2-x)}$$
		De esta forma, tenemos el siguiente sistema de ecuaciones
		$$2A = 1, \quad B-A = 0 \ra A = \dfrac{1}{2}, \quad B = \dfrac{1}{2}$$
		Entonces, tenemos que la descomposición es
		$$\dfrac{1}{x(2-x)} = \dfrac{1}{2x} + \dfrac{1}{2(2-x)}$$
		Luego, la integral original la podemos escribir como
		$$\displaystyle\int \dfrac{dx}{x(2-x)} = \displaystyle\int \dfrac{dx}{2x} + \displaystyle\int \dfrac{dx}{2(2-x)}$$
		$$ = \dfrac{1}{2} \left( \displaystyle\int \dfrac{dx}{x} + \displaystyle\int \dfrac{dx}{2-x} \right)$$
		$$ = \dfrac{1}{2} (ln|x| - ln|2-x| ) + c$$
\end{solucion}
\item Calcular la integral
	$$\displaystyle\int_0^1 \dfrac{x^3+6x^2+13x+10}{x+1}dx$$
\begin{solucion}
Como vemos que el grado del numerador es mayor que el del denominador, realizamos división polinómica, es decir, calculamos
		$$x^3+6x^2+13x+10 : x+1$$
		Esto lo podemos hacer utilizando división sintética
		$$
		\renewcommand\arraystretch{1.5}
		\setlength\doublerulesep{0pt}
		\begin{array}{rrrrr}
		\multicolumn{1}{r|}{1} & 1 & 6 & 13 & 10\\\cline{2-5}
		& & 1& 5 & 8 \\\cline{2-5}
		& 1 & 5& 8 & 2 
		\end{array}
		$$
		De donde obtenemos que
		$$\dfrac{x^3+6x^2+13x+10}{x+1} = x^2 +5x + 8 + \dfrac{2}{x+1}$$
		Luego,
		$$\displaystyle\int_0^1 \dfrac{x^3+6x^2+13x+10}{x+1}dx = \displaystyle\int_0^1 \left(x^2 + 5x + 8 +  \dfrac{2}{x+1}\right)dx$$
		$$= \left( \dfrac{x^3}{3} + \dfrac{5x^2}{2} + 8x + 2ln(x+1)\right) \ev_0^1 = \dfrac{1}{3} + \dfrac{5}{2} + 8 + 2ln(2) = \dfrac{65}{6} + 2 ln(2)$$
\end{solucion}
\item Calcule
$$\displaystyle\int \sqrt[]{1+x^2}dx$$
\begin{solucion}
En primer lugar, notemos que es una integral de la forma $a^2 + x^2$, por lo que debemos usar la sustitución
$$x = \tan(\theta) \ra dx = \sec^2(\theta)d\theta$$
Con esto,
$$\displaystyle\int \sqrt[]{1+x^2}dx =
\displaystyle\int \sqrt[]{1+\tan^2(\theta)}\sec^2(\theta)d\theta =
\displaystyle\int \sqrt[]{\sec^2(\theta)}\sec^2(\theta)d\theta =
\displaystyle\int \sec^3(\theta)d\theta
$$
Por lo tanto, debemos resolver
$$I = \displaystyle\int \sec^3(\theta)d\theta$$
Notemos que
$$\displaystyle\int \sec^3(\theta)d\theta =
\displaystyle\int \sec(\theta)\sec^2(\theta)d\theta
$$
Por lo que podemos usar integración por partes con
$$U = \sec(\theta), \quad dv = \sec^2(\theta)d\theta \ra 
du = \sec(\theta)\tan(\theta)d\theta, \quad V = \tan(\theta)$$
Luego,
$$\displaystyle\int \sec^3(\theta)d\theta = 
\sec(\theta)\tan(\theta) - \displaystyle\int \sec(\theta)\tan^2(\theta)d\theta
$$
$$\displaystyle\int \sec^3(\theta)d\theta = 
\sec(\theta)\tan(\theta) - \displaystyle\int \sec(\theta)(\sec^2(\theta)-1)d\theta
$$
$$\displaystyle\int \sec^3(\theta)d\theta = 
\sec(\theta)\tan(\theta) - \displaystyle\int \sec^3(\theta)d\theta + \displaystyle\int \sec(\theta)d\theta
$$
Notemos que la integral original se encuentra a ambos lados de la igualdad, por lo que podemos escribirlo como
$$I = 
\sec(\theta)\tan(\theta) - I + \displaystyle\int \sec(\theta)d\theta
$$
Despejando,
$$I = 
\dfrac{1}{2}\left(\sec(\theta)\tan(\theta) + \displaystyle\int \sec(\theta)d\theta\right)
$$
Entonces, faltaría solo calcular
$$\displaystyle\int \sec(\theta)d\theta =
\displaystyle\int \sec(\theta)\dfrac{\sec(\theta) + \tan(\theta)}{\sec(\theta) + \tan(\theta)}d\theta =
\displaystyle\int \dfrac{\sec^2(\theta) + \sec(\theta)\tan(\theta)}{\sec(\theta) + \tan(\theta)}d\theta
$$
$$u = \sec(\theta) + \tan(\theta) \ra du = (\sec^2(\theta) + \sec(\theta)\tan(\theta))d\theta$$
$$= \displaystyle\int \dfrac{du}{u} = \ln|u| + c = \ln|\sec(\theta) + \tan(\theta)| + c
$$
Finalmente,
$$I = 
\dfrac{1}{2}\sec(\theta)\tan(\theta) + \dfrac{1}{2}\ln|\sec(\theta) + \tan(\theta)| + c
$$
\end{solucion}
\end{preguntas}
\end{document}