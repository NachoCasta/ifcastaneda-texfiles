\documentclass[12pt]{article}

\usepackage{fullpage}
\usepackage{graphicx}
\usepackage{amssymb}
\usepackage{amsmath}
\usepackage[none]{hyphenat}
\usepackage{parskip}
\usepackage[spanish]{babel}
\usepackage[utf8]{inputenc}
\usepackage{hyperref}
\usepackage{fancyhdr}
\usepackage{tasks}
\usepackage{mdframed}
\usepackage{xcolor}
\usepackage{pgfplots}
\usepackage[makeroom]{cancel}
\usepackage{multicol}
\usepackage[shortlabels]{enumitem}
\usepackage{tabto}

\setlength{\headheight}{10pt}
\setlength{\headsep}{10pt}
\pagestyle{fancy}
\rhead{\ayudantia \ - \alumno}

\newcommand*{\mybox}[2]{\colorbox{#1!30}{\parbox{.98\linewidth}{#2}}}

\newenvironment{solucion}
{\begin{mdframed}[backgroundcolor=black!10]
		{\bf Solución:}\\
	}
	{
	\end{mdframed}
}

\newenvironment{alternativas}[1]
{\begin{multicols}{#1}
		\begin{enumerate}[a)]
		}
		{
		\end{enumerate}
	\end{multicols}
}

\newenvironment{preguntas}
{\begin{enumerate}\itemsep12pt
	}
	{
	\end{enumerate}
}

\newcommand{\ayudantia}{{\sc Ayudantía 1}}
\newcommand{\tituloayu}{Límites}
\newcommand{\fecha}{12 de marzo de 2019}
\newcommand{\sigla}{MAT1610}
\newcommand{\nombre}{Cálculo I}
\newcommand{\profesor}{Amal Taarabt}
\newcommand{\ano}{2019}
\newcommand{\semestre}{1}
\newcommand{\mail}{mat1610@ifcastaneda.cl}
\newcommand{\alumno}{Ignacio Castañeda - \mail}

\newcommand{\ev}{\Big|}
\newcommand{\ra}{\rightarrow}
\newcommand{\lra}{\leftrightarrow}
\newcommand{\N}{\mathbb{N}}
\newcommand{\R}{\mathbb{R}}
\newcommand{\Exp}[1]{\mathcal{E}_{#1}}
\newcommand{\List}[1]{\mathcal{L}_{#1}}
\newcommand{\EN}{\Exp{\N}}
\newcommand{\LN}{\List{\N}}
\newcommand{\comment}[1]{}
\newcommand{\lb}{\\~\\}
\newcommand{\eop}{_{\square}}
\newcommand{\hsig}{\hat{\sigma}}

\begin{document}
\thispagestyle{empty}

\begin{minipage}{2cm}
	\includegraphics[width=2cm]{../../../../img/logo.pdf}
	\vspace{0.5cm}
\end{minipage}
\begin{minipage}{\linewidth}
	\begin{tabular}{lrl}
		{\scriptsize\sc Pontificia Universidad Catolica de Chile} & \hspace*{0.7in}Curso: &
		\sigla  - \nombre\\
		{\sc Facultad de Matemáticas}&
		Profesor: & \profesor \\
		{\sc Semestre \ano-\semestre} & Ayudante: & {Ignacio Castañeda}\\
		& {Mail:} & \texttt{\mail}
	\end{tabular}
\end{minipage}

\vspace{-10mm}
\begin{center}
	{\LARGE\bf \ayudantia}\\
	\vspace{0.1cm}
	{\tituloayu}\\
	\vspace{0.1cm}
	\fecha\\
	\vspace{0.4cm}
\end{center}

\begin{preguntas}
\item Calcule los siguientes límites, en caso de que existan
\begin{tasks}(2)
\task $\lim\limits_{x \ra 5} \dfrac{1}{x-5}$ 
\task $\lim\limits_{x \ra 2} \dfrac{4x}{6-x}$
\task $\lim\limits_{x \ra -1} \dfrac{x^3+1}{x^2-1}$
\task $\lim\limits_{x \ra 3} \dfrac{\sqrt[]{4-x}-1}{x-3}$
\task $\lim\limits_{x \ra 2} \dfrac{\sqrt[]{6-x}-2}{\sqrt[]{3-x}-1}$
\task $\lim\limits_{x \ra 1} \dfrac{x^4+2x^3-3x^2+4x-4}{x^4+2x-3}$
\task $\lim\limits_{x \ra 0} \dfrac{|x|(|x|-x)}{x^2}$
\task $\lim\limits_{x \ra 2}\left(\dfrac{1}{x}-\dfrac{1}{|x|}\right)$
\end{tasks}
\item Aplicando la definición de límite, demostrar que
$$\lim\limits_{x \ra 1} \dfrac{x+1}{2} = 2$$
\item Calcule el siguiente límite, en caso de que exista
$$ \lim\limits_{x \ra 0} x^2e^{\sin{\frac{1}{x}}} $$
\textbf{Hint:} \textit{Utilice el teorema del sandwich}
\end{preguntas}
\end{document}