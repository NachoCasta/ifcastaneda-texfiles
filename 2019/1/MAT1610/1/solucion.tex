\documentclass[12pt]{article}

\usepackage{fullpage}
\usepackage{graphicx}
\usepackage{amssymb}
\usepackage{amsmath}
\usepackage[none]{hyphenat}
\usepackage{parskip}
\usepackage[spanish]{babel}
\usepackage[utf8]{inputenc}
\usepackage{hyperref}
\usepackage{fancyhdr}
\usepackage{tasks}
\usepackage{mdframed}
\usepackage{xcolor}
\usepackage{pgfplots}
\usepackage[makeroom]{cancel}
\usepackage{multicol}
\usepackage[shortlabels]{enumitem}
\usepackage{tabto}

\setlength{\headheight}{10pt}
\setlength{\headsep}{10pt}
\pagestyle{fancy}
\rhead{\ayudantia \ - \alumno}

\newcommand*{\mybox}[2]{\colorbox{#1!30}{\parbox{.98\linewidth}{#2}}}

\newenvironment{solucion}
{\begin{mdframed}[backgroundcolor=black!10]
		{\bf Solución:}\\
	}
	{
	\end{mdframed}
}

\newenvironment{alternativas}[1]
{\begin{multicols}{#1}
		\begin{enumerate}[a)]
		}
		{
		\end{enumerate}
	\end{multicols}
}

\newenvironment{preguntas}
{\begin{enumerate}\itemsep12pt
	}
	{
	\end{enumerate}
}

\newcommand{\ayudantia}{{\sc Ayudantía 1}}
\newcommand{\tituloayu}{Límites}
\newcommand{\fecha}{12 de marzo de 2019}
\newcommand{\sigla}{MAT1610}
\newcommand{\nombre}{Cálculo I}
\newcommand{\profesor}{Amal Taarabt}
\newcommand{\ano}{2019}
\newcommand{\semestre}{1}
\newcommand{\mail}{mat1610@ifcastaneda.cl}
\newcommand{\alumno}{Ignacio Castañeda - \mail}

\newcommand{\ev}{\Big|}
\newcommand{\ra}{\rightarrow}
\newcommand{\lra}{\leftrightarrow}
\newcommand{\N}{\mathbb{N}}
\newcommand{\R}{\mathbb{R}}
\newcommand{\Exp}[1]{\mathcal{E}_{#1}}
\newcommand{\List}[1]{\mathcal{L}_{#1}}
\newcommand{\EN}{\Exp{\N}}
\newcommand{\LN}{\List{\N}}
\newcommand{\comment}[1]{}
\newcommand{\lb}{\\~\\}
\newcommand{\eop}{_{\square}}
\newcommand{\hsig}{\hat{\sigma}}

\begin{document}
\thispagestyle{empty}

\begin{minipage}{2cm}
	\includegraphics[width=2cm]{../../../../img/logo.pdf}
	\vspace{0.5cm}
\end{minipage}
\begin{minipage}{\linewidth}
	\begin{tabular}{lrl}
		{\scriptsize\sc Pontificia Universidad Catolica de Chile} & \hspace*{0.7in}Curso: &
		\sigla  - \nombre\\
		{\sc Facultad de Matemáticas}&
		Profesor: & \profesor \\
		{\sc Semestre \ano-\semestre} & Ayudante: & {Ignacio Castañeda}\\
		& {Mail:} & \texttt{\mail}
	\end{tabular}
\end{minipage}

\vspace{-10mm}
\begin{center}
	{\LARGE\bf \ayudantia}\\
	\vspace{0.1cm}
	{\tituloayu}\\
	\vspace{0.1cm}
	\fecha\\
	\vspace{0.4cm}
\end{center}

\begin{preguntas}
\item Calcule los siguientes límites, en caso de que existan
\begin{tasks}(2)
\task $\lim\limits_{x \ra 5} \dfrac{1}{x-5}$ 
\task $\lim\limits_{x \ra 2} \dfrac{4x}{6-x}$
\task $\lim\limits_{x \ra -1} \dfrac{x^3+1}{x^2-1}$
\task $\lim\limits_{x \ra 3} \dfrac{\sqrt[]{4-x}-1}{x-3}$
\task $\lim\limits_{x \ra 2} \dfrac{\sqrt[]{6-x}-2}{\sqrt[]{3-x}-1}$
\task $\lim\limits_{x \ra 1} \dfrac{x^4+2x^3-3x^2+4x-4}{x^4+2x-3}$
\task $\lim\limits_{x \ra 0} \dfrac{|x|(|x|-x)}{x^2}$
\task $\lim\limits_{x \ra 2}\left(\dfrac{1}{x}-\dfrac{1}{|x|}\right)$
\end{tasks}
\begin{solucion}

\begin{enumerate}[a)]
\item $\lim\limits_{x \ra 5} \dfrac{1}{x-5}$\\
\\
Al evaluar en $x = 5$, podemos ver que se genera una discontinuidad,
$$\lim\limits_{x \ra 5} \dfrac{1}{x-5} = \dfrac{1}{5-5} = \dfrac{1}{0} = \infty = \not\exists$$
Por lo tanto, concluimos que el límite no existe.
\item $\lim\limits_{x \ra 2} \dfrac{4x}{6-x}$\\
\\
Como no se genera ninguna discontinuidad en $x = 2$, podemos evaluar directamente, obteniendo
$$\lim\limits_{x \ra 2} \dfrac{4x}{6-x} = \dfrac{4 \cdot 2}{6-2} = \dfrac{8}{4} = 2$$
\item $\lim\limits_{x \ra -1} \dfrac{x^3+1}{x^2-1}$\\
\\
Al evaluar en $x = -1$, podemos notar que obtendremos $\dfrac{0}{0}$, lo que no nos dice nada. Entonces, tenemos que buscar otra forma de evaluar el límite. Intentemos factorizar.
$$\lim\limits_{x \ra -1} \dfrac{x^3+1}{x^2-1} = \lim\limits_{x \ra -1} \dfrac{(x+1)(x^2-x+1)}{(x+1)(x-1)} = \lim\limits_{x \ra -1} \dfrac{(x^2-x+1)}{(x-1)} = -\dfrac{3}{2}$$
Al factorizar somos capaces de eliminar lo que nos generaba problemas y luego podemos evaluar directamente.
\item $\lim\limits_{x \ra 3} \dfrac{\sqrt[]{4-x}-1}{x-3}$\\
\\
En este caso, nos ocurre algo similar al ejercicio anterior, sin embargo, no es tan sencillo la simplificación que debemos hacer. \\
Para esto, utilicemos el método de la sustitución. Sea $u = \sqrt[]{4-x}$,\\
Luego,
$$x \ra 3$$
$$-x \ra -3$$
$$4-x \ra 1$$
$$\sqrt[]{4-x} \ra 1$$
$$u \ra 1$$
Procedemos a escribir el límite en términos de $u$
$$\lim\limits_{x \ra 3} \dfrac{\sqrt[]{4-x}-1}{x-3}
= \lim\limits_{u \ra 1} \dfrac{u-1}{1-u^2} $$
Ahora, resolvemos como siempre, es decir,
$$\lim\limits_{u \ra 1} \dfrac{u-1}{1-u^2} 
= \lim\limits_{u \ra 1} \dfrac{u-1}{(1+u)(1-u)}
= \lim\limits_{u \ra 1} \dfrac{-(1-u)}{(1+u)(1-u)}
= \lim\limits_{u \ra 1} \dfrac{-1}{(1+u)}
= -\dfrac{1}{2}$$
\item $\lim\limits_{x \ra 3} \dfrac{\sqrt[]{4-x}-1}{x-3}$\\
\\
En este caso podríamos utilizar sustitución como antes, pero dado que hay dos raices distintas, sería complicado. Por lo tanto, usaremos la técnica de la racionalización, esto es
$$\lim\limits_{x \ra 2} \dfrac{\sqrt[]{6-x}-2}{\sqrt[]{3-x}-1}
=\lim\limits_{x \ra 2} \dfrac{\sqrt[]{6-x}-2}{\sqrt[]{3-x}-1} \cdot \dfrac{\sqrt[]{3-x}+1}{\sqrt[]{3-x}+1} 
=\lim\limits_{x \ra 2} \dfrac{(\sqrt[]{6-x}-2)(\sqrt[]{3-x}+1)}{2-x} $$
Ahora, como lo que nos da problemas es el parentesis izquierdo, usaremos la sustitución
$$u = \sqrt[]{6-x} \Longrightarrow u \ra 2$$
Con esto, obtenemos el límite
$$\lim\limits_{u \ra 2} \dfrac{(u-2)(\sqrt[]{u^2-3}+1)}{u^2-4} 
=\lim\limits_{u \ra 2} \dfrac{(u-2)(\sqrt[]{u^2-3}+1)}{(u+2)(u-2)} 
=\lim\limits_{u \ra 2} \dfrac{\sqrt[]{u^2-3}+1}{u+2}
=\dfrac{1}{2}$$
\item $\lim\limits_{x \ra 1} \dfrac{x^4+2x^3-3x^2+4x-4}{x^4+2x-3}$\\
\\
Aquí podemos ver claramente que ambas partes del limite se hacen 0 en $x=1$. Sin embargo, al ser un polinomio de grado 4, se hace dificil factorizarlo \textit{al ojo}.\\
Recordemos que por el Teorema del Resto, sabemos que si un polinomio $P(x)$ cumple con $P(a) = 0$, entonces $P(x)$ es divisible por $(x-a)$. Por lo tanto, ya sabemos que ambas partes son divisibles por $(x-1)$. Para factorizar entonces, podemos utilizar la división polinómica, dividiendo numerador y denominador por $(x-1)$.
En primer lugar, hagamos $x^4+2x^3-3x^2+4x-4 : x-1$, es decir,
\[
\renewcommand\arraystretch{1.5}
\setlength\doublerulesep{0pt}
\begin{array}{rrrrrr}
\multicolumn{1}{r|}{1} & 1 & 2 & -3 & 4 & -4\\\cline{2-6}
& & 1& 3 & 0 & 4\\\cline{2-6}
& 1 & 3& 0 & 4 & 0
\end{array}
\]
por lo que $x^4+2x^3-3x^2+4x-4 : x-1 = x^3+3x^2+4$\\
Luego, $x^4+2x-3 : x-1$, esto es,
\[
\renewcommand\arraystretch{1.5}
\setlength\doublerulesep{0pt}
\begin{array}{rrrrrr}
\multicolumn{1}{r|}{1} & 1 & 0 & 0 & 2 & -3\\\cline{2-6}
& & 1& 1 & 1 & 3\\\cline{2-6}
& 1 & 1& 1 & 3 & 0
\end{array}
\]
por lo que $x^4+2x-3 : x-1 = x^3+x^2+x+3$
Entonces, podemos escribir el límite como
$$\lim\limits_{x \ra 1} \dfrac{(x^3+3x^2+4)(x-1)}{(x^3+x^2+x+3)(x-1)}
= \lim\limits_{x \ra 1} \dfrac{(x^3+3x^2+4)}{(x^3+x^2+x+3)}
= \dfrac{8}{6} = \dfrac{4}{3}$$
\item 
\item 
\end{enumerate}
\end{solucion}
\item Determina $a$ para que el siguiente límite exista:
$$ \lim\limits_{x \ra 2} \dfrac{x^2+ax+6}{x^2-4} $$
\begin{solucion}

\end{solucion}
\item Aplicando la definición de límite, demostrar que
$$\lim\limits_{x \ra 1} \dfrac{x+1}{2} = 2$$
\begin{solucion}
Para demostrar el límite por definición, debemos demostrar que
$$\left| \dfrac{x+3}{2} - 2 \right| < \epsilon \ra |x-1| < \delta$$
para algún $\epsilon$ y $\delta$. Procedamos,
$$\left| \dfrac{x+3}{2} - 2 \right| < \epsilon$$
$$\left| \dfrac{x+3-4}{2} \right| < \epsilon$$
$$\left| \dfrac{x-1}{2} \right| < \epsilon$$
$$ \dfrac{|x-1|}{2} < \epsilon$$
$$ |x-1| < 2\epsilon$$
Luego, basta tomar $\delta = 2 \epsilon$ y se cumple la definición del límite. \\
De sere necesario, se podría demostrar tomando un valor arbitrario suficientemente pequeño de $\epsilon$ y su correspondiente $\delta$
\end{solucion}
\item Calcule el siguiente límite, en caso de que exista
$$ \lim\limits_{x \ra 0} x^2e^{\sin{\frac{1}{x}}} $$
\textit{\textbf{Hint:} Utilice el teorema del sandwich}
\begin{solucion}

\end{solucion}
\end{preguntas}
\end{document}