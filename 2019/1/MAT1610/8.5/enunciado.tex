\documentclass[12pt]{article}

\usepackage{fullpage}
\usepackage{graphicx}
\usepackage{amssymb}
\usepackage{amsmath}
\usepackage[none]{hyphenat}
\usepackage{parskip}
\usepackage[spanish]{babel}
\usepackage[utf8]{inputenc}
\usepackage{hyperref}
\usepackage{fancyhdr}
\usepackage{tasks}
\usepackage{mdframed}
\usepackage{xcolor}
\usepackage{pgfplots}
\usepackage[makeroom]{cancel}
\usepackage{multicol}
\usepackage[shortlabels]{enumitem}
\usepackage{stackrel}
\usepackage{tkz-tab}
\usepackage{xpatch}
\xpatchcmd{\tkzTabLine}{$0$}{$\bullet$}{}{}

\setlength{\headheight}{10pt}
\setlength{\headsep}{10pt}
\pagestyle{fancy}
\rhead{\ayudantia \ - \alumno}
\tikzset{t style/.style={style=solid}}

\newcommand*{\mybox}[2]{\colorbox{#1!30}{\parbox{.98\linewidth}{#2}}}

\newenvironment{solucion}
{\begin{mdframed}[backgroundcolor=black!10]
		{\bf Solución:}\\
	}
	{
	\end{mdframed}
}

\newenvironment{alternativas}[1]
{\begin{multicols}{#1}
		\begin{enumerate}[a)]
		}
		{
		\end{enumerate}
	\end{multicols}
}

\newenvironment{preguntas}
{\begin{enumerate}\itemsep12pt
	}
	{
	\end{enumerate}
}

\newcommand{\ayudantia}{{\sc Ayudantía 8.5}}
\newcommand{\tituloayu}{Compilado I2}
\newcommand{\fecha}{1 de mayo de 2019}
\newcommand{\sigla}{MAT1610}
\newcommand{\nombre}{Cálculo I}
\newcommand{\profesor}{Amal Taarabt}
\newcommand{\ano}{2019}
\newcommand{\semestre}{1}
\newcommand{\mail}{mat1610@ifcastaneda.cl}
\newcommand{\alumno}{Ignacio Castañeda - \mail}

\newcommand{\ev}{\Big|}
\newcommand{\ra}{\rightarrow}
\newcommand{\lra}{\leftrightarrow}
\newcommand{\N}{\mathbb{N}}
\newcommand{\R}{\mathbb{R}}
\newcommand{\Exp}[1]{\mathcal{E}_{#1}}
\newcommand{\List}[1]{\mathcal{L}_{#1}}
\newcommand{\EN}{\Exp{\N}}
\newcommand{\LN}{\List{\N}}
\newcommand{\comment}[1]{}
\newcommand{\lb}{\\~\\}
\newcommand{\eop}{_{\square}}
\newcommand{\hsig}{\hat{\sigma}}
\newcommand{\widesim}[2][1.5]{
	\mathrel{\overset{#2}{\scalebox{#1}[1]{$\sim$}}}
}
\newcommand{\wsim}{\widesim{}}
\newcommand{\lh}{\stackrel{L'H}{=}}

\begin{document}
\thispagestyle{empty}

\begin{minipage}{2cm}
	\includegraphics[width=2cm]{../../../../img/logo.pdf}
	\vspace{0.5cm}
\end{minipage}
\begin{minipage}{\linewidth}
	\begin{tabular}{lrl}
		{\scriptsize\sc Pontificia Universidad Catolica de Chile} & \hspace*{0.7in}Curso: &
		\sigla  - \nombre\\
		{\sc Facultad de Matemáticas}&
		Profesor: & \profesor \\
		{\sc Semestre \ano-\semestre} & Ayudante: & {Ignacio Castañeda}\\
		& {Mail:} & \texttt{\mail}
	\end{tabular}
\end{minipage}

\vspace{-10mm}
\begin{center}
	{\LARGE\bf \ayudantia}\\
	\vspace{0.1cm}
	{\tituloayu}\\
	\vspace{0.1cm}
	\fecha\\
	\vspace{0.4cm}
\end{center}

\begin{preguntas}
\item Calcule la derivada de las siguientes funciones
\begin{tasks}(2)
\task $f(x) = x^4 + 6e^x + 2\cos x$
\task $f(x) = \dfrac{3x^3}{\sin x}$
\task $f(x) = 3ln(x)\tan x$
\task $f(x) = \arcsin x + \dfrac{e^x}{x}$
\end{tasks}
\item Dado $f(x)$, determinar $f'(x)$
\begin{tasks}(2)
\task $f(x) = \sin x(x^4+\cot x)$
\task $f(x) = \cos ^2 (x^3) \sin(x^2)\csc(x)$
\task $f(x) = e^{3x} + ln(3(x+1)^5)$
\task $f(x) = 5^x cos(3x)$
\end{tasks}
\item Sea $f(x)$ una función derivable cuyo gráfico pasa por el punto $(1,1)$ tal que $f'(1) = -2$.\\
Si $g(x) = \dfrac{1}{x^2(f(x))^5}$, calcule $g'(1)$.
\item Si $h(x) = f(x\ f(x))$ donde $f(1)=2$, $f'(1)=4$ y $f'(2) = 5$, encuentre $h'(1)$.
\item Sea $f(x) = ln(x^2+3^x)$. Determine $f'(0) + f''(0)$.
\item Calcular $y'$
\begin{tasks}(2)
\task $x^2+y^2-7=0$
\task $x^2y-xy^2+y^2=4$
\end{tasks}
\item Sea
$$-x^2+xy+y^2=1$$
Encuentre el valor de $y''$ cuando $(x,y)=(1,1)$
\item Determine todos los puntos de la curva $x^2y^2 + e^{3y} = e$ cuya tangente es horizontal.
\item Encuentre las derivadas de las siguientes funciones
\begin{tasks}(2)
\task $f(x) = x^{\sin x}$
\task $f(x) = (x^2+3)^{5x-1}$
\end{tasks}
\item Dada la función invertible $f(x) = x^3 + 3x + 6$, calcular $(f^{-1})'(6)$
\item Sean $f$ y $g$ funciones derivables e invertibles. La tabla adjunta muestra los valores de $f$, $g$ y sus derivadas sobre algunos valores. Calcule $(f \circ f)'(3)$ y $(g^{-1} \circ f)'(2)$
$$
\begin{tabular}{|l|l|l|l|l|}
\hline
  & f & g & f' & g' \\ \hline
1 & 3 & 2 & 4  & 1  \\ \hline
2 & 1 & 3 & 3  & 2  \\ \hline
3 & 2 & 4 & 2  & 4  \\ \hline
4 & 4 & 1 & 1  & 2  \\ \hline
\end{tabular}
$$
\item Sea $f$ una función derivable en un intervalo $(a,b)$ tal que $f'(x) = \dfrac{e^{f(x)}}{1+(f(x))^2}$ para todo $x \in (a,b)$. Demuestre que $f$ es invertible y determine $(f^{-1})'(x)$.
\item En un estanque de forma cónica, con radio basal 5 $m$ y altura 10 $m$, con el vértice hacia abajo, se hace entrar agua a razón de 9 $m^3/min$. ¿Cuán rápido varía el nivel del agua cuando esta tiene una profundidad de 6 metros?
\item La ley de los gases para un gas ideal a la temperatura absoluta $T$ (en Kelvin) y la presión $P$ (en atmósferas) con un volumen $V$ (en litros) es
$$PV = nRT$$
donde $n$ es constante y corresponde al número de moles del gas y $R = 0,0821$ es la constante de los gases.\\
Suponga que en el instante $t_0$ la presión $P$ es igual a 8 $atm$ y que esta aumenta a razón de 0,1 $atm/min$. Además se sabe que en ese mismo instante el volumen $V$ es de 10 litros y que este disminuye a razón de 0,15 $lt/min$.\\
Determine la razón de cambio de $T$, con respecto al tiempoo, en el instante $t_0$, sabiendo que la constante $n = 10\ mol$.
\item Para cada una de las siguientes funciones, encontrar sus máximos y mínimos locales, en caso de haber
\begin{tasks}(2)
\task $f(x) = x^4 - 8x^2 + 3$
\task $f(x)=x+ln(x^2-1)$
\end{tasks}
\item Determine el máximo y mínimo absoluto de la función 
$$f(x) = 2\cos (x) + \sin (2x)$$
en el intervalo $[0, 2\pi]$.
\item Encuentre los máximos y mínimos globales de la función
$$f(x) = \dfrac{|x|}{1+x^2}$$
en el intervalo $x \in [-3,2]$
\item Estudiar, según el valor de la constante $k$, los puntos críticos y los puntos de inflexión de la función
$$f(x) = x^3-3kx^2+12x$$
\item Probar que la ecuación $1+2x+3x^2+4x^3=0$ tiene solución única.
\item Demuestre que la ecuación $\sin (x) = 2x-1$ tiene exáctamente una raíz real.
\item Sea $f$ una función derivable en $[0, \infty)$, tal que $\forall x \in [0, \infty)(f(2x)=2f(x))$.\\
Demuestre que
$$\forall x \in (0, \infty) \exists c > 0 \left(f'(c) = \dfrac{f(x)}{x}\right)$$
\item Si $f$ es una función dos veces derivable en $[a,b]$ tal que $f(a)=f(b)=0$ y $f(c) > 0$ con $a < c < b$, demuestre que
$$\exists \alpha \in (a,b) (f''(\alpha) < 0)$$
\item Determine los siguientes límites, en caso de que existan
\begin{tasks}(3)
\task $\lim\limits_{x\ra 0}\dfrac{(\arcsin x)^2}{1-cos(3x)}$
\task $\lim\limits_{x\ra 0} x^{\frac{1}{\ln (e^x-1)}}$
\task $\lim\limits_{x\ra \infty} \left(1+\dfrac{a}{x}\right)^{bx}$
\end{tasks}
\item Determine las asíntotas de la función $f(x) = xe^{1/x}$.
\end{preguntas}
\end{document}