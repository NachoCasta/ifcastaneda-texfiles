\documentclass[12pt]{article}

\usepackage{fullpage}
\usepackage{graphicx}
\usepackage{amssymb}
\usepackage{amsmath}
\usepackage[none]{hyphenat}
\usepackage{parskip}
\usepackage[spanish]{babel}
\usepackage[utf8]{inputenc}
\usepackage{hyperref}
\usepackage{fancyhdr}
\usepackage{tasks}
\usepackage{mdframed}
\usepackage{xcolor}
\usepackage{pgfplots}
\usepackage[makeroom]{cancel}
\usepackage{multicol}
\usepackage[shortlabels]{enumitem}
\usepackage{stackrel}
\usepackage{tkz-tab}
\usepackage{xpatch}
\xpatchcmd{\tkzTabLine}{$0$}{$\bullet$}{}{}

\setlength{\headheight}{10pt}
\setlength{\headsep}{10pt}
\pagestyle{fancy}
\rhead{\ayudantia \ - \alumno}
\tikzset{t style/.style={style=solid}}

\newcommand*{\mybox}[2]{\colorbox{#1!30}{\parbox{.98\linewidth}{#2}}}

\newenvironment{solucion}
{\begin{mdframed}[backgroundcolor=black!10]
		{\bf Solución:}\\
	}
	{
	\end{mdframed}
}

\newenvironment{alternativas}[1]
{\begin{multicols}{#1}
		\begin{enumerate}[a)]
		}
		{
		\end{enumerate}
	\end{multicols}
}

\newenvironment{preguntas}
{\begin{enumerate}\itemsep12pt
	}
	{
	\end{enumerate}
}

\newcommand{\ayudantia}{{\sc Ayudantía 4}}
\newcommand{\tituloayu}{Derivadas y regla de la cadena}
\newcommand{\fecha}{2 de abril de 2019}
\newcommand{\sigla}{MAT1610}
\newcommand{\nombre}{Cálculo I}
\newcommand{\profesor}{Amal Taarabt}
\newcommand{\ano}{2019}
\newcommand{\semestre}{1}
\newcommand{\mail}{mat1610@ifcastaneda.cl}
\newcommand{\alumno}{Ignacio Castañeda - \mail}

\newcommand{\ev}{\Big|}
\newcommand{\ra}{\rightarrow}
\newcommand{\lra}{\leftrightarrow}
\newcommand{\N}{\mathbb{N}}
\newcommand{\R}{\mathbb{R}}
\newcommand{\Exp}[1]{\mathcal{E}_{#1}}
\newcommand{\List}[1]{\mathcal{L}_{#1}}
\newcommand{\EN}{\Exp{\N}}
\newcommand{\LN}{\List{\N}}
\newcommand{\comment}[1]{}
\newcommand{\lb}{\\~\\}
\newcommand{\eop}{_{\square}}
\newcommand{\hsig}{\hat{\sigma}}
\newcommand{\widesim}[2][1.5]{
	\mathrel{\overset{#2}{\scalebox{#1}[1]{$\sim$}}}
}
\newcommand{\wsim}{\widesim{}}
\newcommand{\lh}{\stackrel{L'H}{=}}

\begin{document}
\thispagestyle{empty}

\begin{minipage}{2cm}
	\includegraphics[width=2cm]{../../../../img/logo.pdf}
	\vspace{0.5cm}
\end{minipage}
\begin{minipage}{\linewidth}
	\begin{tabular}{lrl}
		{\scriptsize\sc Pontificia Universidad Catolica de Chile} & \hspace*{0.7in}Curso: &
		\sigla  - \nombre\\
		{\sc Facultad de Matemáticas}&
		Profesor: & \profesor \\
		{\sc Semestre \ano-\semestre} & Ayudante: & {Ignacio Castañeda}\\
		& {Mail:} & \texttt{\mail}
	\end{tabular}
\end{minipage}

\vspace{-10mm}
\begin{center}
	{\LARGE\bf \ayudantia}\\
	\vspace{0.1cm}
	{\tituloayu}\\
	\vspace{0.1cm}
	\fecha\\
	\vspace{0.4cm}
\end{center}

\begin{preguntas}
\item Calcule la derivada de las siguientes funciones
\begin{tasks}(2)
\task $f(x) = x^4 + 6e^x + 2\cos x$
\task $f(x) = \dfrac{3x^3}{\sin x}$
\task $f(x) = 3ln(x)\tan x$
\task $f(x) = \arcsin x + \dfrac{e^x}{x}$
\end{tasks}
\begin{solucion}

\begin{enumerate}[a)]
\item $f'(x) = 4x^3 + 6e^x - 2\sin x$
\item $f'(x) = \dfrac{9x^2\sin x - 3x^3\cos x}{\sin^2 x}$
\item $f'(x) = 3\dfrac{1}{x}\tan x + 3ln(x) \sec^2 x = \dfrac{3 \tan x}{x} + 3ln(x) \sec^2 x$
\item $f'(x) = \dfrac{1}{\sqrt[]{1-x^2}} + \dfrac{xe^x-e^x}{x^2}$
\end{enumerate}
\end{solucion}
\item Dado $f(x)$, determinar $f'(x)$
\begin{tasks}(2)
\task $f(x) = \sin x(x^4+\cot x)$
\task $f(x) = \cos ^2 (x^3) \sin(x^2)\csc(x)$
\task $f(x) = e^{3x} + ln(3(x+1)^5)$
\task $f(x) = 5^x cos(3x)$
\end{tasks}
\begin{solucion}

\begin{enumerate}[a)]
\item $$\begin{array}{rcl}
	f(x) & = & \sin x(x^4+\cot x) \\
	f'(x) & = & (\sin x(x^4+\cot x))' \\
	f'(x) & = & (\sin x)'(x^4+\cot x) + \sin x(x^4+\cot x)' \\
	f'(x) & = & \cos x(x^4+\cot x) + \sin x(4x^3-\csc^2 x)
	\end{array}$$
\item {\scriptsize$$\begin{array}{rcl}
	f(x) & = & \cos ^2 (x^3) \sin(x^2)\csc(x) \\
	f'(x) & = & (\cos ^2 (x^3) (\sin(x^2)\csc(x)))' \\
	f'(x) & = & \cos ^2 (x^3)' \sin(x^2)\csc(x) + \cos ^2 (x^3) (\sin(x^2)\csc(x))' \\
	f'(x) & = & 2\cos (x^3)\cos (x^3)' \sin(x^2)\csc(x) + \cos ^2 (x^3) (\sin(x^2)'\csc(x) + \sin(x^2)\csc(x)') \\
	f'(x) & = & 2\cos (x^3)\cos (x^3)' \sin(x^2)\csc(x) + \cos ^2 (x^3) (\sin(x^2)'\csc(x) + \sin(x^2)\csc(x)') \\
	f'(x) & = & 2\cos (x^3)(-\sin (x^3))3x^2 \sin(x^2)\csc(x) + \cos ^2 (x^3) (\cos(x^2)2x\csc(x) + \sin(x^2)(-\cot x\csc x)) \\
	f'(x) & = & -2\sin (2x^3)3x^2 \sin(x^2)\csc(x) + \cos ^2 (x^3) (\cos(x^2)2x\csc(x) - \sin(x^2)\cot x\csc x)
	\end{array}$$}
\item $$\begin{array}{rcl}
	f(x) & = & e^{3x} + ln(3(x+1)^5) \\
	f'(x) & = & (e^{3x} + ln(3(x+1)^5))' \\
	f'(x) & = & (e^{3x})' + ln(3(x+1)^5)' \\
	f'(x) & = & 3e^{3x} + \dfrac{1}{3(x+1)^5}(3(x+1)^5)' \\
	f'(x) & = & 3e^{3x} + \dfrac{1}{3(x+1)^5}15(x+1)^4(x+1)' \\
	f'(x) & = & 3e^{3x} + \dfrac{5(x+1)^4}{(x+1)^5} \\
	f'(x) & = & 3e^{3x} + \dfrac{5}{x+1}
	\end{array}$$
\item $$\begin{array}{rcl}
	f(x) & = & 5^x cos(3x) \\
	f'(x) & = & (5^x cos(3x))' \\
	f'(x) & = & (5^x)' cos(3x) + 5^x cos(3x)' \\
	f'(x) & = & 5^x ln(5) cos(3x) + 5^x (-\sin 3x)3 \\
	f'(x) & = & ln(5)5^x cos(3x) - 3\cdot 5^x \sin 3x
	\end{array}$$
\end{enumerate}
\end{solucion}
\item Sea $f(x)$ una función derivable cuyo gráfico pasa por el punto $(1,1)$ tal que $f'(1) = -2$.\\
Si $g(x) = \dfrac{1}{x^2(f(x))^5}$, calcule $g'(1)$.
\begin{solucion}
Del enunciado sabemos que
$$f(1) = 1, \qquad f'(1) = -2$$
Por la regla de la cadena, tenemos que
$$\begin{array}{rcl}
g'(x) & = & \dfrac{-1}{(x^2(f(x))^5)^2}(x^2(f(x))^5)'\\
& = & \dfrac{-1}{x^4(f(x))^{10}}(2x(f(x))^5 + 5(f(x))^4f'(x)x^2)\\
& = & \dfrac{-2x(f(x))^5 - 5(f(x))^4f'(x)x^2}{x^4(f(x))^{10}}
\end{array}$$
Luego, reemplazando con $x=1$, obtenemos

$$g'(1) =\dfrac{-2(f(1))^5 - 5(f(1))^4f'(1)}{(f(1))^{10}} = -2-5\cdot (-2) = 8$$
\end{solucion}
\item Si $h(x) = f(x\ f(x))$ donde $f(1)=2$, $f'(1)=4$ y $f'(2) = 5$, encuentre $h'(1)$.
\begin{solucion}
Por la regla de la cadena, tenemos que
$$h'(x) = f'(x\ f(x)) (x\ f(x))' = f'(x\ f(x))(f(x) + xf'(x))$$
Luego, reemplazando en $x=1$, tenemos que
$$h'(x) = f'(f(1))(f(1) + f'(1))$$
Finalmente, reemplazando con la información del enunciado,
$$h'(x) = f'(2)(2+4) = 5\cdot 6 = 30$$
\end{solucion}
\end{preguntas}
\end{document}