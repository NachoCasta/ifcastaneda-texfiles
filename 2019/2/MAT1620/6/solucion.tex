\documentclass[12pt]{article}

\usepackage{fullpage}
\usepackage{graphicx}
\usepackage{amssymb}
\usepackage{amsmath}
\usepackage[none]{hyphenat}
\usepackage{parskip}
\usepackage[spanish]{babel}
\usepackage[utf8]{inputenc}
\usepackage{hyperref}
\usepackage{fancyhdr}
\usepackage{tasks}
\usepackage{mdframed}
\usepackage{xcolor}
\usepackage{pgfplots}
\usepackage[makeroom]{cancel}
\usepackage{multicol}
\usepackage[shortlabels]{enumitem}
\usepackage{stackrel}
\usepackage{tkz-tab}
\usepackage{xpatch}
\usepackage{tkz-euclide}
\usetkzobj{all}
\usepackage{tabto}
\xpatchcmd{\tkzTabLine}{$0$}{$\bullet$}{}{}

\setlength{\headheight}{10pt}
\setlength{\headsep}{10pt}
\pagestyle{fancy}
\rhead{\ayudantia \ - \alumno}
\tikzset{t style/.style={style=solid}}

\newcommand*{\mybox}[2]{\colorbox{#1!30}{\parbox{.98\linewidth}{#2}}}

\newenvironment{solucion}
{\begin{mdframed}[backgroundcolor=black!10]
		{\bf Solución:}\\
	}
	{
	\end{mdframed}
}

\newenvironment{alternativas}[1]
{\begin{multicols}{#1}
		\begin{enumerate}[a)]
		}
		{
		\end{enumerate}
	\end{multicols}
}

\newenvironment{preguntas}
{\begin{enumerate}\itemsep12pt
	}
	{
	\end{enumerate}
}

\newcommand{\ayudantia}{{\sc Ayudantía 6}}
\newcommand{\tituloayu}{Funciones de varias variables}
\newcommand{\fecha}{17 de septiembre de 2019}
\newcommand{\sigla}{MAT1620}
\newcommand{\nombre}{Cálculo II}
\newcommand{\profesor}{Wolfgang Rivera}
\newcommand{\ano}{2019}
\newcommand{\semestre}{2}
\newcommand{\mail}{mat1620@ifcastaneda.cl}
\newcommand{\alumno}{Ignacio Castañeda - \mail}

\newcommand{\ev}{\Big|}
\newcommand{\ra}{\rightarrow}
\newcommand{\lra}{\leftrightarrow}
\newcommand{\N}{\mathbb{N}}
\newcommand{\R}{\mathbb{R}}
\newcommand{\Exp}[1]{\mathcal{E}_{#1}}
\newcommand{\List}[1]{\mathcal{L}_{#1}}
\newcommand{\EN}{\Exp{\N}}
\newcommand{\LN}{\List{\N}}
\newcommand{\comment}[1]{}
\newcommand{\lb}{\\~\\}
\newcommand{\eop}{_{\square}}
\newcommand{\hsig}{\hat{\sigma}}
\newcommand{\widesim}[2][1.5]{
	\mathrel{\overset{#2}{\scalebox{#1}[1]{$\sim$}}}
}
\newcommand{\wsim}{\widesim{}}
\newcommand{\lh}{\stackrel{L'H}{=}}

\begin{document}
\thispagestyle{empty}

\begin{minipage}{2cm}
	\includegraphics[width=2cm]{../../../../img/logo.pdf}
	\vspace{0.5cm}
\end{minipage}
\begin{minipage}{\linewidth}
	\begin{tabular}{lrl}
		{\scriptsize\sc Pontificia Universidad Catolica de Chile} & \hspace*{0.7in}Curso: &
		\sigla  - \nombre\\
		{\sc Facultad de Matemáticas}&
		Profesor: & \profesor \\
		{\sc Semestre \ano-\semestre} & Ayudante: & {Ignacio Castañeda}\\
		& {Mail:} & \texttt{\mail}
	\end{tabular}
\end{minipage}

\vspace{-10mm}
\begin{center}
	{\LARGE\bf \ayudantia}\\
	\vspace{0.1cm}
	{\tituloayu}\\
	\vspace{0.1cm}
	\fecha\\
	\vspace{0.4cm}
\end{center}

\begin{preguntas}
\item Determina el dominio de las siguientes funciones
\begin{tasks}(2)
\task $f(x, y) = \sqrt[]{x+y} + \ln(x^2+y^2)$
\task $f(x,y) = \dfrac{x+y}{x^2-y^2}$
\task $f(x,y,z) = ln(z+y) - \dfrac{1}{x^2 +z^2}$
\task $f(x,y,z) = \sqrt[]{\ln(x+y+z)}$
\end{tasks}
\begin{solucion}

\begin{enumerate}[a)]
\item 
\item 
\item 
\item 
\end{enumerate}
\end{solucion}
\item Determina el recorrido de las siguientes funciones
\begin{tasks}(2)
\task $f(x,y) = x^2 + y^2 + 3$
\task $f(x,y,z) = \ln(\sqrt[]{x^2+y^2+4}) + z^2$
\end{tasks}
\begin{solucion}

\begin{enumerate}[a)]
\item 
\item 
\end{enumerate}
\end{solucion}
\item Grafique las curvas de nivel de la función 
	$$ f(x,y) = \sqrt[]{9-x^2-y^2}$$
	para $k=0,1,2,3$
\begin{solucion}

		Las curvas de nivel de la función corresponden a
		$$\sqrt[]{9-x^2-y^2} = k \ra 9-x^2-y^2 = k^2 \ra x^2+y^2 = 9-k^2$$
		Notemos que esto corresponde a circunferencias de radio $\sqrt[]{9-k^2}$.
		
		Reemplazando con los valores de $k$ indicados, las curvas de nivel a graficar son
		$$k = 0 \ra x^2+y^2 = 9$$
		$$k = 1 \ra x^2+y^2 = 8$$
		$$k = 2 \ra x^2+y^2 = 5$$
		$$k = 3 \ra x^2+y^2 = 0$$
		Graficando,
		
		\begin{center}
			\includegraphics[scale=0.4]{../../../../img/curvasdenivel}			
		\end{center}
\end{solucion}
\end{preguntas}
\end{document}