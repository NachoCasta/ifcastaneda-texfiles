\documentclass[12pt]{article}

\usepackage{fullpage}
\usepackage{graphicx}
\usepackage{amssymb}
\usepackage{amsmath}
\usepackage[none]{hyphenat}
\usepackage{parskip}
\usepackage[spanish]{babel}
\usepackage[utf8]{inputenc}
\usepackage{hyperref}
\usepackage{fancyhdr}
\usepackage{tasks}
\usepackage{mdframed}
\usepackage{xcolor}
\usepackage{pgfplots}
\usepackage[makeroom]{cancel}
\usepackage{multicol}
\usepackage[shortlabels]{enumitem}
\usepackage{stackrel}
\usepackage{tkz-tab}
\usepackage{xpatch}
\usepackage{tkz-euclide}
\usetkzobj{all}
\usepackage{tabto}
\xpatchcmd{\tkzTabLine}{$0$}{$\bullet$}{}{}

\setlength{\headheight}{10pt}
\setlength{\headsep}{10pt}
\pagestyle{fancy}
\rhead{\ayudantia \ - \alumno}
\tikzset{t style/.style={style=solid}}

\newcommand*{\mybox}[2]{\colorbox{#1!30}{\parbox{.98\linewidth}{#2}}}

\newenvironment{solucion}
{\begin{mdframed}[backgroundcolor=black!10]
		{\bf Solución:}\\
	}
	{
	\end{mdframed}
}

\newenvironment{alternativas}[1]
{\begin{multicols}{#1}
		\begin{enumerate}[a)]
		}
		{
		\end{enumerate}
	\end{multicols}
}

\newenvironment{preguntas}
{\begin{enumerate}\itemsep12pt
	}
	{
	\end{enumerate}
}

\newcommand{\ayudantia}{{\sc Ayudantía 11}}
\newcommand{\tituloayu}{Lagrange e integrales dobles}
\newcommand{\fecha}{5 de diciembre de 2019}
\newcommand{\sigla}{MAT1620}
\newcommand{\nombre}{Cálculo II}
\newcommand{\profesor}{Wolfgang Rivera}
\newcommand{\ano}{2019}
\newcommand{\semestre}{2}
\newcommand{\mail}{mat1620@ifcastaneda.cl}
\newcommand{\alumno}{Ignacio Castañeda - \mail}

\newcommand{\ev}{\Big|}
\newcommand{\ra}{\rightarrow}
\newcommand{\lra}{\leftrightarrow}
\newcommand{\N}{\mathbb{N}}
\newcommand{\R}{\mathbb{R}}
\newcommand{\Exp}[1]{\mathcal{E}_{#1}}
\newcommand{\List}[1]{\mathcal{L}_{#1}}
\newcommand{\EN}{\Exp{\N}}
\newcommand{\LN}{\List{\N}}
\newcommand{\comment}[1]{}
\newcommand{\lb}{\\~\\}
\newcommand{\eop}{_{\square}}
\newcommand{\hsig}{\hat{\sigma}}
\newcommand{\widesim}[2][1.5]{
	\mathrel{\overset{#2}{\scalebox{#1}[1]{$\sim$}}}
}
\newcommand{\wsim}{\widesim{}}
\newcommand{\lh}{\stackrel{L'H}{=}}

\begin{document}
\thispagestyle{empty}

\begin{minipage}{2cm}
	\includegraphics[width=2cm]{../../../../img/logo.pdf}
	\vspace{0.5cm}
\end{minipage}
\begin{minipage}{\linewidth}
	\begin{tabular}{lrl}
		{\scriptsize\sc Pontificia Universidad Catolica de Chile} & \hspace*{0.7in}Curso: &
		\sigla  - \nombre\\
		{\sc Facultad de Matemáticas}&
		Profesor: & \profesor \\
		{\sc Semestre \ano-\semestre} & Ayudante: & {Ignacio Castañeda}\\
		& {Mail:} & \texttt{\mail}
	\end{tabular}
\end{minipage}

\vspace{-10mm}
\begin{center}
	{\LARGE\bf \ayudantia}\\
	\vspace{0.1cm}
	{\tituloayu}\\
	\vspace{0.1cm}
	\fecha\\
	\vspace{0.4cm}
\end{center}

\begin{preguntas}
\item Hallar los valores extremos de $f(x,y) = x^2+2y^2-2x+3$ en el disco cerrado $x^2+y^2 \leq 10$.
\begin{solucion}
Cuando tenemos restricciones, debemos hacer un paso adicional. \\
\\
En primer lugar, encontramos los puntos críticos igual que siempre, es decir, igualando la gradiente a cero y solo guardamos los que cumplan con las restricciones. \\
\\
Luego, utilizamos Lagrange para encontrar los puntos críticos de los bordes. Para esto, definimos cada restricción como una función adicional, despejando toda la restricción hacia un lado de tal manera que se iguale a cero. Entonces, resolvemos el sistema
$$\nabla f = \lambda \nabla g + \mu \nabla h + \dots$$
$$g(x,y) = 0, \quad h(x,y) = 0, \dots$$
para obtener los puntos críticos de los bordes.\\
\\
Finalmente, evaluamos todos los puntos críticos obtenidos para ver cual es el máximo y cual es el mínimo.\\

Procedamos con le ejercicio:\\

En primer lugar, tenemos que
$$\nabla f = \begin{pmatrix} 2x - 2 \\ 4y \end{pmatrix} = \begin{pmatrix} 0 \\ 0 \end{pmatrix}$$
Esto se traduce en el sistema
$$\begin{array}{rcl} 2x-2 & = & 0 \\ 4y & = & 0 \end{array}
\ra 
\begin{array}{rcl} x & = & 1 \\ y & = & 0 \end{array}$$
Por lo que $P_1(1,0)$. Notemos que este punto si cumple con la restricción, ya que $1^2 + 0^2 \leq 10$, por lo que si es un punto crítico.\\
\\
Veamos ahora que sucede con los bordes. Sea
$$g(x,y) = x^2 + y^2 - 10$$
$$\nabla g = \begin{pmatrix} 2x \\ 2y \end{pmatrix}$$
Definimos el sistema
$$\begin{array}{rcl} \nabla f & = & \lambda \nabla g \\ g(x,y) & = & 0 \end{array}
\ra 
\begin{array}{rcl}
2x-2 & = & \lambda 2x \\
4y & = & \lambda 2y \\
x^2+y^2 & = & 10
\end{array}$$
Para resolver este sistema, sería tentador pasar el $y$ diviviendo en la segunda ecuación, sin embargo, recordemos que este podría ser cero, por lo que no podemos hacerlo directamente. Si queremos dividir por algun termino que podría ser cero, debemos ver todos los casos posibles (en esta caso son dos, $y = 0$, $y \neq 0$).
\begin{itemize}
\item $y = 0$\\
En este caso, de la últma ecuación obtenemos
$$x^2 = 10 \ra x = \pm \ \sqrt[]{10}$$
De aquí se desprende
$$P_2(\sqrt[]{10}, 0), \quad P_3(-\ \sqrt[]{10}, 0)$$
\item $y \neq 0$\\
De la segunda ecuación,
$$4 = 2\lambda \ra \lambda = 2$$
Reemplazando esto en la primera,
$$2x - 2 = 4x \ra x = -1$$
Por último, de la última ecuación tenemos que
$$1 + y^2 = 10 \ra y^2 = 9 \ra y_1 = 3, \quad y_2 = -3$$
Por lo tanto, los puntos obtenidos son
$$P_4(-1,3), \quad P_5(-1,-3)$$
\end{itemize}
Finalmente, evaluamos los puntos críticos obtenidos en la función original, esto es
$$f(P_1) = f(1,0) = 1-2+3 =2$$
$$f(P_2) = f(\sqrt[]{10},0) = 10 - 2\ \sqrt[]{10} + 3 = 13 - 2\ \sqrt[]{10}$$
$$f(P_3) = f(-\ \sqrt[]{10},0) = 10 + 2\ \sqrt[]{10} + 3 = 13 + 2\ \sqrt[]{10}$$
$$f(P_4) = f(-1,3) = 1+18+2+3=24$$
$$f(P_5) = f(-1,-3) = 1+18+2+3=24$$
Por lo tanto, concluimos que el máximo es $24$ ($P_4$ y $P_5$) y el mínimo es 2 ($P_1$).
\end{solucion}
\item Determine el máximo y el mínimo valor que alcanza la expresión $z=x^2-4xy-y^2+2y$, para $x\geq0;\quad y\geq0;\quad x+y\leq2$.
\begin{solucion}
En primer lugar, tenemos que
$$\nabla f = \begin{pmatrix} 2x - 4y \\ -4x-2y+2 \end{pmatrix} = \begin{pmatrix} 0 \\ 0 \end{pmatrix}$$
Esto se traduce en el sistema
$$\begin{array}{rcl} 2x-4y & = & 0 \\ -4x+2y+2 & = & 0 \end{array}
\ra 
x = \dfrac{2}{5}, \quad y = \dfrac{1}{5}$$
Por lo que nuestro primer punto es $P_1\left(\dfrac{2}{5}, \dfrac{1}{5}\right)$. Notemos que este punto si cumple con las restricción, ya que $\dfrac{2}{5} + \dfrac{1}{5} \leq 2$ y ambos son positivos, por lo que si es un punto crítico.\\
\\
Veamos ahora que sucede con los bordes. \\
\\
Recordemos para que nos sirve Lagrange. Al armar el sistema de ecuaciones\\ correspondiente, con las restricciones que elijamos, Lagrange nos indicará los puntos críticos que se encuentran cumpliendo \textbf{todas} las restricciones que hayamos agregado al sistema. \\
\\
Al tener múltiples restricciones, debemos hacer Lagrange múltiples veces, con todas las restricciones por separado y combinandolas también con las que tengan sentido, esto es, combinando las restricciones que se intersectan entre si.\\
\\
Recordemos también que en Lagrange, las inecuaciones se convierten en igualdades, ya que estamos analizando solamente el borde y no su contenido.\\

Notemos que las restricciones de este ejercicio corresponden a un triangulo hecho por los ejes coordenados y la recta $y = 2-x$. Luego, los casos que tenemos que analizar son cada uno de los bordes del triangulo y todas las intersecciones de estos, es decir, las esquinas del triangulo. Estos casos corresponden a los siguientes:
\begin{itemize}
	\item $x = 0$
	\item $y = 0$
	\item $x + y = 2$
	\item $x = 0 \wedge y = 0 \ra P_2(0,0)$
	\item $y = 0 \wedge x + y = 2 \ra P_3(0,2)$
	\item $x + y = 2 \wedge x = 0 \ra P_4(2,0)$
\end{itemize}
Como pueden ver, varios casos se resolvieron de manera inmediata, sin embargo, seguimos teniendo 3 casos que analizar. No obstante, dado que las restricciones son bastante sencillas y tenemos solo dos variables, podemos evitar el uso de Lagrange y simplemente reemplazar cada uno de los casos en la función original y encontrar los puntos críticos derivando la ecuación que obtengamos (que será de una variable).\\
\\
Veamos entonces el resto de los casos:
\begin{itemize}
	\item $x = 0$\\
	Al reemplazar en $f(x,y)$, tenemos que
	$$f(x,y) = -y^2 + 2y = f(y)$$
	Derivando,
	$$f'(y) = -2y + 2 = 0 \ra y = 1$$
	Por lo tanto, tenemos $P_5(0,1)$
	\item $y = 0$
	Al reemplazar en $f(x,y)$, tenemos que
	$$f(x,y) = x^2 = f(x)$$
	Derivando,
	$$f'(x) = 2x = 0 \ra x = 0$$
	Sin embargo, este punto ya lo tenemos ($P_2(0,0)$).
	\item $x + y = 2$
	Al reemplazar $y = 2 - x$ en $f(x,y)$, tenemos que
	$$f(x,y) = x^2 - 4x(2-x) - (2-x)^2 + 2(2-x) = 4x^2 - 6x = f(x)$$
	Derivando,
	$$f'(x) = 8x - 6 = 0 \ra x = \dfrac{3}{4}, \quad y = \dfrac{5}{4}$$
	Por lo que $P_6\left(\dfrac{3}{4},\dfrac{5}{4}\right)$.
\end{itemize}
Finalmente, evaluamos todos los puntos en la función dada, esto es
$$f(P_1) = f\left(\dfrac{2}{5}, \dfrac{1}{5}\right) = \dfrac{1}{5}$$
$$f(P_2) = f(0,0) = 0$$
$$f(P_3) = f(0,2) = 0$$
$$f(P_4) = f(2,0) = 4$$
$$f(P_5) = f(0,1) = 1$$
$$f(P_6) = f\left(\dfrac{3}{4}, \dfrac{5}{4}\right) = -\dfrac{9}{4}$$
Entonces, concluimos que el mínimo es $-\dfrac{9}{4}$ y el máximo es $4$.
\end{solucion}
\item Si el plano $x+y+2z=2$ corta al paraboloide $z=x^2+y^2$ se forma una elipse. Encuentre los puntos de la elipse que están más cerca y más lejos del origen.
\begin{solucion}

\end{solucion}
\item Resolver las siguientes integrales múltiples
\begin{tasks}(2)
\task $\displaystyle\int_2^4 \displaystyle\int_1^2 ye^{xy}dxdy$
\task $\displaystyle\int_{-1}^1\displaystyle\int_{-1}^1 \dfrac{xy}{1+x^2+y^2}dxdy$
\end{tasks}
\begin{solucion}

\begin{enumerate}[a)]
\item $\displaystyle\int_2^4 \displaystyle\int_1^2 ye^{xy}dxdy = 
			\displaystyle\int_2^4 e^{xy} \ev_1^2 dy = 
			\displaystyle\int_2^4 e^{2y} - e^{y} dy =
			\dfrac{1}{2}e^{2y}\ev_2^4 - e^{y}\ev_2^4$\\
			$=\dfrac{1}{2}e^8 - \dfrac{1}{2}e^4 - e^4 + e^2 =
			\dfrac{1}{2}e^8 - \dfrac{3}{2}e^4 + e^2$	
\item $\displaystyle\int_{-1}^1\displaystyle\int_{-1}^1 \dfrac{xy}{1+x^2+y^2}dxdy = 
			\displaystyle\int_{-1}^1 \dfrac{y}{2} \displaystyle\int_{-1}^1 \dfrac{2x}{1+x^2+y^2}dxdy$\\
			$= \displaystyle\int_{-1}^1 \dfrac{y}{2} ln(1+x^2+y^2) \ev_{-1}^1 dy = 
			\displaystyle\int_{-1}^1 \dfrac{y}{2}( ln(2+y^2) - ln(2+y^2) ) dy$\\
			$=0$
\end{enumerate}
\end{solucion}
\item Calcule
	$$ \displaystyle\int_0^1 \dfrac{x^b - x^a}{log\ x}dx$$
	sabiendo que $\displaystyle\int_a^b x^y dy = \dfrac{x^b - x^a}{log\ x}$.
\begin{solucion}
Reemplazando con la información dada, tenemos que:
		$$ \displaystyle\int_0^1 \dfrac{x^b - x^a}{log\ x}dx = \displaystyle\int_0^1 \displaystyle\int_a^b x^y dydx $$
		Cambiamos el orden de integración, dado que así es más facil integrar
		$$ = \displaystyle\int_a^b \displaystyle\int_0^1 x^y dxdy $$
		$$ = \displaystyle\int_a^b \dfrac{x^{y+1}}{y+1} \ev_0^1 dy $$
		$$ = \displaystyle\int_a^b \dfrac{dy}{y+1} $$
		$$ = ln|y+1|\ev_a^b $$
		con lo que:
		$$ \displaystyle\int_0^1 \dfrac{x^b - x^a}{log\ x}dx = ln\left|\dfrac{b+1}{a+1}\right| $$
\end{solucion}
\end{preguntas}
\end{document}