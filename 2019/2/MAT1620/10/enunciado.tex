\documentclass[12pt]{article}

\usepackage{fullpage}
\usepackage{graphicx}
\usepackage{amssymb}
\usepackage{amsmath}
\usepackage[none]{hyphenat}
\usepackage{parskip}
\usepackage[spanish]{babel}
\usepackage[utf8]{inputenc}
\usepackage{hyperref}
\usepackage{fancyhdr}
\usepackage{tasks}
\usepackage{mdframed}
\usepackage{xcolor}
\usepackage{pgfplots}
\usepackage[makeroom]{cancel}
\usepackage{multicol}
\usepackage[shortlabels]{enumitem}
\usepackage{stackrel}
\usepackage{tkz-tab}
\usepackage{xpatch}
\usepackage{tkz-euclide}
\usetkzobj{all}
\usepackage{tabto}
\xpatchcmd{\tkzTabLine}{$0$}{$\bullet$}{}{}

\setlength{\headheight}{10pt}
\setlength{\headsep}{10pt}
\pagestyle{fancy}
\rhead{\ayudantia \ - \alumno}
\tikzset{t style/.style={style=solid}}

\newcommand*{\mybox}[2]{\colorbox{#1!30}{\parbox{.98\linewidth}{#2}}}

\newenvironment{solucion}
{\begin{mdframed}[backgroundcolor=black!10]
		{\bf Solución:}\\
	}
	{
	\end{mdframed}
}

\newenvironment{alternativas}[1]
{\begin{multicols}{#1}
		\begin{enumerate}[a)]
		}
		{
		\end{enumerate}
	\end{multicols}
}

\newenvironment{preguntas}
{\begin{enumerate}\itemsep12pt
	}
	{
	\end{enumerate}
}

\newcommand{\ayudantia}{{\sc Ayudantía 10}}
\newcommand{\tituloayu}{Derivación implícita, vector gradiente, derivadas direccionales y optimización}
\newcommand{\fecha}{15 de noviembre de 2019}
\newcommand{\sigla}{MAT1620}
\newcommand{\nombre}{Cálculo II}
\newcommand{\profesor}{Wolfgang Rivera}
\newcommand{\ano}{2019}
\newcommand{\semestre}{2}
\newcommand{\mail}{mat1620@ifcastaneda.cl}
\newcommand{\alumno}{Ignacio Castañeda - \mail}

\newcommand{\ev}{\Big|}
\newcommand{\ra}{\rightarrow}
\newcommand{\lra}{\leftrightarrow}
\newcommand{\N}{\mathbb{N}}
\newcommand{\R}{\mathbb{R}}
\newcommand{\Exp}[1]{\mathcal{E}_{#1}}
\newcommand{\List}[1]{\mathcal{L}_{#1}}
\newcommand{\EN}{\Exp{\N}}
\newcommand{\LN}{\List{\N}}
\newcommand{\comment}[1]{}
\newcommand{\lb}{\\~\\}
\newcommand{\eop}{_{\square}}
\newcommand{\hsig}{\hat{\sigma}}
\newcommand{\widesim}[2][1.5]{
	\mathrel{\overset{#2}{\scalebox{#1}[1]{$\sim$}}}
}
\newcommand{\wsim}{\widesim{}}
\newcommand{\lh}{\stackrel{L'H}{=}}

\begin{document}
\thispagestyle{empty}

\begin{minipage}{2cm}
	\includegraphics[width=2cm]{../../../../img/logo.pdf}
	\vspace{0.5cm}
\end{minipage}
\begin{minipage}{\linewidth}
	\begin{tabular}{lrl}
		{\scriptsize\sc Pontificia Universidad Catolica de Chile} & \hspace*{0.7in}Curso: &
		\sigla  - \nombre\\
		{\sc Facultad de Matemáticas}&
		Profesor: & \profesor \\
		{\sc Semestre \ano-\semestre} & Ayudante: & {Ignacio Castañeda}\\
		& {Mail:} & \texttt{\mail}
	\end{tabular}
\end{minipage}

\vspace{-10mm}
\begin{center}
	{\LARGE\bf \ayudantia}\\
	\vspace{0.1cm}
	{\tituloayu}\\
	\vspace{0.1cm}
	\fecha\\
	\vspace{0.4cm}
\end{center}

\begin{preguntas}
\item Si $z^3 - xz - y = -2$, encuentre $\dfrac{\delta^2z}{\delta x \delta y}$ cuando $(x,y,z) = (-1, 0, -1)$.
\item Determine la gradiente de $f$, evalúela en el punto $P$ y encuentre la razón de cambio de $f$ en $P$ en la dirección del vector $u$.
\begin{enumerate}[a)]
\item $f(x,y) = sen(2x+3y)$\tab$P=(-6,4), u=\left(\sqrt[]{3}i - j\right)$
\item $f(x,y) = \dfrac{y^2}{x}$\tab$P=(1,2), u=\left(2i + \sqrt[]{5}j\right)$
\end{enumerate}
\item Suponga que $f(x,y)$ es una función con derivadas parciales continuas en el punto $(1,1)$. Asumir que la derivada direccional en $(1,1)$ en la dirección $\langle3,4\rangle$ 			es 1 y en la dirección $\langle5,12\rangle$ es $-1$.
\begin{enumerate}[a)]
\item Encontrar la ecuación cartesiana del plano tangente en $(1,1,f(1,1))$.
\item Encontrar la derivada direccional de $f(x,y)$ en $(1,1)$ en dirección al origen.	
\end{enumerate}
\item La temperatura en el punto $(x,y)$ de una lamina metálica viene dada por la función $T(x,y) = \dfrac{x}{x^2+y^2}$. Hallar la razón de crecimiento máximo de la temperatura en el punto $(3,4$) y la dirección en que ella ocurre.
\item Encuentre y clasifíque los puntos críticos de $f(x,y)=x^3+3xy^2-15x-12y$.
\item Sea la función
	$$f(x,y)=ax^2y+bxy^2+\dfrac{a^2y^2}{2}+2y$$
	determine los valores de $a$ y $b$ para que la función posea un punto silla en $(1,1)$.
\end{preguntas}
\end{document}