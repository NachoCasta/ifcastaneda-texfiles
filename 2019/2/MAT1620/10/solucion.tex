\documentclass[12pt]{article}

\usepackage{fullpage}
\usepackage{graphicx}
\usepackage{amssymb}
\usepackage{amsmath}
\usepackage[none]{hyphenat}
\usepackage{parskip}
\usepackage[spanish]{babel}
\usepackage[utf8]{inputenc}
\usepackage{hyperref}
\usepackage{fancyhdr}
\usepackage{tasks}
\usepackage{mdframed}
\usepackage{xcolor}
\usepackage{pgfplots}
\usepackage[makeroom]{cancel}
\usepackage{multicol}
\usepackage[shortlabels]{enumitem}
\usepackage{stackrel}
\usepackage{tkz-tab}
\usepackage{xpatch}
\usepackage{tkz-euclide}
\usetkzobj{all}
\usepackage{tabto}
\xpatchcmd{\tkzTabLine}{$0$}{$\bullet$}{}{}

\setlength{\headheight}{10pt}
\setlength{\headsep}{10pt}
\pagestyle{fancy}
\rhead{\ayudantia \ - \alumno}
\tikzset{t style/.style={style=solid}}

\newcommand*{\mybox}[2]{\colorbox{#1!30}{\parbox{.98\linewidth}{#2}}}

\newenvironment{solucion}
{\begin{mdframed}[backgroundcolor=black!10]
		{\bf Solución:}\\
	}
	{
	\end{mdframed}
}

\newenvironment{alternativas}[1]
{\begin{multicols}{#1}
		\begin{enumerate}[a)]
		}
		{
		\end{enumerate}
	\end{multicols}
}

\newenvironment{preguntas}
{\begin{enumerate}\itemsep12pt
	}
	{
	\end{enumerate}
}

\newcommand{\ayudantia}{{\sc Ayudantía 10}}
\newcommand{\tituloayu}{Derivación implícita, vector gradiente, derivadas direccionales y optimización}
\newcommand{\fecha}{15 de noviembre de 2019}
\newcommand{\sigla}{MAT1620}
\newcommand{\nombre}{Cálculo II}
\newcommand{\profesor}{Wolfgang Rivera}
\newcommand{\ano}{2019}
\newcommand{\semestre}{2}
\newcommand{\mail}{mat1620@ifcastaneda.cl}
\newcommand{\alumno}{Ignacio Castañeda - \mail}

\newcommand{\ev}{\Big|}
\newcommand{\ra}{\rightarrow}
\newcommand{\lra}{\leftrightarrow}
\newcommand{\N}{\mathbb{N}}
\newcommand{\R}{\mathbb{R}}
\newcommand{\Exp}[1]{\mathcal{E}_{#1}}
\newcommand{\List}[1]{\mathcal{L}_{#1}}
\newcommand{\EN}{\Exp{\N}}
\newcommand{\LN}{\List{\N}}
\newcommand{\comment}[1]{}
\newcommand{\lb}{\\~\\}
\newcommand{\eop}{_{\square}}
\newcommand{\hsig}{\hat{\sigma}}
\newcommand{\widesim}[2][1.5]{
	\mathrel{\overset{#2}{\scalebox{#1}[1]{$\sim$}}}
}
\newcommand{\wsim}{\widesim{}}
\newcommand{\lh}{\stackrel{L'H}{=}}

\begin{document}
\thispagestyle{empty}

\begin{minipage}{2cm}
	\includegraphics[width=2cm]{../../../../img/logo.pdf}
	\vspace{0.5cm}
\end{minipage}
\begin{minipage}{\linewidth}
	\begin{tabular}{lrl}
		{\scriptsize\sc Pontificia Universidad Catolica de Chile} & \hspace*{0.7in}Curso: &
		\sigla  - \nombre\\
		{\sc Facultad de Matemáticas}&
		Profesor: & \profesor \\
		{\sc Semestre \ano-\semestre} & Ayudante: & {Ignacio Castañeda}\\
		& {Mail:} & \texttt{\mail}
	\end{tabular}
\end{minipage}

\vspace{-10mm}
\begin{center}
	{\LARGE\bf \ayudantia}\\
	\vspace{0.1cm}
	{\tituloayu}\\
	\vspace{0.1cm}
	\fecha\\
	\vspace{0.4cm}
\end{center}

\begin{preguntas}
\item Si $z^3 - xz - y = -2$, encuentre $\dfrac{\delta^2z}{\delta y \delta x}$ cuando $(x,y,z) = (-1, 0, -1)$.
\begin{solucion}
En primer lugar, notemos que lo que nos estan pidiendo es equivalente a $z_{xy}$ (mucho ojo con la notación).\\
\\
Tenemos que
$$z^3 -xz -y = -2$$
Derivamos en $x$,
$$3z^2z_x - z - xz_x = 0$$
Despejando,
$$z_x = \dfrac{z}{3z^2 - x}$$
Para obtener $z_{yx}$, derivamos la ecuación anterior (no la despejada) en $y$, es decir
$$3z^2z_x - z - xz_x = 0$$
Derivando,
$$6zz_yz_x + 3z^2z_{xy} - z_y - xz_{xy} = 0$$
Despejando,
$$z_{xy} = \dfrac{z_y - 6zz_yz_x}{3z^2 -x}$$
Notemos que para calcular el valor de $z_{xy}$, necesitamos también saber $z_y$, por lo que tomamos
$$z^3 -xz -y = -2$$
y derivamos en $y$,
$$3z^2z_y - xz_y - 1 = 0$$
Despejando,
$$z_y = \dfrac{1}{3z^2 - x}$$
Reemplazando con $(-1,0,1)$, tenemos que
$$z_x(-1,0,-1) = \dfrac{-1}{3+1} = -\dfrac{1}{4}$$
$$z_y(-1,0,-1) = \dfrac{1}{3+1} = \dfrac{1}{4}$$
Luego, reemplazando con los valores obtenidos de $z_x$ y $z_y$,
$$z_{xy}(-1,0,-1) = \dfrac{\dfrac{1}{4} - 6 \cdot (-1)\cdot \dfrac{1}{4} \cdot \left(-\dfrac{1}{4}\right)}{3\cdot (-1)^2 + 1} = -\dfrac{1}{32}$$


\end{solucion}
\item Determine la gradiente de $f$, evalúela en el punto $P$ y encuentre la razón de cambio de $f$ en $P$ en la dirección del vector $u$.
\begin{enumerate}[a)]
\item $f(x,y) = sen(2x+3y)$\tab$P=(-6,4), u=\left(\sqrt[]{3}i - j\right)$
\item $f(x,y) = \dfrac{y^2}{x}$\tab$P=(1,2), u=\left(2i + \sqrt[]{5}j\right)$
\end{enumerate}
\begin{solucion}
Recordemos que la derivada direccional de $f$ en la dirección $u$ en el punto $P(x_0, y_0)$, corresponde a
$$D_u(x_0, y_0) = \nabla f(x_0, y_0) \cdot \hat{u}$$
\begin{enumerate}[a)]
\item $f(x,y) = sen(2x+3y)$\tab$P=(-6,4), u=\left(\sqrt[]{3}i - j\right)$\\
\\
En primer lugar, calculemos el gradiente, esto es
$$\nabla f = \begin{pmatrix}
2\cos(2x + 3y) \\
3\cos(3x + 3y)
\end{pmatrix} \ra 
\nabla f(-6,4) = \begin{pmatrix}
2 \\
3
\end{pmatrix}
$$
Luego, normalicemos $u$, esto es
$$\hat{u} = \dfrac{u}{|u|} = \dfrac{\begin{pmatrix}
\sqrt[]{3} \\
-1
\end{pmatrix}}{2} = \begin{pmatrix}
\sqrt[]{3}/2 \\
-1/2
\end{pmatrix}$$
Finalmente,
$$D_u = 
\begin{pmatrix}
2 \\
3
\end{pmatrix}
\cdot 
\begin{pmatrix}
\sqrt[]{3}/2 \\
-1/2
\end{pmatrix} = \sqrt[]{3} - \dfrac{3}{2}$$
\item $f(x,y) = \dfrac{y^2}{x}$\tab$P=(1,2), u=\left(2i + \sqrt[]{5}j\right)$\\
\\
En primer lugar, calculemos el gradiente, esto es
$$\nabla f = \begin{pmatrix}
-y^2/x^2 \\
2y/x
\end{pmatrix} \ra 
\nabla f(1,2) = \begin{pmatrix}
-4 \\
4
\end{pmatrix}
$$
Luego, normalicemos $u$, esto es
$$\hat{u} = \dfrac{u}{|u|} = \dfrac{\begin{pmatrix}
2 \\
\sqrt[]{5}
\end{pmatrix}}{3} = \begin{pmatrix}
2/3 \\
\sqrt[]{5}/3
\end{pmatrix}$$
Finalmente,
$$D_u = 
\begin{pmatrix}
-4 \\
4
\end{pmatrix}
\cdot 
\begin{pmatrix}
2/3 \\
\sqrt[]{5}/3
\end{pmatrix} = \dfrac{-8 + 4\ \sqrt[]{5}}{3}$$
\end{enumerate}
\end{solucion}
\item Suponga que $f(x,y)$ es una función con derivadas parciales continuas en el punto $(1,1)$. Asumir que la derivada direccional en $(1,1)$ en la dirección $\langle3,4\rangle$ 			es 1 y en la dirección $\langle5,12\rangle$ es $-1$.
\begin{enumerate}[a)]
\item Encontrar la ecuación cartesiana del plano tangente en $(1,1,f(1,1))$.
\item Encontrar la derivada direccional de $f(x,y)$ en $(1,1)$ en dirección al origen.	
\end{enumerate}
\begin{solucion}
Recordemos que 
$$D_u(x,y) = \nabla f(x,y) \cdot \hat{u}$$
Luego, como sabemos que $D_{(3,4)}(1,1) = 1$ donde $u = (3,4) \ra \hat{u} = (\frac{3}{5}, \frac{4}{5})$, tenemos que
$$D_{(3,4)}(1,1) = \begin{pmatrix}f_x(1,1) \\ f_y(1,1)\end{pmatrix} \cdot \begin{pmatrix}\frac{3}{5} \\ \frac{4}{5}\end{pmatrix} = 1$$
$$\dfrac{3}{5}f_x(1,1) + \dfrac{4}{5}f_y(1,1) = 1$$
$$3f_x(1,1) + 4f_y(1,1) = 5$$
Análogamente, como $D_{(5,12)}(1,1) = -1$, donde $u=(5,12) \ra \hat{u} = (\frac{5}{13}, \frac{12}{13})$, tenemos que
$$D_{(5,12)}(1,1) = \begin{pmatrix}f_x(1,1) \\ f_y(1,1)\end{pmatrix} \cdot \begin{pmatrix}\frac{5}{13} \\ \frac{12}{13}\end{pmatrix} = 1$$
$$\dfrac{5}{13}f_x(1,1) + \dfrac{12}{13}f_y(1,1) = 1$$
$$5f_x(1,1) + 12f_y(1,1) = 13$$
Resolviendo este sistema de ecuaciones, obtenemos que
$$f_x(1,1) = 7 \qquad f_y(1,1) = -4$$
\begin{enumerate}[a)]
\item Para obtener el plano tangente, lo podemos hacer con la formula, esto es
$$z - f(1,1) = f_x(1,1)(x-1) + f_y(1,1)(y-1)$$
$$z - f(1,1) = 7(x-1) - 4(y-1)$$
$$7x - 4y - z = 3 - f(1,1)$$
\item $D_{(1,1)}(1,1) = \begin{pmatrix}f_x(1,1) \\ f_y(1,1)\end{pmatrix} \cdot \begin{pmatrix}-\frac{1}{\sqrt[]{2}} \\ -\frac{1}{\sqrt[]{2}}\end{pmatrix}
= \begin{pmatrix}7 \\ -4\end{pmatrix} \cdot \begin{pmatrix}-\frac{1}{\sqrt[]{2}} \\ -\frac{1}{\sqrt[]{2}}\end{pmatrix}
= -\frac{3}{\sqrt[]{2}}$
\end{enumerate}
\end{solucion}
\item La temperatura en el punto $(x,y)$ de una lamina metálica viene dada por la función $T(x,y) = \dfrac{x}{x^2+y^2}$. Hallar la razón de crecimiento máximo de la temperatura en el punto $(3,4$) y la dirección en que ella ocurre.
\begin{solucion}
En primer lugar, debemos calcular la gradiente, esto es
$$\nabla T = \left( 
\dfrac{(x^2+y^2) - 2x\cdot x}{(x^2+y^2)^2}, 
\dfrac{-x}{(x^2+y^2)^2}2y
\right)$$
$$\nabla T = \left( 
\dfrac{y^2 - x^2}{(x^2+y^2)^2}, 
\dfrac{-2xy}{(x^2+y^2)^2}
\right)$$
Luego, la dirección de máximo crecimiento viene dada por
$$\nabla T(3,4) = \left( 
\dfrac{7}{625}, 
-\dfrac{24}{625}
\right)$$
Notese que la dirección podría ser cualquier múltiplo de este vector, por lo que $(7,-24)$ también sería válido.\\
\\
Ahora, la razón de máximo crecimiento, viene dada por
$$||\nabla T(3,4)|| 
= \sqrt[]{\dfrac{7^2 + 24^2}{625^2}}
= \dfrac{\sqrt[]{7^2 + 24^2}}{625}
= \dfrac{\sqrt[]{625}}{625}
= \dfrac{1}{25}$$
Ojo que en este caso debemos usar la gradiente exacta, no un múltiplo de esta, ya que eso alteraría el resultado.\\

Si nos pidieran la razón de decrecimiento máximo, habría que cambiarle el signo al resultado.
\end{solucion}
\item Encuentre y clasifíque los puntos críticos de $f(x,y)=x^3+3xy^2-15x-12y$.
\begin{solucion}
Para encontrar y clasificar los puntos críticos de una función, se debe hacer lo siguiente:\\

Para clasificarlos, calculamos el gradiente de la función y lo igualamos a 0. Todos los puntos que cumplan con esta condición serán los puntos críticos.\\

Para clasificarlos, debemos evaluarlos en el determinante de la matriz Hessiana, que en el caso de dos variables es
$$H(x,y) = f_{xx}f_{yy} - f_{xy}^2$$
Aqui se cumplirá lo siguiente:
\begin{itemize}
\item $H > 0$
\begin{itemize}
    \item $f_{xx} > 0 \ra$ Mínimo local
    \item $f_{xx} < 0 \ra$ Máximo local
\end{itemize}
\item $H < 0 \ra$ Punto silla
\end{itemize}
Veamos ahora el ejercicio.\\

Tenemos que
$$f(x,y)=x^3+3xy^2-15x-12y$$
Luego,
$$\nabla f = \begin{pmatrix} 3x^2 + 3y^2 - 15 \\ 6xy - 12 \end{pmatrix} = \vec{0}$$
Luego, tenemos que resolver
$$3x^2 + 3y^2 - 15 = 0$$
$$6xy - 12 = 0$$
Simplificando,
$$x^2 + y^2 = 5$$
$$xy = 2$$
Sumando dos veces la segunda a la primera ecuación, obtenemos
$$x^2 + 2xy + y^2 = 9$$
Luego,
$$(x+y)^2 = 9 \ra x + y = \pm3$$
Ahora, debemos ver cada uno de estos casos y despejar reemplazando en la otra ecuación.
En primer lugar, veamos el caso donde $x+y = 3$, es decir, $y = 3 - x$.\\
\\
Reemplazando,
$$x(3-x) = 2 \ra x^2-3x+2=0 \ra (x-1)(x-2) = 0$$
De aqui tenemos que $x_1 = 1\ra y_1 = 2$ y $x_2 = 2 \ra y_2 = 1$, por lo que $P_1(1,2)$ y $P_2(2,1)$.\\
\\
Veamos ahora que pasa con $x+y = -3$, es decir $y = -3-x$\\
\\
Reemplazando,
$$x(-3-x) = 2 \ra x^2+3x+2=0 \ra (x+1)(x+2) = 0$$
De aqui tenemos que $x_3 = -1 \ra y_3 = -2$ y $x_4 = -2 \ra y_4 = -1$, por lo que $P_3(-1,-2)$ y $P_4(-2,-1)$.\\
\\
Entonces, los puntos críticos son
$$P_1(1,2), \quad P_2(2,1), \quad P_3(-1,-2), \quad P_4(-2,-1)$$
Calculemos ahora el determinante del Hessiano, es decir,
$$H(x,y) = f_{xx}f_{yy} - f_{xy}^2$$
Notemos que
$$f_{xx} = 6x \quad f_{yy} = 6x \quad f_{xy} = 6y$$
Luego,
$$H(x,y) = 36(x^2-y^2)$$
Finalmente,
$$H(P_1) = H(1,2) = 36(1^2-2^2) < 0 \ra \text{Punto silla}$$
$$H(P_2) = H(2,1) = 36(2^2-1^2) > 0$$
$$f_{xx}(2,1) = 12 > 0 \ra \text{Mínimo local}$$
$$H(P_3) = H(-1,-2) = 36((-1)^2-(-2)^2) < 0 \ra \text{Punto silla}$$
$$H(P_4) = H(-2,-1) = 36((-2)^2-(-1)^2) > 0$$
$$f_{xx}(2,1) = -12 < 0 \ra \text{Máximo local}$$
\end{solucion}
\item Sea la función
	$$f(x,y)=ax^2y+bxy^2+\dfrac{a^2y^2}{2}+2y$$
	determine los valores de $a$ y $b$ para que la función posea un punto silla en $(1,1)$.
\begin{solucion}
Para que exista un punto silla en $(1,1)$, deben cumplirse dos cosas:
$$\nabla f(1,1) = 0 \wedge H(1,1) < 0$$
Tenemos que
$$\nabla f = \begin{pmatrix} 2axy + by^2 \\ ax^2 + 2bxy + a^2y\end{pmatrix}$$
Evaluando en $(1,1)$,
$$\nabla f(1,1) = \begin{pmatrix} 2a + b \\ a + 2b + a^2\end{pmatrix} = 0$$
Luego,
$$2a + b = 0$$
$$a + 2b + a^2= 0$$
Reemplazando la primera ecuación ($b = -2a$) en la segunda obtenemos,
$$a^2 - 3a +2 = 0 \ra (a-2)(a-1) = 0$$
De aquí obtememos
$$a = 1 \ra b = -2, \qquad a = 2 \ra b = -4$$
Veamos ahora cuál de estos valores de $a$ y $b$ cumplen la restricción del Hessiano.
$$f_{xx} = 2ay, \quad f_{yy} = 2bx + a^2, \quad f_{xy} = 2ax + 2by$$
Luego,
$$H(x,y) = 2ay(2bx+a^2) - (2ax +2by)^2$$
Evaluamos,
$$H(1,1) = 2a(2b+a^2) - (2a+2b)^2$$
Probamos con $a=1$ y $b=-2$,
$$H = 2(-4+1)-(2-4)^2 = -6 - 4 = -10 < 0$$
Por lo que si es un punto silla.\\
\\
Probemos ahora ocn $a = 2$ y $b = -4$
$$H = 4(-8-4)-(4-8)^2 = -16 - 16 = -32 < 0$$
Por lo que si es un punto silla.\\

Finalmente, los valores que puede tomar $a$ y $b$ son $(1, -2)$ y $(2, -4)$
\end{solucion}
\end{preguntas}
\end{document}