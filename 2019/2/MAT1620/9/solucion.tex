\documentclass[12pt]{article}

\usepackage{fullpage}
\usepackage{graphicx}
\usepackage{amssymb}
\usepackage{amsmath}
\usepackage[none]{hyphenat}
\usepackage{parskip}
\usepackage[spanish]{babel}
\usepackage[utf8]{inputenc}
\usepackage{hyperref}
\usepackage{fancyhdr}
\usepackage{tasks}
\usepackage{mdframed}
\usepackage{xcolor}
\usepackage{pgfplots}
\usepackage[makeroom]{cancel}
\usepackage{multicol}
\usepackage[shortlabels]{enumitem}
\usepackage{stackrel}
\usepackage{tkz-tab}
\usepackage{xpatch}
\usepackage{tkz-euclide}
\usetkzobj{all}
\usepackage{tabto}
\xpatchcmd{\tkzTabLine}{$0$}{$\bullet$}{}{}

\setlength{\headheight}{10pt}
\setlength{\headsep}{10pt}
\pagestyle{fancy}
\rhead{\ayudantia \ - \alumno}
\tikzset{t style/.style={style=solid}}

\newcommand*{\mybox}[2]{\colorbox{#1!30}{\parbox{.98\linewidth}{#2}}}

\newenvironment{solucion}
{\begin{mdframed}[backgroundcolor=black!10]
		{\bf Solución:}\\
	}
	{
	\end{mdframed}
}

\newenvironment{alternativas}[1]
{\begin{multicols}{#1}
		\begin{enumerate}[a)]
		}
		{
		\end{enumerate}
	\end{multicols}
}

\newenvironment{preguntas}
{\begin{enumerate}\itemsep12pt
	}
	{
	\end{enumerate}
}

\newcommand{\ayudantia}{{\sc Ayudantía 9}}
\newcommand{\tituloayu}{Repaso I2}
\newcommand{\fecha}{8 de noviembre de 2019}
\newcommand{\sigla}{MAT1620}
\newcommand{\nombre}{Cálculo II}
\newcommand{\profesor}{Wolfgang Rivera}
\newcommand{\ano}{2019}
\newcommand{\semestre}{2}
\newcommand{\mail}{mat1620@ifcastaneda.cl}
\newcommand{\alumno}{Ignacio Castañeda - \mail}

\newcommand{\ev}{\Big|}
\newcommand{\ra}{\rightarrow}
\newcommand{\lra}{\leftrightarrow}
\newcommand{\N}{\mathbb{N}}
\newcommand{\R}{\mathbb{R}}
\newcommand{\Exp}[1]{\mathcal{E}_{#1}}
\newcommand{\List}[1]{\mathcal{L}_{#1}}
\newcommand{\EN}{\Exp{\N}}
\newcommand{\LN}{\List{\N}}
\newcommand{\comment}[1]{}
\newcommand{\lb}{\\~\\}
\newcommand{\eop}{_{\square}}
\newcommand{\hsig}{\hat{\sigma}}
\newcommand{\widesim}[2][1.5]{
	\mathrel{\overset{#2}{\scalebox{#1}[1]{$\sim$}}}
}
\newcommand{\wsim}{\widesim{}}
\newcommand{\lh}{\stackrel{L'H}{=}}

\begin{document}
\thispagestyle{empty}

\begin{minipage}{2cm}
	\includegraphics[width=2cm]{../../../../img/logo.pdf}
	\vspace{0.5cm}
\end{minipage}
\begin{minipage}{\linewidth}
	\begin{tabular}{lrl}
		{\scriptsize\sc Pontificia Universidad Catolica de Chile} & \hspace*{0.7in}Curso: &
		\sigla  - \nombre\\
		{\sc Facultad de Matemáticas}&
		Profesor: & \profesor \\
		{\sc Semestre \ano-\semestre} & Ayudante: & {Ignacio Castañeda}\\
		& {Mail:} & \texttt{\mail}
	\end{tabular}
\end{minipage}

\vspace{-10mm}
\begin{center}
	{\LARGE\bf \ayudantia}\\
	\vspace{0.1cm}
	{\tituloayu}\\
	\vspace{0.1cm}
	\fecha\\
	\vspace{0.4cm}
\end{center}

\begin{preguntas}
\item Calcular la siguiente suma
	$$1 - ln(2) + \dfrac{(ln2)^2}{2!}-\dfrac{(ln2)^3}{3!}+\cdots$$
\begin{solucion}
En primer lugar, escribamos la suma como una serie, esto es
$$\sum\limits_{0}^{\infty} \dfrac{(\ln(2))^n}{n!}$$
Recordemos que
$$e^x = \sum\limits_{0}^{\infty} \dfrac{x^n}{n!}$$
Luego,
$$\sum\limits_{0}^{\infty} \dfrac{(\ln(2))^n}{n!} = e^{\ln(2)} = 2$$
\end{solucion}
\item Evalúe la integral indefinida como una serie infinita
	$$\int x\ \arctan(3x)dx$$
\begin{solucion}
Lo que nos genera problemas en la integral es el $\arctan(3x)$, por lo que convertiremos eso en una serie.\\
\\
Sea
$$f(x) = \arctan(3x)$$
Derivando,
$$f'(x) = \dfrac{3}{1+9x^2} = \dfrac{3}{1-(-9x^2)} = \sum\limits_0^{\infty}(-9x^2)^n = \sum\limits_0^{\infty}(-1)^n 9^n x^{2n}$$
Integrando,
$$f(x) = \sum\limits_0^{\infty}(-1)^n 9^n \int x^{2n} = \sum\limits_0^{\infty}(-1)^n \dfrac{9^n}{2n+1}x^{2n+1} + c$$
Evaluando en 0 en la seria y la función original,
$$f(0) = c = 0$$
Luego,
$$\int x\ \arctan(3x)dx
= \int x\ \sum\limits_0^{\infty}(-1)^n \dfrac{9^n}{2n+1}x^{2n+1}dx
= \int \sum\limits_0^{\infty}(-1)^n \dfrac{9^n}{2n+1}x^{2n+2}dx
$$
$$
= \sum\limits_0^{\infty}(-1)^n \dfrac{9^n}{(2n+1)(2n+3)}x^{2n+3}dx
$$
\end{solucion}
\item Demuestre que
$$\int_0^1 \dfrac{1-cos(x)}{x^2} = \sum\limits_{n=1} \dfrac{(-1)^n}{(2n)!(2n-1)} $$
\begin{solucion}

\end{solucion}
\item Sea la función
	$$f(x,y)=
	\begin{cases}
	\dfrac{xy}{\sqrt[]{x^2+y^2}} & (x,y) \neq (0,0)\\
	0 & (x,y)=(0,0)
	\end{cases}
	$$
\begin{enumerate}[a)]
\item Determinar si $f$ es continua en $(0,0)$.
\item Calcular sus derivadas parciales en $(0,0)$, en caso de que existan.
\item Determinar si $f$ es diferenciable en $(0,0)$.
\end{enumerate}
\begin{solucion}

\begin{enumerate}[a)]
\item Para que la función sea continua en $(0,0)$, debe cumplirse que
$$\lim\limits_{(x,y) \ra (0,0)} \dfrac{xy}{\sqrt[]{x^2+y^2}} = f(0,0) = 0$$
Analizando el límite, vemos que tiene grado 2 en el numerador y grado 1 en el denominador, por lo que "al ojo" sería razonable creer que existe. Por esto, intentaremos hacerlo con coordenadas polares de inmediato, esto es,
$$x = r\cos(\theta) \quad y = r\cos(\theta) \ra x^2 + y^2 = r^2$$
Luego, el límite queda
$$\lim\limits_{r \ra 0} \dfrac{r^2\cos(\theta)\sin(\theta)}{\sqrt[]{r^2}}
= \lim\limits_{r \ra 0}  r\cos(\theta)\sin(\theta) = 0$$
Por lo tanto, la función es continua en $(0,0)$.
\item Para obtener las derivadas parciales de $f$, debemos hacerlo por definición, ya que no sabemos si la función es diferenciable en $(0,0)$.
$$f_x(x,y) = \lim\limits_{h \ra 0} \dfrac{f(x+h,y)-f(x,y)}{h}
= \lim\limits_{h \ra 0} \dfrac{\dfrac{(x+h)y}{\sqrt[]{(x+h)^2+y^2}} - \dfrac{xy}{\sqrt[]{x^2+y^2}}}{h}$$
$$= \lim\limits_{h \ra 0} \dfrac{\dfrac{xy}{\sqrt[]{(x+h)^2+y^2}} + \dfrac{hy}{\sqrt[]{(x+h)^2+y^2}} - \dfrac{xy}{\sqrt[]{x^2+y^2}}}{h} $$
$$=\lim\limits_{h \ra 0} \dfrac{\dfrac{hy}{\sqrt[]{(x+h)^2+y^2}}}{h}
=\lim\limits_{h \ra 0} \dfrac{y}{\sqrt[]{(x+h)^2+y^2}}
=\dfrac{y}{\sqrt[]{x^2+y^2}}
$$
$$f_y(x,y) = \lim\limits_{h \ra 0} \dfrac{f(x,y+h)-f(x,y)}{h}
= \lim\limits_{h \ra 0} \dfrac{\dfrac{x(y+h)}{\sqrt[]{x^2+(y+h)^2}} - \dfrac{xy}{\sqrt[]{x^2+y^2}}}{h}$$
$$= \lim\limits_{h \ra 0} \dfrac{\dfrac{xy}{\sqrt[]{x^2+(y+h)^2}} + \dfrac{hx}{\sqrt[]{x^2+(y+h)^2}} - \dfrac{xy}{\sqrt[]{x^2+y^2}}}{h} $$
$$=\lim\limits_{h \ra 0} \dfrac{\dfrac{hx}{\sqrt[]{x^2+(y+h)^2}}}{h}
=\lim\limits_{h \ra 0} \dfrac{x}{\sqrt[]{x^2+(y+h)^2}}
=\dfrac{x}{\sqrt[]{x^2+y^2}}
$$
\item Para que la función sea diferenciable en $(0,0)$, se debe cumplir que
$$\lim\limits_{(h,k)\ra(0,0)} \dfrac{f(h,k)-f(0,0)-hf_x(0,0) - kf_y(0,0)}{\sqrt[]{h^2+k^2}} = 0$$
De la parte b), tenemos que
$$f_x(0,0) = 0 \quad f_y(0,0) = 0$$
Luego,
$$\lim\limits_{(h,k)\ra(0,0)} \dfrac{\dfrac{hk}{\sqrt[]{h^2+k^2}-0-0-0}{\sqrt[]{h^2+k^2}}}
= \lim\limits_{(h,k)\ra(0,0)} \dfrac{hk}{h^2+k^2} $$
$$h = r\cos(\theta) \quad k = r\sin(\theta)$$
$$= \lim\limits_{r \ra 0} \dfrac{r^2\cos(\theta)\sin(\theta)}{r^2}
= \cos(\theta)\sin(\theta) \neq 0 $$
Por lo tanto, la función no es diferenciable.
\end{enumerate}
\end{solucion}
\item Encontrar el plano tangente a la superficie $z = y\cos\left(\dfrac{x}{2}\right)$ en el punto $(\pi, 1, 0)$.
\begin{solucion}
Recordemos que el plano tangente viene dado por
$$z - z_0 = f_x(x_0, y_0)(x-x_0) + f_y(x_0,y_0)(y-y_0)$$
Notemos que
$$f_x(x,y) = -\dfrac{1}{2}y\sin\left(\dfrac{x}{2}\right) \qquad f(\pi, 1) = -\dfrac{1}{2}$$
$$f_y(x,y) = \cos\left(\dfrac{x}{2}\right) \qquad f(\pi, 1) = 0$$
Luego,
$$z = -\dfrac{1}{2}(x-\pi) \ra 2z + x = \pi$$
Donde $\Pi: 2z + x = \pi$ corresponde al plano tangente.
\end{solucion}
\item Hallar las ecuaciones de los planos tangentes a la superficie $x^2 + 2y^2 + 3z^2 = 21$ que sean paralelos al plano $x + 4y + 6z = 0$.
\begin{solucion}
Cuando no podemos despejar $z$, podemos usar la ecuación general del plano tangente,
En primer lugar, despejemos $z$, esto es
$$f_x(x_0,y_0,z_0)(x-x_0) +f_y(x_0,y_0,z_0)(y-y_0) + f_z(x_0,y_0,z_0)(z-z_0)$$
Sea 
$$f(x,y,z) = x^2 + 2y^2 + 3z^2 -21$$
Ahora, calculamos las derivadas, esto es
$$f_x = 2x \qquad f_y = 4y \qquad f_z = 6z$$
Para saber en que puntos estan los planos tangentes, hay que notar que las derivadas parciales serán las que definan los coeficientes de nuestro plano tangente y por lo tanto, podemos concluir que el vector normal del plano tangente será
$$\vec{n} = \begin{pmatrix}2x \\ 4y\\ 6z\end{pmatrix}$$
Para que este plano sea tangente al plano dado, su vector normal debe ser múltiplo del vector normal del plano dado, esto es,
$$\begin{pmatrix}2x \\ 4y\\ 6z\end{pmatrix} = k \begin{pmatrix}1 \\ 4\\ 6\end{pmatrix}$$
Luego,
$$x = \dfrac{k}{2}, \qquad y = \dfrac{k}{4}, \qquad z = \dfrac{k}{2}$$
Como es un plano paralelo a $x + 4y + 6z = 0$, será de la forma
$$x + 4y + 6z = D$$
Reemplazamos con los valores de $x,y,z$
$$\left(\dfrac{k}{2}\right)^2 + 4\left(\dfrac{k}{4}\right)^2 + 6\left(\dfrac{k}{2}\right)^2$$
\end{solucion}
\item El cambio de variables 
	$\begin{cases}
	x=u+v\\
	y=uv^2
	\end{cases}$
	transforma a $z=f(x,y)$ en $z=g(u,v)$. Calcular el valor de $\dfrac{\delta^2z}{\delta u \delta v}$ en el punto $(u,v) = (1,1)$, sabiendo que en aquel punto se cumple$$\dfrac{\delta f}			{\delta y} = \dfrac{\delta^2 f}{\delta x^2}=\dfrac{\delta^2 f}{\delta y^2}=\dfrac{\delta^2 f}{\delta x \delta y}=\dfrac{\delta^2 f}{\delta y \delta x}=1$$
\begin{solucion}

\end{solucion}
\end{preguntas}
\end{document}