\documentclass[12pt]{article}

\usepackage{fullpage}
\usepackage{graphicx}
\usepackage{amssymb}
\usepackage{amsmath}
\usepackage[none]{hyphenat}
\usepackage{parskip}
\usepackage[spanish]{babel}
\usepackage[utf8]{inputenc}
\usepackage{hyperref}
\usepackage{fancyhdr}
\usepackage{tasks}
\usepackage{mdframed}
\usepackage{xcolor}
\usepackage{pgfplots}
\usepackage[makeroom]{cancel}
\usepackage{multicol}
\usepackage[shortlabels]{enumitem}
\usepackage{stackrel}
\usepackage{tkz-tab}
\usepackage{xpatch}
\usepackage{tkz-euclide}
\usetkzobj{all}
\xpatchcmd{\tkzTabLine}{$0$}{$\bullet$}{}{}

\setlength{\headheight}{10pt}
\setlength{\headsep}{10pt}
\pagestyle{fancy}
\rhead{\ayudantia \ - \alumno}
\tikzset{t style/.style={style=solid}}

\newcommand*{\mybox}[2]{\colorbox{#1!30}{\parbox{.98\linewidth}{#2}}}

\newenvironment{solucion}
{\begin{mdframed}[backgroundcolor=black!10]
		{\bf Solución:}\\
	}
	{
	\end{mdframed}
}

\newenvironment{alternativas}[1]
{\begin{multicols}{#1}
		\begin{enumerate}[a)]
		}
		{
		\end{enumerate}
	\end{multicols}
}

\newenvironment{preguntas}
{\begin{enumerate}\itemsep12pt
	}
	{
	\end{enumerate}
}

\newcommand{\ayudantia}{{\sc Ayudantía 2}}
\newcommand{\tituloayu}{Sucesiones}
\newcommand{\fecha}{20 de agosto de 2019}
\newcommand{\sigla}{MAT1620}
\newcommand{\nombre}{Cálculo II}
\newcommand{\profesor}{Wolfgang Rivera}
\newcommand{\ano}{2019}
\newcommand{\semestre}{2}
\newcommand{\mail}{mat1620@ifcastaneda.cl}
\newcommand{\alumno}{Ignacio Castañeda - \mail}

\newcommand{\ev}{\Big|}
\newcommand{\ra}{\rightarrow}
\newcommand{\lra}{\leftrightarrow}
\newcommand{\N}{\mathbb{N}}
\newcommand{\R}{\mathbb{R}}
\newcommand{\Exp}[1]{\mathcal{E}_{#1}}
\newcommand{\List}[1]{\mathcal{L}_{#1}}
\newcommand{\EN}{\Exp{\N}}
\newcommand{\LN}{\List{\N}}
\newcommand{\comment}[1]{}
\newcommand{\lb}{\\~\\}
\newcommand{\eop}{_{\square}}
\newcommand{\hsig}{\hat{\sigma}}
\newcommand{\widesim}[2][1.5]{
	\mathrel{\overset{#2}{\scalebox{#1}[1]{$\sim$}}}
}
\newcommand{\wsim}{\widesim{}}
\newcommand{\lh}{\stackrel{L'H}{=}}

\begin{document}
\thispagestyle{empty}

\begin{minipage}{2cm}
	\includegraphics[width=2cm]{../../../../img/logo.pdf}
	\vspace{0.5cm}
\end{minipage}
\begin{minipage}{\linewidth}
	\begin{tabular}{lrl}
		{\scriptsize\sc Pontificia Universidad Catolica de Chile} & \hspace*{0.7in}Curso: &
		\sigla  - \nombre\\
		{\sc Facultad de Matemáticas}&
		Profesor: & \profesor \\
		{\sc Semestre \ano-\semestre} & Ayudante: & {Ignacio Castañeda}\\
		& {Mail:} & \texttt{\mail}
	\end{tabular}
\end{minipage}

\vspace{-10mm}
\begin{center}
	{\LARGE\bf \ayudantia}\\
	\vspace{0.1cm}
	{\tituloayu}\\
	\vspace{0.1cm}
	\fecha\\
	\vspace{0.4cm}
\end{center}

\begin{preguntas}
\item Calcule el límite de las siguientes sucesiónes
\begin{tasks}(2)
\task $a_n = \dfrac{3^n+7}{5^n-3}$
\task $a_n = \dfrac{n^3-n}{7n^3+6}$
\task $a_n = \dfrac{(-1)^n n}{n^3+4}$
\task $a_n=\ln(n+1)-\ln(n)$
\end{tasks}
\begin{solucion}

\begin{enumerate}[a)]
\item $a_n = \dfrac{3^n+7}{5^n-3}$\\
			\\
			Como antes, vemos el límite del término general
			$$\lim\limits_{n\ra \infty} a_n = \lim\limits_{n\ra \infty} \dfrac{3^n+7}{5^n-3} \stackrel{L'H}{=} \lim\limits_{n\ra \infty} \dfrac{ln(3)3^n}{ln(5)5^n} = \lim\limits_{n\ra \infty} \dfrac{ln(3)}{ln(5)} \left(\dfrac{3}{5}\right)^n = 0$$
\item $a_n = \dfrac{n^3-n}{7n^3+6}$\\
			\\
			$$\lim\limits_{n\ra \infty} a_n = \lim\limits_{n\ra \infty} \dfrac{n^3-n}{7n^3+6} \stackrel{L'H}{=} \lim\limits_{n\ra \infty} \dfrac{3n^2-1}{21n^2}
			\stackrel{L'H}{=} \lim\limits_{n\ra \infty} \dfrac{6n}{42n} = \dfrac{1}{7}$$
\item 
\item 
\end{enumerate}
\end{solucion}
\item Determine si las siguientes sucesiones convergen y en caso de hacerlo, calcular su límite.
\begin{tasks}(2)
\task $\left\{1, -\dfrac{2}{3}, \dfrac{4}{9}, -\dfrac{8}{27}, \dots \right\}$
\task $\left\{\sqrt[]{2}, \sqrt[]{2\ \sqrt[]{2}}, \sqrt[]{2\ \sqrt[]{2\ \sqrt[]{2}}}, \dots \right\}$
\end{tasks}
\begin{solucion}

\begin{enumerate}[a)]
\item $\left\{1, -\dfrac{2}{3}, \dfrac{4}{9}, -\dfrac{8}{27}, \dots \right\}$\\
			\\
			Observando la sucesión, resulta evidente que el numerador es $2^n$ y el\\ denominador $3^n$. Además, vemos que es alternante, por lo que el término general será
			$$a_n = (-1)^n \dfrac{2^{n}}{3^{n}}$$
			Luego, el límite de la sucesión es igual al límite del termino general, que sería
			$$\lim\limits_{n\ra \infty} a_n = \lim\limits_{n\ra \infty} (-1)^n \dfrac{2^{n}}{3^{n}} = 
			\lim\limits_{n\ra \infty} (-1)^n \left(\dfrac{2}{3}\right)^n = 0$$
\item $\left\{\sqrt[]{2}, \sqrt[]{2\ \sqrt[]{2}}, \sqrt[]{2\ \sqrt[]{2\ \sqrt[]{2}}}, \dots \right\}$
\end{enumerate}
\end{solucion}
\item Sea $a_n = \dfrac{n!}{n^n}$, calcular $\lim\limits_{n \ra \infty} \dfrac{a_{n+1}}{a_n}$
\begin{solucion}
$$\lim\limits_{n \ra \infty} \dfrac{a_{n+1}}{a_n} = \lim\limits_{n \ra \infty} \dfrac{\dfrac{(n+1)!}{(n+1)^{n+1}}}{\dfrac{n!}{n^n}} = \lim\limits_{n \ra \infty} \dfrac{(n+1)!}{n!} \cdot \dfrac{n^n}{(n+1)^{n+1}} =\lim\limits_{n \ra \infty} (n+1) \cdot \dfrac{n^n}{(n+1)^{n+1}}$$
		$$=\lim\limits_{n \ra \infty} \dfrac{n^n}{(n+1)^{n}} = \lim\limits_{n \ra \infty} \left(\dfrac{n}{n+1}\right)^n = \lim\limits_{n \ra \infty} \dfrac{1}{\left(\dfrac{n+1}{n}\right)^n} = \lim\limits_{n \ra \infty} \dfrac{1}{\left(1+\dfrac{1}{n}\right)^n}$$
		$$ = \dfrac{1}{e}$$
\end{solucion}
\item La sucesión $\{a_n\}$ se define con 
		$$a_1=1\quad y\quad a_{n+1} = 3-\dfrac{1}{a_n},\ \forall n\geq 1$$\\
Se sabe que $\{a_n\}$ es monótona creciente. Prueba que $\{a_n\}$ es convergente y calcule su límite.
\begin{solucion}
Para que una sucesión sea convergente, basta que esta sea monótona creciente y acotada, por lo que debemos demostrar que $\{a_n\}$ es acotada.\\

Demostremos por inducción que $\{a_n\}$ esta acotada superiormente por 3. \\
$$n = 1 \ra a_1 = 1 \leq 3$$
Hipótesis de inducción:
$$n = k \ra a_k \leq 3$$
Tésis de inducción:
$$n = k+1$$
Por demostrar que $a_{n+1} \leq 3$
$$\a_k \leq 3$$
$$\dfrac{1}{a_k} \geq \dfrac{1}{3}$$
$$-\dfrac{1}{a_k} \leq -\dfrac{1}{3}$$
$$3-\dfrac{1}{a_k} \leq 3-\dfrac{1}{3}$$
$$a_{k+1} \leq 3-\dfrac{1}{3}$$
$$a_{k+1} \leq 3$$
Luego, al ser monótona creciente y acotada, la sucesión es convergente.\\

Para calcular el límite de esta, digamos que
$$L = \lim\limits_{n \ra \infty} a_n$$
Luego,
$$L = 3 - \dfrac{1}{L}$$
$$L^2 = 3L - 1$$
$$L^2 - 3L + 1 = 0$$
$$L = \dfrac{1\pm\ \sqrt[]{5}}{2}$$
Pero $a_n > 0$, por lo que
$$L = \dfrac{1 + \sqrt[]{5}}{2}$$

\end{solucion}
\end{preguntas}
\end{document}