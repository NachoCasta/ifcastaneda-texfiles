\documentclass[12pt]{article}

\usepackage{fullpage}
\usepackage{graphicx}
\usepackage{amssymb}
\usepackage{amsmath}
\usepackage[none]{hyphenat}
\usepackage{parskip}
\usepackage[spanish]{babel}
\usepackage[utf8]{inputenc}
\usepackage{hyperref}
\usepackage{fancyhdr}
\usepackage{tasks}
\usepackage{mdframed}
\usepackage{xcolor}
\usepackage{pgfplots}
\usepackage[makeroom]{cancel}
\usepackage{multicol}
\usepackage[shortlabels]{enumitem}
\usepackage{stackrel}
\usepackage{tkz-tab}
\usepackage{xpatch}
\usepackage{tkz-euclide}
\usetkzobj{all}
\xpatchcmd{\tkzTabLine}{$0$}{$\bullet$}{}{}

\setlength{\headheight}{10pt}
\setlength{\headsep}{10pt}
\pagestyle{fancy}
\rhead{\ayudantia \ - \alumno}
\tikzset{t style/.style={style=solid}}

\newcommand*{\mybox}[2]{\colorbox{#1!30}{\parbox{.98\linewidth}{#2}}}

\newenvironment{solucion}
{\begin{mdframed}[backgroundcolor=black!10]
		{\bf Solución:}\\
	}
	{
	\end{mdframed}
}

\newenvironment{alternativas}[1]
{\begin{multicols}{#1}
		\begin{enumerate}[a)]
		}
		{
		\end{enumerate}
	\end{multicols}
}

\newenvironment{preguntas}
{\begin{enumerate}\itemsep12pt
	}
	{
	\end{enumerate}
}

\newcommand{\ayudantia}{{\sc Ayudantía 3.5}}
\newcommand{\tituloayu}{Compilado I1}
\newcommand{\fecha}{28 de agosto de 2019}
\newcommand{\sigla}{MAT1620}
\newcommand{\nombre}{Cálculo II}
\newcommand{\profesor}{Wolfgang Rivera}
\newcommand{\ano}{2019}
\newcommand{\semestre}{2}
\newcommand{\mail}{mat1620@ifcastaneda.cl}
\newcommand{\alumno}{Ignacio Castañeda - \mail}

\newcommand{\ev}{\Big|}
\newcommand{\ra}{\rightarrow}
\newcommand{\lra}{\leftrightarrow}
\newcommand{\N}{\mathbb{N}}
\newcommand{\R}{\mathbb{R}}
\newcommand{\Exp}[1]{\mathcal{E}_{#1}}
\newcommand{\List}[1]{\mathcal{L}_{#1}}
\newcommand{\EN}{\Exp{\N}}
\newcommand{\LN}{\List{\N}}
\newcommand{\comment}[1]{}
\newcommand{\lb}{\\~\\}
\newcommand{\eop}{_{\square}}
\newcommand{\hsig}{\hat{\sigma}}
\newcommand{\widesim}[2][1.5]{
	\mathrel{\overset{#2}{\scalebox{#1}[1]{$\sim$}}}
}
\newcommand{\wsim}{\widesim{}}
\newcommand{\lh}{\stackrel{L'H}{=}}

\begin{document}
\thispagestyle{empty}

\begin{minipage}{2cm}
	\includegraphics[width=2cm]{../../../../img/logo.pdf}
	\vspace{0.5cm}
\end{minipage}
\begin{minipage}{\linewidth}
	\begin{tabular}{lrl}
		{\scriptsize\sc Pontificia Universidad Catolica de Chile} & \hspace*{0.7in}Curso: &
		\sigla  - \nombre\\
		{\sc Facultad de Matemáticas}&
		Profesor: & \profesor \\
		{\sc Semestre \ano-\semestre} & Ayudante: & {Ignacio Castañeda}\\
		& {Mail:} & \texttt{\mail}
	\end{tabular}
\end{minipage}

\vspace{-10mm}
\begin{center}
	{\LARGE\bf \ayudantia}\\
	\vspace{0.1cm}
	{\tituloayu}\\
	\vspace{0.1cm}
	\fecha\\
	\vspace{0.4cm}
\end{center}

\begin{preguntas}
\item Determinar si las siguientes integrales impropias convergen o divergen y calcularlas en caso de que converjan.
\begin{tasks}(3)
\task $\displaystyle\int_{0}^{3}\dfrac{1}{x\ \sqrt[]{x}}dx$
\task $\displaystyle\int_{e}^{\infty}\dfrac{1}{xlnx}dx$
\task $\displaystyle\int_{0}^{\infty}\dfrac{1}{\sqrt[]{x}(1+x)}dx$
\end{tasks}
\item La Trompeta de Torricelli se consigue al rotar la curva $y=\dfrac{1}{x},\ x \geq 1$ en torno al eje $x$.
\begin{enumerate}[a)]
\item Calcular la superficie de la trompeta entre $1$ y $L$ con la siguiente formula:
		$$ S = 2\pi \displaystyle\int_{1}^{L} f(x)\ \sqrt[]{1+[f'(x)]^2}dx, \quad con\ L \ra \infty $$
\item Calcular el volumen de la trompeta entre $1$ y $L$ con la siguiente formula:
		$$ V = \pi \displaystyle\int_{1}^{L} f(x)^2dx, \quad con\ L \ra \infty$$
\end{enumerate}
\item Determinar el valor de la constante $C$ para la cual la integral
	$$\displaystyle\int_{0}^{\infty} \left( \dfrac{x}{x^2+1} - \dfrac{C}{3x+1}\right)dx$$
	converge. Evalúe la integral para este valor de $C$.
\item Estudiar la convergencia de la siguiente integral
$$\displaystyle\int_{0}^{\infty} e^{-x} \dfrac{sen^2(x)}{x^{5/2}}dx$$
\item Calcule el límite de las siguientes sucesiónes
\begin{tasks}(2)
\task $a_n = \dfrac{3^n+7}{5^n-3}$
\task $a_n = \dfrac{n^3-n}{7n^3+6}$
\task $a_n = \dfrac{(-1)^n n}{n^3+4}$
\task $a_n=\ln(n+1)-\ln(n)$
\end{tasks}
\item Determine si las siguientes sucesiones convergen y en caso de hacerlo, calcular su límite.
\begin{tasks}(2)
\task $\left\{1, -\dfrac{2}{3}, \dfrac{4}{9}, -\dfrac{8}{27}, \dots \right\}$
\task $\left\{\sqrt[]{2}, \sqrt[]{2\ \sqrt[]{2}}, \sqrt[]{2\ \sqrt[]{2\ \sqrt[]{2}}}, \dots \right\}$
\end{tasks}
\item Sea $a_n = \dfrac{n!}{n^n}$, calcular $\lim\limits_{n \ra \infty} \dfrac{a_{n+1}}{a_n}$
\item La sucesión $\{a_n\}$ se define con 
		$$a_1=1\quad y\quad a_{n+1} = 3-\dfrac{1}{a_n},\ \forall n\geq 1$$\\
Se sabe que $\{a_n\}$ es monótona creciente. Prueba que $\{a_n\}$ es convergente y calcule su límite.
\item Determine si las siguientes series convergen o divergen.
\begin{tasks}(3)
\task $\sum\limits_{n=1}^{\infty}\dfrac{5n^3}{7n+n^3-1}$
\task $\sum\limits_{n=1}^{\infty}\left(sen\left(\dfrac{n\pi}{2}\right)\right)^2$
\task $\sum\limits_{n=1}^{\infty}2^{2n}3^{1-3n}$
\task $\sum\limits_{n=1}^{\infty}\dfrac{1 + n + \sin(n)}{3n^4 + \ln(n)}$
\task $\sum\limits_{n=20}^{\infty}\dfrac{1}{nln(n)ln(ln(n))}$
\task $\sum\limits_{n=1}^{\infty}ln\left(1+\dfrac{1}{n}\right)$
\end{tasks}
\item Demuestre que si $\sum\limits_{n=1}^{\infty}a_n$ es convergente, entonces el limite de la sucesión $b_n=ln(1+a_n)$ en el infinito es $0$.
\item Sea $a_n$ una sucesión tal que $a_n \neq 0,\ \forall n \in \N$.\\
	Demuestre que si $\sum\limits_{n=1}^{\infty}a_n$ converge, entonces $\sum\limits_{n=1}^{\infty}\dfrac{1}{a_n}$ diverge.
\item Considere una función $f$ continua en $\R$, decreciente y no negativa tal que
	$$\lim_{x\ra\infty}\dfrac{f(x)}{e^{-x}}=5$$
	Analice la convergencia de la serie $\sum\limits_{n=1}^{\infty}f(n)$
\item Considere la representación decimal de un número,
$$0,d_1d_2d_3... = \dfrac{d_1}{10} + \dfrac{d_2}{10^2} + \dfrac{d_3}{10^3}$$
donde $d_i$ es alguno de los dígitos entre 0 y 9. Pruebe que la serie anterior es convergente.
\end{preguntas}
\end{document}