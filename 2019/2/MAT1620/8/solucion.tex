\documentclass[12pt]{article}

\usepackage{fullpage}
\usepackage{graphicx}
\usepackage{amssymb}
\usepackage{amsmath}
\usepackage[none]{hyphenat}
\usepackage{parskip}
\usepackage[spanish]{babel}
\usepackage[utf8]{inputenc}
\usepackage{hyperref}
\usepackage{fancyhdr}
\usepackage{tasks}
\usepackage{mdframed}
\usepackage{xcolor}
\usepackage{pgfplots}
\usepackage[makeroom]{cancel}
\usepackage{multicol}
\usepackage[shortlabels]{enumitem}
\usepackage{stackrel}
\usepackage{tkz-tab}
\usepackage{xpatch}
\usepackage{tkz-euclide}
\usetkzobj{all}
\usepackage{tabto}
\xpatchcmd{\tkzTabLine}{$0$}{$\bullet$}{}{}

\setlength{\headheight}{10pt}
\setlength{\headsep}{10pt}
\pagestyle{fancy}
\rhead{\ayudantia \ - \alumno}
\tikzset{t style/.style={style=solid}}

\newcommand*{\mybox}[2]{\colorbox{#1!30}{\parbox{.98\linewidth}{#2}}}

\newenvironment{solucion}
{\begin{mdframed}[backgroundcolor=black!10]
		{\bf Solución:}\\
	}
	{
	\end{mdframed}
}

\newenvironment{alternativas}[1]
{\begin{multicols}{#1}
		\begin{enumerate}[a)]
		}
		{
		\end{enumerate}
	\end{multicols}
}

\newenvironment{preguntas}
{\begin{enumerate}\itemsep12pt
	}
	{
	\end{enumerate}
}

\newcommand{\ayudantia}{{\sc Ayudantía 8}}
\newcommand{\tituloayu}{Derivadas parciales y regla de la cadena}
\newcommand{\fecha}{1 de noviembre de 2019}
\newcommand{\sigla}{MAT1620}
\newcommand{\nombre}{Cálculo II}
\newcommand{\profesor}{Wolfgang Rivera}
\newcommand{\ano}{2019}
\newcommand{\semestre}{2}
\newcommand{\mail}{mat1620@ifcastaneda.cl}
\newcommand{\alumno}{Ignacio Castañeda - \mail}

\newcommand{\ev}{\Big|}
\newcommand{\ra}{\rightarrow}
\newcommand{\lra}{\leftrightarrow}
\newcommand{\N}{\mathbb{N}}
\newcommand{\R}{\mathbb{R}}
\newcommand{\Exp}[1]{\mathcal{E}_{#1}}
\newcommand{\List}[1]{\mathcal{L}_{#1}}
\newcommand{\EN}{\Exp{\N}}
\newcommand{\LN}{\List{\N}}
\newcommand{\comment}[1]{}
\newcommand{\lb}{\\~\\}
\newcommand{\eop}{_{\square}}
\newcommand{\hsig}{\hat{\sigma}}
\newcommand{\widesim}[2][1.5]{
	\mathrel{\overset{#2}{\scalebox{#1}[1]{$\sim$}}}
}
\newcommand{\wsim}{\widesim{}}
\newcommand{\lh}{\stackrel{L'H}{=}}

\begin{document}
\thispagestyle{empty}

\begin{minipage}{2cm}
	\includegraphics[width=2cm]{../../../../img/logo.pdf}
	\vspace{0.5cm}
\end{minipage}
\begin{minipage}{\linewidth}
	\begin{tabular}{lrl}
		{\scriptsize\sc Pontificia Universidad Catolica de Chile} & \hspace*{0.7in}Curso: &
		\sigla  - \nombre\\
		{\sc Facultad de Matemáticas}&
		Profesor: & \profesor \\
		{\sc Semestre \ano-\semestre} & Ayudante: & {Ignacio Castañeda}\\
		& {Mail:} & \texttt{\mail}
	\end{tabular}
\end{minipage}

\vspace{-10mm}
\begin{center}
	{\LARGE\bf \ayudantia}\\
	\vspace{0.1cm}
	{\tituloayu}\\
	\vspace{0.1cm}
	\fecha\\
	\vspace{0.4cm}
\end{center}

\begin{preguntas}
\item Determine la derivada por definición en $x$ de la función
	$$f(x,y) = 2xy + x^2y + x + y$$
\begin{solucion}
Para determinar la derivada en $x$ debemos resolver el siguiente límite
		$$f_x = \lim\limits_{h \ra 0} \dfrac{f(x+h,y) - f(x,y)}{h}$$
		En nuestro caso,
		$$f_x = \lim\limits_{h \ra 0} \dfrac{2(x+h)y + (x+h)^2y + (x+h) + y - (2xy + x^2y + x + y)}{h}$$
		$$= \lim\limits_{h \ra 0} \dfrac{2xy+2hy + x^2y+2xhy + yh^2 + x+h + y - 2xy - x^2y - x - y}{h}$$
		$$= \lim\limits_{h \ra 0} \dfrac{2hy+2xhy + h^2 +h}{h}
		= \lim\limits_{h \ra 0} \dfrac{h(2y+2xy + hy +1)}{h}$$
		$$= \lim\limits_{h \ra 0} 2y+2xy + hy +1
		= 2y+2xy+1$$
\end{solucion}
\item Para las siguientes funciones, calcular $f_x$ y $f_y$
\begin{enumerate}[a)]
\item $f(x,y) = \dfrac{xy}{x-y}$
\item $f(x,y) = (x^2+y^2)sen\left(\dfrac{1}{x^2+y^2}\right)$
\end{enumerate}
\begin{solucion}

\begin{enumerate}[a)]
\item $f(x,y) = \dfrac{xy}{x-y}$\\
			\\
			$$f_x = \dfrac{y(x-y)-xy}{(x-y)^2} = -\dfrac{y^2}{(x-y)^2}$$
			$$f_y = \dfrac{x(x-y)+xy}{(x-y)^2} = \dfrac{x^2}{(x-y)^2}$$
\item $f(x,y) = (x^2+y^2)sen\left(\dfrac{1}{x^2+y^2}\right)$
			$$f_x = 2x\ sen\left(\dfrac{1}{x^2+y^2}\right) + cos\left(\dfrac{1}{x^2+y^2}\right) \cdot \dfrac{-1}{(x^2+y^2)^2} \cdot 2x \cdot (x^2+y^2)$$
			$$ = 2x\ sen\left(\dfrac{1}{x^2+y^2}\right) - cos\left(\dfrac{1}{x^2+y^2}\right) \cdot \dfrac{2x}{x^2+y^2}$$
			$$f_y = 2x\ sen\left(\dfrac{1}{x^2+y^2}\right) + cos\left(\dfrac{1}{x^2+y^2}\right) \cdot \dfrac{-1}{(x^2+y^2)^2} \cdot 2y \cdot (x^2+y^2)$$
			$$ = 2x\ sen\left(\dfrac{1}{x^2+y^2}\right) - cos\left(\dfrac{1}{x^2+y^2}\right) \cdot \dfrac{2y}{x^2+y^2}$$
			Notemos que ambas derivadas son practicamente iguales, pero se cambiando de lugar las variables. Esto es porque la función cumple que $f(x,y) = f(y,x)$
\end{enumerate}
\end{solucion}
\item Una función armónica es aquella que cumple con $f_{xx} + f_{yy} = 0$. Determina si la siguiente funcion es armónica
	$$f(x,y) = xy + 3x^2 -y^3$$
\begin{solucion}
Derivando,
		$$f_x = y + 6x \ra f_{xx} = 6$$
		$$f_y = x -3y^2 \ra f_{yy} = -6y$$
		Como $f_{xx} + f_{yy} = 6 - 6y \neq 0$, concluimos que la función no es armónica.
\end{solucion}
\item Busque $\dfrac{\delta z}{\delta t}$ o $\dfrac{\delta w}{\delta t}$, según corresponda.
\begin{enumerate}[a)]
\item $z = x^2+y^2+xy$\tab$x=sen(t), y=e^t$
\item $w=xe^{y/z}$\tab$x=t^2, y=1-t, z=1+2t$
\item $w=ln(\sqrt[]{x^2+y^2+z^2})$\tab$x=sen(t), y=cos(t), z=tan(t)$
\end{enumerate}
\begin{solucion}

\begin{enumerate}[a)]
\item $z = x^2+y^2+xy$\tab$x=sen(t), y=e^t$\\
			\\
			$z_t = z_x x_t + z_y y_z = (2x+y)cos(t) + (2y+x)e^t$
\item $w=xe^{y/z}$\tab$x=t^2, y=1-t, z=1+2t$\\
			\\
			{\small$w_t = w_x x_t + w_y y_t + w_z z_t = e^{y/z} 2t + xe^{y/z}\dfrac{1}{z}\cdot(-1) + xe^{y/z} \cdot \dfrac{-1}{z^2}\cdot 2 = e^{y/z}\left(2t - \dfrac{x}{z} - \dfrac{2x}{z^2}\right)$}
\item $w=ln(\sqrt[]{x^2+y^2+z^2})$\tab$x=sen(t), y=cos(t), z=tan(t)$\\
			\\
			$w_t = w_x x_t + w_y y_t + w_z z_t$
			$$ = \dfrac{1}{\sqrt[]{x^2+y^2+z^2}}\cdot \dfrac{1}{2\ \sqrt[]{x^2+y^2+z^2}} (2x\ cos(t) - 2y\ sen(t) + 2z\ sec^2(t))$$
			$$ = \dfrac{1}{x^2+y^2+z^2} (x\ cos(t) - y\ sen(t) + z\ sec^2(t)) \qquad \qquad \qquad \qquad \qquad$$
\end{enumerate}
\end{solucion}
\item Sea $f$ una función con segundas derivadas parciales continuas en todo $R^2$. El cambio de variables $x=uv$, $y=\dfrac{u^2-v^2}{2}$ transforma la función $f(x,y)$ en la función $g(u,v)$.
\begin{enumerate}[a)]
\item Calcule $\dfrac{\delta g}{\delta u}, \dfrac{\delta g}{\delta v}$ en terminos de las derivadas parciales de $f$.
\item Si $f_{xx}(x,y) + f_{yy}(x,y) = 2$ para todo $(x,y) \in R^2$, determine las constantes $a,b \in \R$ tales que
		$$a\dfrac{\delta^2g}{\delta u^2} - b\dfrac{\delta^2 g}{\delta v^2} = u^2 + v^2$$
\end{enumerate}
\begin{solucion}

\begin{enumerate}[a)]
\item 
\item 
\end{enumerate}
\end{solucion}
\end{preguntas}
\end{document}