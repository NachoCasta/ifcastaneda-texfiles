\documentclass[12pt]{article}

\usepackage{fullpage}
\usepackage{graphicx}
\usepackage{amssymb}
\usepackage{amsmath}
\usepackage[none]{hyphenat}
\usepackage{parskip}
\usepackage[spanish]{babel}
\usepackage[utf8]{inputenc}
\usepackage{hyperref}
\usepackage{fancyhdr}
\usepackage{tasks}
\usepackage{mdframed}
\usepackage{xcolor}
\usepackage{pgfplots}
\usepackage[makeroom]{cancel}
\usepackage{multicol}
\usepackage[shortlabels]{enumitem}
\usepackage{stackrel}
\usepackage{tkz-tab}
\usepackage{xpatch}
\usepackage{tkz-euclide}
\usetkzobj{all}
\xpatchcmd{\tkzTabLine}{$0$}{$\bullet$}{}{}

\setlength{\headheight}{10pt}
\setlength{\headsep}{10pt}
\pagestyle{fancy}
\rhead{\ayudantia \ - \alumno}
\tikzset{t style/.style={style=solid}}

\newcommand*{\mybox}[2]{\colorbox{#1!30}{\parbox{.98\linewidth}{#2}}}

\newenvironment{solucion}
{\begin{mdframed}[backgroundcolor=black!10]
		{\bf Solución:}\\
	}
	{
	\end{mdframed}
}

\newenvironment{alternativas}[1]
{\begin{multicols}{#1}
		\begin{enumerate}[a)]
		}
		{
		\end{enumerate}
	\end{multicols}
}

\newenvironment{preguntas}
{\begin{enumerate}\itemsep12pt
	}
	{
	\end{enumerate}
}

\newcommand{\ayudantia}{{\sc Ayudantía 4}}
\newcommand{\tituloayu}{Series II}
\newcommand{\fecha}{3 de septiembre de 2019}
\newcommand{\sigla}{MAT1620}
\newcommand{\nombre}{Cálculo II}
\newcommand{\profesor}{Wolfgang Rivera}
\newcommand{\ano}{2019}
\newcommand{\semestre}{2}
\newcommand{\mail}{mat1620@ifcastaneda.cl}
\newcommand{\alumno}{Ignacio Castañeda - \mail}

\newcommand{\ev}{\Big|}
\newcommand{\ra}{\rightarrow}
\newcommand{\lra}{\leftrightarrow}
\newcommand{\N}{\mathbb{N}}
\newcommand{\R}{\mathbb{R}}
\newcommand{\Exp}[1]{\mathcal{E}_{#1}}
\newcommand{\List}[1]{\mathcal{L}_{#1}}
\newcommand{\EN}{\Exp{\N}}
\newcommand{\LN}{\List{\N}}
\newcommand{\comment}[1]{}
\newcommand{\lb}{\\~\\}
\newcommand{\eop}{_{\square}}
\newcommand{\hsig}{\hat{\sigma}}
\newcommand{\widesim}[2][1.5]{
	\mathrel{\overset{#2}{\scalebox{#1}[1]{$\sim$}}}
}
\newcommand{\wsim}{\widesim{}}
\newcommand{\lh}{\stackrel{L'H}{=}}

\begin{document}
\thispagestyle{empty}

\begin{minipage}{2cm}
	\includegraphics[width=2cm]{../../../../img/logo.pdf}
	\vspace{0.5cm}
\end{minipage}
\begin{minipage}{\linewidth}
	\begin{tabular}{lrl}
		{\scriptsize\sc Pontificia Universidad Catolica de Chile} & \hspace*{0.7in}Curso: &
		\sigla  - \nombre\\
		{\sc Facultad de Matemáticas}&
		Profesor: & \profesor \\
		{\sc Semestre \ano-\semestre} & Ayudante: & {Ignacio Castañeda}\\
		& {Mail:} & \texttt{\mail}
	\end{tabular}
\end{minipage}

\vspace{-10mm}
\begin{center}
	{\LARGE\bf \ayudantia}\\
	\vspace{0.1cm}
	{\tituloayu}\\
	\vspace{0.1cm}
	\fecha\\
	\vspace{0.4cm}
\end{center}

\begin{preguntas}
\item Determina si la siguiente serie converge o diverge
$$\sum\limits_{n=1}^{\infty}\dfrac{n!}{n^n}$$
\begin{solucion}
$\sum\limits_{n=1}^{\infty}\dfrac{n!}{n^n}$\\
			\\
			Recordemos que en el infinito,
			$$n^n > n! > a^n > n^a > ln(n)$$
			El límite de la sucesión es
			$$\lim\limits_{n\ra\infty}\dfrac{n!}{n^n} = 0$$
			Usando el criterio de la razón,
			$$\lim\limits_{n \ra \infty} \dfrac{a_{n+1}}{a_n}
			= \lim\limits_{n \ra \infty} \dfrac{\dfrac{(n+1)!}{(n+1)^{n+1}}}{\dfrac{n!}{n^n}}
			= \lim\limits_{n \ra \infty} \dfrac{(n+1)!}{n!}\dfrac{n^n}{(n+1)^{n+1}}$$
			$$= \lim\limits_{n \ra \infty} \dfrac{(n+1)}{1}\dfrac{n^n}{(n+1)^{n+1}}
			= \lim\limits_{n \ra \infty} \dfrac{n^n}{(n+1)^{n}}
			= \lim\limits_{n \ra \infty} \left(\dfrac{n}{n+1}\right)^n$$
			$$= \lim\limits_{n \ra \infty} \dfrac{1}{\left(\dfrac{n+1}{n}\right)^n}
			= \dfrac{1}{e} < 1$$
			Por criterio de la razón, la serie es convergente.
\end{solucion}
\item Estudiar, según el valor de $\lambda \in \R$, la convergencia de la serie
	$$\sum\limits_{n=1}^{\infty}ln(1+n^{-\lambda})$$
\begin{solucion}
Notemos que la serie se puede escribir como
$$\sum\limits_{n=1}^{\infty}ln\left(1+\dfrac{1}{n^{\lambda}}\right)$$
Sea $b_n = \dfrac{1}{n^{\lambda}}$
$$\lim\limits_{n \ra \infty} \dfrac{a_n}{b_n}
= \lim\limits_{n \ra \infty} ln(1+\dfrac{1}{n^{\lambda}})n^{\lambda}
= \lim\limits_{n \ra \infty} ln\left(\left(1+\dfrac{1}{n^{\lambda}}\right)^{n^{\lambda}}\right)
= ln(e) = 1 \neq 0$$
Luego, por CCL, $\sum\limits_{n=1}^{\infty}ln\left(1+\dfrac{1}{n^{\lambda}}\right)$ se comportará igual a $\sum\limits_{n=1}^{\infty} \dfrac{1}{n^{\lambda}}$.\\

Notemos que esta última es una serie-p, por lo que convergerá para $\lambda > 1$ y divergerá para $\lambda \leq 1$.\\

Finalmente, $\sum\limits_{n=1}^{\infty}ln(1+n^{-\lambda})$ converge para $\lambda > 1$ y diverge para $\lambda \leq 1$
\end{solucion}
\item Determine si las siguientes series convergen condicionalmente, absolutamente o divergen.
\begin{tasks}(3)
\task $\sum\limits_{n=1}^{\infty}\dfrac{(-1)^{n+1}}{\sqrt[]{n}}$
\task $\sum\limits_{n=2}^{\infty}\dfrac{(-1)^{n-1}(2n-1)}{(\sqrt[]{2})^n}$
\task $\sum\limits_{n=2}^{\infty}\dfrac{(-1)^{n-1}(n+1)}{n}$
\end{tasks}
\begin{solucion}

\begin{enumerate}[a)]
\item $\sum\limits_{n=1}^{\infty}\dfrac{(-1)^{n+1}}{\sqrt[]{n}}$\\
			\\
			Veamos el límite de la sucesión,
			$$\lim_{n\ra\infty} |a_n| = \lim_{n\ra\infty} \dfrac{1}{\sqrt[]{n}} = 0$$
			Además, notemos que 
			$$\dfrac{1}{\sqrt[]{n}} > \dfrac{1}{\sqrt[]{n+1}} \ra \dfrac{1}{n} > \dfrac{1}{n+1} \ra n < n+1$$
			Por lo que la sucesión es decreciente.\\
			\\
			Dicho esto, por el criterio de Leibniz, la serie converge.\\
			\\
			Veamos ahora que pasa con la serie no alternante 
			$$\sum\limits_{n=1}^{\infty}\dfrac{1}{\sqrt[]{n}} = \sum\limits_{n=1}^{\infty}\dfrac{1}{n^{1/2}}$$
			Es una serie-p con $p<1$, por lo que es divergente.\\
			\\
			Finalmente, la serie converge condicionalmente.
\item $\sum\limits_{n=2}^{\infty}\dfrac{(-1)^{n-1}(2n-1)}{(\sqrt[]{2})^n}$\\
			\\
			Veamos el límite de la sucesión,
			$$\lim_{n\ra\infty} |a_n| = \lim_{n\ra\infty} \dfrac{2n-1}{(\sqrt[]{2})^n} = 0$$
			Usando el criterio de la razón,
			$$\lim\limits_{n \ra \infty} \left|\dfrac{a_{n+1}}{a_n}\right|
			= \lim\limits_{n \ra \infty} \dfrac{2n+2-1}{(\sqrt[]{2})^{\cancel{n+1}}} \cdot \dfrac{(\cancel{\sqrt[]{2})^n}}{2n-1}
			= \lim\limits_{n \ra \infty} \dfrac{2n+1}{(2n-1)(\sqrt[]{2})} = \dfrac{1}{\sqrt[]{2}} < 1$$
			Por lo tanto, la serie no alternante converge, lo que implica que la alternante también. En conclusióm, la serie converge absolutamente.
\item $\sum\limits_{n=2}^{\infty}\dfrac{(-1)^{n-1}(n+1)}{n}$\\
			\\
			Veamos el límite de la sucesión,
			$$\lim_{n\ra\infty} |a_n| = \lim_{n\ra\infty}\dfrac{(n+1)}{n} = 1 \neq 0$$
			Por la prueba de la divergencia, la serie es divergente.
\end{enumerate}
\end{solucion}
\item Determine el radio y los intervalos de convergencia de las siguientes series
\begin{tasks}(3)
\task $\sum\limits_{n=1}^{\infty}\dfrac{(-1)^{n-1}(x+3)^n}{3n}$
\task $\sum\limits_{n=2}^{\infty}\dfrac{2(x-4)^n}{n}$
\task $\sum\limits_{n=2}^{\infty}\dfrac{(x-2)^n}{2^{n+1}}$
\end{tasks}
\begin{solucion}
Sea $\sum\limits_{n=1}^{\infty} a_n (x-c)^n$ una serie de potencias, existen dos métodos para obtener el intervalo y radio de convergencia.\\

El primer método para obtener el intervalo es notando que la serie sera convergente siempre que
$$\lim\limits_{n \ra \infty} \left|\dfrac{a_{n+1}}{a_n}\right| < 1$$
Los bordes debemos evaluarlos de manera individual y el radio será la mitad del largo del intervalo.\\
Notemos que en este caso debemos incluir $(x-c)^n$ en el término general.\\
\\
El segundo método (en mi opinión más facil) consiste en notar que el centro del intervalo siempre será $c$ y que el radio cumple con
$$\dfrac{1}{R} = \lim\limits_{n \ra \infty} \dfrac{a_{n+1}}{a_n}$$
Luego, igual que antes, se evaluan los bordes de manera indiviual. Estos serán $c-R$ y $c+R$.
\begin{enumerate}[a)]
\item $\sum\limits_{n=1}^{\infty}\dfrac{(-1)^{n-1}(x+3)^n}{3n}$\\
\\
Para que la serie converga, debe ocurrir que
$$\lim\limits_{n \ra \infty} \left|\dfrac{a_{n+1}}{a_n}\right| < 1$$
$$\lim\limits_{n \ra \infty} \left|\dfrac{(x+3)^{n+1}}{3n+3}\dfrac{3n}{(x+3)^n}\right| < 1$$
$$|x+3| < 1$$
$$-1 < x+3 < 1$$
$$-4 < x < -2$$
De aqui podemos ver que el radio de convergencia es 1. Para determinar el intervalo de convergencia debemos ver que ocurre en los bordes.

\begin{itemize}
	\item $x = -4$\\
	\\
	$\sum\limits_{n=1}^{\infty}(-1)^{n-1}\dfrac{(-1)^n}{3n}
	= \sum\limits_{n=1}^{\infty}-\dfrac{1}{3n} 
	= -\dfrac{1}{3n} \sum\limits_{n=1}^{\infty}\dfrac{1}{n} 
	\ra \text{diverge}
	$
	\item $x = -2$\\
	\\
	$\sum\limits_{n=1}^{\infty}(-1)^{n-1}\dfrac{1}{3n}
	= \dfrac{1}{3} \sum\limits_{n=1}^{\infty}(-1)^{n-1}\dfrac{1}{n}
	\ra \text{converge por Leibniz}
	$
\end{itemize}
Finalmente, el intervalo de convergencia es $]-4, 2]$ y su radio de convergencia es 1.
\item $\sum\limits_{n=2}^{\infty}\dfrac{2(x-4)^n}{n}$\\
\\
$$\dfrac{1}{R} = \lim\limits_{n \ra \infty} \dfrac{a_{n+1}}{a_n}
= \lim\limits_{n \ra \infty} \dfrac{2}{n+1}\dfrac{n}{2} = 1 \ra R = 1$$
Notemos que el centro es $4$, por lo que debemos evaluar en $x=3$ y $x=5$.
\begin{itemize}
	\item $x=3$\\
	\\
	$\sum\limits_{n=2}^{\infty}(-1)^n\dfrac{2}{n} \ra \text{converge por Leibniz}$	\item $x=5$\\
	\\
	$\sum\limits_{n=2}^{\infty}\dfrac{2}{n} \ra \text{diverge}$
\end{itemize}
Finalmente, el intervalo es $[3,5[$
\item $\sum\limits_{n=2}^{\infty}\dfrac{2(x-4)^n}{n}$\\
\\
$$\dfrac{1}{R} = \lim\limits_{n \ra \infty} \dfrac{a_{n+1}}{a_n}
= \lim\limits_{n \ra \infty} \dfrac{2}{n+1}\dfrac{n}{2} = 1 \ra R = 1$$
Notemos que el centro es $4$, por lo que debemos evaluar en $x=3$ y $x=5$.
\begin{itemize}
	\item $x=3$\\
	\\
	$\sum\limits_{n=2}^{\infty}(-1)^n\dfrac{2}{n} \ra \text{converge por Leibniz}$	\item $x=5$\\
	\\
	$\sum\limits_{n=2}^{\infty}\dfrac{2}{n} \ra \text{diverge}$
\end{itemize}
Finalmente, el intervalo es $[3,5[$
\item $\sum\limits_{n=2}^{\infty}\dfrac{(x-2)^n}{2^{n+1}}$\\
\\
$$\dfrac{1}{R} = \lim\limits_{n \ra \infty} \dfrac{a_{n+1}}{a_n}
= \lim\limits_{n \ra \infty} \dfrac{1}{2^{n+2}}\dfrac{2^{n+1}}{1} = \dfrac{1}{2} \ra R = 2$$
Notemos que el centro es $2$, por lo que debemos evaluar en $x=0$ y $x=4$.
\begin{itemize}
	\item $x=0$\\
	\\
	$\sum\limits_{n=2}^{\infty}\dfrac{(-2)^n}{2^{n+1}} = \sum\limits_{n=2}^{\infty}(-1)^n\dfrac{2^n}{2^{n+1}} = \sum\limits_{n=2}^{\infty}(-1)^n\dfrac{1}{2} \ra \text{divergente}$
	\item $x=4$\\
	\\
	$\sum\limits_{n=2}^{\infty}\dfrac{2^n}{2^{n+1}} = \sum\limits_{n=2}^{\infty}\dfrac{1}{2} \ra \text{divergente}$
\end{itemize}
Finalmente, el intervalo de convergencia corresponde a $]0, 4[$.
\end{enumerate}
\end{solucion}
\end{preguntas}
\end{document}