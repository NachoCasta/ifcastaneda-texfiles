\documentclass[12pt]{article}

\usepackage{fullpage}
\usepackage{graphicx}
\usepackage{amssymb}
\usepackage{amsmath}
\usepackage[none]{hyphenat}
\usepackage{parskip}
\usepackage[spanish]{babel}
\usepackage[utf8]{inputenc}
\usepackage{hyperref}
\usepackage{fancyhdr}
\usepackage{tasks}
\usepackage{mdframed}
\usepackage{xcolor}
\usepackage{pgfplots}
\usepackage[makeroom]{cancel}
\usepackage{multicol}
\usepackage[shortlabels]{enumitem}
\usepackage{stackrel}
\usepackage{tkz-tab}
\usepackage{xpatch}
\usepackage{tkz-euclide}
\usetkzobj{all}
\usepackage{tabto}
\xpatchcmd{\tkzTabLine}{$0$}{$\bullet$}{}{}

\setlength{\headheight}{10pt}
\setlength{\headsep}{10pt}
\pagestyle{fancy}
\rhead{\ayudantia \ - \alumno}
\tikzset{t style/.style={style=solid}}

\newcommand*{\mybox}[2]{\colorbox{#1!30}{\parbox{.98\linewidth}{#2}}}

\newenvironment{solucion}
{\begin{mdframed}[backgroundcolor=black!10]
		{\bf Solución:}\\
	}
	{
	\end{mdframed}
}

\newenvironment{alternativas}[1]
{\begin{multicols}{#1}
		\begin{enumerate}[a)]
		}
		{
		\end{enumerate}
	\end{multicols}
}

\newenvironment{preguntas}
{\begin{enumerate}\itemsep12pt
	}
	{
	\end{enumerate}
}

\newcommand{\ayudantia}{{\sc Ayudantía 7}}
\newcommand{\tituloayu}{Límites de varias variables, derivadas parciales y regla de la cadena}
\newcommand{\fecha}{24 de septiembre de 2019}
\newcommand{\sigla}{MAT1620}
\newcommand{\nombre}{Cálculo II}
\newcommand{\profesor}{Wolfgang Rivera}
\newcommand{\ano}{2019}
\newcommand{\semestre}{2}
\newcommand{\mail}{mat1620@ifcastaneda.cl}
\newcommand{\alumno}{Ignacio Castañeda - \mail}

\newcommand{\ev}{\Big|}
\newcommand{\ra}{\rightarrow}
\newcommand{\lra}{\leftrightarrow}
\newcommand{\N}{\mathbb{N}}
\newcommand{\R}{\mathbb{R}}
\newcommand{\Exp}[1]{\mathcal{E}_{#1}}
\newcommand{\List}[1]{\mathcal{L}_{#1}}
\newcommand{\EN}{\Exp{\N}}
\newcommand{\LN}{\List{\N}}
\newcommand{\comment}[1]{}
\newcommand{\lb}{\\~\\}
\newcommand{\eop}{_{\square}}
\newcommand{\hsig}{\hat{\sigma}}
\newcommand{\widesim}[2][1.5]{
	\mathrel{\overset{#2}{\scalebox{#1}[1]{$\sim$}}}
}
\newcommand{\wsim}{\widesim{}}
\newcommand{\lh}{\stackrel{L'H}{=}}

\begin{document}
\thispagestyle{empty}

\begin{minipage}{2cm}
	\includegraphics[width=2cm]{../../../../img/logo.pdf}
	\vspace{0.5cm}
\end{minipage}
\begin{minipage}{\linewidth}
	\begin{tabular}{lrl}
		{\scriptsize\sc Pontificia Universidad Catolica de Chile} & \hspace*{0.7in}Curso: &
		\sigla  - \nombre\\
		{\sc Facultad de Matemáticas}&
		Profesor: & \profesor \\
		{\sc Semestre \ano-\semestre} & Ayudante: & {Ignacio Castañeda}\\
		& {Mail:} & \texttt{\mail}
	\end{tabular}
\end{minipage}

\vspace{-10mm}
\begin{center}
	{\LARGE\bf \ayudantia}\\
	\vspace{0.1cm}
	{\tituloayu}\\
	\vspace{0.1cm}
	\fecha\\
	\vspace{0.4cm}
\end{center}

\begin{preguntas}
\item Determinar si los siguientes limites existen o no. En caso de que existan, calcule su valor
\begin{tasks}(4)
\task $\lim\limits_{(x,y) \to (0,0)} \dfrac{x^2}{x^2+y^2}$
\task $\lim\limits_{(x,y) \to (0,0)} \dfrac{5x^2y}{x^2+y^2}$
\task $\lim\limits_{(x,y) \to (0,0)} \dfrac{xy}{x^2+y^2}$
\task $\lim\limits_{(x,y) \to (0,0)} \dfrac{x^2ye^y}{x^4+y^2}$
\end{tasks}
\begin{solucion}

\begin{enumerate}[a)]
\item $\lim\limits_{(x,y) \to (0,0)} \dfrac{x^2}{x^2+y^2}$\\
			\\
			Notemos que a simple vista el grado del numerador y el denominador son iguales, por lo que es de esperarse que el límite no exista.
			
			Probemos desde distintas direcciones
			$$x = 0 \ra \lim\limits_{(0,y) \to (0,0)} \dfrac{0^2}{0^2+y^2}
			= 0$$
			$$y = 0 \ra \lim\limits_{(x,0) \to (0,0)} \dfrac{x^2}{x^2+0^2}
			= 1$$
			Como $0 \neq 1$, concluimos que el límite no existe.
\item $\lim\limits_{(x,y) \to (0,0)} \dfrac{5x^2y}{x^2+y^2}$\\
			\\
			En este caso, el grado del numerador es 3 y el del denominador es 2 por lo que, siendo el numerador mayor al denominador, es de esperarse que el límite exista (y probablemente sea igual a 0).
			
			Probemos desde distintas direcciones.
			$$x = 0 \ra \lim\limits_{(0,y) \to (0,0)} \dfrac{5\cdot0^2y}{0^2+y^2} = 0$$
			$$y = 0 \ra \lim\limits_{(x,0) \to (0,0)} \dfrac{5x^2\cdot 0}{x^2+0^2} = 0$$
			$$y = mx \ra \lim\limits_{(x,mx) \to (0,0)} \dfrac{5x^2 mx}{x^2+(mx)^2}
			= \lim\limits_{(x,mx) \to (0,0)} \dfrac{5mx^3}{x^2+m^2x^2}$$
			$$\qquad \quad = \lim\limits_{(x,mx) \to (0,0)} \dfrac{5mx^3}{x^2(1+m^2)}
			= \lim\limits_{(x,mx) \to (0,0)} \dfrac{5mx}{1+m^2}$$
			$$= \dfrac{0}{1+m^2} = 0$$
			Hasta el momento todo indica que el límite existe, sin embargo, para estar seguros es necesario comprobar con coordenadas polares, por lo que haremos el cambio de variables
			$$x = rcos(\theta), \quad y = rsen(\theta)$$
			Notemos que con este cambio,
			$$x^2+y^2 = r^2, \quad (x,y)\ra 0 \Longrightarrow r \ra 0$$
			Luego, debemos resolver el límite
			$$\lim\limits_{r\ra 0} \dfrac{5r^2cos^2(\theta)rsen(\theta)}{r^2}
			= \lim\limits_{r\ra 0} 5rcos^2(\theta)sen(\theta) = 0$$
			Como las funciones $sen$ y $cos$ son funciones acotadas, por teorema del sandwich concluimos que el límite existe y es cero.
\item $\lim\limits_{(x,y) \to (0,0)} \dfrac{xy}{x^2+y^2}$\\
			\\
			En este caso, el grado del numerador y el denominador son iguales, por lo que es de esperarse que el límite no exista.
			
			Probemos desde distintas direcciones.
			$$x = 0 \ra \lim\limits_{(0,y) \to (0,0)} \dfrac{0\cdot y}{0^2+y^2} = 0$$
			$$y = 0 \ra \lim\limits_{(x,0) \to (0,0)} \dfrac{x\cdot 0}{x^2+0^2} = 0$$
			$$y = mx \ra \lim\limits_{(x,mx) \to (0,0)} \dfrac{x mx}{x^2+(mx)^2}
			= \lim\limits_{(x,mx) \to (0,0)} \dfrac{mx^2}{x^2+m^2x^2}$$
			$$ = \lim\limits_{(x,mx) \to (0,0)} \dfrac{mx^2}{x^2(1+m^2)}
			= \dfrac{m}{1+m^2}$$
			Como el resultado depende de $m$, para distintos valores de $m$ el valor será distinto, por lo que concluimos que el límite no existe.
\item $\lim\limits_{(x,y) \to (0,0)} \dfrac{x^2ye^y}{x^4+y^2}$\\
			\\
			A simple vista, el grado del numerador es 3 y el del denominador es 4 por lo que, dado que el numerador es menor que el denominador, es de esperarse que el límite no exista.
			
			Probemos desde distintas direcciones.
			$$x = 0 \ra \lim\limits_{(0,y) \to (0,0)} \dfrac{0^2ye^y}{0^4+y^2} = 0$$
			$$y = 0 \ra \lim\limits_{(x,0) \to (0,0)} \dfrac{x^2\cdot \cdot e^0}{x^4+0^2} = 0$$
			$$y = mx \ra \lim\limits_{(x,mx) \to (0,0)} \dfrac{mx^3e^{mx}}{x^4+m^2x^2} = 
			\lim\limits_{(x,mx) \to (0,0)} \dfrac{mxe^{mx}}{x^2+m^2} = 0$$
			Hasta el momento todo indica que el límite si existe y es cero, sin embargo, como intuitivamente dijimos que no debía existir, busquemos otro acercamiento al límite. Notemos que el exponente del $x$ y el $y$ en el denominador son $4$ y $2$, respectivamente, por lo que probaremos con una función que iguale estos exponente. Esta puede ser, cualquiera de la forma $y = mx^2$.
			$$y = mx^2 \ra \lim\limits_{(x,mx^2) \to (0,0)} \dfrac{mx^4e^{mx^2}}{x^4+m^2x^4} = 
			\lim\limits_{(x,mx) \to (0,0)} \dfrac{me^{mx^2}}{1+m^2} = \dfrac{m}{1+m^2} \neq 0$$
			Finalmente, concluimos que el límite no existe.
\end{enumerate}
\end{solucion}
\item Determine la derivada por definición en $x$ de la función
	$$f(x,y) = 2xy + x^2y + x + y$$
\begin{solucion}
Para determinar la derivada en $x$ debemos resolver el siguiente límite
		$$f_x = \lim\limits_{h \ra 0} \dfrac{f(x+h,y) - f(x,y)}{h}$$
		En nuestro caso,
		$$f_x = \lim\limits_{h \ra 0} \dfrac{2(x+h)y + (x+h)^2y + (x+h) + y - (2xy + x^2y + x + y)}{h}$$
		$$= \lim\limits_{h \ra 0} \dfrac{2xy+2hy + x^2y+2xhy + yh^2 + x+h + y - 2xy - x^2y - x - y}{h}$$
		$$= \lim\limits_{h \ra 0} \dfrac{2hy+2xhy + h^2 +h}{h}
		= \lim\limits_{h \ra 0} \dfrac{h(2y+2xy + hy +1)}{h}$$
		$$= \lim\limits_{h \ra 0} 2y+2xy + hy +1
		= 2y+2xy+1$$
\end{solucion}
\item Para las siguientes funciones, calcular $f_x$ y $f_y$
\begin{enumerate}[a)]
\item $f(x,y) = \dfrac{xy}{x-y}$
\item $f(x,y) = (x^2+y^2)sen\left(\dfrac{1}{x^2+y^2}\right)$
\end{enumerate}
\begin{solucion}

\begin{enumerate}[a)]
\item $f(x,y) = \dfrac{xy}{x-y}$\\
			\\
			$$f_x = \dfrac{y(x-y)-xy}{(x-y)^2} = -\dfrac{y^2}{(x-y)^2}$$
			$$f_y = \dfrac{x(x-y)+xy}{(x-y)^2} = \dfrac{x^2}{(x-y)^2}$$
\item $f(x,y) = (x^2+y^2)sen\left(\dfrac{1}{x^2+y^2}\right)$
			$$f_x = 2x\ sen\left(\dfrac{1}{x^2+y^2}\right) + cos\left(\dfrac{1}{x^2+y^2}\right) \cdot \dfrac{-1}{(x^2+y^2)^2} \cdot 2x \cdot (x^2+y^2)$$
			$$ = 2x\ sen\left(\dfrac{1}{x^2+y^2}\right) - cos\left(\dfrac{1}{x^2+y^2}\right) \cdot \dfrac{2x}{x^2+y^2}$$
			$$f_y = 2x\ sen\left(\dfrac{1}{x^2+y^2}\right) + cos\left(\dfrac{1}{x^2+y^2}\right) \cdot \dfrac{-1}{(x^2+y^2)^2} \cdot 2y \cdot (x^2+y^2)$$
			$$ = 2x\ sen\left(\dfrac{1}{x^2+y^2}\right) - cos\left(\dfrac{1}{x^2+y^2}\right) \cdot \dfrac{2y}{x^2+y^2}$$
			Notemos que ambas derivadas son practicamente iguales, pero se cambiando de lugar las variables. Esto es porque la función cumple que $f(x,y) = f(y,x)$
\end{enumerate}
\end{solucion}
\item Una función armónica es aquella que cumple con $f_{xx} + f_{yy} = 0$. Determina si la siguiente funcion es armónica
	$$f(x,y) = xy + 3x^2 -y^3$$
\begin{solucion}
Derivando,
		$$f_x = y + 6x \ra f_{xx} = 6$$
		$$f_y = x -3y^2 \ra f_{yy} = -6y$$
		Como $f_{xx} + f_{yy} = 6 - 6y \neq 0$, concluimos que la función no es armónica.
\end{solucion}
\item Busque $\dfrac{\delta z}{\delta t}$ o $\dfrac{\delta w}{\delta t}$, según corresponda.
\begin{enumerate}[a)]
\item $z = x^2+y^2+xy$\tab$x=sen(t), y=e^t$
\item $w=xe^{y/z}$\tab$x=t^2, y=1-t, z=1+2t$
\item $w=ln(\sqrt[]{x^2+y^2+z^2})$\tab$x=sen(t), y=cos(t), z=tan(t)$
\end{enumerate}
\begin{solucion}

\begin{enumerate}[a)]
\item $z = x^2+y^2+xy$\tab$x=sen(t), y=e^t$\\
			\\
			$z_t = z_x x_t + z_y y_z = (2x+y)cos(t) + (2y+x)e^t$
\item $w=xe^{y/z}$\tab$x=t^2, y=1-t, z=1+2t$\\
			\\
			{\small$w_t = w_x x_t + w_y y_t + w_z z_t = e^{y/z} 2t + xe^{y/z}\dfrac{1}{z}\cdot(-1) + xe^{y/z} \cdot \dfrac{-1}{z^2}\cdot 2 = e^{y/z}\left(2t - \dfrac{x}{z} - \dfrac{2x}{z^2}\right)$}
\item $w=ln(\sqrt[]{x^2+y^2+z^2})$\tab$x=sen(t), y=cos(t), z=tan(t)$\\
			\\
			$w_t = w_x x_t + w_y y_t + w_z z_t$
			$$ = \dfrac{1}{\sqrt[]{x^2+y^2+z^2}}\cdot \dfrac{1}{2\ \sqrt[]{x^2+y^2+z^2}} (2x\ cos(t) - 2y\ sen(t) + 2z\ sec^2(t))$$
			$$ = \dfrac{1}{x^2+y^2+z^2} (x\ cos(t) - y\ sen(t) + z\ sec^2(t)) \qquad \qquad \qquad \qquad \qquad$$
\end{enumerate}
\end{solucion}
\end{preguntas}
\end{document}