\documentclass[12pt]{article}

\usepackage{fullpage}
\usepackage{graphicx}
\usepackage{amssymb}
\usepackage{amsmath}
\usepackage[none]{hyphenat}
\usepackage{parskip}
\usepackage[spanish]{babel}
\usepackage[utf8]{inputenc}
\usepackage{hyperref}
\usepackage{fancyhdr}
\usepackage{tasks}
\usepackage{mdframed}
\usepackage{xcolor}
\usepackage{pgfplots}
\usepackage[makeroom]{cancel}
\usepackage{multicol}
\usepackage[shortlabels]{enumitem}
\usepackage{stackrel}
\usepackage{tkz-tab}
\usepackage{xpatch}
\usepackage{tkz-euclide}
\usetkzobj{all}
\usepackage{tabto}
\xpatchcmd{\tkzTabLine}{$0$}{$\bullet$}{}{}

\setlength{\headheight}{10pt}
\setlength{\headsep}{10pt}
\pagestyle{fancy}
\rhead{\ayudantia \ - \alumno}
\tikzset{t style/.style={style=solid}}

\newcommand*{\mybox}[2]{\colorbox{#1!30}{\parbox{.98\linewidth}{#2}}}

\newenvironment{solucion}
{\begin{mdframed}[backgroundcolor=black!10]
		{\bf Solución:}\\
	}
	{
	\end{mdframed}
}

\newenvironment{alternativas}[1]
{\begin{multicols}{#1}
		\begin{enumerate}[a)]
		}
		{
		\end{enumerate}
	\end{multicols}
}

\newenvironment{preguntas}
{\begin{enumerate}\itemsep12pt
	}
	{
	\end{enumerate}
}

\newcommand{\ayudantia}{{\sc Ayudantía 7}}
\newcommand{\tituloayu}{Limites de varias variables y continuidad}
\newcommand{\fecha}{24 de septiembre de 2019}
\newcommand{\sigla}{MAT1620}
\newcommand{\nombre}{Cálculo II}
\newcommand{\profesor}{Wolfgang Rivera}
\newcommand{\ano}{2019}
\newcommand{\semestre}{2}
\newcommand{\mail}{mat1620@ifcastaneda.cl}
\newcommand{\alumno}{Ignacio Castañeda - \mail}

\newcommand{\ev}{\Big|}
\newcommand{\ra}{\rightarrow}
\newcommand{\lra}{\leftrightarrow}
\newcommand{\N}{\mathbb{N}}
\newcommand{\R}{\mathbb{R}}
\newcommand{\Exp}[1]{\mathcal{E}_{#1}}
\newcommand{\List}[1]{\mathcal{L}_{#1}}
\newcommand{\EN}{\Exp{\N}}
\newcommand{\LN}{\List{\N}}
\newcommand{\comment}[1]{}
\newcommand{\lb}{\\~\\}
\newcommand{\eop}{_{\square}}
\newcommand{\hsig}{\hat{\sigma}}
\newcommand{\widesim}[2][1.5]{
	\mathrel{\overset{#2}{\scalebox{#1}[1]{$\sim$}}}
}
\newcommand{\wsim}{\widesim{}}
\newcommand{\lh}{\stackrel{L'H}{=}}

\begin{document}
\thispagestyle{empty}

\begin{minipage}{2cm}
	\includegraphics[width=2cm]{../../../../img/logo.pdf}
	\vspace{0.5cm}
\end{minipage}
\begin{minipage}{\linewidth}
	\begin{tabular}{lrl}
		{\scriptsize\sc Pontificia Universidad Catolica de Chile} & \hspace*{0.7in}Curso: &
		\sigla  - \nombre\\
		{\sc Facultad de Matemáticas}&
		Profesor: & \profesor \\
		{\sc Semestre \ano-\semestre} & Ayudante: & {Ignacio Castañeda}\\
		& {Mail:} & \texttt{\mail}
	\end{tabular}
\end{minipage}

\vspace{-10mm}
\begin{center}
	{\LARGE\bf \ayudantia}\\
	\vspace{0.1cm}
	{\tituloayu}\\
	\vspace{0.1cm}
	\fecha\\
	\vspace{0.4cm}
\end{center}

\begin{preguntas}
\item Determinar si los siguientes limites existen o no. En caso de que existan, calcule su valor
\begin{tasks}(2)
\task $\lim\limits_{(x,y) \to (0,0)} \dfrac{x^2}{x^2+y^2}$
\task $\lim\limits_{(x,y) \to (0,0)} \dfrac{5x^2y}{x^2+y^2}$
\task $\lim\limits_{(x,y) \to (0,0)} \dfrac{xy}{x^2+y^2}$
\task $\lim\limits_{(x,y) \to (0,0)} \dfrac{x^2ye^y}{x^4+y^2}$
\task $\lim\limits_{(x,y) \to (0,0)} \dfrac{sen(x^2+y^2)}{x^2+y^2}$
\task $\lim\limits_{(x,y) \to (0,0)} \dfrac{sen(xy)}{xy}$
\end{tasks}
\begin{solucion}

\begin{enumerate}[a)]
\item $\lim\limits_{(x,y) \to (0,0)} \dfrac{x^2}{x^2+y^2}$\\
			\\
			Notemos que a simple vista el grado del numerador y el denominador son iguales, por lo que es de esperarse que el límite no exista.
			
			Probemos desde distintas direcciones
			$$x = 0 \ra \lim\limits_{(0,y) \to (0,0)} \dfrac{0^2}{0^2+y^2}
			= 0$$
			$$y = 0 \ra \lim\limits_{(x,0) \to (0,0)} \dfrac{x^2}{x^2+0^2}
			= 1$$
			Como $0 \neq 1$, concluimos que el límite no existe.
\item $\lim\limits_{(x,y) \to (0,0)} \dfrac{5x^2y}{x^2+y^2}$\\
			\\
			En este caso, el grado del numerador es 3 y el del denominador es 2 por lo que, siendo el numerador mayor al denominador, es de esperarse que el límite exista (y probablemente sea igual a 0).
			
			Probemos desde distintas direcciones.
			$$x = 0 \ra \lim\limits_{(0,y) \to (0,0)} \dfrac{5\cdot0^2y}{0^2+y^2} = 0$$
			$$y = 0 \ra \lim\limits_{(x,0) \to (0,0)} \dfrac{5x^2\cdot 0}{x^2+0^2} = 0$$
			$$y = mx \ra \lim\limits_{(x,mx) \to (0,0)} \dfrac{5x^2 mx}{x^2+(mx)^2}
			= \lim\limits_{(x,mx) \to (0,0)} \dfrac{5mx^3}{x^2+m^2x^2}$$
			$$\qquad \quad = \lim\limits_{(x,mx) \to (0,0)} \dfrac{5mx^3}{x^2(1+m^2)}
			= \lim\limits_{(x,mx) \to (0,0)} \dfrac{5mx}{1+m^2}$$
			$$= \dfrac{0}{1+m^2} = 0$$
			Hasta el momento todo indica que el límite existe, sin embargo, para estar seguros es necesario comprobar con coordenadas polares, por lo que haremos el cambio de variables
			$$x = rcos(\theta), \quad y = rsen(\theta)$$
			Notemos que con este cambio,
			$$x^2+y^2 = r^2, \quad (x,y)\ra 0 \Longrightarrow r \ra 0$$
			Luego, debemos resolver el límite
			$$\lim\limits_{r\ra 0} \dfrac{5r^2cos^2(\theta)rsen(\theta)}{r^2}
			= \lim\limits_{r\ra 0} 5rcos^2(\theta)sen(\theta) = 0$$
			Como las funciones $sen$ y $cos$ son funciones acotadas, por teorema del sandwich concluimos que el límite existe y es cero.
\item $\lim\limits_{(x,y) \to (0,0)} \dfrac{xy}{x^2+y^2}$\\
			\\
			En este caso, el grado del numerador y el denominador son iguales, por lo que es de esperarse que el límite no exista.
			
			Probemos desde distintas direcciones.
			$$x = 0 \ra \lim\limits_{(0,y) \to (0,0)} \dfrac{0\cdot y}{0^2+y^2} = 0$$
			$$y = 0 \ra \lim\limits_{(x,0) \to (0,0)} \dfrac{x\cdot 0}{x^2+0^2} = 0$$
			$$y = mx \ra \lim\limits_{(x,mx) \to (0,0)} \dfrac{x mx}{x^2+(mx)^2}
			= \lim\limits_{(x,mx) \to (0,0)} \dfrac{mx^2}{x^2+m^2x^2}$$
			$$ = \lim\limits_{(x,mx) \to (0,0)} \dfrac{mx^2}{x^2(1+m^2)}
			= \dfrac{m}{1+m^2}$$
			Como el resultado depende de $m$, para distintos valores de $m$ el valor será distinto, por lo que concluimos que el límite no existe.
\item $\lim\limits_{(x,y) \to (0,0)} \dfrac{x^2ye^y}{x^4+y^2}$\\
			\\
			A simple vista, el grado del numerador es 3 y el del denominador es 4 por lo que, dado que el numerador es menor que el denominador, es de esperarse que el límite no exista.
			
			Probemos desde distintas direcciones.
			$$x = 0 \ra \lim\limits_{(0,y) \to (0,0)} \dfrac{0^2ye^y}{0^4+y^2} = 0$$
			$$y = 0 \ra \lim\limits_{(x,0) \to (0,0)} \dfrac{x^2\cdot \cdot e^0}{x^4+0^2} = 0$$
			$$y = mx \ra \lim\limits_{(x,mx) \to (0,0)} \dfrac{mx^3e^{mx}}{x^4+m^2x^2} = 
			\lim\limits_{(x,mx) \to (0,0)} \dfrac{mxe^{mx}}{x^2+m^2} = 0$$
			Hasta el momento todo indica que el límite si existe y es cero, sin embargo, como intuitivamente dijimos que no debía existir, busquemos otro acercamiento al límite. Notemos que el exponente del $x$ y el $y$ en el denominador son $4$ y $2$, respectivamente, por lo que probaremos con una función que iguale estos exponente. Esta puede ser, cualquiera de la forma $y = mx^2$.
			$$y = mx^2 \ra \lim\limits_{(x,mx^2) \to (0,0)} \dfrac{mx^4e^{mx^2}}{x^4+m^2x^4} = 
			\lim\limits_{(x,mx) \to (0,0)} \dfrac{me^{mx^2}}{1+m^2} = \dfrac{m}{1+m^2} \neq 0$$
			Finalmente, concluimos que el límite no existe.
\item $\lim\limits_{(x,y) \to (0,0)} \dfrac{sen(x^2+y^2)}{x^2+y^2}$\\
\\
Probemos desde distintas direcciones.
$$x = 0 \ra \lim\limits_{(0,y) \to (0,0)} \dfrac{sen(0^2+y^2)}{0^2+y^2} 
= \lim\limits_{y \to 0} \dfrac{sen(y^2)}{y^2}
= 1$$
$$y = 0 \ra \lim\limits_{(x,0) \to (0,0)} \dfrac{sen(x^2+0^2)}{x^2+0^2} 
= \lim\limits_{x \to 0} \dfrac{sen(x^2)}{x^2}
= 1$$
$$y = mx \ra \lim\limits_{(x,mx) \to (0,0)} \dfrac{sen(x^2+(mx)^2)}{x^2+(mx)^2} 
= \lim\limits_{x \to 0} \dfrac{sen(x^2(1+m^2))}{x^2(1+m^2)}
= 1$$
Ahora, comprobamos en polares
$$x = rcos(\theta), \quad y = rsen(\theta)$$
$$x^2 + y^2 = r^2$$
Con esto, el límite queda
$$\lim\limits_{r\ra 0} \dfrac{sen(r^2)}{r^2} = 1$$
\item $\lim\limits_{(x,y) \to (0,0)} \dfrac{sen(xy)}{xy}$\\
\\
Probemos directamente con polares,
$$x = rcos(\theta), \quad y = rsen(\theta)$$
Con esto, el límite queda
$$\lim\limits_{r\ra 0} \dfrac{sen(rcos(\theta) rsen(\theta))}{rcos(\theta) rsen(\theta)}$$
Uno tendería a pensar que esto es 1, sin embargo, si evaluamos con $\theta = 0$ antes de evaluar $r$, nos queda $\dfrac{0}{0}$, lo que es indefinido. Ojo que esto no es lo mismo que un límite que al evaluarlo nos queda $\dfrac{0}{0}$, ya que eso sería $\dfrac{casi\ 0}{casi\ 0}$. Aquí lo que ocurre es un real $\dfrac{0}{0}$, cosa que es imposible.\\

Por lo tanto, el límite no existe.\\

Como conclusión, para que el límite exista, este debe ser igual para cualquier valor de $\theta$, incluso antes de evaluar el límite.
\end{enumerate}
\end{solucion}
\item Dada la función 
	$$ f(x) = 
	\begin{cases}
	\dfrac{x^4y^3}{3x^2+2y^2} & si\ (x,y) \neq (0,0) \\
	0 & si\ (x,y) = (0,0)
	\end{cases}
	$$
	determinar si es continua en todo $\R^2$ o no.
\begin{solucion}
Para que la función sea continua debe ocurrir que 
$$\lim\limits_{(x,y) \to (0,0)} f(x,y) = f(0,0) = 0$$

$\lim\limits_{(x,y) \to (0,0)} \dfrac{5x^2y}{x^2+y^2}$\\
			\\
			En este caso, el grado del numerador es 3 y el del denominador es 2 por lo que, siendo el numerador mayor al denominador, es de esperarse que el límite exista (y probablemente sea igual a 0).
			
			Probemos desde distintas direcciones.
$$x = 0 \ra \lim\limits_{(0,y) \to (0,0)} \dfrac{0^4y^3}{3\cdot0^2+2y^2} = 0$$
$$y = 0 \ra \lim\limits_{(x,0) \to (0,0)} \dfrac{x^40^3}{3x^2+2\cdot 0^2} = 0$$
$$y = mx \ra \lim\limits_{(x,mx) \to (0,0)} \dfrac{x^4 (mx)^3}{3x^2+2m^2x^2}
= \lim\limits_{(x,mx) \to (0,0)} \dfrac{m^3x^7}{x^2(3+2m^2)}
= \lim\limits_{(x,mx) \to (0,0)} \dfrac{m^3x^5}{3+2m^2} = 0$$
			Hasta el momento todo indica que el límite existe, sin embargo, para estar seguros es necesario comprobar con coordenadas polares, por lo que haremos el cambio de variables
$$x = \sqrt[]{2}rcos(\theta), \quad y = \sqrt[]{3}rsen(\theta)$$
Notemos que con este cambio,
$$3x^2+2y^2 = 6r^2, \quad (x,y)\ra 0 \Longrightarrow r \ra 0$$
Luego, debemos resolver el límite
$$\lim\limits_{r\ra 0} \dfrac{4r^4cos^4(\theta)3\ \sqrt[]{3}r^3sen^3(\theta)}{6r^2}
			= \lim\limits_{r\ra 0} \dfrac{12\ \sqrt[]{3}}{6}r^5cos^4(\theta)sen^3(\theta) = 0$$
			Como las funciones $sen$ y $cos$ son funciones acotadas, por teorema del sandwich concluimos que el límite existe y es cero.\\
\\
Finalmente, la función es continua.
\end{solucion}
\item Determine el conjunto de puntos en los cuales la función es continua
	$$f(x,y) = 
	\begin{cases}
	\dfrac{x^3y^2}{x^2+2y^2} & (x,y) \neq (0,0)\\
	1 & (x,y) = (0,0)
	\end{cases}
	$$
\begin{solucion}
Notemos que para todo $(x,y) \neq (0,0)$, $f(x,y)$ es continua por ser composición de funciones continuas.\\

Luego, solo nos queda ver que ocurre en $(0,0)$.\\

Para que la función sea continua en $(0,0)$ debe ocurrir que 
$$\lim\limits_{(x,y) \to (0,0)} f(x,y) = f(0,0) = 1$$
			
			Probemos desde distintas direcciones.
$$x = 0 \ra \lim\limits_{(0,y) \to (0,0)} \dfrac{0^3y^2}{0^2+2y^2} = 0$$
$$y = 0 \ra \lim\limits_{(x,0) \to (0,0)} \dfrac{x^30^2}{x^2+2\cdot 0^2} = 0$$
$$y = mx \ra \lim\limits_{(x,mx) \to (0,0)} \dfrac{x^3 (mx)^2}{x^2+2m^2x^2}
= \lim\limits_{(x,mx) \to (0,0)} \dfrac{m^2x^5}{x^2(1+2m^2)}
= \lim\limits_{(x,mx) \to (0,0)} \dfrac{m^2x^3}{1+2m^2} = 0$$
			Hasta el momento todo indica que el límite existe, sin embargo, para estar seguros es necesario comprobar con coordenadas polares, por lo que haremos el cambio de variables
$$x = rcos(\theta), \quad y = \sqrt[]{2}rsen(\theta)$$
Notemos que con este cambio,
$$2x^2+y^2 = 2r^2, \quad (x,y)\ra 0 \Longrightarrow r \ra 0$$
Luego, debemos resolver el límite
$$\lim\limits_{r\ra 0} \dfrac{r^3cos^3(\theta)2r^2sen^2(\theta)}{2r^2}
			= \lim\limits_{r\ra 0} \dfrac{2}{2}r^3cos^3(\theta)sen^2(\theta) = 0$$
			Como las funciones $sen$ y $cos$ son funciones acotadas, por teorema del sandwich concluimos que el límite existe y es cero.\\
\\
Por lo tanto, la función es continua en $\R^2 - (0,0)$.
\end{solucion}
\end{preguntas}
\end{document}