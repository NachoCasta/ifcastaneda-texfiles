\documentclass[12pt]{article}

\usepackage{fullpage}
\usepackage{graphicx}
\usepackage{amssymb}
\usepackage{amsmath}
\usepackage[none]{hyphenat}
\usepackage{parskip}
\usepackage[spanish]{babel}
\usepackage[utf8]{inputenc}
\usepackage{hyperref}
\usepackage{fancyhdr}
\usepackage{tasks}
\usepackage{mdframed}
\usepackage{xcolor}
\usepackage{pgfplots}
\usepackage[makeroom]{cancel}
\usepackage{multicol}
\usepackage[shortlabels]{enumitem}
\usepackage{stackrel}
\usepackage{tkz-tab}
\usepackage{xpatch}
\usepackage{tkz-euclide}
\usetkzobj{all}
\usepackage{tabto}
\xpatchcmd{\tkzTabLine}{$0$}{$\bullet$}{}{}

\setlength{\headheight}{10pt}
\setlength{\headsep}{10pt}
\pagestyle{fancy}
\rhead{\ayudantia \ - \alumno}
\tikzset{t style/.style={style=solid}}

\newcommand*{\mybox}[2]{\colorbox{#1!30}{\parbox{.98\linewidth}{#2}}}

\newenvironment{solucion}
{\begin{mdframed}[backgroundcolor=black!10]
		{\bf Solución:}\\
	}
	{
	\end{mdframed}
}

\newenvironment{alternativas}[1]
{\begin{multicols}{#1}
		\begin{enumerate}[a)]
		}
		{
		\end{enumerate}
	\end{multicols}
}

\newenvironment{preguntas}
{\begin{enumerate}\itemsep12pt
	}
	{
	\end{enumerate}
}

\newcommand{\ayudantia}{{\sc Ayudantía 12}}
\newcommand{\tituloayu}{Cambio de orden de integración y uso de coordenadas polares}
\newcommand{\fecha}{12 de diciembre de 2019}
\newcommand{\sigla}{MAT1620}
\newcommand{\nombre}{Cálculo II}
\newcommand{\profesor}{Wolfgang Rivera}
\newcommand{\ano}{2019}
\newcommand{\semestre}{2}
\newcommand{\mail}{mat1620@ifcastaneda.cl}
\newcommand{\alumno}{Ignacio Castañeda - \mail}

\newcommand{\ev}{\Big|}
\newcommand{\ra}{\rightarrow}
\newcommand{\lra}{\leftrightarrow}
\newcommand{\N}{\mathbb{N}}
\newcommand{\R}{\mathbb{R}}
\newcommand{\Exp}[1]{\mathcal{E}_{#1}}
\newcommand{\List}[1]{\mathcal{L}_{#1}}
\newcommand{\EN}{\Exp{\N}}
\newcommand{\LN}{\List{\N}}
\newcommand{\comment}[1]{}
\newcommand{\lb}{\\~\\}
\newcommand{\eop}{_{\square}}
\newcommand{\hsig}{\hat{\sigma}}
\newcommand{\widesim}[2][1.5]{
	\mathrel{\overset{#2}{\scalebox{#1}[1]{$\sim$}}}
}
\newcommand{\wsim}{\widesim{}}
\newcommand{\lh}{\stackrel{L'H}{=}}

\begin{document}
\thispagestyle{empty}

\begin{minipage}{2cm}
	\includegraphics[width=2cm]{../../../../img/logo.pdf}
	\vspace{0.5cm}
\end{minipage}
\begin{minipage}{\linewidth}
	\begin{tabular}{lrl}
		{\scriptsize\sc Pontificia Universidad Catolica de Chile} & \hspace*{0.7in}Curso: &
		\sigla  - \nombre\\
		{\sc Facultad de Matemáticas}&
		Profesor: & \profesor \\
		{\sc Semestre \ano-\semestre} & Ayudante: & {Ignacio Castañeda}\\
		& {Mail:} & \texttt{\mail}
	\end{tabular}
\end{minipage}

\vspace{-10mm}
\begin{center}
	{\LARGE\bf \ayudantia}\\
	\vspace{0.1cm}
	{\tituloayu}\\
	\vspace{0.1cm}
	\fecha\\
	\vspace{0.4cm}
\end{center}

\begin{preguntas}
\item Dibuje la región de integración de $\displaystyle\int_0^1 \displaystyle\int_x^{2x} dydx$ y luego cambie el orden de integración.
\item Evalúe la integral $\displaystyle\int_1^2 \displaystyle\int_x^{x^2} 12x dydx$ y luego dibuje la región de integración y exprese la integral en el orden $dxdy$. Integre nuevamente.
\item Calcule la integral doble $\displaystyle\iint\limits_R e^{x/y} dA$ donde $R$ es la región en $\R^2$ encerrada por las curvas $y=\sqrt[]{x}$ e $y=\sqrt[3]{x}$.
\item Cambie el orden de integración y calcule cuando sea posible
\begin{tasks}(2)
\task $\displaystyle\int_0^1 \displaystyle\int_0^{\sqrt[]{x}} \dfrac{2xy}{1-y^4} dydx$
\task $\displaystyle\int_0^1 \displaystyle\int_{z^2}^z ze^{-y^2} dydz$
\task $\displaystyle\int_0^1 \displaystyle\int_{arcsen(y)}^{\pi/2} cos(x)\ \sqrt[]{1+cos^2(x)} dxdy$
\task $\displaystyle\int_0^1 \displaystyle\int_{\sqrt[]{x}}^1 \dfrac{x}{\sqrt[]{x^2+y^2}} dydx$
\end{tasks}
\item Utilizando coordenadas polares, calcule:
	$$ \displaystyle\iint\limits_D \dfrac{x^2y^2}{(x^2+y^2)^2}dxdy $$
	siendo $D = \{ (x,y) \in \R^2 : 1 < x^2 + y^2 < 2\}$.
\end{preguntas}
\end{document}