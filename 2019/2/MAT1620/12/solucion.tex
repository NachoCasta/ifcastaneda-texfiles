\documentclass[12pt]{article}

\usepackage{fullpage}
\usepackage{graphicx}
\usepackage{amssymb}
\usepackage{amsmath}
\usepackage[none]{hyphenat}
\usepackage{parskip}
\usepackage[spanish]{babel}
\usepackage[utf8]{inputenc}
\usepackage{hyperref}
\usepackage{fancyhdr}
\usepackage{tasks}
\usepackage{mdframed}
\usepackage{xcolor}
\usepackage{pgfplots}
\usepackage[makeroom]{cancel}
\usepackage{multicol}
\usepackage[shortlabels]{enumitem}
\usepackage{stackrel}
\usepackage{tkz-tab}
\usepackage{xpatch}
\usepackage{tkz-euclide}
\usetkzobj{all}
\usepackage{tabto}
\xpatchcmd{\tkzTabLine}{$0$}{$\bullet$}{}{}

\setlength{\headheight}{10pt}
\setlength{\headsep}{10pt}
\pagestyle{fancy}
\rhead{\ayudantia \ - \alumno}
\tikzset{t style/.style={style=solid}}

\newcommand*{\mybox}[2]{\colorbox{#1!30}{\parbox{.98\linewidth}{#2}}}

\newenvironment{solucion}
{\begin{mdframed}[backgroundcolor=black!10]
		{\bf Solución:}\\
	}
	{
	\end{mdframed}
}

\newenvironment{alternativas}[1]
{\begin{multicols}{#1}
		\begin{enumerate}[a)]
		}
		{
		\end{enumerate}
	\end{multicols}
}

\newenvironment{preguntas}
{\begin{enumerate}\itemsep12pt
	}
	{
	\end{enumerate}
}

\newcommand{\ayudantia}{{\sc Ayudantía 12}}
\newcommand{\tituloayu}{Cambio de orden de integración y uso de coordenadas polares}
\newcommand{\fecha}{12 de diciembre de 2019}
\newcommand{\sigla}{MAT1620}
\newcommand{\nombre}{Cálculo II}
\newcommand{\profesor}{Wolfgang Rivera}
\newcommand{\ano}{2019}
\newcommand{\semestre}{2}
\newcommand{\mail}{mat1620@ifcastaneda.cl}
\newcommand{\alumno}{Ignacio Castañeda - \mail}

\newcommand{\ev}{\Big|}
\newcommand{\ra}{\rightarrow}
\newcommand{\lra}{\leftrightarrow}
\newcommand{\N}{\mathbb{N}}
\newcommand{\R}{\mathbb{R}}
\newcommand{\Exp}[1]{\mathcal{E}_{#1}}
\newcommand{\List}[1]{\mathcal{L}_{#1}}
\newcommand{\EN}{\Exp{\N}}
\newcommand{\LN}{\List{\N}}
\newcommand{\comment}[1]{}
\newcommand{\lb}{\\~\\}
\newcommand{\eop}{_{\square}}
\newcommand{\hsig}{\hat{\sigma}}
\newcommand{\widesim}[2][1.5]{
	\mathrel{\overset{#2}{\scalebox{#1}[1]{$\sim$}}}
}
\newcommand{\wsim}{\widesim{}}
\newcommand{\lh}{\stackrel{L'H}{=}}

\begin{document}
\thispagestyle{empty}

\begin{minipage}{2cm}
	\includegraphics[width=2cm]{../../../../img/logo.pdf}
	\vspace{0.5cm}
\end{minipage}
\begin{minipage}{\linewidth}
	\begin{tabular}{lrl}
		{\scriptsize\sc Pontificia Universidad Catolica de Chile} & \hspace*{0.7in}Curso: &
		\sigla  - \nombre\\
		{\sc Facultad de Matemáticas}&
		Profesor: & \profesor \\
		{\sc Semestre \ano-\semestre} & Ayudante: & {Ignacio Castañeda}\\
		& {Mail:} & \texttt{\mail}
	\end{tabular}
\end{minipage}

\vspace{-10mm}
\begin{center}
	{\LARGE\bf \ayudantia}\\
	\vspace{0.1cm}
	{\tituloayu}\\
	\vspace{0.1cm}
	\fecha\\
	\vspace{0.4cm}
\end{center}

\begin{preguntas}
\item Dibuje la región de integración de $\displaystyle\int_0^1 \displaystyle\int_x^{2x} dydx$ y luego cambie el orden de integración.
\begin{solucion}
\begin{center}
		\begin{tikzpicture}
		\begin{axis}[
		axis lines = left,
		xlabel = $x$,
		ylabel = $y$,
		]
		\addplot [
		domain=0:1,  
		color=red,
		]
		{2*x};
		\addplot [
		domain=0:1, 
		color=red,
		]
		{x};
		\addplot [
		domain=0.5:1, 
		color=blue,
		]
		{1};
		\addplot [
		domain=0:1,  
		color=green,
		]
		coordinates {(1,0) (1,2)};
		
		\end{axis}
		\end{tikzpicture}
		\end{center}
		Como queremos cambiar el orden de integración, despejamos estas funciones en función de la otra variable, es decir:
		$$ y = 2x \ra x = \dfrac{y}{2} \quad; \quad y=x \ra x=y$$
		Luego, cambiamos el orden de integración, teniendo cuidado con los intervalos donde estamos abajo y arriba de la linea azul.
		$$\displaystyle\int_0^1 \displaystyle\int_x^{2x} dydx = 
		\displaystyle\int_0^1 \displaystyle\int_{y/2}^{y} dxdy + 
		\displaystyle\int_1^2 \displaystyle\int_{y/2}^{1} dxdy$$
\end{solucion}
\item Evalúe la integral $\displaystyle\int_1^2 \displaystyle\int_x^{x^2} 12x dydx$ y luego dibuje la región de integración y exprese la integral en el orden $dxdy$. Integre nuevamente.
\begin{solucion}
$$\displaystyle\int_1^2 \displaystyle\int_x^{x^2} 12x dydx =
		\displaystyle\int_1^2 12x(x^2-x)dx = \displaystyle\int_1^2 12(x^3-x^2)dx = 17$$
		\begin{center}
			\begin{tikzpicture}
			\begin{axis}[
			axis lines = left,
			xlabel = $x$,
			ylabel = $y$,
			]
			\addplot [
			domain=0:2,  
			color=red,
			]
			{x};
			\addplot [
			domain=0:2, 
			color=red,
			]
			{x^2};
			\addplot [
			domain=0:2, 
			color=blue,
			]
			{2};
			\addplot [
			domain=0:2,  
			color=green,
			]
			coordinates {(2,0) (2,4)};
			
			\end{axis}
			\end{tikzpicture}
		\end{center}
		Despejamos estas funciones en función de la otra variable
		$$ y = x^2 \ra x = \sqrt[]{y} \quad; \quad y=x \ra x=y$$
		y cambiamos el orden de integración
		$$\displaystyle\int_1^2 \displaystyle\int_x^{x^2} 12x dydx = 
		\displaystyle\int_1^2 \displaystyle\int_{\sqrt[]{y}}^{y} 12x dxdy + 
		\displaystyle\int_2^4 \displaystyle\int_{\sqrt[]{y}}^{2} 12x dxdy$$
		Resolvemos ahora esta integral y obtenemos que
		$$\displaystyle\int_1^2 \displaystyle\int_{\sqrt[]{y}}^{y} 12x dxdy + 
		\displaystyle\int_2^4 \displaystyle\int_{\sqrt[]{y}}^{2} 12x dxdy = 17$$
\end{solucion}
\item Calcule la integral doble $\displaystyle\iint\limits_R e^{x/y} dA$ donde $R$ es la región en $\R^2$ encerrada por las curvas $y=\sqrt[]{x}$ e $y=\sqrt[3]{x}$.
\begin{solucion}
Para tener más claro qué es lo que estamos integrando, vamos a graficar ambas curvas:
		\begin{center}
			\begin{tikzpicture}
			\begin{axis}[
			axis lines = left,
			xlabel = $x$,
			ylabel = $y$,
			]
			\addplot [
			domain=0:1,  
			color=red,
			]
			{x^(1/3)};
			\addplot [
			domain=0:1, 
			color=blue,
			]
			{x^(1/2)};
			\end{axis}
			\end{tikzpicture}
		\end{center}
		Vemos claramente que se intersectan en $x=0$ y $x=1$. Además, podemos observar que $y=\sqrt[]{x}$ es menor a $y=\sqrt[3]{x}$ en ese intervalo. Por lo tanto, nos queda la integral siguiente:
		$$\displaystyle\iint\limits_R e^{x/y} dA = 
		\displaystyle\int_0^1 \displaystyle\int_{\sqrt[]{x}}^{\sqrt[3]{x}} e^{x/y} dydx$$
		Sin embargo, esta integral no es sencilla de resolver, ya que para eso tendríamos que integrar $e^{1/y}$. Dado esto, vamos a cambiar el orden de integración, utilizando el grafico de arriba. Para esto, despejamos $x$ en ambas funciones, obteniendo
		$$ y = \sqrt[]{x} \ra x = y^2 \quad; \quad y = \sqrt[3]{x} \ra x=y^3$$
		Luego, podemos escribir la integral como 
		$$\displaystyle\int_0^1 \displaystyle\int_{\sqrt[]{x}}^{\sqrt[3]{x}} e^{x/y} dydx = 
		\displaystyle\int_0^1 \displaystyle\int_{y^3}^{y^2} e^{x/y} dxdy$$
		Por último, resolvemos esta integral
		$$= \displaystyle\int_0^1 y e^{x/y} \ev_{y^3}^{y^2} dy$$
		$$= \displaystyle\int_0^1 y\left(e^{y^2/y} - e^{y^3/y}\right) dy$$
		$$= \displaystyle\int_0^1 ye^y - ye^{y^2} dy$$
		Para el primer termino se aplica integración por partes $(u=y, dv=e^y)$ y para el segundo termino se utiliza la sustitución $u=y^2$, obteniendo asi:
		$$\displaystyle\iint\limits_R e^{x/y} dA = \dfrac{1}{2}(e+1)$$
\end{solucion}
\item Cambie el orden de integración y calcule cuando sea posible
\begin{tasks}(2)
\task $\displaystyle\int_0^1 \displaystyle\int_0^{\sqrt[]{x}} \dfrac{2xy}{1-y^4} dydx$
\task $\displaystyle\int_0^1 \displaystyle\int_{z^2}^z ze^{-y^2} dydz$
\task $\displaystyle\int_0^1 \displaystyle\int_{arcsen(y)}^{\pi/2} cos(x)\ \sqrt[]{1+cos^2(x)} dxdy$
\task $\displaystyle\int_0^1 \displaystyle\int_{\sqrt[]{x}}^1 \dfrac{x}{\sqrt[]{x^2+y^2}} dydx$
\end{tasks}
\begin{solucion}

\begin{enumerate}[a)]
\item $\displaystyle\int_0^1 \displaystyle\int_0^{\sqrt[]{x}} \dfrac{2xy}{1-y^4} dydx$\\
			\\
			En primer lugar, vamos a graficar nuestra area de integración:
			\begin{center}
				\begin{tikzpicture}
				\begin{axis}[
				axis lines = left,
				xlabel = $x$,
				ylabel = $y$,
				]
				\addplot [
				domain=0:1,  
				color=red,
				]
				{x^(1/2)};
				\addplot [
				domain=0:1, 
				color=red,
				]
				{0};
				\addplot [
				domain=0:1, 
				color=green,
				]
				coordinates {(1,0) (1,1)};
				\end{axis}
				\end{tikzpicture}
			\end{center}
			Tenemos que $y=\sqrt[]{x} \ra x = y^2$ y podemos ver que la coordenada y se mueve entre $0$ y $1$, por lo tanto:
			$$\displaystyle\int_0^1 \displaystyle\int_0^{\sqrt[]{x}} \dfrac{2xy}{1-y^4} dydx
			= \displaystyle\int_0^1 \displaystyle\int_{y^2}^1 \dfrac{2xy}{1-y^4} dxdy$$
			Ahora, la resolvemos:
			$$= \displaystyle\int_0^1 \displaystyle\int_{y^2}^1 \dfrac{2xy}{1-y^4} dxdy$$
			$$= \displaystyle\int_0^1 \dfrac{y}{1-y^4} \left(\displaystyle\int_{y^2}^1 2x dx \right) dy$$
			$$= \displaystyle\int_0^1 \dfrac{y}{1-y^4} x^2 \ev_{y^2}^1 dy$$
			$$= \displaystyle\int_0^1 \dfrac{y}{1-y^4} (1-y^4) dy$$
			$$= \displaystyle\int_0^1 y dy$$
			$$= \dfrac{1}{2}$$
\item $\displaystyle\int_0^1 \displaystyle\int_{z^2}^z ze^{-y^2} dydz$\\
			\\
			Graficamos la región:
			\begin{center}
				\begin{tikzpicture}
				\begin{axis}[
				axis lines = left,
				xlabel = $z$,
				ylabel = $y$,
				]
				\addplot [
				domain=0:1,  
				color=red,
				]
				{x^2};
				\addplot [
				domain=0:1, 
				color=red,
				]
				{x};
				\end{axis}
				\end{tikzpicture}
			\end{center}
			Tenemos que $ y = z^2 \ra z = \sqrt[]{y}$ y $y=z \ra z=y$. Viendo el gráfico podemos ver que al hacer el cambio del orden de integración, seguiremos teniendo una sola integral, con lo que
			$$\displaystyle\int_0^1 \displaystyle\int_{z^2}^z ze^{-y^2} dydz = 
			\displaystyle\int_0^1 \displaystyle\int_{y}^{\sqrt[]{y}} ze^{-y^2} dzdy$$
			Evaluando esta integral, tenemos que
			$$= \displaystyle\int_0^1 \displaystyle\int_{y}^{\sqrt[]{y}} ze^{-y^2} dzdy$$
			$$= \displaystyle\int_0^1 e^{-y^2} \left(\displaystyle\int_{y}^{\sqrt[]{y}} z dz \right) dy$$
			$$= \displaystyle\int_0^1 e^{-y^2} (y-y^2) dy$$
			Podemos seguir desarrollando, pero eventualmente tendremos que calcular $\displaystyle\int_0^1 e^{-y^2}dy$, que no tiene primitiva (no se puede calcular).
\item $\displaystyle\int_0^1 \displaystyle\int_{arcsen(y)}^{\pi/2} cos(x)\ \sqrt[]{1+cos^2(x)} dxdy$\\
			\\
			Partimos graficando la región. Notemos que para graficar con los ejes que estamos acostumbrados, debemos graficar la función $y=sen(x)$, la cual es equivalente a $x = arcsen(y)$
			\begin{center}
				\begin{tikzpicture}
				\begin{axis}[
				axis lines = left,
				xlabel = $x$,
				ylabel = $y$,
				xtick={
					0.78539, 1.5707
				},
				xticklabels={
					$\frac{\pi}{4}$, $\frac{\pi}{2}$
				}
				]
				\addplot [
				domain=0:1.5707,  
				color=blue,
				]
				{sin(deg(x))};
				\addplot [
				domain=0:1.5707, 
				color=blue,
				]
				{0};
				\addplot [
				domain=0:1.5707, 
				color=blue,
				]
				coordinates {(1.5707,0) (1.5707,1)};
				\end{axis}
				\end{tikzpicture}
			\end{center}		
			Podemos ver que $x$ va entre $0$ y $\dfrac{\pi}{2}$. Además, $y$ va entre $0$ y $sen(x)$. Dicho esto, tenemos que
			$$\displaystyle\int_0^1 \displaystyle\int_{arcsen(y)}^{\pi/2} cos(x)\ \sqrt[]{1+cos^2(x)} dxdy=
			\displaystyle\int_0^{\pi/2} \displaystyle\int_0^{sen(x)} cos(x)\ \sqrt[]{1+cos^2(x)} dydx$$
			Resolviendo:
			$$=\displaystyle\int_0^{\pi/2} \displaystyle\int_0^{sen(x)} cos(x)\ \sqrt[]{1+cos^2(x)} dydx$$
			$$=\displaystyle\int_0^{\pi/2} cos(x)\ \sqrt[]{1+cos^2(x)} \left( \displaystyle\int_0^{sen(x)} dy \right) dx$$
			$$=\displaystyle\int_0^{\pi/2} sen(x) cos(x)\ \sqrt[]{1+cos^2(x)} dx$$
			Hacemos el cambio de variables
			$$ u = cos^2(x) \ra du = -2cos(x)sen(x)dx $$
			Con lo que nos queda:
			$$=\dfrac{1}{2}\displaystyle\int_0^1 \sqrt[]{1+u} du$$
			$$=\dfrac{1}{2} \cdot \dfrac{2}{3}(1+u)^{3/2} \ev_0^1$$
			$$=\dfrac{1}{3} (2^{3/2}-1)$$
\item $\displaystyle\int_0^1 \displaystyle\int_{\sqrt[]{x}}^1 \dfrac{x}{\sqrt[]{x^2+y^2}} dydx$\\
			\\
			Graficamos:
			\begin{center}
				\begin{tikzpicture}
				\begin{axis}[
				axis lines = left,
				xlabel = $x$,
				ylabel = $y$,
				]
				\addplot [
				domain=0:1,  
				color=green,
				]
				{x^(1/2)};
				\addplot [
				domain=0:1, 
				color=green,
				]
				{1};
				\addplot [
				domain=0:1, 
				color=green,
				]
				coordinates {(0,0) (0,1)};
				\end{axis}
				\end{tikzpicture}
			\end{center}
			Tenemos que $y=\sqrt[]{x} \ra x = y^2$. De esta forma,
			$$\displaystyle\int_0^1 \displaystyle\int_{\sqrt[]{x}}^1 \dfrac{x}{\sqrt[]{x^2+y^2}} dydx =
			\displaystyle\int_0^1 \displaystyle\int_0^{y^2} \dfrac{x}{\sqrt[]{x^2+y^2}} dxdy$$
			Resolviendo esta integral, tenemos
			$$= \displaystyle\int_0^1 \displaystyle\int_0^{y^2} \dfrac{x}{\sqrt[]{x^2+y^2}} dxdy$$
			$$= \displaystyle\int_0^1 \left( \displaystyle\int_0^{y^2} \dfrac{1}{2} \dfrac{2x}{\sqrt[]{x^2+y^2}} dx \right) dy$$
			Haciendo el cambio de variable $u=x^2 \ra du = 2xdx$ en la integral de más adentro
			$$= \displaystyle\int_0^1 \left( \displaystyle\int_0^{y^4} \dfrac{du}{2\ \sqrt[]{u+y^2}} \right) dy$$
			$$= \displaystyle\int_0^1\sqrt[]{u+y^2} \ev_0^{y^4} dy$$
			$$= \displaystyle\int_0^1\sqrt[]{y^4+y^2} -|y| dy$$
			Como el intervalo de integración es positivo, sabemos que $|y| = y$
			$$= \displaystyle\int_0^1 y\ \sqrt[]{y^2+1} -y dy$$
			Haciendo $u=y^2 \ra du = 2ydy$
			$$= \dfrac{1}{2}\displaystyle\int_0^1 y\ \sqrt[]{u+1} du - \dfrac{1}{2}$$
			$$= \dfrac{1}{2} \cdot \dfrac{2}{3} (2^{3/2}-1) - \dfrac{1}{2}$$
			$$= \dfrac{2\ \sqrt[]{2}}{3} - \dfrac{5}{6}$$
\end{enumerate}
\end{solucion}
\item Utilizando coordenadas polares, calcule:
	$$ \displaystyle\iint\limits_D \dfrac{x^2y^2}{(x^2+y^2)^2}dxdy $$
	siendo $D = \{ (x,y) \in \R^2 : 1 < x^2 + y^2 < 2\}$.
\begin{solucion}
Utilizando coordenadas polares, tenemos que
		$$x = rcos(\theta) \quad , \quad y = rsen(\theta)$$
		$$dxdy \ra rdrd\theta$$
		Es evidente entonces que
		$$ 1 < x^2 + y^2 < 2 \ra 1 < r^2 < 2 \ra 1 < r < \sqrt[]{2}$$
		Como no hay restricciones que involucren a $\theta$, tenemos que $\theta \in [0, 2\pi]$
		Luego, 
		$$\displaystyle\iint\limits_D \dfrac{x^2y^2}{(x^2+y^2)^2}dxdy =
		 \displaystyle\int_0^{2\pi} \displaystyle\int_1^{\sqrt[]{2}} \dfrac{r^2 cos^2(\theta) r^2 sen^2(\theta)}{r^4}rdrd\theta$$
		Simplificando nos queda
		$$ =\displaystyle\int_0^{2\pi} \displaystyle\int_1^{\sqrt[]{2}} r (cos(\theta) sen(\theta))^2drd\theta$$
		$$ =\displaystyle\int_0^{2\pi} \displaystyle\int_1^{\sqrt[]{2}} r \left(\dfrac{1}{2} sen(2\theta) \right)^2drd\theta$$
		Notemos que la funcion es separable, por lo que nos queda
		$$ =\dfrac{1}{4} \left(\displaystyle\int_0^{2\pi} sen^2(2\theta)d\theta \right) \left(\displaystyle\int_1^{\sqrt[]{2}} rdr \right)$$
		$$ =\dfrac{1}{4} \left(\displaystyle\int_0^{2\pi} \dfrac{1-cos(4\theta)}{2} d\theta \right) \dfrac{1}{2}$$
		$$ = \dfrac{2\pi}{16}$$
		$$ = \dfrac{\pi}{8}$$
\end{solucion}
\end{preguntas}
\end{document}