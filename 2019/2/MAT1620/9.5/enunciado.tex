\documentclass[12pt]{article}

\usepackage{fullpage}
\usepackage{graphicx}
\usepackage{amssymb}
\usepackage{amsmath}
\usepackage[none]{hyphenat}
\usepackage{parskip}
\usepackage[spanish]{babel}
\usepackage[utf8]{inputenc}
\usepackage{hyperref}
\usepackage{fancyhdr}
\usepackage{tasks}
\usepackage{mdframed}
\usepackage{xcolor}
\usepackage{pgfplots}
\usepackage[makeroom]{cancel}
\usepackage{multicol}
\usepackage[shortlabels]{enumitem}
\usepackage{stackrel}
\usepackage{tkz-tab}
\usepackage{xpatch}
\usepackage{tkz-euclide}
\usetkzobj{all}
\usepackage{tabto}
\xpatchcmd{\tkzTabLine}{$0$}{$\bullet$}{}{}

\setlength{\headheight}{10pt}
\setlength{\headsep}{10pt}
\pagestyle{fancy}
\rhead{\ayudantia \ - \alumno}
\tikzset{t style/.style={style=solid}}

\newcommand*{\mybox}[2]{\colorbox{#1!30}{\parbox{.98\linewidth}{#2}}}

\newenvironment{solucion}
{\begin{mdframed}[backgroundcolor=black!10]
		{\bf Solución:}\\
	}
	{
	\end{mdframed}
}

\newenvironment{alternativas}[1]
{\begin{multicols}{#1}
		\begin{enumerate}[a)]
		}
		{
		\end{enumerate}
	\end{multicols}
}

\newenvironment{preguntas}
{\begin{enumerate}\itemsep12pt
	}
	{
	\end{enumerate}
}

\newcommand{\ayudantia}{{\sc Ayudantía 9.5}}
\newcommand{\tituloayu}{Compilado I2}
\newcommand{\fecha}{9 de noviembre de 2019}
\newcommand{\sigla}{MAT1620}
\newcommand{\nombre}{Cálculo II}
\newcommand{\profesor}{Wolfgang Rivera}
\newcommand{\ano}{2019}
\newcommand{\semestre}{2}
\newcommand{\mail}{mat1620@ifcastaneda.cl}
\newcommand{\alumno}{Ignacio Castañeda - \mail}

\newcommand{\ev}{\Big|}
\newcommand{\ra}{\rightarrow}
\newcommand{\lra}{\leftrightarrow}
\newcommand{\N}{\mathbb{N}}
\newcommand{\R}{\mathbb{R}}
\newcommand{\Exp}[1]{\mathcal{E}_{#1}}
\newcommand{\List}[1]{\mathcal{L}_{#1}}
\newcommand{\EN}{\Exp{\N}}
\newcommand{\LN}{\List{\N}}
\newcommand{\comment}[1]{}
\newcommand{\lb}{\\~\\}
\newcommand{\eop}{_{\square}}
\newcommand{\hsig}{\hat{\sigma}}
\newcommand{\widesim}[2][1.5]{
	\mathrel{\overset{#2}{\scalebox{#1}[1]{$\sim$}}}
}
\newcommand{\wsim}{\widesim{}}
\newcommand{\lh}{\stackrel{L'H}{=}}

\begin{document}
\thispagestyle{empty}

\begin{minipage}{2cm}
	\includegraphics[width=2cm]{../../../../img/logo.pdf}
	\vspace{0.5cm}
\end{minipage}
\begin{minipage}{\linewidth}
	\begin{tabular}{lrl}
		{\scriptsize\sc Pontificia Universidad Catolica de Chile} & \hspace*{0.7in}Curso: &
		\sigla  - \nombre\\
		{\sc Facultad de Matemáticas}&
		Profesor: & \profesor \\
		{\sc Semestre \ano-\semestre} & Ayudante: & {Ignacio Castañeda}\\
		& {Mail:} & \texttt{\mail}
	\end{tabular}
\end{minipage}

\vspace{-10mm}
\begin{center}
	{\LARGE\bf \ayudantia}\\
	\vspace{0.1cm}
	{\tituloayu}\\
	\vspace{0.1cm}
	\fecha\\
	\vspace{0.4cm}
\end{center}

\begin{preguntas}
\item Determina si la siguiente serie converge o diverge
$$\sum\limits_{n=1}^{\infty}\dfrac{n!}{n^n}$$
\item Determine si las siguientes series convergen condicionalmente, absolutamente o divergen.
\begin{tasks}(3)
\task $\sum\limits_{n=1}^{\infty}\dfrac{(-1)^{n+1}}{\sqrt[]{n}}$
\task $\sum\limits_{n=2}^{\infty}\dfrac{(-1)^{n-1}(2n-1)}{(\sqrt[]{2})^n}$
\task $\sum\limits_{n=2}^{\infty}\dfrac{(-1)^{n-1}(n+1)}{n}$
\end{tasks}
\item Determine el radio y los intervalos de convergencia de las siguientes series
\begin{tasks}(3)
\task $\sum\limits_{n=1}^{\infty}\dfrac{(-1)^{n-1}(x+3)^n}{3n}$
\task $\sum\limits_{n=2}^{\infty}\dfrac{2(x-4)^n}{n}$
\task $\sum\limits_{n=2}^{\infty}\dfrac{(x-2)^n}{2^{n+1}}$
\end{tasks}
\item Escribir las siguientes funciones como una serie de potencias
\begin{tasks}(2)
\task $f(x) = \dfrac{1}{x^2-4x+20}$
\task $f(x) = ln(1+x)$
\end{tasks}
\item Expresar las siguientes series de potencias como una función
\begin{tasks}(2)
\task $\sum\limits_{n=2}^\infty \dfrac{x^n}{2^{n-1}}$
\task $\sum\limits_{n=1}^\infty \dfrac{(-1)^n (3x+1)^{n-1}}{5^n}$
\task $\sum\limits_{n=0}^\infty \dfrac{x^{2n}}{4^n}$
\task $\sum\limits_{n=1}^\infty \dfrac{(2x-3)^n}{n2^n}$
\end{tasks}
\item Determinar, utilizando series de potencias, el valor de
	$$\sum\limits_{n=1}^\infty \frac{1}{3^n}$$
\item Determinar el valor de la siguiente serie
	$$ \sum\limits_{n=2}^{\infty}\dfrac{n^2+n}{3^{n-1}} $$
\item Para cada función, encontrar la serie de Maclaurin que la representa
\begin{tasks}(2)
\task $f(x) = cos(x)$
\task $f(x) = \dfrac{1}{1-x}$
\end{tasks}
\item Encontrar una aproximación de $\sqrt[]{101}$ con un error máximo de $10^{-3}$
\item Determina el dominio de las siguientes funciones
\begin{tasks}(2)
\task $f(x, y) = \sqrt[]{x+y} + \ln(x^2+y^2)$
\task $f(x,y) = \dfrac{x+y}{x^2-y^2}$
\task $f(x,y,z) = ln(z+y) - \dfrac{1}{x^2 +z^2}$
\task $f(x,y,z) = \sqrt[]{\ln(x+y+z)}$
\end{tasks}
\item Determina el recorrido de las siguientes funciones
\begin{tasks}(2)
\task $f(x,y) = x^2 + y^2 + 3$
\task $f(x,y,z) = \ln(\sqrt[]{x^2+y^2+4}) + z^2$
\end{tasks}
\item Grafique las curvas de nivel de la función 
	$$ f(x,y) = \sqrt[]{9-x^2-y^2}$$
	para $k=0,1,2,3$
\item Determinar si los siguientes limites existen o no. En caso de que existan, calcule su valor
\begin{tasks}(2)
\task $\lim\limits_{(x,y) \to (0,0)} \dfrac{x^2}{x^2+y^2}$
\task $\lim\limits_{(x,y) \to (0,0)} \dfrac{5x^2y}{x^2+y^2}$
\task $\lim\limits_{(x,y) \to (0,0)} \dfrac{xy}{x^2+y^2}$
\task $\lim\limits_{(x,y) \to (0,0)} \dfrac{x^2ye^y}{x^4+y^2}$
\task $\lim\limits_{(x,y) \to (0,0)} \dfrac{sen(x^2+y^2)}{x^2+y^2}$
\task $\lim\limits_{(x,y) \to (0,0)} \dfrac{sen(xy)}{xy}$
\end{tasks}
\item Dada la función 
	$$ f(x) = 
	\begin{cases}
	\dfrac{x^4y^3}{3x^2+2y^2} & si\ (x,y) \neq (0,0) \\
	0 & si\ (x,y) = (0,0)
	\end{cases}
	$$
	determinar si es continua en todo $\R^2$ o no.
\item Determine el conjunto de puntos en los cuales la función es continua
	$$f(x,y) = 
	\begin{cases}
	\dfrac{x^3y^2}{x^2+2y^2} & (x,y) \neq (0,0)\\
	1 & (x,y) = (0,0)
	\end{cases}
	$$
\item Determine la derivada por definición en $x$ de la función
	$$f(x,y) = 2xy + x^2y + x + y$$
\item Para las siguientes funciones, calcular $f_x$ y $f_y$
\begin{enumerate}[a)]
\item $f(x,y) = \dfrac{xy}{x-y}$
\item $f(x,y) = (x^2+y^2)sen\left(\dfrac{1}{x^2+y^2}\right)$
\end{enumerate}
\item Una función armónica es aquella que cumple con $f_{xx} + f_{yy} = 0$. Determina si la siguiente funcion es armónica
	$$f(x,y) = xy + 3x^2 -y^3$$
\item Busque $\dfrac{\delta z}{\delta t}$ o $\dfrac{\delta w}{\delta t}$, según corresponda.
\begin{enumerate}[a)]
\item $z = x^2+y^2+xy$\tab$x=sen(t), y=e^t$
\item $w=xe^{y/z}$\tab$x=t^2, y=1-t, z=1+2t$
\item $w=ln(\sqrt[]{x^2+y^2+z^2})$\tab$x=sen(t), y=cos(t), z=tan(t)$
\end{enumerate}
\item Sea $f$ una función con segundas derivadas parciales continuas en todo $R^2$. El cambio de variables $x=uv$, $y=\dfrac{u^2-v^2}{2}$ transforma la función $f(x,y)$ en la función $g(u,v)$.
\begin{enumerate}[a)]
\item Calcule $\dfrac{\delta g}{\delta u}, \dfrac{\delta g}{\delta v}$ en terminos de las derivadas parciales de $f$.
\item Si $f_{xx}(x,y) + f_{yy}(x,y) = 2$ para todo $(x,y) \in R^2$, determine las constantes $a,b \in \R$ tales que
		$$a\dfrac{\delta^2g}{\delta u^2} - b\dfrac{\delta^2 g}{\delta v^2} = u^2 + v^2$$
\end{enumerate}
\end{preguntas}
\end{document}