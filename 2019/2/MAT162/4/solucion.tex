\documentclass[12pt]{article}

\usepackage{fullpage}
\usepackage{graphicx}
\usepackage{amssymb}
\usepackage{amsmath}
\usepackage[none]{hyphenat}
\usepackage{parskip}
\usepackage[spanish]{babel}
\usepackage[utf8]{inputenc}
\usepackage{hyperref}
\usepackage{fancyhdr}
\usepackage{tasks}
\usepackage{mdframed}
\usepackage{xcolor}
\usepackage{pgfplots}
\usepackage[makeroom]{cancel}
\usepackage{multicol}
\usepackage[shortlabels]{enumitem}
\usepackage{stackrel}
\usepackage{tkz-tab}
\usepackage{xpatch}
\usepackage{tkz-euclide}
\usetkzobj{all}
\xpatchcmd{\tkzTabLine}{$0$}{$\bullet$}{}{}

\setlength{\headheight}{10pt}
\setlength{\headsep}{10pt}
\pagestyle{fancy}
\rhead{\ayudantia \ - \alumno}
\tikzset{t style/.style={style=solid}}

\newcommand*{\mybox}[2]{\colorbox{#1!30}{\parbox{.98\linewidth}{#2}}}

\newenvironment{solucion}
{\begin{mdframed}[backgroundcolor=black!10]
		{\bf Solución:}\\
	}
	{
	\end{mdframed}
}

\newenvironment{alternativas}[1]
{\begin{multicols}{#1}
		\begin{enumerate}[a)]
		}
		{
		\end{enumerate}
	\end{multicols}
}

\newenvironment{preguntas}
{\begin{enumerate}\itemsep12pt
	}
	{
	\end{enumerate}
}

\newcommand{\ayudantia}{{\sc Ayudantía 4}}
\newcommand{\tituloayu}{Series II}
\newcommand{\fecha}{3 de septiembre de 2019}
\newcommand{\sigla}{MAT210E}
\newcommand{\nombre}{Cálculo I}
\newcommand{\profesor}{Claudio Fernandez}
\newcommand{\ano}{2016}
\newcommand{\semestre}{1}
\newcommand{\mail}{mat210e@ifcastaneda.cl}
\newcommand{\alumno}{Ignacio Castañeda - \mail}

\newcommand{\ev}{\Big|}
\newcommand{\ra}{\rightarrow}
\newcommand{\lra}{\leftrightarrow}
\newcommand{\N}{\mathbb{N}}
\newcommand{\R}{\mathbb{R}}
\newcommand{\Exp}[1]{\mathcal{E}_{#1}}
\newcommand{\List}[1]{\mathcal{L}_{#1}}
\newcommand{\EN}{\Exp{\N}}
\newcommand{\LN}{\List{\N}}
\newcommand{\comment}[1]{}
\newcommand{\lb}{\\~\\}
\newcommand{\eop}{_{\square}}
\newcommand{\hsig}{\hat{\sigma}}
\newcommand{\widesim}[2][1.5]{
	\mathrel{\overset{#2}{\scalebox{#1}[1]{$\sim$}}}
}
\newcommand{\wsim}{\widesim{}}
\newcommand{\lh}{\stackrel{L'H}{=}}

\begin{document}
\thispagestyle{empty}

\begin{minipage}{2cm}
	\includegraphics[width=2cm]{../../../../img/logo.pdf}
	\vspace{0.5cm}
\end{minipage}
\begin{minipage}{\linewidth}
	\begin{tabular}{lrl}
		{\scriptsize\sc Pontificia Universidad Catolica de Chile} & \hspace*{0.7in}Curso: &
		\sigla  - \nombre\\
		{\sc Facultad de Matemáticas}&
		Profesor: & \profesor \\
		{\sc Semestre \ano-\semestre} & Ayudante: & {Ignacio Castañeda}\\
		& {Mail:} & \texttt{\mail}
	\end{tabular}
\end{minipage}

\vspace{-10mm}
\begin{center}
	{\LARGE\bf \ayudantia}\\
	\vspace{0.1cm}
	{\tituloayu}\\
	\vspace{0.1cm}
	\fecha\\
	\vspace{0.4cm}
\end{center}

\begin{preguntas}
\item Determine si las siguientes series convergen o divergen.
\begin{tasks}(3)
\task $\sum\limits_{n=1}^{\infty}\dfrac{n}{n^4+1}$
\task $\sum\limits_{n=2}^{\infty}\dfrac{n}{(n+1)^2ln(n)}$
\task $\sum\limits_{n=1}^{\infty}\dfrac{n!}{n^n}$
\end{tasks}
\begin{solucion}

\begin{enumerate}[a)]
\item $\sum\limits_{n=1}^{\infty}\dfrac{n}{n^4+1}$\\
			\\
			En primer lugar, veamos cual es el límite de la sucesión, fijandonos bien en el término que se genera antes de evaluar
			$$\lim\limits_{n \ra \infty} a_n
			= \lim\limits_{n \ra \infty} \dfrac{n}{n^4+1}
			= \lim\limits_{n \ra \infty} \dfrac{n}{n^4\left(1+\dfrac{1}{n^4}\right)}
			= \lim\limits_{n \ra \infty} \dfrac{1}{n^3} = 0$$
			Tomemos ese último termino y usemoslo como $b_n$. Sea $b_n = \dfrac{1}{n^3}$,
			$$\lim\limits_{n \ra \infty} \dfrac{a_n}{b_n}
			= \lim\limits_{n \ra \infty} \dfrac{\dfrac{n}{n^4+1}}{\dfrac{1}{n^3}}
			= \lim\limits_{n \ra \infty} \dfrac{\dfrac{1}{n^3}}{\dfrac{1}{n^3}} = 1 \neq 0$$
			Por lo tanto, $\sum\limits^{\infty} a_n$ se comporta igual que $\sum\limits^{\infty} b_n$.\\
			\\
			Como $\sum\limits^{\infty} \dfrac{1}{n^3}$ es una serie-$p$ con $p=3>1$, la serie converge. Luego, por criterio de comparación al límite, la serie $\lim\limits_{n \ra \infty} \dfrac{n}{n^4+1}$ también converge.
\item $\sum\limits_{n=2}^{\infty}\dfrac{n}{(n+1)^2ln(n)}$\\
			\\
			Igual que antes, veamos el límite de la sucesión,
			$$\lim\limits_{n \ra \infty} a_n
			= \lim\limits_{n \ra \infty} \dfrac{n}{(n+1)^2ln(n)}
			= \lim\limits_{n \ra \infty} \dfrac{n}{\left(n\left(1+\dfrac{1}{n}\right)\right)^2ln(n)}
			= \lim\limits_{n \ra \infty} \dfrac{n}{n^2ln(n)}$$
			$$= \lim\limits_{n \ra \infty} \dfrac{1}{nln(n)} = 0$$
			Sea $b_n = \dfrac{1}{nln(n)}$,
			$$\lim\limits_{n \ra \infty} \dfrac{a_n}{b_n}
			=\lim\limits_{n \ra \infty} \dfrac{\dfrac{n}{(n+1)^2ln(n)}}{\dfrac{1}{nln(n)}}
			=\lim\limits_{n \ra \infty} \dfrac{\dfrac{1}{nln(n)}}{\dfrac{1}{nln(n)}} = 1 \neq 0$$
			Por lo tanto, $\sum\limits^{\infty} a_n$ se comporta igual que $\sum\limits^{\infty} b_n$.\\
			\\
			Veamos ahora que pasa con $\sum\limits_{n=2}^{\infty} \dfrac{1}{nln(n)}$. Usando el criterio de la integral, tenemos que la serie se comportara igual a $\displaystyle\int_2^{\infty} \dfrac{1}{xln(x)}$. \\
			\\
			Calculemos entonces esta integral impropia,
			$$\displaystyle\int_2^{\infty} \dfrac{1}{xln(x)} =  ln(ln(x)) \ev_2^{\infty} = \infty$$
			Luego, $\displaystyle\int_2^{\infty} \dfrac{1}{xln(x)}$ diverge, por lo que por criterio de la integral, $\sum\limits_{n=2}^{\infty} \dfrac{1}{nln(n)}$ tambiém diverge. Finalmente, por el criterio de comparación al limite, la serie $\sum\limits_{n=2}^{\infty}\dfrac{n}{(n+1)^2ln(n)}$ es divergente.
\item $\sum\limits_{n=1}^{\infty}\dfrac{n!}{n^n}$\\
			\\
			Recordemos que en el infinito,
			$$n^n > n! > a^n > n > ln(n)$$
			El límite de la sucesión es
			$$\lim\limits_{n\ra\infty}\dfrac{n!}{n^n} = 0$$
			Usando el criterio de la razón,
			$$\lim\limits_{n \ra \infty} \dfrac{a_{n+1}}{a_n}
			= \lim\limits_{n \ra \infty} \dfrac{\dfrac{(n+1)!}{(n+1)^{n+1}}}{\dfrac{n!}{n^n}}
			= \lim\limits_{n \ra \infty} \dfrac{(n+1)!}{n!}\dfrac{n^n}{(n+1)^{n+1}}$$
			$$= \lim\limits_{n \ra \infty} \dfrac{(n+1)}{1}\dfrac{n^n}{(n+1)^{n+1}}
			= \lim\limits_{n \ra \infty} \dfrac{n^n}{(n+1)^{n}}
			= \lim\limits_{n \ra \infty} \left(\dfrac{n}{n+1}\right)^n$$
			$$= \lim\limits_{n \ra \infty} \dfrac{1}{\left(\dfrac{n+1}{n}\right)^n}
			= \dfrac{1}{e} < 1$$
			Por criterio de la razón, la serie es convergente.
\end{enumerate}
\end{solucion}
\item Determine si las siguientes series convergen condicionalmente, absolutamente o divergen.
\begin{tasks}(3)
\task $\sum\limits_{n=1}^{\infty}\dfrac{(-1)^{n+1}}{\sqrt[]{n}}$
\task $\sum\limits_{n=2}^{\infty}\dfrac{(-1)^{n-1}(2n-1)}{(\sqrt[]{2})^n}$
\task $\sum\limits_{n=2}^{\infty}\dfrac{(-1)^{n-1}(n+1)}{n}$
\end{tasks}
\begin{solucion}

\begin{enumerate}[a)]
\item $\sum\limits_{n=1}^{\infty}\dfrac{(-1)^{n+1}}{\sqrt[]{n}}$\\
			\\
			Veamos el límite de la sucesión,
			$$\lim_{n\ra\infty} |a_n| = \lim_{n\ra\infty} \dfrac{1}{\sqrt[]{n}} = 0$$
			Además, notemos que 
			$$\dfrac{1}{\sqrt[]{n}} > \dfrac{1}{\sqrt[]{n+1}} \ra \dfrac{1}{n} > \dfrac{1}{n+1} \ra n < n+1$$
			Por lo que la sucesión es decreciente.\\
			\\
			Dicho esto, por el criterio de Leibniz, la serie converge.\\
			\\
			Veamos ahora que pasa con la serie no alternante 
			$$\sum\limits_{n=1}^{\infty}\dfrac{1}{\sqrt[]{n}} = \sum\limits_{n=1}^{\infty}\dfrac{1}{n^{1/2}}$$
			Es una serie-p con $p<1$, por lo que es divergente.\\
			\\
			Finalmente, la serie converge condicionalmente.
\item $\sum\limits_{n=2}^{\infty}\dfrac{(-1)^{n-1}(2n-1)}{(\sqrt[]{2})^n}$\\
			\\
			Veamos el límite de la sucesión,
			$$\lim_{n\ra\infty} |a_n| = \lim_{n\ra\infty} \dfrac{2n-1}{(\sqrt[]{2})^n} = 0$$
			Usando el criterio de la razón,
			$$\lim\limits_{n \ra \infty} \left|\dfrac{a_{n+1}}{a_n}\right|
			= \lim\limits_{n \ra \infty} \dfrac{2n+2-1}{(\sqrt[]{2})^{\cancel{n+1}}} \cdot \dfrac{(\cancel{\sqrt[]{2})^n}}{2n-1}
			= \lim\limits_{n \ra \infty} \dfrac{2n+1}{(2n-1)(\sqrt[]{2})} = \dfrac{1}{\sqrt[]{2}} < 1$$
			Por lo tanto, la serie no alternante converge, lo que implica que la alternante también. En conclusióm, la serie converge absolutamente.
\item $\sum\limits_{n=2}^{\infty}\dfrac{(-1)^{n-1}(n+1)}{n}$\\
			\\
			Veamos el límite de la sucesión,
			$$\lim_{n\ra\infty} |a_n| = \lim_{n\ra\infty}\dfrac{(n+1)}{n} = 1 \neq 0$$
			Por la prueba de la divergencia, la serie es divergente.
\end{enumerate}
\end{solucion}
\item Considere una función $f$ continua en $\R$, decreciente y no negativa tal que
	$$\lim_{x\ra\infty}\dfrac{f(x)}{e^{-x}}=5$$
	Analice la convergencia de la serie $\sum\limits_{n=1}^{\infty}f(n)$
\begin{solucion}
Sabemos que
		$$\lim_{x\ra\infty}\dfrac{f(x)}{e^{-x}}=5 \neq 0$$
		Por lo tanto, por el criterio de comparación al límite, $\sum\limits_{n=1}^{\infty}f(n)$ se comporta igual que $\sum\limits_{n=1}^{\infty}e^{-n}$\\
		\\
		Veamos la convergencia de 
		$$\sum\limits_{n=1}^{\infty}e^{-n}$$
		Usando el criterio de la razón,
		$$\lim\limits_{n \ra \infty} \dfrac{a_{n+1}}{a_n}
		= \lim\limits_{n \ra \infty} \dfrac{e^{-n-1}}{e^{-n}}
		= e^{-1} < 1$$
		Por lo tanto, es convergente. De la misma forma, la serie $\sum\limits_{n=1}^{\infty}f(n)$ también lo es.
\end{solucion}
\end{preguntas}
\end{document}