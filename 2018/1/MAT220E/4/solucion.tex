\documentclass[12pt]{article}

\usepackage{fullpage}
\usepackage{graphicx}
\usepackage{amssymb}
\usepackage{amsmath}
\usepackage[none]{hyphenat}
\usepackage{parskip}
\usepackage[spanish]{babel}
\usepackage[utf8]{inputenc}
\usepackage{hyperref}
\usepackage{fancyhdr}
\usepackage{tasks}
\usepackage{mdframed}
\usepackage{xcolor}
\usepackage{pgfplots}
\usepackage[makeroom]{cancel}
\usepackage{multicol}
\usepackage[shortlabels]{enumitem}
\usepackage{stackrel}
\usepackage{tkz-tab}
\usepackage{xpatch}
\xpatchcmd{\tkzTabLine}{$0$}{$\bullet$}{}{}

\setlength{\headheight}{10pt}
\setlength{\headsep}{10pt}
\pagestyle{fancy}
\rhead{\ayudantia \ - \alumno}
\tikzset{t style/.style={style=solid}}

\newcommand*{\mybox}[2]{\colorbox{#1!30}{\parbox{.98\linewidth}{#2}}}

\newenvironment{solucion}
{\begin{mdframed}[backgroundcolor=black!10]
		{\bf Solución:}\\
	}
	{
	\end{mdframed}
}

\newenvironment{alternativas}[1]
{\begin{multicols}{#1}
		\begin{enumerate}[a)]
		}
		{
		\end{enumerate}
	\end{multicols}
}

\newenvironment{preguntas}
{\begin{enumerate}\itemsep12pt
	}
	{
	\end{enumerate}
}

\newcommand{\ayudantia}{{\sc Ayudantía 4}}
\newcommand{\tituloayu}{Repaso I1}
\newcommand{\fecha}{2 de abril de 2018}
\newcommand{\sigla}{MAT220E}
\newcommand{\nombre}{Cálculo II}
\newcommand{\profesor}{Vania Ramirez}
\newcommand{\ano}{2018}
\newcommand{\semestre}{1}
\newcommand{\mail}{mat220e@ifcastaneda.cl}
\newcommand{\alumno}{Ignacio Castañeda - \mail}

\newcommand{\ev}{\Big|}
\newcommand{\ra}{\rightarrow}
\newcommand{\lra}{\leftrightarrow}
\newcommand{\N}{\mathbb{N}}
\newcommand{\R}{\mathbb{R}}
\newcommand{\Exp}[1]{\mathcal{E}_{#1}}
\newcommand{\List}[1]{\mathcal{L}_{#1}}
\newcommand{\EN}{\Exp{\N}}
\newcommand{\LN}{\List{\N}}
\newcommand{\comment}[1]{}
\newcommand{\lb}{\\~\\}
\newcommand{\eop}{_{\square}}
\newcommand{\hsig}{\hat{\sigma}}
\newcommand{\widesim}[2][1.5]{
	\mathrel{\overset{#2}{\scalebox{#1}[1]{$\sim$}}}
}
\newcommand{\wsim}{\widesim{}}
\newcommand{\lh}{\stackrel{L'H}{=}}

\begin{document}
\thispagestyle{empty}

\begin{minipage}{2cm}
	\includegraphics[width=2cm]{../../../../img/logo.pdf}
	\vspace{0.5cm}
\end{minipage}
\begin{minipage}{\linewidth}
	\begin{tabular}{lrl}
		{\scriptsize\sc Pontificia Universidad Catolica de Chile} & \hspace*{0.7in}Curso: &
		\sigla  - \nombre\\
		{\sc Facultad de Matemáticas}&
		Profesor: & \profesor \\
		{\sc Semestre \ano-\semestre} & Ayudante: & {Ignacio Castañeda}\\
		& {Mail:} & \texttt{\mail}
	\end{tabular}
\end{minipage}

\vspace{-10mm}
\begin{center}
	{\LARGE\bf \ayudantia}\\
	\vspace{0.1cm}
	{\tituloayu}\\
	\vspace{0.1cm}
	\fecha\\
	\vspace{0.4cm}
\end{center}

\begin{preguntas}
\item Calcule el siguiente límite
	$$\lim\limits_{x \ra \infty} \dfrac{\displaystyle\int_0^{arctan(x)} sen(ln(t^2+t+1)) dt}{x}$$
\begin{solucion}
Al evaluar podemos ver que el límite tiene la forma $\dfrac{0}{0}$, por lo que podemos usar $L'Hopital$
		$$\lim\limits_{x \ra \infty} \dfrac{\displaystyle\int_0^{arctan(x)} sen(ln(t^2+t+1)) dt}{x}
		\stackrel{L'H}{=} \lim\limits_{x \ra \infty} \dfrac{sen(ln(arctan^2(x)+arctan(x)+1))}{1}$$
		$$= \lim\limits_{x \ra \infty} sen(ln(arctan^2(x)+arctan(x)+1)) = sen(ln(1)) = sen(0) = 0$$
\end{solucion}
\item Resuelva la siguiente integral
	$$\displaystyle\int \dfrac{dx}{x(200-x)}$$
\begin{solucion}
En primer lugar, utilizamos fracciones parciales para descomponer la integral
		$$\dfrac{1}{x(2-x)} = \dfrac{A}{x} + \dfrac{B}{2-x}$$
		$$\dfrac{1}{x(2-x)} = \dfrac{A(2-x) + Bx}{x(2-x)}$$
		$$\dfrac{1}{x(2-x)} = \dfrac{2A-Ax + Bx}{x(2-x)}$$
		$$\dfrac{1}{x(2-x)} = \dfrac{2A+ (B-A)x}{x(2-x)}$$
		De esta forma, tenemos el siguiente sistema de ecuaciones
		$$2A = 1, \quad B-A = 0 \ra A = \dfrac{1}{2}, \quad B = \dfrac{1}{2}$$
		Entonces, tenemos que la descomposición es
		$$\dfrac{1}{x(2-x)} = \dfrac{1}{2x} + \dfrac{1}{2(2-x)}$$
		Luego, la integral original la podemos escribir como
		$$\displaystyle\int \dfrac{dx}{x(2-x)} = \displaystyle\int \dfrac{dx}{2x} + \displaystyle\int \dfrac{dx}{2(2-x)}$$
		$$ = \dfrac{1}{2} \left( \displaystyle\int \dfrac{dx}{x} + \displaystyle\int \dfrac{dx}{2-x} \right)$$
		$$ = \dfrac{1}{2} (ln|x| - ln|2-x| ) + c$$
\end{solucion}
\item Calcular la integral
	$$\displaystyle\int_0^1 \dfrac{x^3+6x^2+13x+10}{x+1}dx$$
\begin{solucion}
Como vemos que el grado del numerador es mayor que el del denominador, realizamos división polinómica, es decir, calculamos
		$$x^3+6x^2+13x+10 : x+1$$
		Esto lo podemos hacer utilizando división sintética
		$$
		\renewcommand\arraystretch{1.5}
		\setlength\doublerulesep{0pt}
		\begin{array}{rrrrr}
		\multicolumn{1}{r|}{1} & 1 & 6 & 13 & 10\\\cline{2-5}
		& & 1& 5 & 8 \\\cline{2-5}
		& 1 & 5& 8 & 2 
		\end{array}
		$$
		De donde obtenemos que
		$$\dfrac{x^3+6x^2+13x+10}{x+1} = x^2 +5x + 8 + \dfrac{2}{x+1}$$
		Luego,
		$$\displaystyle\int_0^1 \dfrac{x^3+6x^2+13x+10}{x+1}dx = \displaystyle\int_0^1 \left(x^2 + 5x + 8 +  \dfrac{2}{x+1}\right)dx$$
		$$= \left( \dfrac{x^3}{3} + \dfrac{5x^2}{2} + 8x + 2ln(x+1)\right) \ev_0^1 = \dfrac{1}{3} + \dfrac{5}{2} + 8 + 2ln(2) = \dfrac{65}{6} + 2 ln(2)$$
\end{solucion}
\item Calcule el área de la región que se encuentra bajo la gráfica de la función $f(x) = e^xx^2$, donde $x \in [1,2]$
\begin{solucion}
Esta área esta dada por la integral
		$$\displaystyle \int_1^2 e^x x^2dx$$
		Calculemos primero la integral indefinida. Para esto, utilizamos integración por partes, con
		$$U = x^2 \ra du = 2xdx$$
		$$dv = e^xdx \ra V = e^x$$
		$$\displaystyle \int e^x x^2dx = x^2e^x - 2 \displaystyle \int e^x xdx$$
		Para la segunda integral usamos nuevamente integración por partes, con
		$$U = x \ra du = dx$$
		$$dv = e^xdx \ra V = e^x$$
		Luego,
		$$\displaystyle \int e^x xdx = xe^x - \displaystyle \int e^x dx = xe^x - e^x$$
		Reemplazando en la primera, tenemos que
		$$\displaystyle \int e^x x^2dx = x^2e^x - 2 (xe^x - e^x) = e^x(x^2-2x+2)$$
		Finalmente,
		$$\displaystyle \int_1^2 e^x x^2dx = e^x(x^2-2x+2) \ev_1^2 = 2e^2 - e$$
\end{solucion}
\item Calcule la siguiente integral
	$$\displaystyle\int sen^3(x)cos^4(x)dx$$
\begin{solucion}
Como el exponente del seno es impar, lo separamos y dejamos un seno aislado
		$$\displaystyle\int sen^2(x)cos^4(x) sen(x)dx$$
		Luego, pasamos todo a coseno
		$$\displaystyle\int (1-cos^2(x))cos^4(x) sen(x)dx$$
		A continuación, utilizamos la sustitución $u = cos(x) \ra du = -sen(x)dx$
		$$\displaystyle\int (1-u^2)u^4(- du) = -\displaystyle\int (u^4-u^6)du = -(\dfrac{u^5}{5} - \dfrac{u^7}{7}) +c= \dfrac{u^7}{7}-\dfrac{u^5}{5} + c$$
		Por último, volvemos a la variable original, con lo que
		$$\displaystyle\int sen^3(x)cos^4(x)dx = \dfrac{cos^7(x)}{7}-\dfrac{cos^5(x)}{5} + c$$
\end{solucion}
\item $$\displaystyle\int x^2\ arctan(x)dx$$
\begin{solucion}
Usamos integración por partes, con
		$$U = arctan(x) \ra du = \dfrac{dx}{1+x^2}$$
		$$dv = x^2 \ra V = \dfrac{x^3}{3}dx$$
		Luego,
		$$\displaystyle\int x^2\ arctan(x)dx = \dfrac{x^3}{3}arctan(x) - \dfrac{1}{3} \displaystyle\int \dfrac{x^3}{x^2+1}dx$$
		Para la integral del lado derecho, usamos sustitución con $m = 1+x^2 \ra dm = 2xdx$, con lo que
		$$\displaystyle\int \dfrac{x^3}{x^2+1}dx
		= \dfrac{1}{2}\displaystyle\int \dfrac{m-1}{m}dm
		= \dfrac{1}{2}\displaystyle\int \left(1-\dfrac{1}{m}\right)dm = \dfrac{1}{2}(m-ln|m|)$$
		Volvemos a la variable original
		$$= \dfrac{1}{2}(1+x^2-ln(1+x^2))$$
		Finalmente, reemplazando en el inicio,
		$$\displaystyle\int x^2\ arctan(x)dx = \dfrac{x^3}{3}arctan(x) - \dfrac{1}{6} (1+x^2-ln(1+x^2))$$
\end{solucion}
\end{preguntas}
\end{document}