\documentclass[12pt]{article}

\usepackage{fullpage}
\usepackage{graphicx}
\usepackage{amssymb}
\usepackage{amsmath}
\usepackage[none]{hyphenat}
\usepackage{parskip}
\usepackage[spanish]{babel}
\usepackage[utf8]{inputenc}
\usepackage{hyperref}
\usepackage{fancyhdr}
\usepackage{tasks}
\usepackage{mdframed}
\usepackage{xcolor}
\usepackage{pgfplots}
\usepackage[makeroom]{cancel}
\usepackage{multicol}
\usepackage[shortlabels]{enumitem}
\usepackage{tabto}

\setlength{\headheight}{10pt}
\setlength{\headsep}{10pt}
\pagestyle{fancy}
\rhead{\ayudantia \ - \alumno}

\newcommand*{\mybox}[2]{\colorbox{#1!30}{\parbox{.98\linewidth}{#2}}}

\newenvironment{solucion}
{\begin{mdframed}[backgroundcolor=black!10]
		{\bf Solución:}\\
	}
	{
	\end{mdframed}
}

\newenvironment{alternativas}[1]
{\begin{multicols}{#1}
		\begin{enumerate}[a)]
		}
		{
		\end{enumerate}
	\end{multicols}
}

\newenvironment{preguntas}
{\begin{enumerate}\itemsep12pt
	}
	{
	\end{enumerate}
}

\newcommand{\ayudantia}{{\sc Ayudantía 7}}
\newcommand{\tituloayu}{Integrales impropias y series}
\newcommand{\fecha}{2018-05-07}
\newcommand{\alumno}{Ignacio Castañeda - ifcastaneda@uc.cl}
\newcommand{\sigla}{MAT220E}
\newcommand{\nombre}{Cálculo II}
\newcommand{\profesor}{Vania Ramirez}
\newcommand{\ano}{2018}
\newcommand{\semestre}{1}

\newcommand{\ev}{\Big|}
\newcommand{\ra}{\rightarrow}
\newcommand{\lra}{\leftrightarrow}
\newcommand{\N}{\mathbb{N}}
\newcommand{\R}{\mathbb{R}}
\newcommand{\Exp}[1]{\mathcal{E}_{#1}}
\newcommand{\List}[1]{\mathcal{L}_{#1}}
\newcommand{\EN}{\Exp{\N}}
\newcommand{\LN}{\List{\N}}
\newcommand{\comment}[1]{}
\newcommand{\lb}{\\~\\}
\newcommand{\eop}{_{\square}}
\newcommand{\hsig}{\hat{\sigma}}

\begin{document}
\thispagestyle{empty}

\begin{minipage}{2cm}
	\includegraphics[width=2cm]{../../../../img/logo.pdf}
	\vspace{0.5cm}
\end{minipage}
\begin{minipage}{\linewidth}
	\begin{tabular}{lrl}
		{\scriptsize\sc Pontificia Universidad Catolica de Chile} & \hspace*{0.7in}Curso: &
		\sigla - \nombre\\
		{\sc Facultad de Matemáticas}&
		Profesor: & \profesor \\
		{\sc Semestre \ano-\semestre} & Ayudante: & {Ignacio Castañeda}\\
		& {Mail:} & \texttt{ifcastaneda@uc.cl}
	\end{tabular}
\end{minipage}

\begin{preguntas}
\item Determinar si las siguientes integrales impropias convergen o divergen y calcularlas en caso de que converjan.
\begin{tasks}(3)
\task $\displaystyle\int_{0}^{3}\dfrac{1}{x\ \sqrt[]{x}}dx$
\task $\displaystyle\int_{e}^{\infty}\dfrac{1}{xlnx}dx$
\task $\displaystyle\int_{0}^{\infty}\dfrac{1}{\sqrt[]{x}(1+x)}dx$
\end{tasks}
\begin{solucion}

\begin{enumerate}[a)]
\item$\displaystyle\int_{0}^{3}\dfrac{1}{x\ \sqrt[]{x}}dx
			= \lim\limits_{z\ra 0} \displaystyle\int_{z}^{3}\dfrac{1}{x^{3/2}}dx 
			= \lim\limits_{z\ra 0} \dfrac{x^{-1/2}}{-\dfrac{1}{2}}\ev_z^3 
			= \lim\limits_{z\ra 0} -2x^{-1/2} \ev_z^3$\\
			$= \lim\limits_{z\ra 0} -2(3^{-1/2}-z^{-1/2})
			 = -2\left(\dfrac{1}{3^{1/2}} - \cancelto{\infty}{\dfrac{1}{z^{1/2}}}\right) = \infty = \not \exists$\\
			Luego, la integral diverge.
\item$\displaystyle\int_{e}^{\infty}\dfrac{1}{xlnx}dx$\\
			\\
			Usamos la sustitución $u=ln(x) \ra du = \dfrac{dx}{x}$,
			$$\displaystyle\int_{e}^{\infty}\dfrac{1}{xlnx}dx
			= \lim\limits_{a\ra \infty} \int_1^{ln(a)} \dfrac{du}{u}
			= \lim\limits_{a\ra \infty} ln(u) \ev_1^{ln(a)}
			= \lim\limits_{a\ra \infty} ln(ln(a)) - ln(1)$$
			$$= ln(ln(\infty)) = ln(\infty) = \infty = \not \exists$$
			Con lo que concluimos que la integral diverge
\item$\displaystyle\int_{0}^{\infty}\dfrac{1}{\sqrt[]{x}(1+x)}dx$\\
			\\
			Notemos que hay problemas en ambos límites de integración, por lo que debemos separar la integral en dos,
			$$\displaystyle\int_{0}^{\infty}\dfrac{1}{\sqrt[]{x}(1+x)}dx = \displaystyle\int_{0}^{1}\dfrac{1}{\sqrt[]{x}(1+x)}dx + \displaystyle\int_{1}^{\infty}\dfrac{1}{\sqrt[]{x}(1+x)}dx$$
			Calculemos primero la integral indefinida
			$$\displaystyle\int \dfrac{1}{\sqrt[]{x}(1+x)}dx$$
			Usando $u = \sqrt[]{x} \ra \dfrac{dx}{2\ \sqrt[]{x}}$,
			$$\int \dfrac{2du}{1+u^2} = 2\int \dfrac{du}{1+u^2} = 2arctan(u) = 2arctan(\sqrt[]{x})$$
			Luego,
			$$\displaystyle\int_{0}^{1}\dfrac{1}{\sqrt[]{x}(1+x)}dx + \displaystyle\int_{1}^{\infty}\dfrac{1}{\sqrt[]{x}(1+x)}dx$$
			$$= \lim\limits_{a\ra 0} 2arctan(\sqrt[]{x}) \ev_a^1 + \lim\limits_{a\ra \infty} 2arctan(\sqrt[]{x}) \ev_1^a$$
			$$= \lim\limits_{a\ra 0} 2(arctan(1) -arctan(\sqrt[]{a}))  + \lim\limits_{a\ra \infty} 2(arctan(\sqrt[]{a}) - arctan(1)) \ev_1^a$$
			$$= 2\left(\dfrac{\pi}{4} - 0\right) + 2\left(\dfrac{\pi}{2} - \dfrac{\pi}{4}\right)$$
			$$= \pi$$
			Por lo que la integral converge a $\pi$.
\end{enumerate}
\end{solucion}
\item La Trompeta de Torricelli se consigue al rotar la curva $y=\dfrac{1}{x},\ x \geq 1$ en torno al eje $x$.
\begin{enumerate}[a)]
\item Calcular la superficie de la trompeta entre $1$ y $L$ con la siguiente formula:
		$$ S = 2\pi \displaystyle\int_{1}^{L} f(x)\ \sqrt[]{1+[f'(x)]^2}dx, \quad con\ L \ra \infty $$
\item Calcular el volumen de la trompeta entre $1$ y $L$ con la siguiente formula:
		$$ V = \pi \displaystyle\int_{1}^{L} f(x)^2dx, \quad con\ L \ra \infty$$
\end{enumerate}
\begin{solucion}
En primer lugar,
			$$f(x) = \dfrac{1}{x} \ra f'(x) = -\dfrac{1}{x^2}$$
\begin{enumerate}[a)]
\itemCalcular la superficie de la trompeta entre $1$ y $L$ con la siguiente formula:
			$$ S = 2\pi \displaystyle\int_{1}^{L} f(x)\ \sqrt[]{1+[f'(x)]^2}dx, \quad con\ L \ra \infty$$
			Usando la fórmula que nos dan,
			$$S = 2\pi \displaystyle\int_{1}^{\infty} \dfrac{1}{x}\ \sqrt[]{1+\left(-\dfrac{1}{x^2}\right)^2}dx 
			= 2\pi \displaystyle\int_{1}^{\infty} \dfrac{\sqrt[]{1+\frac{1}{x^4}}}{x}dx$$
			Notemos que
			$$\dfrac{\sqrt[]{1+\frac{1}{x^4}}}{x} > \dfrac{1}{x}$$
			Luego,
			$$2\pi \displaystyle\int_{1}^{\infty} \dfrac{\sqrt[]{1+\frac{1}{x^4}}}{x}dx
			>  2\pi \displaystyle\int_{1}^{\infty} \dfrac{1}{x}dx$$
			$$2\pi \displaystyle\int_{1}^{\infty} \dfrac{\sqrt[]{1+\frac{1}{x^4}}}{x}dx
			>  2\pi ln(x) \ev_1^{\infty}$$
			$$2\pi \displaystyle\int_{1}^{\infty} \dfrac{\sqrt[]{1+\frac{1}{x^4}}}{x}dx
			>  \infty$$
			Finalmente,
			$$S = \infty$$
\itemCalcular el volumen de la trompeta entre $1$ y $L$ con la siguiente formula:
			$$ V = \pi \displaystyle\int_{1}^{L} f(x)^2dx, \quad con\ L \ra \infty$$
			Usando la fórmula dada,
			$$V = \pi \displaystyle\int_{1}^{\infty} \left(\dfrac{1}{x}\right)^2dx
			=  \pi \displaystyle\int_{1}^{\infty} \dfrac{1}{x^2}dx
			= \pi \left(-\dfrac{1}{x}\ev_1^{\infty}\right)$$
			$$V = \pi$$
\end{enumerate}
\end{solucion}
\item Determinar el valor de la constante $C$ para la cual la integral
	$$\displaystyle\int_{0}^{\infty} \left( \dfrac{x}{x^2+1} - \dfrac{C}{3x+1}\right)dx$$
	converge. Evalúe la integral para este valor de $C$.
\begin{solucion}
Como habrán notado, al igual que los límites al infinito, en algunas integrales impropias podemos predecir si estas convergeran o divergeran por la diferencia de grados entre el numerador y el denominador. De esta forma, una integral como $\displaystyle\int_1^{\infty} \dfrac{x^2+1}{x^4}dx$ tendría la misma convergencia que $\displaystyle\int_1^{\infty} \dfrac{1}{x^2}dx$, ya que su diferencia de grados es 2. Esta última sabemos que converge, ya que es una integral $p$ con $p>1$. Vamos ahora al problema.\\
		\\
		En primer lugar, para ver como se comporta esta integral, juntemosla en una sola,
		\small$$\displaystyle\int_{0}^{\infty} \left( \dfrac{x}{x^2+1} - \dfrac{C}{3x+1}\right)dx
		= \displaystyle\int_{0}^{\infty} \dfrac{3x^2+x-Cx^2-C}{(x^2+1)(3x+1)}dx
		= \displaystyle\int_{0}^{\infty} \dfrac{x^2(3-C) + x - C}{(x^2+1)(3x+1)}dx$$
		Podemos ver que la diferencia de grado es $1$, por lo que se comportará igual a una integral $p$ con $p=1$, la cual diverge. Sin embargo, si $C=3$, el término $x^2$ se anula y la diferencia de grados es $2$, por lo que se comportará como la integral $p$ con $p=2>1$, la cual converge.
		
		De esta manera, la integral converge para $C=3$.
		
		Reemplazando, debemos ahora resolver
		$$I = \displaystyle\int_{0}^{\infty} \left( \dfrac{x}{x^2+1} - \dfrac{3}{3x+1}\right)dx$$
		Para resolver nos conviene dejarlas cada una por separado.
		
		En primer lugar,
		$$\displaystyle\int \dfrac{x}{x^2+1}$$
		Usamos $u = x^2+1 \ra du=2xdx$,
		$$\displaystyle\int \dfrac{x}{x^2+1} = \int\dfrac{du}{2u} = \dfrac{1}{2}ln(u) = \dfrac{1}{2}ln(x^2+1) = ln(\sqrt[]{x^2+1})$$
		En segundo lugar,
		$$\int \dfrac{3}{3x+1}dx$$
		Usamos $u=3x+1 \ra du = 3dx$,
		$$\int \dfrac{3}{3x+1}dx = \int \dfrac{du}{u} = ln(u) = ln(3x+1)$$
		Volviendo al problema original,
		$$I = \lim\limits_{a\ra \infty} (ln(\sqrt[]{x^2+1})-ln(3x+1)) \ev_0^a$$
		$$= \lim\limits_{a\ra \infty} ln \left( \dfrac{\sqrt[]{x^2+1}}{3x+1} \right) \ev_0^a$$
		$$= \lim\limits_{a\ra \infty} ln \left( \dfrac{\sqrt[]{a^2+1}}{3a+1} \right) - ln(1)$$
		$$I = ln\left(\dfrac{1}{3}\right) = -ln(3)$$
\end{solucion}
\item Determine si las siguientes series convergen o divergen.
\begin{tasks}(3)
\task $\sum\limits_{n=1}^{\infty}\dfrac{5n^3}{7n+n^3-1}$
\task $\sum\limits_{n=1}^{\infty}\left(sen\left(\dfrac{n\pi}{2}\right)\right)^2$
\task $\sum\limits_{n=1}^{\infty}2^{2n}3^{1-n}$
\task $\sum\limits_{n=1}^{\infty}\dfrac{1}{n^4}$
\task $\sum\limits_{n=20}^{\infty}\dfrac{1}{nln(n)ln(ln(n))}$
\task $\sum\limits_{n=1}^{\infty}ln\left(1+\dfrac{1}{n}\right)$
\end{tasks}
\begin{solucion}

\begin{enumerate}[a)]
\item$\sum\limits_{n=1}^{\infty}\dfrac{5n^3}{7n+n^3-1}$\\
			\\
			En primer lugar, veamos el límite de la sucesión, esto es
			$$\lim\limits_{n\ra \infty} \dfrac{5n^3}{7n+n^3-1} = 5 \neq 0$$
			Entonces, por la prueba de la divergencia, concluimos que la serie diverge.
\item$\sum\limits_{n=1}^{\infty}\left(sen\left(\dfrac{n\pi}{2}\right)\right)^2$\\
			\\
			Veamos el límite de la sucesión
			$$\lim\limits_{n\ra \infty} \left(sen\left(\dfrac{n\pi}{2}\right)\right)^2 \ra alternante \neq 0$$
			Por la prueba de la divergencia, la serie diverge.
\item$\sum\limits_{n=1}^{\infty}2^{2n}3^{1-n}$\\
			\\
			Notemos que esta es una serie geométrica, por lo que debemos escribirla de la forma $\sum\limits_1^{\infty} Ar^n$ para saber como se comporta.
			$$\sum\limits_{n=1}^{\infty}2^{2n}3^{1-n}
			= \sum\limits_{n=1}^{\infty}2^{2n}\left(\dfrac{1}{3}\right)^{n-1}
			= \sum\limits_{n=1}^{\infty}2^{2n}3\left(\dfrac{1}{3}\right)^n
			= \sum\limits_{n=1}^{\infty}4^{n}3\left(\dfrac{1}{3}\right)^n
			= \sum\limits_{n=1}^{\infty}3\left(\dfrac{4}{3}\right)^n$$
			Como $r = \dfrac{4}{3} \geq 1$, la serie diverge.
\item$\sum\limits_{n=1}^{\infty}\dfrac{1}{n^4}$\\
			\\
			Notemos que esta es una $serie-p$ con $p=4>1$, por lo que la serie es convergente.
\item$\sum\limits_{n=20}^{\infty}\dfrac{1}{nln(n)ln(ln(n))}$\\
			\\
			Sea
			$$a_n = \dfrac{1}{nln(n)ln(ln(n))} \ra f(x) = \dfrac{1}{xln(x)ln(ln(x))}$$
			Usando el criterio de la integral, $\displaystyle\int_{20}^{\infty}f(x)dx$ se comportará igual que $\sum\limits_{n=20}^{\infty} a_n$, por lo que debemos ver la convergencia de la integral en cuestión
			$$\int_{20}^{\infty} \dfrac{1}{xln(x)ln(ln(x))} dx$$
			Usando $u=ln(ln(x)) \ra du = \dfrac{dx}{xln(x)}$,
			$$\int_{20}^{\infty} \dfrac{1}{xln(x)ln(ln(x))} dx
			= \int_{ln(ln(20))}^{\infty} \dfrac{du}{u} = ln(u) \ev_{ln(ln(20))}^{\infty}$$
			$$ = ln(\infty) - ln(ln(ln(20))) = \infty = \not \exists$$
			Finalmente, por el criterio de la integral, la serie converge.
\item $\sum\limits_{n=1}^{\infty}ln\left(1+\dfrac{1}{n}\right)$\\
			\\
			Notemos que
			$$\sum\limits_{n=1}^{\infty} ln\left(1+\dfrac{1}{n}\right)
			= \sum\limits_{n=1}^{\infty} ln\left(\dfrac{n+1}{n}\right)
			= \sum\limits_{n=1}^{\infty} ln(n+1)-ln(n)$$
			Es una serie telescopica. Expandiendo términos
			$$= (ln(2)-ln(1)) + (ln(3)-ln(2)) + \dots + (ln(n) - ln(n-1)) + (ln(n+1) - ln(n))$$
			$$= (\cancel{ln(2)}-ln(1)) + (\cancel{ln(3)}-\cancel{ln(2)}) + \cancel{\dots} + (\cancel{ln(n)} - \cancel{ln(n-1)}) + (ln(n+1) - \cancel{ln(n)})$$
			$$= ln(n+1) - ln(1)$$
			$$\lim\limits_{n \ra \infty} ln(n+1) = ln(\infty) = \infty = \not \exists$$
			Finalmente, la serie es divergente.
\end{enumerate}
\end{solucion}
\end{preguntas}
\end{document}