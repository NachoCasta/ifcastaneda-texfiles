\documentclass[12pt]{article}

\usepackage{fullpage}
\usepackage{graphicx}
\usepackage{amssymb}
\usepackage{amsmath}
\usepackage[none]{hyphenat}
\usepackage{parskip}
\usepackage[spanish]{babel}
\usepackage[utf8]{inputenc}
\usepackage{hyperref}
\usepackage{fancyhdr}
\usepackage{tasks}
\usepackage{mdframed}
\usepackage{xcolor}
\usepackage{pgfplots}
\usepackage[makeroom]{cancel}
\usepackage{multicol}
\usepackage[shortlabels]{enumitem}
\usepackage{stackrel}
\usepackage{tkz-tab}
\usepackage{xpatch}
\xpatchcmd{\tkzTabLine}{$0$}{$\bullet$}{}{}

\setlength{\headheight}{10pt}
\setlength{\headsep}{10pt}
\pagestyle{fancy}
\rhead{\ayudantia \ - \alumno}
\tikzset{t style/.style={style=solid}}

\newcommand*{\mybox}[2]{\colorbox{#1!30}{\parbox{.98\linewidth}{#2}}}

\newenvironment{solucion}
{\begin{mdframed}[backgroundcolor=black!10]
		{\bf Solución:}\\
	}
	{
	\end{mdframed}
}

\newenvironment{alternativas}[1]
{\begin{multicols}{#1}
		\begin{enumerate}[a)]
		}
		{
		\end{enumerate}
	\end{multicols}
}

\newenvironment{preguntas}
{\begin{enumerate}\itemsep12pt
	}
	{
	\end{enumerate}
}

\newcommand{\ayudantia}{{\sc Ayudantía 2}}
\newcommand{\tituloayu}{Integrales inmediatas y sustitucion}
\newcommand{\fecha}{19 de marzo de 2018}
\newcommand{\sigla}{MAT220E}
\newcommand{\nombre}{Cálculo II}
\newcommand{\profesor}{Vania Ramirez}
\newcommand{\ano}{2018}
\newcommand{\semestre}{1}
\newcommand{\mail}{mat220e@ifcastaneda.cl}
\newcommand{\alumno}{Ignacio Castañeda - \mail}

\newcommand{\ev}{\Big|}
\newcommand{\ra}{\rightarrow}
\newcommand{\lra}{\leftrightarrow}
\newcommand{\N}{\mathbb{N}}
\newcommand{\R}{\mathbb{R}}
\newcommand{\Exp}[1]{\mathcal{E}_{#1}}
\newcommand{\List}[1]{\mathcal{L}_{#1}}
\newcommand{\EN}{\Exp{\N}}
\newcommand{\LN}{\List{\N}}
\newcommand{\comment}[1]{}
\newcommand{\lb}{\\~\\}
\newcommand{\eop}{_{\square}}
\newcommand{\hsig}{\hat{\sigma}}
\newcommand{\widesim}[2][1.5]{
	\mathrel{\overset{#2}{\scalebox{#1}[1]{$\sim$}}}
}
\newcommand{\wsim}{\widesim{}}
\newcommand{\lh}{\stackrel{L'H}{=}}

\begin{document}
\thispagestyle{empty}

\begin{minipage}{2cm}
	\includegraphics[width=2cm]{../../../../img/logo.pdf}
	\vspace{0.5cm}
\end{minipage}
\begin{minipage}{\linewidth}
	\begin{tabular}{lrl}
		{\scriptsize\sc Pontificia Universidad Catolica de Chile} & \hspace*{0.7in}Curso: &
		\sigla  - \nombre\\
		{\sc Facultad de Matemáticas}&
		Profesor: & \profesor \\
		{\sc Semestre \ano-\semestre} & Ayudante: & {Ignacio Castañeda}\\
		& {Mail:} & \texttt{\mail}
	\end{tabular}
\end{minipage}

\vspace{-10mm}
\begin{center}
	{\LARGE\bf \ayudantia}\\
	\vspace{0.1cm}
	{\tituloayu}\\
	\vspace{0.1cm}
	\fecha\\
	\vspace{0.4cm}
\end{center}

\begin{preguntas}
\item Determine el valor de $r$ de modo que $y(x) = e^{rx}$ sea solución de la ecuación:
	$$3y'' + 3y' - 4y=0$$
\item Calcule $\lim\limits_{x \ra 0+} x^{x}
\item Sea $a_n = \dfrac{n!}{n^n}$, calcular $\lim\limits_{n \ra \infty} \dfrac{a_{n+1}}{a_n}$
\item Calcular $F'(0)$, siendo
	$$F(x) = \displaystyle\int_0^{5x+1} \dfrac{e^{t^2}}{1+t^4}dt$$
\item Resolver las siguientes integrales indefinidas
\begin{tasks}(2)
\task $\displaystyle\int (3x^2 + 2x + 1)dx$
\task $\displaystyle\int x(x+1)(x+2)dx$
\task $\displaystyle\int sen^2(x)dx$
\task $\displaystyle\int (1+e)^xdx$
\end{tasks}
\item Resolver las siguientes integrales indefinidas
\begin{tasks}(2)
\task $\displaystyle\int \dfrac{e^{ln(x)}}{x^2+7}dx$
\task $\displaystyle\int (x+2)sen(x^2+4x-6)dx$
\task $\displaystyle\int \dfrac{3x}{\sqrt[3]{x^2+3}}dx$
\task $\displaystyle\int \dfrac{e^x-1}{e^x+1}dx$
\end{tasks}
\item Resolver las siguientes integrales definidas
\begin{tasks}(2)
\task $\displaystyle\int_0^2 \dfrac{x^2}{x^3+8} dx$
\task $\displaystyle\int_{-1}^1 xsen(1-x^2)dx$
\end{tasks}
\end{preguntas}
\end{document}