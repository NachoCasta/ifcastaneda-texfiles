\documentclass[12pt]{article}

\usepackage{fullpage}
\usepackage{graphicx}
\usepackage{amssymb}
\usepackage{amsmath}
\usepackage[none]{hyphenat}
\usepackage{parskip}
\usepackage[spanish]{babel}
\usepackage[utf8]{inputenc}
\usepackage{hyperref}
\usepackage{fancyhdr}
\usepackage{tasks}
\usepackage{mdframed}
\usepackage{xcolor}
\usepackage{pgfplots}
\usepackage[makeroom]{cancel}
\usepackage{multicol}
\usepackage[shortlabels]{enumitem}
\usepackage{stackrel}
\usepackage{tkz-tab}
\usepackage{xpatch}
\xpatchcmd{\tkzTabLine}{$0$}{$\bullet$}{}{}

\setlength{\headheight}{10pt}
\setlength{\headsep}{10pt}
\pagestyle{fancy}
\rhead{\ayudantia \ - \alumno}
\tikzset{t style/.style={style=solid}}

\newcommand*{\mybox}[2]{\colorbox{#1!30}{\parbox{.98\linewidth}{#2}}}

\newenvironment{solucion}
{\begin{mdframed}[backgroundcolor=black!10]
		{\bf Solución:}\\
	}
	{
	\end{mdframed}
}

\newenvironment{alternativas}[1]
{\begin{multicols}{#1}
		\begin{enumerate}[a)]
		}
		{
		\end{enumerate}
	\end{multicols}
}

\newenvironment{preguntas}
{\begin{enumerate}\itemsep12pt
	}
	{
	\end{enumerate}
}

\newcommand{\ayudantia}{{\sc Ayudantía 2}}
\newcommand{\tituloayu}{Integrales inmediatas y sustitucion}
\newcommand{\fecha}{19 de marzo de 2018}
\newcommand{\sigla}{MAT220E}
\newcommand{\nombre}{Cálculo II}
\newcommand{\profesor}{Vania Ramirez}
\newcommand{\ano}{2018}
\newcommand{\semestre}{1}
\newcommand{\mail}{mat220e@ifcastaneda.cl}
\newcommand{\alumno}{Ignacio Castañeda - \mail}

\newcommand{\ev}{\Big|}
\newcommand{\ra}{\rightarrow}
\newcommand{\lra}{\leftrightarrow}
\newcommand{\N}{\mathbb{N}}
\newcommand{\R}{\mathbb{R}}
\newcommand{\Exp}[1]{\mathcal{E}_{#1}}
\newcommand{\List}[1]{\mathcal{L}_{#1}}
\newcommand{\EN}{\Exp{\N}}
\newcommand{\LN}{\List{\N}}
\newcommand{\comment}[1]{}
\newcommand{\lb}{\\~\\}
\newcommand{\eop}{_{\square}}
\newcommand{\hsig}{\hat{\sigma}}
\newcommand{\widesim}[2][1.5]{
	\mathrel{\overset{#2}{\scalebox{#1}[1]{$\sim$}}}
}
\newcommand{\wsim}{\widesim{}}
\newcommand{\lh}{\stackrel{L'H}{=}}

\begin{document}
\thispagestyle{empty}

\begin{minipage}{2cm}
	\includegraphics[width=2cm]{../../../../img/logo.pdf}
	\vspace{0.5cm}
\end{minipage}
\begin{minipage}{\linewidth}
	\begin{tabular}{lrl}
		{\scriptsize\sc Pontificia Universidad Catolica de Chile} & \hspace*{0.7in}Curso: &
		\sigla  - \nombre\\
		{\sc Facultad de Matemáticas}&
		Profesor: & \profesor \\
		{\sc Semestre \ano-\semestre} & Ayudante: & {Ignacio Castañeda}\\
		& {Mail:} & \texttt{\mail}
	\end{tabular}
\end{minipage}

\vspace{-10mm}
\begin{center}
	{\LARGE\bf \ayudantia}\\
	\vspace{0.1cm}
	{\tituloayu}\\
	\vspace{0.1cm}
	\fecha\\
	\vspace{0.4cm}
\end{center}

\begin{preguntas}
\item Determine el valor de $r$ de modo que $y(x) = e^{rx}$ sea solución de la ecuación:
	$$3y'' + 3y' - 4y=0$$
\begin{solucion}
En primer lugar, calculamos las primeras dos derivadas de $y(x)$
		$$y(x) = e^{rx}$$
		$$y'(x) = re^{rx}$$
		$$y''(x) = r^2e^{rx}$$
		y procedemos a reemplazar en la ecuación
		$$3r^2e^{rx} + 3re^{rx} - 4e^{rx} = 0$$
		$$3r^2 + 3r - 4 = 0$$
		$$r = \dfrac{-3 \pm \sqrt[]{57}}{6}$$
\end{solucion}
\item Calcule $\lim\limits_{x \ra 0+} x^{x}
\begin{solucion}
Para calcular este límite, se usa la propiedad siguiente
		$$e^{ln(algo)} = algo$$
		Utilizando esta propiedad, 
		$$\lim\limits_{x \ra 0+} x^{x} = \lim\limits_{x \ra 0+} e^{ln(x^x)} = \lim\limits_{x \ra 0+} e^{xln(x)}$$
		Como $e$ es una constante, tenemos que
		$$\lim\limits_{x \ra 0+} e^{xln(x)} = e^{\lim\limits_{x \ra 0+} xln(x)}$$
		Por lo que en primer lugar calcularemos el límite del exponente, es decir
		$$\lim\limits_{x \ra 0+} xln(x) = \lim\limits_{x \ra 0+} \dfrac{ln(x)}{\frac{1}{x}} \stackrel{L'H}{=} \lim\limits_{x \ra 0+} \dfrac{\frac{1}{x}}{-\frac{1}{x^2}} = \lim\limits_{x \ra 0+} -x = 0 $$
\end{solucion}
\item Sea $a_n = \dfrac{n!}{n^n}$, calcular $\lim\limits_{n \ra \infty} \dfrac{a_{n+1}}{a_n}$
\begin{solucion}
$$\lim\limits_{n \ra \infty} \dfrac{a_{n+1}}{a_n} = \lim\limits_{n \ra \infty} \dfrac{\dfrac{(n+1)!}{(n+1)^{n+1}}}{\dfrac{n!}{n^n}} = \lim\limits_{n \ra \infty} \dfrac{(n+1)!}{n!} \cdot \dfrac{n^n}{(n+1)^{n+1}} =\lim\limits_{n \ra \infty} (n+1) \cdot \dfrac{n^n}{(n+1)^{n+1}}$$
		$$=\lim\limits_{n \ra \infty} \dfrac{n^n}{(n+1)^{n}} = \lim\limits_{n \ra \infty} \left(\dfrac{n}{n+1}\right)^n = \lim\limits_{n \ra \infty} \dfrac{1}{\left(\dfrac{n+1}{n}\right)^n} = \lim\limits_{n \ra \infty} \dfrac{1}{\left(1+\dfrac{1}{n}\right)^n}$$
		$$ = \dfrac{1}{e}$$
\end{solucion}
\end{preguntas}
\end{document}