\documentclass[12pt]{article}

\usepackage{fullpage}
\usepackage{graphicx}
\usepackage{amssymb}
\usepackage{amsmath}
\usepackage[none]{hyphenat}
\usepackage{parskip}
\usepackage[spanish]{babel}
\usepackage[utf8]{inputenc}
\usepackage{hyperref}
\usepackage{fancyhdr}
\usepackage{tasks}
\usepackage{mdframed}
\usepackage{xcolor}
\usepackage{pgfplots}
\usepackage[makeroom]{cancel}
\usepackage{multicol}
\usepackage[shortlabels]{enumitem}
\usepackage{stackrel}

\setlength{\headheight}{10pt}
\setlength{\headsep}{10pt}
\pagestyle{fancy}
\rhead{\ayudantia \ - \alumno}

\newcommand*{\mybox}[2]{\colorbox{#1!30}{\parbox{.98\linewidth}{#2}}}

\newenvironment{solucion}
{\begin{mdframed}[backgroundcolor=black!10]
		{\bf Solución:}\\
	}
	{
	\end{mdframed}
}

\newenvironment{alternativas}[1]
{\begin{multicols}{#1}
		\begin{enumerate}[a)]
		}
		{
		\end{enumerate}
	\end{multicols}
}

\newenvironment{preguntas}
{\begin{enumerate}\itemsep12pt
	}
	{
	\end{enumerate}
}

\newcommand{\ayudantia}{{\sc Ayudantía 5}}
\newcommand{\tituloayu}{Matrices elementales y temas pendientes}
\newcommand{\fecha}{4 de abril de 2018}
\newcommand{\sigla}{MAT1203}
\newcommand{\nombre}{Álgebra Lineal}
\newcommand{\profesor}{Rodrigo Rubio Varas}
\newcommand{\ano}{2018}
\newcommand{\semestre}{1}
\newcommand{\mail}{mat1203@ifcastaneda.cl}
\newcommand{\alumno}{Ignacio Castañeda - \mail}

\newcommand{\ev}{\Big|}
\newcommand{\ra}{\rightarrow}
\newcommand{\lra}{\leftrightarrow}
\newcommand{\N}{\mathbb{N}}
\newcommand{\R}{\mathbb{R}}
\newcommand{\Exp}[1]{\mathcal{E}_{#1}}
\newcommand{\List}[1]{\mathcal{L}_{#1}}
\newcommand{\EN}{\Exp{\N}}
\newcommand{\LN}{\List{\N}}
\newcommand{\comment}[1]{}
\newcommand{\lb}{\\~\\}
\newcommand{\eop}{_{\square}}
\newcommand{\hsig}{\hat{\sigma}}
\newcommand{\widesim}[2][1.5]{
	\mathrel{\overset{#2}{\scalebox{#1}[1]{$\sim$}}}
}
\newcommand{\wsim}{\widesim{}}

\begin{document}
\thispagestyle{empty}

\begin{minipage}{2cm}
	\includegraphics[width=2cm]{../../../../img/logo.pdf}
	\vspace{0.5cm}
\end{minipage}
\begin{minipage}{\linewidth}
	\begin{tabular}{lrl}
		{\scriptsize\sc Pontificia Universidad Catolica de Chile} & \hspace*{0.7in}Curso: &
		\sigla  - \nombre\\
		{\sc Facultad de Matemáticas}&
		Profesor: & \profesor \\
		{\sc Semestre \ano-\semestre} & Ayudante: & {Ignacio Castañeda}\\
		& {Mail:} & \texttt{\mail}
	\end{tabular}
\end{minipage}

\vspace{-10mm}
\begin{center}
	{\LARGE\bf \ayudantia}\\
	\vspace{0.1cm}
	{\tituloayu}\\
	\vspace{0.1cm}
	\fecha\\
	\vspace{0.4cm}
\end{center}

\begin{preguntas}
\item Sea
	$$A = \begin{bmatrix}
	1 & 0 & 6\\
	-1 & 2 & 0\\
	0 & 5 & -1
	\end{bmatrix}$$
	Escriba $A$ como un producto de matrices elementales
\begin{solucion}
Para esto, escalonamos la matriz $A$ hasta reducirla a la matriz identidad. Es importante que hagamos esto paso por paso, sin realizar mas de una operación por fila a la vez.
		$$\begin{bmatrix}
		1 & 0 & 6\\
		-1 & 2 & 0\\
		0 & 5 & -1
		\end{bmatrix} \widesim{F_2+F_1}
		\begin{bmatrix}
		1 & 0 & 6\\
		0 & 2 & 6\\
		0 & 5 & -1
		\end{bmatrix} \widesim{F_3-\frac{5}{2}F_2}
		\begin{bmatrix}
		1 & 0 & 6\\
		0 & 2 & 6\\
		0 & 0 & -16
		\end{bmatrix} \widesim{F_3 \leftarrow -\frac{1}{16}F_3}
		\begin{bmatrix}
		1 & 0 & 6\\
		0 & 2 & 6\\
		0 & 0 & 1
		\end{bmatrix}$$
		$$\widesim{F_2 \leftarrow \frac{1}{2}F_2}
		\begin{bmatrix}
		1 & 0 & 6\\
		0 & 1 & 3\\
		0 & 0 & 1
		\end{bmatrix} \widesim{F_2 - 3F_3}
		\begin{bmatrix}
		1 & 0 & 6\\
		0 & 1 & 0\\
		0 & 0 & 1
		\end{bmatrix} \widesim{F_1 - 6F_3}
		\begin{bmatrix}
		1 & 0 & 0\\
		0 & 1 & 0\\
		0 & 0 & 1
		\end{bmatrix}$$
		Notemos que todas estas operaciones las podemos escribir como matrices elementales, aplicando la operación a la matriz identidad Sean
		$$E_1 = \begin{bmatrix}
		1 & 0 & 0\\
		1 & 1 & 0\\
		0 & 0 & 1
		\end{bmatrix}, \quad E_2 = \begin{bmatrix}
		1 & 0 & 0\\
		0 & 1 & 0\\
		0 & -\frac{5}{2} & 1
		\end{bmatrix}, E_3 = \begin{bmatrix}
		1 & 0 & 0\\
		0 & 1 & 0\\
		0 & 0 & -\frac{1}{16}
		\end{bmatrix}$$
		$$E_4 = \begin{bmatrix}
		1 & 0 & 0\\
		0 & \frac{1}{2} & 0\\
		0 & 0 & 1
		\end{bmatrix}, \quad E_5 = \begin{bmatrix}
		1 & 0 & 0\\
		0 & 1 & -3\\
		0 & 0 & 1
		\end{bmatrix}, E_6 = \begin{bmatrix}
		1 & 0 & -6\\
		0 & 1 & 0\\
		0 & 0 & 1
		\end{bmatrix}$$
		tenemos que
		$$E_6E_5E_4E_3E_2E_1A = I$$
		por lo que
		$$A = E_1^{-1}E_2^{-1}E_3^{-1}E_4^{-1}E_5^{-1}E_6^{-1}$$
		donde
		$$E_1^{-1} = \begin{bmatrix}
		1 & 0 & 0\\
		-1 & 1 & 0\\
		0 & 0 & 1
		\end{bmatrix}, \quad E_2^{-1} = \begin{bmatrix}
		1 & 0 & 0\\
		0 & 1 & 0\\
		0 & \frac{5}{2} & 1
		\end{bmatrix}, E_3^{-1} = \begin{bmatrix}
		1 & 0 & 0\\
		0 & 1 & 0\\
		0 & 0 & -16
		\end{bmatrix}$$
		$$E_4^{-1} = \begin{bmatrix}
		1 & 0 & 0\\
		0 & 2 & 0\\
		0 & 0 & 1
		\end{bmatrix}, \quad E_5^{-1} = \begin{bmatrix}
		1 & 0 & 0\\
		0 & 1 & 3\\
		0 & 0 & 1
		\end{bmatrix}, E_6^{-1} = \begin{bmatrix}
		1 & 0 & 6\\
		0 & 1 & 0\\
		0 & 0 & 1
		\end{bmatrix}$$
		Finalmente,
		$$A = \begin{bmatrix}
		1 & 0 & 0\\
		-1 & 1 & 0\\
		0 & 0 & 1
		\end{bmatrix}\begin{bmatrix}
		1 & 0 & 0\\
		0 & 1 & 0\\
		0 & \frac{5}{2} & 1
		\end{bmatrix}\begin{bmatrix}
		1 & 0 & 0\\
		0 & 1 & 0\\
		0 & 0 & \frac{1}{16}
		\end{bmatrix}\begin{bmatrix}
		1 & 0 & 0\\
		0 & -\frac{1}{2} & 0\\
		0 & 0 & 1
		\end{bmatrix}\begin{bmatrix}
		1 & 0 & 0\\
		0 & 1 & 3\\
		0 & 0 & 1
		\end{bmatrix}\begin{bmatrix}
		1 & 0 & 6\\
		0 & 1 & 0\\
		0 & 0 & 1
		\end{bmatrix}$$
\end{solucion}
\item Determinar las condiciones en $a$ para que la matriz
	$$A = \begin{bmatrix}
	a & 2a & 0 & 0 \\
	0 & 1 & 0 & 3a-1\\
	0 & 1 & a-1 & 2a-1\\
	a & 2a & 0 & a
	\end{bmatrix}$$
	tenga solución única
\begin{solucion}
En primer lugar, debemos pivotear $A$ para dejarla en su forma escalonada:
		$$\begin{bmatrix}
		a & 2a & 0 & 0 \\
		0 & 1 & 0 & 3a-1\\
		0 & 1 & a-1 & 2a-1\\
		a & 2a & 0 & a
		\end{bmatrix} \widesim{F_4-F_1}
		\begin{bmatrix}
		a & 2a & 0 & 0 \\
		0 & 1 & 0 & 3a-1\\
		0 & 1 & a-1 & 2a-1\\
		0 & 0 & 0 & a
		\end{bmatrix} \widesim{F_3-F_2}
		\begin{bmatrix}
		a & 2a & 0 & 0 \\
		0 & 1 & 0 & 3a-1\\
		0 & 0 & a-1 & -a\\
		0 & 0 & 0 & a
		\end{bmatrix}$$
		Notemos que la última fila nos puede dar problemas. Si $a=0$, el sistema va a tener solo 3 pivotes por lo que, en caso de ser consistente, tendría infinitas soluciones. Dicho esto, una restricción es $a\neq 0$
		
		En la tercera fila, debemos fijarnos que $a-1 \neq 0$, ya que de lo contrario, esta fila sería un múltiplo de la cuarta fila, por lo que habrían 3 pivotes y tendríamos soluciones infinitas. Dicho esto, tenemos que $a \neq 1$.
		
		De esta forma, el sistema siempre tendrá 4 pivotes y por lo tanto, tendrá solución única.
		
		En resumen, las condiciones son
		$$a \neq 0, \quad a \neq 1$$
\end{solucion}
\item Demuestre que el conjunto $\{u, v, w\}$ es L.I. si y solo si el conjunto $\{u+v, u+w, v+w\}$ es L.I.
\begin{solucion}
Como nos dicen si y solo si, debemos demostrar en ambas direcciones
		$$(\Longrightarrow)$$
		Digamos que $\{u, v, w\}$ es L.I.\\
		\\
		P.D. $\{u+v, u+w, v+w\}$ es L.I.\\
		\\
		Para que $\{u+v, u+w, v+w\}$ sea L.I., el sistema
		$$\alpha (u+v) + \beta (u+w) + \gamma (v+w) = 0$$
		debe tener solución única $\alpha = 0$, $\beta = 0$, $\gamma = 0$
		Reordenando,
		$$\alpha (u+v) + \beta (u+w) + \gamma (v+w) = 0$$
		$$(\alpha + \beta) u + (\alpha + \gamma) v + (\beta + \gamma) w = 0$$
		Como sabemos que el conjunto $\{u, v, w\}$ es L.I., entonces tenemos que
		$$\begin{array}{rl}
		\alpha + \beta & = 0\\
		\alpha + \gamma & = 0\\
		\beta + \gamma & = 0
		\end{array}$$
		Este sistema tiene por solución
		$$\alpha = 0, \quad \beta = 0, \quad \gamma = 0$$
		Con lo que demostramos que $\{u+v, u+w, v+w\}$ es L.I.
		$$(\Longleftarrow)$$
		Digamos que $\{u+v, u+w, v+w\}$ es L.I.\\
		\\
		P.D. $\{u, v, w\}$ es L.I.\\
		\\
		Para que $\{u, v, w\}$ sea L.I., el sistema
		$$c_1u + c_2v + c_3w$$
		debe tener solución única $c_1 = 0$, $c_2 = 0$, $c_3 = 0$\\
		\\
		Como $\{u+v, u+w, v+w\}$ es L.I., sabemos que el sistema
		$$\alpha (u+v) + \beta (u+w) + \gamma (v+w) = 0$$
		tiene solución única $\alpha = 0$, $\beta = 0$, $\gamma = 0$
		Reordenando,
		$$\alpha (u+v) + \beta (u+w) + \gamma (v+w) = 0$$
		$$(\alpha + \beta) u + (\alpha + \gamma) v + (\beta + \gamma) w = 0$$
		Con lo que tenemos que
		$$c_1 = \alpha + \beta = 0, \quad c_2 = \alpha + \gamma = 0, \quad c_3 = \beta + \gamma = 0$$
		es la única solución del sistema. \\
		\\
		Luego, $\{u, v, w\}$ es L.I.
		$$q.e.d$$
\end{solucion}
\end{preguntas}
\end{document}