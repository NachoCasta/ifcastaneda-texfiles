\documentclass[12pt]{article}

\usepackage{fullpage}
\usepackage{graphicx}
\usepackage{amssymb}
\usepackage{amsmath}
\usepackage[none]{hyphenat}
\usepackage{parskip}
\usepackage[spanish]{babel}
\usepackage[utf8]{inputenc}
\usepackage{hyperref}
\usepackage{fancyhdr}
\usepackage{tasks}
\usepackage{mdframed}
\usepackage{xcolor}
\usepackage{pgfplots}
\usepackage[makeroom]{cancel}
\usepackage{multicol}
\usepackage[shortlabels]{enumitem}
\usepackage{stackrel}

\setlength{\headheight}{10pt}
\setlength{\headsep}{10pt}
\pagestyle{fancy}
\rhead{\ayudantia \ - \alumno}

\newcommand*{\mybox}[2]{\colorbox{#1!30}{\parbox{.98\linewidth}{#2}}}

\newenvironment{solucion}
{\begin{mdframed}[backgroundcolor=black!10]
		{\bf Solución:}\\
	}
	{
	\end{mdframed}
}

\newenvironment{alternativas}[1]
{\begin{multicols}{#1}
		\begin{enumerate}[a)]
		}
		{
		\end{enumerate}
	\end{multicols}
}

\newenvironment{preguntas}
{\begin{enumerate}\itemsep12pt
	}
	{
	\end{enumerate}
}

\newcommand{\ayudantia}{{\sc Ayudantía 6}}
\newcommand{\tituloayu}{Repaso I1}
\newcommand{\fecha}{11 de abril de 2018}
\newcommand{\sigla}{MAT1203}
\newcommand{\nombre}{Álgebra Lineal}
\newcommand{\profesor}{Rodrigo Rubio Varas}
\newcommand{\ano}{2018}
\newcommand{\semestre}{1}
\newcommand{\mail}{mat1203@ifcastaneda.cl}
\newcommand{\alumno}{Ignacio Castañeda - \mail}

\newcommand{\ev}{\Big|}
\newcommand{\ra}{\rightarrow}
\newcommand{\lra}{\leftrightarrow}
\newcommand{\N}{\mathbb{N}}
\newcommand{\R}{\mathbb{R}}
\newcommand{\Exp}[1]{\mathcal{E}_{#1}}
\newcommand{\List}[1]{\mathcal{L}_{#1}}
\newcommand{\EN}{\Exp{\N}}
\newcommand{\LN}{\List{\N}}
\newcommand{\comment}[1]{}
\newcommand{\lb}{\\~\\}
\newcommand{\eop}{_{\square}}
\newcommand{\hsig}{\hat{\sigma}}
\newcommand{\widesim}[2][1.5]{
	\mathrel{\overset{#2}{\scalebox{#1}[1]{$\sim$}}}
}
\newcommand{\wsim}{\widesim{}}

\begin{document}
\thispagestyle{empty}

\begin{minipage}{2cm}
	\includegraphics[width=2cm]{../../../../img/logo.pdf}
	\vspace{0.5cm}
\end{minipage}
\begin{minipage}{\linewidth}
	\begin{tabular}{lrl}
		{\scriptsize\sc Pontificia Universidad Catolica de Chile} & \hspace*{0.7in}Curso: &
		\sigla  - \nombre\\
		{\sc Facultad de Matemáticas}&
		Profesor: & \profesor \\
		{\sc Semestre \ano-\semestre} & Ayudante: & {Ignacio Castañeda}\\
		& {Mail:} & \texttt{\mail}
	\end{tabular}
\end{minipage}

\vspace{-10mm}
\begin{center}
	{\LARGE\bf \ayudantia}\\
	\vspace{0.1cm}
	{\tituloayu}\\
	\vspace{0.1cm}
	\fecha\\
	\vspace{0.4cm}
\end{center}

\begin{preguntas}
\item Determine si el siguiente sistema de ecuaciones posee solución única, infinitas soluciones o no tiene solución
	$$
	\begin{array}{rcr}
	x+y-z& = & 1\\
	3x+2y+z& = & 1\\
	5x+3y+4z& = & 2\\
	-2x -y +5z & = & 6
	\end{array}$$
\begin{solucion}
Busquemos la forma escalonada de la matriz aumentada correspondiente al sistema
		$$
		\left[
		\begin{array}{ccc|c}
		1 & 1 & -1 & 1\\
		3 & 2 & 1 & 1\\
		5 & 3 & 4 & 2\\
		-2 & -1 & 5 & 6
		\end{array}
		\right] \wsim 
		\left[
		\begin{array}{ccc|c}
		1 & 1 & -1 & 1\\
		0 & -1 & 4 & 2\\
		0 & 0 & 1 & 1\\
		0 & 0 & 0 & -1
		\end{array}
		\right]$$
		Podemos ver que la última linea equivale a $0 = -1$, por lo que el sistema es inconsistente, es decir, no tiene solución.
\end{solucion}
\item Determine condiciones sobre $\alpha$ y $\beta$ de modo que \\
	$\begin{pmatrix}
	\alpha-\beta\\
	\alpha + \beta\\
	2\alpha - \beta
	\end{pmatrix}$ pertenezca a $Gen\left\{\begin{pmatrix}
	-1\\
	-1\\
	1
	\end{pmatrix}\, \begin{pmatrix}
	1\\
	2\\
	1
	\end{pmatrix}, \begin{pmatrix}
	1\\
	3\\
	3
	\end{pmatrix}\right\}$
\begin{solucion}
Lo que debemos ver es para que valores de $\alpha$ y $\beta$, existe una combinación lineal de los vectores del generado que genera el vector dado. Esto equivale a ver para que valores de $\alpha$ y $\beta$, el siguiente sistema es consistente:
		$$
		\left[
		\begin{array}{ccc|c}
		-1 & 1 & 1 & \alpha - \beta\\
		-1 & 2 & 3 & \alpha + \beta\\
		1 & 1 & 3 & 2\alpha - \beta
		\end{array}
		\right] \stackbin[F_3+F_1]{F_2-F_1}{\wsim}
		\left[
		\begin{array}{ccc|c}
		-1 & 1 & 1 & \alpha - \beta\\
		0 & 1 & 2 & 2\beta\\
		0 & 0 & 0 & 3\alpha - 6\beta
		\end{array}	\right]$$
		Como la última fila se anula, el sistema será consistente cuando
		$$3\alpha - 6\beta = 0$$
		Es decir, la condición para $\alpha$ y $\beta$, tal que 
		$$\begin{pmatrix}
		\alpha-\beta\\
		\alpha + \beta\\
		2\alpha - \beta
		\end{pmatrix} \in Gen\left\{\begin{pmatrix}
		-1\\
		-1\\
		1
		\end{pmatrix}\, \begin{pmatrix}
		1\\
		2\\
		1
		\end{pmatrix}, \begin{pmatrix}
		1\\
		3\\
		3
		\end{pmatrix}\right\}$$
		es $\alpha = 2\beta$
\end{solucion}
\item Determine todas las matrices $A$ tales que
	$$ A\begin{bmatrix}
	3\\
	2
	\end{bmatrix}=\begin{bmatrix}
	2\\
	1
	\end{bmatrix}$$
\begin{solucion}
En primer lugar, notemos que dada las dimensiones de los vectores, la matriz $A$ es de $2 \times 2$. Luego,
		$$A = \begin{bmatrix}
		\alpha & \beta \\
		\gamma &\delta
		\end{bmatrix}$$
		Reemplazando,
		$$A\begin{bmatrix}
		3\\
		2
		\end{bmatrix}=\begin{bmatrix}
		2\\
		1
		\end{bmatrix} \Longrightarrow 
		\begin{bmatrix}
		\alpha & \beta \\
		\gamma &\delta
		\end{bmatrix}
		\begin{bmatrix}
			3\\
			2
		\end{bmatrix}=\begin{bmatrix}
			2\\
			1
		\end{bmatrix} \Longrightarrow
		\begin{array}{rc}
		3\alpha + 2\beta & = 2\\
		3\gamma + 2 \delta & = 1
		\end{array} \Longrightarrow
		\begin{array}{rc}
		3\alpha & = 2 - 2\beta\\
		3\gamma & = 1 - 2 \delta
		\end{array}$$
		Con lo que
		$$\alpha = \dfrac{2}{3} - \dfrac{2\beta}{3}$$
		$$\gamma = \dfrac{1}{3} - \dfrac{2\delta}{3}$$
		Finalmente,
		$$A = \begin{bmatrix}
		\frac{2}{3} - \frac{2\beta}{3} & \beta \\
		\frac{1}{3} - \frac{2\delta}{3} &\delta
		\end{bmatrix}$$
\end{solucion}
\item Sea $T: \R^2 \ra \R^3$ la transformación lineal definida por
	$$ T\left(\begin{bmatrix}
	x_1\\
	x_2
	\end{bmatrix}\right) = \begin{bmatrix}
	5x_1 - 3x_2\\
	4x_1 -x_2\\
	2x_1 +3x_2
	\end{bmatrix}$$
	encuentre $x = \begin{bmatrix}
	x_1\\
	x_2
	\end{bmatrix}$ tal que $T(x) = \begin{bmatrix}
	13\\
	9\\
	1
	\end{bmatrix}$
\begin{solucion}
Notemos que la matriz de la transformación es
		$$ A = \begin{bmatrix}
		5 &- 3\\
		4& -1\\
		2& 3
		\end{bmatrix}$$
		Por que debemos resolver el sistema
		$$ \begin{bmatrix}
		5 &- 3\\
		4& -1\\
		2& 3
		\end{bmatrix}\begin{bmatrix}
		x_1\\
		x_2
		\end{bmatrix} = \begin{bmatrix}
		13\\
		9\\
		1
		\end{bmatrix}$$
		Viendolo en forma matricial,
		$$\left[
		\begin{array}{cc|c}
		5 &- 3 & 13\\
		4 & -1 & 9\\
		2 & 3 & 1
		\end{array}
		\right] \stackrel{F.E.R}{\wsim} \left[
		\begin{array}{cc|c}
		1 & 0 & 2\\
		0 & 1 & -1\\
		0 & 0 & 0
		\end{array}
		\right]$$
		Por lo que
		$$x_1 = 2, \quad x_2 = -1$$
		Finalmente,
		$$x = \begin{bmatrix}
		2 \\ -1
		\end{bmatrix}$$
\end{solucion}
\item Sea $A$ y $B$ dos matrices de $m\times n$ y $n \times p$, respectivamente. Demuestre que si las columnas de $B$ son linealmente dependientes, entonces también lo son las columnas de $AB$.
\begin{solucion}
P.D. que si $B$ es $L.D.$, $AB$ también lo es\\
		\\
		Si las columnas de $B$ son $L.D.$, entonces
		$$\exists u \neq \vec{0}\ ( Bu = \vec{0})$$
		Luego,
		$$(AB)u = A(Bu) = A\vec{0} = \vec{0}$$
		$$(AB)u = \vec{0}$$
		Recordemos que $u \neq \vec{0}$, por lo que hay una combinación lineal no trivial de las columnas de $AB$ que da $\vec{0}$. Entonces, las columnas de $AB$ son $L.D.$
\end{solucion}
\item En cada caso, determine si la afirmación es verdadera o falsa y justifique su respuesta.
\begin{enumerate}[a)]
\item Si el conjunto $\{v_1, v_2, v_3, v_4\}$ es L.I., entonces $\{v_1, v_2, v_3\}$ también lo es.
\item Si $A$ es una matriz de $3\times 5$ y $T$ es la transformación lineal definida por $T(x) = Ax$, entonces $T$ tiene dominio $\R^3$.
\item Un sistema con más ecuaciones que incógnitas es siempre consistente.
\end{enumerate}
\begin{solucion}

\begin{enumerate}[a)]
\item Si el conjunto $\{v_1, v_2, v_3, v_4\}$ es L.I., entonces $\{v_1, v_2, v_3\}$ también lo es.\\
			\\
			Digamos que $\{v_1, v_2, v_3, v_4\}$ es L.I. y que $\{v_1, v_2, v_3\}$ es L.D.\\
			Entonces, $\exists$ una combinación lineal no trivial tal que 
			$$\alpha_1 v_1 + \alpha_2 v_2 + \alpha_4 v_4 = 0$$
			Definamos ahora $alpha_3 = 0$ y agreguemos a la ecuación anterior $alpha_3 v_3$ (esto será $\vec{0}$ asi que no afecta).\\
			$$\alpha_1 v_1 + \alpha_2 v_2 + \alpha_3 v_3 + \alpha_4 v_4 = 0$$
			Luego, esta también será una combinación lineal no trivial de $\{v_1, v_2, v_3, v_4\}$.\\
			Pero $\{v_1, v_2, v_3, v_4\}$ es L.I., por lo que esto contradice nuestro supuesto.
			
			
			Finalmente, $\{v_1, v_2, v_4\}$ es L.I., por lo que la afirmación es {\bf Verdadera}.
\item Si $A$ es una matriz de $3\times 5$ y $T$ es la transformación lineal definida por $T(x) = Ax$, entonces $T$ tiene dominio $\R^3$.\\
			\\
			$x$ debe tener tantas filas como columnas tiene $A$, es decir 5, por lo que el dominio de $T$ es $\R^5$.
			
			Entonces, la afirmación es {\bf Falsa}.
\item Un sistema con más ecuaciones que incógnitas es siempre consistente.\\
			\\
			Intentemos buscar un caso donde no se cumpla esto.
			$$\begin{array}{rcl}
			x + y & = & 1 \\
			y & = & 1 \\
			y & = & 2
			\end{array} \Longrightarrow
			\begin{bmatrix}
			1 & 1 & 1\\
			0 & 1 & 1\\
			0 & 1 & 2
			\end{bmatrix} \wsim 
			\begin{bmatrix}
			1 & 1 & 1\\
			0 & 1 & 1\\
			0 & 0 & 1
			\end{bmatrix}$$
			Este sistema es claramente inconsistente.
			
			Entonces, la afirmación es {\bf Falsa}.
\end{enumerate}
\end{solucion}
\end{preguntas}
\end{document}