\documentclass[12pt]{article}

\usepackage{fullpage}
\usepackage{graphicx}
\usepackage{amssymb}
\usepackage{amsmath}
\usepackage[none]{hyphenat}
\usepackage{parskip}
\usepackage[spanish]{babel}
\usepackage[utf8]{inputenc}
\usepackage{hyperref}
\usepackage{fancyhdr}
\usepackage{tasks}
\usepackage{mdframed}
\usepackage{xcolor}
\usepackage{pgfplots}
\usepackage[makeroom]{cancel}
\usepackage{multicol}
\usepackage[shortlabels]{enumitem}
\usepackage{stackrel}

\setlength{\headheight}{10pt}
\setlength{\headsep}{10pt}
\pagestyle{fancy}
\rhead{\ayudantia \ - \alumno}

\newcommand*{\mybox}[2]{\colorbox{#1!30}{\parbox{.98\linewidth}{#2}}}

\newenvironment{solucion}
{\begin{mdframed}[backgroundcolor=black!10]
		{\bf Solución:}\\
	}
	{
	\end{mdframed}
}

\newenvironment{alternativas}[1]
{\begin{multicols}{#1}
		\begin{enumerate}[a)]
		}
		{
		\end{enumerate}
	\end{multicols}
}

\newenvironment{preguntas}
{\begin{enumerate}\itemsep12pt
	}
	{
	\end{enumerate}
}

\newcommand{\ayudantia}{{\sc Ayudantía 6}}
\newcommand{\tituloayu}{Repaso I1}
\newcommand{\fecha}{11 de abril de 2018}
\newcommand{\sigla}{MAT1203}
\newcommand{\nombre}{Álgebra Lineal}
\newcommand{\profesor}{Rodrigo Rubio Varas}
\newcommand{\ano}{2018}
\newcommand{\semestre}{1}
\newcommand{\mail}{mat1203@ifcastaneda.cl}
\newcommand{\alumno}{Ignacio Castañeda - \mail}

\newcommand{\ev}{\Big|}
\newcommand{\ra}{\rightarrow}
\newcommand{\lra}{\leftrightarrow}
\newcommand{\N}{\mathbb{N}}
\newcommand{\R}{\mathbb{R}}
\newcommand{\Exp}[1]{\mathcal{E}_{#1}}
\newcommand{\List}[1]{\mathcal{L}_{#1}}
\newcommand{\EN}{\Exp{\N}}
\newcommand{\LN}{\List{\N}}
\newcommand{\comment}[1]{}
\newcommand{\lb}{\\~\\}
\newcommand{\eop}{_{\square}}
\newcommand{\hsig}{\hat{\sigma}}
\newcommand{\widesim}[2][1.5]{
	\mathrel{\overset{#2}{\scalebox{#1}[1]{$\sim$}}}
}
\newcommand{\wsim}{\widesim{}}

\begin{document}
\thispagestyle{empty}

\begin{minipage}{2cm}
	\includegraphics[width=2cm]{../../../../img/logo.pdf}
	\vspace{0.5cm}
\end{minipage}
\begin{minipage}{\linewidth}
	\begin{tabular}{lrl}
		{\scriptsize\sc Pontificia Universidad Catolica de Chile} & \hspace*{0.7in}Curso: &
		\sigla  - \nombre\\
		{\sc Facultad de Matemáticas}&
		Profesor: & \profesor \\
		{\sc Semestre \ano-\semestre} & Ayudante: & {Ignacio Castañeda}\\
		& {Mail:} & \texttt{\mail}
	\end{tabular}
\end{minipage}

\vspace{-10mm}
\begin{center}
	{\LARGE\bf \ayudantia}\\
	\vspace{0.1cm}
	{\tituloayu}\\
	\vspace{0.1cm}
	\fecha\\
	\vspace{0.4cm}
\end{center}

\begin{preguntas}
\item Determine si el siguiente sistema de ecuaciones posee solución única, infinitas soluciones o no tiene solución
	$$
	\begin{array}{rcr}
	x+y-z& = & 1\\
	3x+2y+z& = & 1\\
	5x+3y+4z& = & 2\\
	-2x -y +5z & = & 6
	\end{array}$$
\item Determine condiciones sobre $\alpha$ y $\beta$ de modo que \\
	$\begin{pmatrix}
	\alpha-\beta\\
	\alpha + \beta\\
	2\alpha - \beta
	\end{pmatrix}$ pertenezca a $Gen\left\{\begin{pmatrix}
	-1\\
	-1\\
	1
	\end{pmatrix}\, \begin{pmatrix}
	1\\
	2\\
	1
	\end{pmatrix}, \begin{pmatrix}
	1\\
	3\\
	3
	\end{pmatrix}\right\}$
\item Determine todas las matrices $A$ tales que
	$$ A\begin{bmatrix}
	3\\
	2
	\end{bmatrix}=\begin{bmatrix}
	2\\
	1
	\end{bmatrix}$$
\item Sea $T: \R^2 \ra \R^3$ la transformación lineal definida por
	$$ T\left(\begin{bmatrix}
	x_1\\
	x_2
	\end{bmatrix}\right) = \begin{bmatrix}
	5x_1 - 3x_2\\
	4x_1 -x_2\\
	2x_1 +3x_2
	\end{bmatrix}$$
	encuentre $x = \begin{bmatrix}
	x_1\\
	x_2
	\end{bmatrix}$ tal que $T(x) = \begin{bmatrix}
	13\\
	9\\
	1
	\end{bmatrix}$
\item Sea $A$ y $B$ dos matrices de $m\times n$ y $n \times p$, respectivamente. Demuestre que si las columnas de $B$ son linealmente dependientes, entonces también lo son las columnas de $AB$.
\item En cada caso, determine si la afirmación es verdadera o falsa y justifique su respuesta.
\begin{enumerate}[a)]
\item Si el conjunto $\{v_1, v_2, v_3, v_4\}$ es L.I., entonces $\{v_1, v_2, v_3\}$ también lo es.
\item Si $A$ es una matriz de $3\times 5$ y $T$ es la transformación lineal definida por $T(x) = Ax$, entonces $T$ tiene dominio $\R^3$.
\item Un sistema con más ecuaciones que incógnitas es siempre consistente.
\end{enumerate}
\end{preguntas}
\end{document}