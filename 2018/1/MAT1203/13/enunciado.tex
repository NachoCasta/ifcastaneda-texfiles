\documentclass[12pt]{article}

\usepackage{fullpage}
\usepackage{graphicx}
\usepackage{amssymb}
\usepackage{amsmath}
\usepackage[none]{hyphenat}
\usepackage{parskip}
\usepackage[spanish]{babel}
\usepackage[utf8]{inputenc}
\usepackage{hyperref}
\usepackage{fancyhdr}
\usepackage{tasks}
\usepackage{mdframed}
\usepackage{xcolor}
\usepackage{pgfplots}
\usepackage[makeroom]{cancel}
\usepackage{multicol}
\usepackage[shortlabels]{enumitem}
\usepackage{stackrel}
\usepackage{tkz-tab}
\usepackage{xpatch}
\xpatchcmd{\tkzTabLine}{$0$}{$\bullet$}{}{}

\setlength{\headheight}{10pt}
\setlength{\headsep}{10pt}
\pagestyle{fancy}
\rhead{\ayudantia \ - \alumno}
\tikzset{t style/.style={style=solid}}

\newcommand*{\mybox}[2]{\colorbox{#1!30}{\parbox{.98\linewidth}{#2}}}

\newenvironment{solucion}
{\begin{mdframed}[backgroundcolor=black!10]
		{\bf Solución:}\\
	}
	{
	\end{mdframed}
}

\newenvironment{alternativas}[1]
{\begin{multicols}{#1}
		\begin{enumerate}[a)]
		}
		{
		\end{enumerate}
	\end{multicols}
}

\newenvironment{preguntas}
{\begin{enumerate}\itemsep12pt
	}
	{
	\end{enumerate}
}

\newcommand{\ayudantia}{{\sc Ayudantía 13}}
\newcommand{\tituloayu}{Formas cuadráticas y descomposición en valores singulares}
\newcommand{\fecha}{20 de junio de 2018}
\newcommand{\sigla}{MAT1203}
\newcommand{\nombre}{Álgebra Lineal}
\newcommand{\profesor}{Rodrigo Rubio Varas}
\newcommand{\ano}{2018}
\newcommand{\semestre}{1}
\newcommand{\mail}{mat1203@ifcastaneda.cl}
\newcommand{\alumno}{Ignacio Castañeda - \mail}

\newcommand{\ev}{\Big|}
\newcommand{\ra}{\rightarrow}
\newcommand{\lra}{\leftrightarrow}
\newcommand{\N}{\mathbb{N}}
\newcommand{\R}{\mathbb{R}}
\newcommand{\Exp}[1]{\mathcal{E}_{#1}}
\newcommand{\List}[1]{\mathcal{L}_{#1}}
\newcommand{\EN}{\Exp{\N}}
\newcommand{\LN}{\List{\N}}
\newcommand{\comment}[1]{}
\newcommand{\lb}{\\~\\}
\newcommand{\eop}{_{\square}}
\newcommand{\hsig}{\hat{\sigma}}
\newcommand{\widesim}[2][1.5]{
	\mathrel{\overset{#2}{\scalebox{#1}[1]{$\sim$}}}
}
\newcommand{\wsim}{\widesim{}}
\newcommand{\lh}{\stackrel{L'H}{=}}

\begin{document}
\thispagestyle{empty}

\begin{minipage}{2cm}
	\includegraphics[width=2cm]{../../../../img/logo.pdf}
	\vspace{0.5cm}
\end{minipage}
\begin{minipage}{\linewidth}
	\begin{tabular}{lrl}
		{\scriptsize\sc Pontificia Universidad Catolica de Chile} & \hspace*{0.7in}Curso: &
		\sigla  - \nombre\\
		{\sc Facultad de Matemáticas}&
		Profesor: & \profesor \\
		{\sc Semestre \ano-\semestre} & Ayudante: & {Ignacio Castañeda}\\
		& {Mail:} & \texttt{\mail}
	\end{tabular}
\end{minipage}

\vspace{-10mm}
\begin{center}
	{\LARGE\bf \ayudantia}\\
	\vspace{0.1cm}
	{\tituloayu}\\
	\vspace{0.1cm}
	\fecha\\
	\vspace{0.4cm}
\end{center}

\begin{preguntas}
\item Sea $A = \begin{bmatrix}1 & -3 \\ -3 & 9\end{bmatrix}$
\begin{enumerate}[a)]
\item Encuentre una matriz $L$ cuadrada, triangular inferior con numeros 1 en la diagonal, y una matriz diagonal $D$ tal que $A = LDL^T$.
\item Realice un cambio de variable $x=Py$ que transforme la forma cuadrática
		$$Q\left(\begin{bmatrix}x_1\\x_2\end{bmatrix}\right) = \begin{bmatrix}x_1& x_2\end{bmatrix}A\begin{bmatrix}x_1\\ x_2\end{bmatrix}$$
		en una sin términos con producto cruzado.
\end{enumerate}
\item Sea $A$ la matriz simétrica que representa a la forma cuadrática
	$$Q(x) = 9x_1^2 + 7x_2^2 + 11x_3^2 - 8x_1x_2 + 8x_1x_3$$
\begin{enumerate}[a)]
\item Encuentre una matriz ortogonal $P$ tal que el cambio de variable $x = Py$ transforma la forma $x^TAx$ en una forma cuadrática sin productos cruzados.
\item Escriba la forma cuadrática en las nuevas variables y clasifíquela.
\end{enumerate}
\item Calcule la descomposición en valores singulares de la matriz $A = \begin{bmatrix}4 & 11 & 14\\ 8 & 7 & -2\end{bmatrix}$
\item Sea la matriz
	$A = \begin{bmatrix}
	1 & 0 \\1 & 1 \\
	-1 & 1
	\end{bmatrix}$
	, determine su descomposición en valores singulares.
\end{preguntas}
\end{document}