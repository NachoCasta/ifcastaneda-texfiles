\documentclass[12pt]{article}

\usepackage{fullpage}
\usepackage{graphicx}
\usepackage{amssymb}
\usepackage{amsmath}
\usepackage[none]{hyphenat}
\usepackage{parskip}
\usepackage[spanish]{babel}
\usepackage[utf8]{inputenc}
\usepackage{hyperref}
\usepackage{fancyhdr}
\usepackage{tasks}
\usepackage{mdframed}
\usepackage{xcolor}
\usepackage{pgfplots}
\usepackage[makeroom]{cancel}
\usepackage{multicol}
\usepackage[shortlabels]{enumitem}
\usepackage{stackrel}
\usepackage{tkz-tab}
\usepackage{xpatch}
\xpatchcmd{\tkzTabLine}{$0$}{$\bullet$}{}{}

\setlength{\headheight}{10pt}
\setlength{\headsep}{10pt}
\pagestyle{fancy}
\rhead{\ayudantia \ - \alumno}
\tikzset{t style/.style={style=solid}}

\newcommand*{\mybox}[2]{\colorbox{#1!30}{\parbox{.98\linewidth}{#2}}}

\newenvironment{solucion}
{\begin{mdframed}[backgroundcolor=black!10]
		{\bf Solución:}\\
	}
	{
	\end{mdframed}
}

\newenvironment{alternativas}[1]
{\begin{multicols}{#1}
		\begin{enumerate}[a)]
		}
		{
		\end{enumerate}
	\end{multicols}
}

\newenvironment{preguntas}
{\begin{enumerate}\itemsep12pt
	}
	{
	\end{enumerate}
}

\newcommand{\ayudantia}{{\sc Ayudantía 9}}
\newcommand{\tituloayu}{Espacios y subespacios vectoriales}
\newcommand{\fecha}{9 de mayo de 2018}
\newcommand{\sigla}{MAT1203}
\newcommand{\nombre}{Álgebra Lineal}
\newcommand{\profesor}{Rodrigo Rubio Varas}
\newcommand{\ano}{2018}
\newcommand{\semestre}{1}
\newcommand{\mail}{mat1203@ifcastaneda.cl}
\newcommand{\alumno}{Ignacio Castañeda - \mail}

\newcommand{\ev}{\Big|}
\newcommand{\ra}{\rightarrow}
\newcommand{\lra}{\leftrightarrow}
\newcommand{\N}{\mathbb{N}}
\newcommand{\R}{\mathbb{R}}
\newcommand{\Exp}[1]{\mathcal{E}_{#1}}
\newcommand{\List}[1]{\mathcal{L}_{#1}}
\newcommand{\EN}{\Exp{\N}}
\newcommand{\LN}{\List{\N}}
\newcommand{\comment}[1]{}
\newcommand{\lb}{\\~\\}
\newcommand{\eop}{_{\square}}
\newcommand{\hsig}{\hat{\sigma}}
\newcommand{\widesim}[2][1.5]{
	\mathrel{\overset{#2}{\scalebox{#1}[1]{$\sim$}}}
}
\newcommand{\wsim}{\widesim{}}
\newcommand{\lh}{\stackrel{L'H}{=}}

\begin{document}
\thispagestyle{empty}

\begin{minipage}{2cm}
	\includegraphics[width=2cm]{../../../../img/logo.pdf}
	\vspace{0.5cm}
\end{minipage}
\begin{minipage}{\linewidth}
	\begin{tabular}{lrl}
		{\scriptsize\sc Pontificia Universidad Catolica de Chile} & \hspace*{0.7in}Curso: &
		\sigla  - \nombre\\
		{\sc Facultad de Matemáticas}&
		Profesor: & \profesor \\
		{\sc Semestre \ano-\semestre} & Ayudante: & {Ignacio Castañeda}\\
		& {Mail:} & \texttt{\mail}
	\end{tabular}
\end{minipage}

\vspace{-10mm}
\begin{center}
	{\LARGE\bf \ayudantia}\\
	\vspace{0.1cm}
	{\tituloayu}\\
	\vspace{0.1cm}
	\fecha\\
	\vspace{0.4cm}
\end{center}

\begin{preguntas}
\item Sea 
	$$U=\{p(x) \in \mathbb{P}_2 : p(1) + p(0) = p(-1)\}$$
	Determinar una base de $U$.
\item Use vectores de coordenadas para probar la independencia lineal del conjunto de polinomios $\{1-2t^2-t^3,t+2t^3,1+t-2t^2\}$.
\item Sean $p_1(t) = 1-t^2, p_2(t) = 1+t, p_3(t) = 1+t+t^2$. Se sabe que $\{p_1(t), p_2(t), p_3(t)\}$ es una base de $\mathbb{P}_2$.
\begin{enumerate}[a)]
\item Exprese los polinomios $f(t) = 3-5t + 2t^2$ y $g(t) = 1-3t$ como combinaciones lineales de $\{p_1(t), p_2(t), p_3(t)\}$.
\item Use los vectores de coordenadas encontrados en la parte anterior para determinar si el conjunto $\{p_1(t), f(t), g(t)\}$ es L.I. o L.D.
\end{enumerate}
\item Sea $A$ una matriz tal que $dim(Nul(A)) = 3$ y
	$$A^T \begin{pmatrix}
	1 \\ -2 \\ 1 \\ 1
	\end{pmatrix} = \begin{pmatrix}
	0 \\ 0 \\ 0 \\ 0 \\ 0
	\end{pmatrix}, \quad A^T \begin{pmatrix}
	2 \\ -1 \\ -1 \\ 1
	\end{pmatrix} = \begin{pmatrix}
	0 \\ 0 \\ 0 \\ 0 \\ 0
	\end{pmatrix}$$
\begin{enumerate}[a)]
\item Determine las dimensiones de $Col(A)$, de $Fila(A)$ y de $Nul(A^T)$
\item Determine, en los casos que sea posible, bases para los espacios $Nul(A^T)$ y $Fila(A)$.
\end{enumerate}
\end{preguntas}
\end{document}