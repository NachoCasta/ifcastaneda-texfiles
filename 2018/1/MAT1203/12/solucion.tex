\documentclass[12pt]{article}

\usepackage{fullpage}
\usepackage{graphicx}
\usepackage{amssymb}
\usepackage{amsmath}
\usepackage[none]{hyphenat}
\usepackage{parskip}
\usepackage[spanish]{babel}
\usepackage[utf8]{inputenc}
\usepackage{hyperref}
\usepackage{fancyhdr}
\usepackage{tasks}
\usepackage{mdframed}
\usepackage{xcolor}
\usepackage{pgfplots}
\usepackage[makeroom]{cancel}
\usepackage{multicol}
\usepackage[shortlabels]{enumitem}
\usepackage{stackrel}
\usepackage{tkz-tab}
\usepackage{xpatch}
\xpatchcmd{\tkzTabLine}{$0$}{$\bullet$}{}{}

\setlength{\headheight}{10pt}
\setlength{\headsep}{10pt}
\pagestyle{fancy}
\rhead{\ayudantia \ - \alumno}
\tikzset{t style/.style={style=solid}}

\newcommand*{\mybox}[2]{\colorbox{#1!30}{\parbox{.98\linewidth}{#2}}}

\newenvironment{solucion}
{\begin{mdframed}[backgroundcolor=black!10]
		{\bf Solución:}\\
	}
	{
	\end{mdframed}
}

\newenvironment{alternativas}[1]
{\begin{multicols}{#1}
		\begin{enumerate}[a)]
		}
		{
		\end{enumerate}
	\end{multicols}
}

\newenvironment{preguntas}
{\begin{enumerate}\itemsep12pt
	}
	{
	\end{enumerate}
}

\newcommand{\ayudantia}{{\sc Ayudantía 12}}
\newcommand{\tituloayu}{Conjuntos ortogonales, proyecciones y mínimos cuadrados}
\newcommand{\fecha}{6 de junio de 2018}
\newcommand{\sigla}{MAT1203}
\newcommand{\nombre}{Álgebra Lineal}
\newcommand{\profesor}{Rodrigo Rubio Varas}
\newcommand{\ano}{2018}
\newcommand{\semestre}{1}
\newcommand{\mail}{mat1203@ifcastaneda.cl}
\newcommand{\alumno}{Ignacio Castañeda - \mail}

\newcommand{\ev}{\Big|}
\newcommand{\ra}{\rightarrow}
\newcommand{\lra}{\leftrightarrow}
\newcommand{\N}{\mathbb{N}}
\newcommand{\R}{\mathbb{R}}
\newcommand{\Exp}[1]{\mathcal{E}_{#1}}
\newcommand{\List}[1]{\mathcal{L}_{#1}}
\newcommand{\EN}{\Exp{\N}}
\newcommand{\LN}{\List{\N}}
\newcommand{\comment}[1]{}
\newcommand{\lb}{\\~\\}
\newcommand{\eop}{_{\square}}
\newcommand{\hsig}{\hat{\sigma}}
\newcommand{\widesim}[2][1.5]{
	\mathrel{\overset{#2}{\scalebox{#1}[1]{$\sim$}}}
}
\newcommand{\wsim}{\widesim{}}
\newcommand{\lh}{\stackrel{L'H}{=}}

\begin{document}
\thispagestyle{empty}

\begin{minipage}{2cm}
	\includegraphics[width=2cm]{../../../../img/logo.pdf}
	\vspace{0.5cm}
\end{minipage}
\begin{minipage}{\linewidth}
	\begin{tabular}{lrl}
		{\scriptsize\sc Pontificia Universidad Catolica de Chile} & \hspace*{0.7in}Curso: &
		\sigla  - \nombre\\
		{\sc Facultad de Matemáticas}&
		Profesor: & \profesor \\
		{\sc Semestre \ano-\semestre} & Ayudante: & {Ignacio Castañeda}\\
		& {Mail:} & \texttt{\mail}
	\end{tabular}
\end{minipage}

\vspace{-10mm}
\begin{center}
	{\LARGE\bf \ayudantia}\\
	\vspace{0.1cm}
	{\tituloayu}\\
	\vspace{0.1cm}
	\fecha\\
	\vspace{0.4cm}
\end{center}

\begin{preguntas}
\item Sean $u, v$ dos vectores ortogonales en $\R^n$ tales que $||u|| = 1, ||v|| = \sqrt[]{3/2}$. Demuestre que el conjunto 
	$$B = \{u-v, 3u+2v\}$$
	es ortogonal y encuentre las coordenadas del vector $4u - 9v$ respecto al conjunto $B$.
\begin{solucion}
Para demostrar que $B$ es ortogonal, debemos demostrar que todos sus elementos son ortogonales entre si, es decir, debemos demoestrar que
		$$(u-v) \cdot (3u+2v) = 0$$
		Notemos que como $u$ y $v$ son ortogonales entre si, $u \cdot v = 0$. Luego,
		$$\begin{array}{rcl}
		(u-v) \cdot (3u+2v) & = & 3u \cdot u + 2 u \cdot v - 3u \cdot v - 2v \cdot v\\
		& = & 3||u||^2 + 2 (u \cdot v) - 3(u \cdot v) - 2||v||^2 \\
		& = & 3 + 2 \cdot 0 - 3 \cdot 0 - 2(\ \sqrt[]{3/2})^2 \\
		& = & 3 - 3 \\
		& = & 0
		\end{array}$$
		$$\blacksquare$$
		Ahora, para buscar las coordenadas de $4u - 9v$ respecto al conjunto $B$, debemos buscar $\alpha, \beta \in \R$, tal que
		$$\alpha(u-v) + \beta(3u+2v) = 4u-9v$$
		Reordenando,
		$$(\alpha+3\beta)u + (2\beta -\alpha)v= 4u -9v$$
		Luego, debemos resolver el sistema
		$$\begin{array}{rcl}
		\alpha+3\beta & = & 4\\
		-\alpha+2\beta & = & -9
		\end{array}$$
		Resolviendolo, obtenemos que
		$$\alpha = 7 \quad y \quad \beta = -1$$
		Finalmente,
		$$[4u-9v]_B = \begin{pmatrix}
		7 \\ -1
		\end{pmatrix}$$
\end{solucion}
\item Sea 
	$U = Gen\left\{
	\begin{bmatrix}1\\0\\1\\1\end{bmatrix},
	\begin{bmatrix}0\\1\\2\\1\end{bmatrix},
	\begin{bmatrix}1\\1\\3\\2\end{bmatrix}
	\right\}$.
\begin{enumerate}[a)]
\item Encuentre una base ortonormal de $U$.
\item Encuentre la distancia de $b = 
		\begin{bmatrix}1\\1\\1\\1\end{bmatrix}$ a $U$.
\end{enumerate}
\begin{solucion}

\begin{enumerate}[a)]
\item Encuentre una base ortonormal de $U$.\\
			\\
			En primer lugar, debemos buscar una base común y corriente de $U$. Para esto, tomemos los vectores L.I. del conjunto que lo genera. Notemos que
			$$\begin{bmatrix}
			1 & 0 & 1 \\
			0 & 1 & 1 \\
			1 & 2 & 3 \\
			1 & 1 & 2
			\end{bmatrix} \sim
			\begin{bmatrix}
			1 & 0 & 1 \\
			0 & 1 & 1 \\
			0 & 0 & 0 \\
			0 & 0 & 0
			\end{bmatrix}$$
			Por lo que concluimos que los dos vectores del generado son L.I. Entonces, una base de $U$ corresponde a
			$$B = \left\{ \begin{pmatrix}
			1 \\ 0 \\ 1 \\ 1
			\end{pmatrix},\begin{pmatrix}
			0 \\ 1 \\ 2 \\ 1
			\end{pmatrix} \right\}$$
			Recordemos que Gramm-Schmidt forma una base ortogonal a partir de una base cualquiera y funciona de la siguiente forma:
			
			Sea $B = \{v_1, v_2, \dots \}$ una base cualquiera de un espacio vectorial,
			
			$$\begin{array}{rcl}
			u_1 & = & v_1 \\\\
			u_2 & = & v_2 - \dfrac{v_2u_1}{u_1u_1}u_1 \\\\
			u_3 & = & v_3 - \dfrac{v_3u_1}{u_1u_1}u_1 - \dfrac{v_3u_2}{u_2u_2}u_2\\
			& \vdots & 			
			\end{array}$$
			Luego, $B^{\perp} = \{u_1, u_2, \dots \}$ es una base ortogonal de ese espacio vectorial.
			
			Hagamos esto con nuestra base:
			
			En primer lugar,
			$$u_1 = v_1 = \begin{pmatrix}
			1 \\ 0 \\ 1 \\ 1
			\end{pmatrix}$$
			
			Luego,
			$$u_2 = v_2 - \dfrac{v_2u_1}{u_1u_1}u_1
			= \begin{pmatrix}
			0 \\ 1 \\ 2 \\ 1
			\end{pmatrix} - \dfrac{\begin{pmatrix}
				0 \\ 1 \\ 2 \\ 1
				\end{pmatrix}\begin{pmatrix}
				1 \\ 0 \\ 1 \\ 1
				\end{pmatrix}}{\begin{pmatrix}
				1 \\ 0 \\ 1 \\ 1
				\end{pmatrix}\begin{pmatrix}
				1 \\ 0 \\ 1 \\ 1
				\end{pmatrix}}\begin{pmatrix}
			1 \\ 0 \\ 1 \\ 1
			\end{pmatrix} 
			= \begin{pmatrix}
			0 \\ 1 \\ 2 \\ 1
			\end{pmatrix} - \dfrac{3}{3}\begin{pmatrix}
			1 \\ 0 \\ 1 \\ 1
			\end{pmatrix} 
			= \begin{pmatrix}
			-1 \\ 1 \\ 1 \\ 0
			\end{pmatrix} $$
			De esta forma,
			$$B^{\perp} = \left\{\begin{pmatrix}
			1 \\ 0 \\ 1 \\ 1
			\end{pmatrix}, \begin{pmatrix}
			-1 \\ 1 \\ 1 \\ 0
			\end{pmatrix}\right\}$$
			Sin embargo, necesitamos una base ortonormal. Para esto, basta con dividir todos los vectores de la base por su norma, obteniendo
			$$\hat{B}^{\perp} = \left\{\begin{pmatrix}
			1/\ \sqrt[]{3} \\ 0 \\ 1/\ \sqrt[]{3} \\ 1/\ \sqrt[]{3}
			\end{pmatrix}, \begin{pmatrix}
			-1/\ \sqrt[]{3} \\ 1/\ \sqrt[]{3} \\ 1/\ \sqrt[]{3} \\ 0
			\end{pmatrix}\right\}$$
\item Encuentre la distancia de $b = 
			\begin{bmatrix}1\\1\\1\\1\end{bmatrix}$ a $U$.
			
			Sea $A$ una matriz con los vectores de una base de $U$ en sus columnas y $x$ un vector coordenada de un elemento cualquier perteneciente a $U$, entonces la distancia entre ese elemento y $b$ corresponde a $||Ax-b||$.
			
			Luego, la distancia entre $b$ y $U$ corresponde a la distancia más pequeña entre $b$ y un elemento de $U$, es decir,
			$$min||Ax-b||$$
			Resolver este problema es equivalente a resolver el sistema
			$$A^TAx = A^Tb$$
			Utilicemos
			$$A = \begin{bmatrix}
			1 & 0 \\
			0 & 1 \\
			1 & 2 \\
			1 & 1
			\end{bmatrix}$$
			Entonces,
			$$A^TA = \begin{bmatrix}
			3 & 3 \\
			3 & 6
			\end{bmatrix}, \quad A^Tb = \begin{pmatrix}
			3 \\ 4
			\end{pmatrix}$$
			Por lo que tenemos que resolver el sistema
			$$\begin{bmatrix}
			3 & 3 \\
			3 & 6
			\end{bmatrix}x = \begin{pmatrix}
			3 \\ 4
			\end{pmatrix} \ra x = \begin{pmatrix}
			\frac{2}{3}\\ \frac{1}{3}
			\end{pmatrix}$$
			Finalmente, la distancia corresponde a
			$$||Ax-b|| = \left|\left|\begin{bmatrix}
			1 & 0 \\
			0 & 1 \\
			1 & 2 \\
			1 & 1
			\end{bmatrix}\begin{pmatrix}
			\frac{2}{3}\\ \frac{1}{3}
			\end{pmatrix} - \begin{pmatrix}1\\1\\1\\1\end{pmatrix}\right|\right|
			= \left|\left|\begin{pmatrix}
			-\frac{1}{3}\\
			-\frac{2}{3}\\
			\frac{1}{3}\\
			0
			\end{pmatrix}\right|\right| = \sqrt[]{\dfrac{2}{3}}$$
\end{enumerate}
\end{solucion}
\item Demuestre que si $P$ es una matriz ortogonal de  $n \times n$, entonces para todo $x, y \in \R^n$ se tiene que $Px \cdot Py = x \cdot y$
\begin{solucion}
Como $P$ es ortogonal, se cumple que $P^TP = I$. Luego,
		$$Px \cdot Py = (Px)^T(Py) = x^TP^TPy = x^TIx= x^Ty = x \cdot y$$
\end{solucion}
\item Diagonalice ortogonalmente
	$$M = \begin{bmatrix}
	1 & 0 & 1 \\
	0 & 2 & 0 \\
	1 & 0 & 1
	\end{bmatrix}$$
\begin{solucion}
Para hacer esto debemos hacer un procedimiento similar al que realizamos cuando queremos diagonalizar una matriz, es decir, buscar $P$ y $D$ tal que $M = PDP^{-1}$, con la diferencia de que en este caso $P$ debe ser ortogonal.
		
		Comenzamos buscando los valores propios, con lo que obtendremos
		$$\lambda_1 = 0 \ra \text{multiplicidad 1}$$
		$$\lambda_2 = 2 \ra \text{multiplicidad 2}$$
		Luego, buscamos los vectores propios asociados a cada uno de estos valores propios, los que son
		$$\lambda_1 = 0 \ra v_1 = \begin{pmatrix}
		1 \\ 0 \\ -1
		\end{pmatrix}$$
		$$\lambda_2 = 0 \ra v_2 = \begin{pmatrix}
		0 \\ 1 \\ 0
		\end{pmatrix}, v_3 = \begin{pmatrix}
		1 \\ 0 \\ 1
		\end{pmatrix}$$
		Notemos que todos estos vectores son ortogonales entre si, sin embargo, necesitamos que sean ortonormales. Recordemos que un vector propio será cualquier multiplo de los vectores calculados anteriormente, por lo que para obtener vectores ortonormales, basta con dividir cada uno por su modulo. De esta forma, nuestros nuevos vectores serán
		$$v_1 = \begin{pmatrix}
		1/\ \sqrt[]{2} \\ 0 \\ -1/\ \sqrt[]{2}
		\end{pmatrix}, v_2 = \begin{pmatrix}
		0 \\ 1 \\ 0
		\end{pmatrix}, v_3 = \begin{pmatrix}
		1/\ \sqrt[]{2} \\ 0 \\ 1/\ \sqrt[]{2}
		\end{pmatrix}$$
		Luego, $M$ se diagonaliza con
		$$P = \begin{bmatrix}
		1/\ \sqrt[]{2} & 0 & 1/\ \sqrt[]{2} \\
		0 & 1 & 0 \\
		-1/\ \sqrt[]{2} & 0 & 1/\ \sqrt[]{2}
		\end{bmatrix}, \quad D = \begin{bmatrix}
		0 & 0 & 0 \\
		0 & 2 & 0 \\
		0 & 0 & 2
		\end{bmatrix}$$
\end{solucion}
\item Un cierto experimento genera los datos $(1,3)$, $(2,5)$ y $(3,4)$. Describa el modelo que da un ajuste de mínimos cuadrados de esos puntos mediante una recta de la forma $y = \beta_0 +  \beta_1x$
\begin{solucion}
Los valores del experimento se pueden ver en las siguiente tabla:\\
		\\
		\begin{center}
		\begin{tabular}{|l|l|}
			\hline
			\textbf{y} & \textbf{x} \\ \hline
			3          & 1          \\ \hline
			5          & 2          \\ \hline
			4          & 3          \\ \hline
		\end{tabular}
		\end{center}
		Estamos buscando $\beta_0, \beta_1$ tal que
		$$y = \beta_0 +  \beta_1x$$
		sea un ajuste por mínimos cuadrados. Esto lo podemos escribir como
		$$Y = X\beta$$
		donde
		$$Y = \begin{pmatrix}
		3 \\ 5 \\ 4
		\end{pmatrix}, \quad 
		X = \begin{bmatrix}
		1 & 1 \\
		1 & 2 \\
		1 & 3
		\end{bmatrix}, \quad
		\beta = \begin{pmatrix}
		\beta_0 \\ \beta_1
		\end{pmatrix}$$
		Para encontrar $\beta$, debemos resolver el sistema
		$$X^TX\beta = X^TY \ra \beta = (X^TX)^{-1}X^TY$$
		$$(X^TX)^{-1} = \left(\begin{bmatrix}
		1 & 1 & 1\\
		1 & 2 & 3
		\end{bmatrix}
		\begin{bmatrix}
		1 & 1 \\
		1 & 2 \\
		1 & 3
		\end{bmatrix}\right)^{-1}
		 = 
		 \begin{bmatrix}
		 3 & 6 \\
		 6 & 14
		 \end{bmatrix}^{-1} = \begin{bmatrix}
		 \frac{7}{3} & -1 \\
		 -1 & \frac{1}{2}
		 \end{bmatrix}$$
		 $$ X^TY = \begin{bmatrix}
		 1 & 1 & 1\\
		 1 & 2 & 3
		 \end{bmatrix} \begin{pmatrix}
		 3 \\ 5 \\ 4
		 \end{pmatrix} = 
		 \begin{pmatrix}
		 12 \\ 25
		 \end{pmatrix}$$
		 Luego,
		 $$\beta = (X^TX)^{-1}X^TY
		 = \begin{bmatrix}
		 \frac{7}{3} & -1 \\
		 -1 & \frac{1}{2}
		 \end{bmatrix}\begin{pmatrix}
		 12 \\ 25
		 \end{pmatrix}
		 = \begin{pmatrix}
		 3 \\ \frac{1}{2}
		 \end{pmatrix}
		 $$
		 Finalmente, el ajuste por mínimos cuadrados es
		 $$y = 3 + \dfrac{1}{2}x$$
\end{solucion}
\end{preguntas}
\end{document}