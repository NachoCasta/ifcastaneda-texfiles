\documentclass[12pt]{article}

\usepackage{fullpage}
\usepackage{graphicx}
\usepackage{amssymb}
\usepackage{amsmath}
\usepackage[none]{hyphenat}
\usepackage{parskip}
\usepackage[spanish]{babel}
\usepackage[utf8]{inputenc}
\usepackage{hyperref}
\usepackage{fancyhdr}
\usepackage{tasks}
\usepackage{mdframed}
\usepackage{xcolor}
\usepackage{pgfplots}
\usepackage[makeroom]{cancel}
\usepackage{multicol}
\usepackage[shortlabels]{enumitem}
\usepackage{stackrel}

\setlength{\headheight}{10pt}
\setlength{\headsep}{10pt}
\pagestyle{fancy}
\rhead{\ayudantia \ - \alumno}

\newcommand*{\mybox}[2]{\colorbox{#1!30}{\parbox{.98\linewidth}{#2}}}

\newenvironment{solucion}
{\begin{mdframed}[backgroundcolor=black!10]
		{\bf Solución:}\\
	}
	{
	\end{mdframed}
}

\newenvironment{alternativas}[1]
{\begin{multicols}{#1}
		\begin{enumerate}[a)]
		}
		{
		\end{enumerate}
	\end{multicols}
}

\newenvironment{preguntas}
{\begin{enumerate}\itemsep12pt
	}
	{
	\end{enumerate}
}

\newcommand{\ayudantia}{{\sc Ayudantía 3}}
\newcommand{\tituloayu}{Conjunto solución, independencia lineal y transformaciones lineales}
\newcommand{\fecha}{21 de marzo de 2018}
\newcommand{\sigla}{MAT1203}
\newcommand{\nombre}{Álgebra Lineal}
\newcommand{\profesor}{Rodrigo Rubio Varas}
\newcommand{\ano}{2018}
\newcommand{\semestre}{1}
\newcommand{\mail}{mat1203@ifcastaneda.cl}
\newcommand{\alumno}{Ignacio Castañeda - \mail}

\newcommand{\ev}{\Big|}
\newcommand{\ra}{\rightarrow}
\newcommand{\lra}{\leftrightarrow}
\newcommand{\N}{\mathbb{N}}
\newcommand{\R}{\mathbb{R}}
\newcommand{\Exp}[1]{\mathcal{E}_{#1}}
\newcommand{\List}[1]{\mathcal{L}_{#1}}
\newcommand{\EN}{\Exp{\N}}
\newcommand{\LN}{\List{\N}}
\newcommand{\comment}[1]{}
\newcommand{\lb}{\\~\\}
\newcommand{\eop}{_{\square}}
\newcommand{\hsig}{\hat{\sigma}}
\newcommand{\widesim}[2][1.5]{
	\mathrel{\overset{#2}{\scalebox{#1}[1]{$\sim$}}}
}
\newcommand{\wsim}{\widesim{}}

\begin{document}
\thispagestyle{empty}

\begin{minipage}{2cm}
	\includegraphics[width=2cm]{../../../../img/logo.pdf}
	\vspace{0.5cm}
\end{minipage}
\begin{minipage}{\linewidth}
	\begin{tabular}{lrl}
		{\scriptsize\sc Pontificia Universidad Catolica de Chile} & \hspace*{0.7in}Curso: &
		\sigla  - \nombre\\
		{\sc Facultad de Matemáticas}&
		Profesor: & \profesor \\
		{\sc Semestre \ano-\semestre} & Ayudante: & {Ignacio Castañeda}\\
		& {Mail:} & \texttt{\mail}
	\end{tabular}
\end{minipage}

\vspace{-10mm}
\begin{center}
	{\LARGE\bf \ayudantia}\\
	\vspace{0.1cm}
	{\tituloayu}\\
	\vspace{0.1cm}
	\fecha\\
	\vspace{0.4cm}
\end{center}

\begin{preguntas}
\item Sean los vectores 
	$$
	v_1 = \begin{pmatrix}
	1\\
	1\\
	2
	\end{pmatrix};\qquad
	v_2 = \begin{pmatrix}
	2\\
	2\\
	-3
	\end{pmatrix}; \qquad
	v_3 = \begin{pmatrix}
	0\\
	1\\
	3
	\end{pmatrix}$$
	determinar si $Gen\{v_1, v_2, v_3\} = R^3$.
\item Sea la matriz $A=
	\begin{bmatrix}
	3 & 5 & -4\\
	-3 & -2 & 4\\
	6 & 1 & -8
	\end{bmatrix}$
\begin{enumerate}[a)]
\item Determinar el conjunto solución de su sistema homogeneo.
\item Describir todas las soluciones de $Ax=b$ con $b=
		\begin{pmatrix}
		7\\
		-1\\
		-4
		\end{pmatrix}$ 
\end{enumerate}
\item Sea la matriz $A=
	\begin{bmatrix}
	1 & 3 & 4\\
	-4 & 2 & -6\\
	-3 & -2 & -7
	\end{bmatrix}
	$ y $b$ un vector en $R^3$. ¿La ecuación $Ax=b$ es consistente para todo $b$?
\item Describa todas las soluciones del siguiente sistema de ecuaciones y comparelas con las de su sistema homogeneo
	$$
	\begin{array}{rcr}
	x_1 +2x_2-3x_3& = & 5\\
	2x_1 + x_2 - 3x_3& = & 13\\
	-x_1 + x_2 & = & -8
	\end{array}$$
\item Sean $\{u, v, w\}$ un conjunto de vectores linealmente independientes. Demuestre que el conjunto $\{u+v, u+2w, v+3u+w\}$ es linealmente independiente.
\item Sea $L: P_2(\R) \ra \R^2$ una transformación lineal tal que
	$$L(1+x) = \begin{pmatrix}
	1\\
	1
	\end{pmatrix}, \quad L(1+x+x^2) = \begin{pmatrix}
	1\\
	-1
	\end{pmatrix}\ y \ L(1+2x) = \begin{pmatrix}
	1\\
	2
	\end{pmatrix}$$
	determine $L(a+bx+cx^2)$ para todo $a,b,c \in \R$.
\end{preguntas}
\end{document}