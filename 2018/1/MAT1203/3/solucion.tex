\documentclass[12pt]{article}

\usepackage{fullpage}
\usepackage{graphicx}
\usepackage{amssymb}
\usepackage{amsmath}
\usepackage[none]{hyphenat}
\usepackage{parskip}
\usepackage[spanish]{babel}
\usepackage[utf8]{inputenc}
\usepackage{hyperref}
\usepackage{fancyhdr}
\usepackage{tasks}
\usepackage{mdframed}
\usepackage{xcolor}
\usepackage{pgfplots}
\usepackage[makeroom]{cancel}
\usepackage{multicol}
\usepackage[shortlabels]{enumitem}
\usepackage{stackrel}

\setlength{\headheight}{10pt}
\setlength{\headsep}{10pt}
\pagestyle{fancy}
\rhead{\ayudantia \ - \alumno}

\newcommand*{\mybox}[2]{\colorbox{#1!30}{\parbox{.98\linewidth}{#2}}}

\newenvironment{solucion}
{\begin{mdframed}[backgroundcolor=black!10]
		{\bf Solución:}\\
	}
	{
	\end{mdframed}
}

\newenvironment{alternativas}[1]
{\begin{multicols}{#1}
		\begin{enumerate}[a)]
		}
		{
		\end{enumerate}
	\end{multicols}
}

\newenvironment{preguntas}
{\begin{enumerate}\itemsep12pt
	}
	{
	\end{enumerate}
}

\newcommand{\ayudantia}{{\sc Ayudantía 3}}
\newcommand{\tituloayu}{Conjunto solución, independencia lineal y transformaciones lineales}
\newcommand{\fecha}{21 de marzo de 2018}
\newcommand{\sigla}{MAT1203}
\newcommand{\nombre}{Álgebra Lineal}
\newcommand{\profesor}{Rodrigo Rubio Varas}
\newcommand{\ano}{2018}
\newcommand{\semestre}{1}
\newcommand{\mail}{mat1203@ifcastaneda.cl}
\newcommand{\alumno}{Ignacio Castañeda - \mail}

\newcommand{\ev}{\Big|}
\newcommand{\ra}{\rightarrow}
\newcommand{\lra}{\leftrightarrow}
\newcommand{\N}{\mathbb{N}}
\newcommand{\R}{\mathbb{R}}
\newcommand{\Exp}[1]{\mathcal{E}_{#1}}
\newcommand{\List}[1]{\mathcal{L}_{#1}}
\newcommand{\EN}{\Exp{\N}}
\newcommand{\LN}{\List{\N}}
\newcommand{\comment}[1]{}
\newcommand{\lb}{\\~\\}
\newcommand{\eop}{_{\square}}
\newcommand{\hsig}{\hat{\sigma}}
\newcommand{\widesim}[2][1.5]{
	\mathrel{\overset{#2}{\scalebox{#1}[1]{$\sim$}}}
}
\newcommand{\wsim}{\widesim{}}

\begin{document}
\thispagestyle{empty}

\begin{minipage}{2cm}
	\includegraphics[width=2cm]{../../../../img/logo.pdf}
	\vspace{0.5cm}
\end{minipage}
\begin{minipage}{\linewidth}
	\begin{tabular}{lrl}
		{\scriptsize\sc Pontificia Universidad Catolica de Chile} & \hspace*{0.7in}Curso: &
		\sigla  - \nombre\\
		{\sc Facultad de Matemáticas}&
		Profesor: & \profesor \\
		{\sc Semestre \ano-\semestre} & Ayudante: & {Ignacio Castañeda}\\
		& {Mail:} & \texttt{\mail}
	\end{tabular}
\end{minipage}

\vspace{-10mm}
\begin{center}
	{\LARGE\bf \ayudantia}\\
	\vspace{0.1cm}
	{\tituloayu}\\
	\vspace{0.1cm}
	\fecha\\
	\vspace{0.4cm}
\end{center}

\begin{preguntas}
\item Sean los vectores 
	$$
	v_1 = \begin{pmatrix}
	1\\
	1\\
	2
	\end{pmatrix};\qquad
	v_2 = \begin{pmatrix}
	2\\
	2\\
	-3
	\end{pmatrix}; \qquad
	v_3 = \begin{pmatrix}
	0\\
	1\\
	3
	\end{pmatrix}$$
	determinar si $Gen\{v_1, v_2, v_3\} = R^3$.
\begin{solucion}
Para ver si estos tres vectores generan $\R^3$, tiene que pasar que todos sean L.I. entre si. Una forma de ver esto, es formar una matriz con los vectores y ver la cantidad de pivotes que hay, es decir
		$$\begin{bmatrix}
			\\
			v_1 & v_2 & v_3\\
			&&
		\end{bmatrix} \sim 
		\begin{bmatrix}
		1 & 2 & 0\\
		1 & 2 & 1\\
		2 & -3 & 3
		\end{bmatrix} \sim
		\begin{bmatrix}
		1 & 2 & 0\\
		0 & 0 & 1\\
		0 & -7 & 3
		\end{bmatrix} \sim
		\begin{bmatrix}
		1 & 2 & 0\\
		0 & -7 & 3\\
		0 & 0 & 1
		\end{bmatrix}$$
		Como la matriz tiene 3 pivotes, quiere decir que hay 3 vectores L.I., es decir, todos son linealmente independientes entre si, formando asi $\R^3$
\end{solucion}
\item Sea la matriz $A=
	\begin{bmatrix}
	3 & 5 & -4\\
	-3 & -2 & 4\\
	6 & 1 & -8
	\end{bmatrix}$
\begin{enumerate}[a)]
\item Determinar el conjunto solución de su sistema homogeneo.
\item Describir todas las soluciones de $Ax=b$ con $b=
		\begin{pmatrix}
		7\\
		-1\\
		-4
		\end{pmatrix}$ 
\end{enumerate}
\begin{solucion}

\begin{enumerate}[a)]
\item Determinar el conjunto solución de su sistema homogeneo.\\\\
			El sistema homogeneo corresponde a la ecuación $Ax=\vec{0}$, por lo que nuestra matriz aumentada sería
			$$\left[
			\begin{array}{ccc|c}
			3 & 5 & -4 & 0\\
			-3 & -2 & 4 & 0\\
			6 & 1 & -8 & 0
			\end{array}
			\right] \sim \left[
			\begin{array}{ccc|c}
			3 & 5 & -4 & 0\\
			0 & 3 & 0 & 0\\
			0 & -9 & 0 & 0
			\end{array}
			\right] \sim \left[
			\begin{array}{ccc|c}
			3 & 5 & -4 & 0\\
			0 & 3 & 0 & 0\\
			0 & 0 & 0 & 0
			\end{array}
			\right]$$
			$$\sim \left[
			\begin{array}{ccc|c}
			1 & \frac{5}{3} & -\frac{4}{3} & 0\\
			0 & 1 & 0 & 0\\
			0 & 0 & 0 & 0
			\end{array}
			\right] \sim \left[
			\begin{array}{ccc|c}
			1 & 0 & -\frac{4}{3} & 0\\
			0 & 1 & 0 & 0\\
			0 & 0 & 0 & 0
			\end{array}
			\right]$$
			Esto corresponde al sistema
			$$\begin{array}{rcl}
			x_1 - \frac{4}{3}x_3 & = & 0\\
			x_2 & = & 0\\
			0 & = & 0
			\end{array} \ra \begin{array}{rcl}
			x_1 & = & \frac{4}{3}x_3\\
			x_2 & = & 0\\
			x_3 & = & x_3
			\end{array} \ra x = \begin{pmatrix}
			\frac{4}{3}x_3\\
			0\\
			x_3
			\end{pmatrix} = x_3\begin{pmatrix}
			\frac{4}{3}\\
			0\\
			1
			\end{pmatrix}$$
			Luego, la solución del sistema homogeneo es
			$$S = Gen\left\{\begin{pmatrix}
			\frac{4}{3}\\
			0\\
			1
			\end{pmatrix}\right\}$$
\item Describir todas las soluciones de $Ax=b$ con $b=
			\begin{pmatrix}
			7\\
			-1\\
			-4
			\end{pmatrix}$ \\\\
			Para esto, hacemos lo mismo que en la parte a), pero aumentando la matriz por el vector $b$, es decir
			$$\left[
			\begin{array}{ccc|c}
			3 & 5 & -4 & 7\\
			-3 & -2 & 4 & -1\\
			6 & 1 & -8 & -4
			\end{array}
			\right] \sim \left[
			\begin{array}{ccc|c}
			3 & 5 & -4 & 7\\
			0 & 3 & 0 & 6\\
			0 & -9 & 0 & -18
			\end{array}
			\right] \sim \left[
			\begin{array}{ccc|c}
			3 & 5 & -4 & 7\\
			0 & 3 & 0 & 6\\
			0 & 0 & 0 & 0
			\end{array}
			\right]$$
			$$\sim \left[
			\begin{array}{ccc|c}
			1 & \frac{5}{3} & -\frac{4}{3} & \frac{7}{3}\\
			0 & 1 & 0 & 2\\
			0 & 0 & 0 & 0
			\end{array}
			\right] \sim \left[
			\begin{array}{ccc|c}
			1 & 0 & -\frac{4}{3} & -1\\
			0 & 1 & 0 & 2\\
			0 & 0 & 0 & 0
			\end{array}
			\right]$$
			Esto corresponde al sistema
			$$\begin{array}{rcl}
			x_1 - \frac{4}{3}x_3 & = & -1\\
			x_2 & = & 2\\
			0 & = & 0
			\end{array} \ra \begin{array}{rcl}
			x_1 & = & -1 + \frac{4}{3}x_3\\
			x_2 & = & 2\\
			x_3 & = & x_3
			\end{array}$$
			$$ x = \begin{pmatrix}
			-1 + \frac{4}{3}x_3\\
			2\\
			x_3
			\end{pmatrix} = \begin{pmatrix}
			-1\\2\\0
			\end{pmatrix} + x_3\begin{pmatrix}
			\frac{4}{3}\\
			0\\
			1
			\end{pmatrix}$$
			Luego, la solución del sistema $Ax = b$ es
			$$S = \begin{pmatrix}
			-1\\2\\0
			\end{pmatrix} + Gen\left\{\begin{pmatrix}
			\frac{4}{3}\\
			0\\
			1
			\end{pmatrix}\right\}$$
			Finalmente, podriamos amplificar el vector del generado por 3, para que quede más bonito, con lo que
			$$S = \begin{pmatrix}
			-1\\2\\0
			\end{pmatrix} + Gen\left\{\begin{pmatrix}
			4\\
			0\\
			3
			\end{pmatrix}\right\}$$
\end{enumerate}
\end{solucion}
\item Sea la matriz $A=
	\begin{bmatrix}
	1 & 3 & 4\\
	-4 & 2 & -6\\
	-3 & -2 & -7
	\end{bmatrix}
	$ y $b$ un vector en $R^3$. ¿La ecuación $Ax=b$ es consistente para todo $b$?
\begin{solucion}
Diremos que $b = \begin{pmatrix}
	b_1 \\ b_2 \\ b_3
	\end{pmatrix}$. Luego, el sistema $Ax = b$ se puede representar como
	$$\left[
	\begin{array}{ccc|c}
	1 & 3 & 4 &b_1\\
	-4 & 2 & -6 & b_2\\
	-3 & -2 & -7 & b_3
	\end{array}
	\right] \sim \left[
	\begin{array}{ccc|c}
	1 & 3 & 4 &b_1\\
	0 & 14 & 10 & b_2 + 4b_1\\
	0 & 7 & 5 & b_3 + 3b_1
	\end{array}
	\right]$$
	$$ \sim \left[
	\begin{array}{ccc|c}
	1 & 3 & 4 &b_1\\
	0 & 14 & 10 & b_2 + 4b_1\\
	0 & 0 & 0 & b_3 + 3b_1 - \frac{1}{2}(b_2+4b_1)
	\end{array}
	\right] \sim \left[
	\begin{array}{ccc|c}
	1 & 3 & 4 &b_1\\
	0 & 14 & 10 & b_2 + 4b_1\\
	0 & 0 & 0 & b_1 - \frac{1}{2}b_2 + 4b_3
	\end{array}
	\right] $$
	Luego, el sistema será consistente para $b_1 - \dfrac{1}{2}b_2 + b_3 = 0$. Esto se puede representar como
	$$\begin{array}{rcl}
	b_1 & = & \frac{1}{2}b_2 - b_3\\
	b_2 & = & b_2 \\
	b_3 & = & b_3
	\end{array} \ra b = \begin{pmatrix}
	\frac{1}{2}b_2 - b_3 \\
	b_2\\
	b_3 
	\end{pmatrix} = b_2\begin{pmatrix}
	\frac{1}{2} \\
	1\\
	0 
	\end{pmatrix} +  b_3\begin{pmatrix}
	-1 \\
	0\\
	1 
	\end{pmatrix} $$
	Finalmente,
	$$b = Gen \left\{ \begin{pmatrix}
	\frac{1}{2} \\
	1\\
	0 
	\end{pmatrix}, \begin{pmatrix}
	-1 \\
	0\\
	1
	\end{pmatrix}\right\}$$
\end{solucion}
\item Describa todas las soluciones del siguiente sistema de ecuaciones y comparelas con las de su sistema homogeneo
	$$
	\begin{array}{rcr}
	x_1 +2x_2-3x_3& = & 5\\
	2x_1 + x_2 - 3x_3& = & 13\\
	-x_1 + x_2 & = & -8
	\end{array}$$
\begin{solucion}
La matriz aumentada asociada es
		$$\left[
		\begin{array}{ccc|c}
		1 & 2 & -3 & 5\\
		2 & 1 & -3 & 13\\
		-1 & 1 & 0 & -8
		\end{array}
		\right] \sim \left[
		\begin{array}{ccc|c}
		1 & 2 & -3 & 5\\
		0 & -3 & 3 & 3\\
		0 & 3 & -3 & -3
		\end{array}
		\right] \sim \left[
		\begin{array}{ccc|c}
		1 & 2 & -3 & 5\\
		0 & -3 & 3 & 3\\
		0 & 0 & 0 & 0
		\end{array}
		\right]$$
		$$ \sim \left[
		\begin{array}{ccc|c}
		1 & 2 & -3 & 5\\
		0 & 1 & -1 & -1\\
		0 & 0 & 0 & 0
		\end{array}
		\right]  \sim \left[
		\begin{array}{ccc|c}
		1 & 0 & -1 & 7\\
		0 & 1 & -1 & -1\\
		0 & 0 & 0 & 0
		\end{array}
		\right] $$
		Esto corresponde al sistema
		$$\begin{array}{rcl}
		x_1 - x_3 & = & 7\\
		x_2 - x_3 & = & -1\\
		0 & = & 0
		\end{array} \ra \begin{array}{rcl}
		x_1 & = & x_3 + 7\\
		x_2 & = & x_3 - 1\\
		x_3 & = & x_3
		\end{array} $$
		Luego,
		$$x = \begin{pmatrix}
		x_3 + 7\\
		x_3 - 1\\
		x_3
		\end{pmatrix} = \begin{pmatrix}
		7\\
		- 1\\
		0
		\end{pmatrix} + x_3 \begin{pmatrix}
		1\\
		1\\
		1
		\end{pmatrix}$$
		Por último, la solución es
		$$S = \begin{pmatrix}
		7\\
		- 1\\
		0
		\end{pmatrix} + Gen\left\{\begin{pmatrix}
		1\\
		1\\
		1
		\end{pmatrix}\right\}$$
		Notemos que la solución del sistema homogeneo será lo mismo pero quitando el vector que no forma parte del generado, es decir
		$$S_H = Gen\left\{\begin{pmatrix}
		1\\
		1\\
		1
		\end{pmatrix}\right\}$$
\end{solucion}
\item Sean $\{u, v, w\}$ un conjunto de vectores linealmente independientes. Demuestre que el conjunto $\{u+v, u+2w, v+3u+w\}$ es linealmente independiente.
\begin{solucion}
$$P.D. \quad \{y, v, w\}\quad L.I. \ra \{u+v, u+2w, v+3u+w\}\quad L.I.$$
		Para que $\{u+v, u+2w, v+3u+w\}$ sea L.I., tiene que cumplirse que el sistema
		$$\alpha(u+v) + \beta(u+2w) + \gamma(v+3u+w) = 0$$
		tenga solucion única $\begin{pmatrix}
		\alpha\\ \beta \\ \gamma
		\end{pmatrix} = \begin{pmatrix} 0\\0\\0\end{pmatrix}$\\
		Trabajemos entonces con el sistema
		$$\alpha(u+v) + \beta(u+2w) + \gamma(v+3u+w) = 0$$
		$$\alpha u+ \alpha v + \beta u+2\beta w + \gamma v+3\gamma u+\gamma w = 0$$
		$$(\alpha + \beta + 3 \gamma)u + (\alpha + \gamma)v + (2\beta + \gamma)w = 0$$
		Recordemos que ${u, v, w}$ es L.I, por lo que debe cumplirse que
		$$\begin{array}{rcl}
		\alpha + \beta + 3 \gamma & = & 0\\
		\alpha + \gamma & = & 0\\
		2\beta + \gamma & = & 0
		\end{array}$$
		Este sistema lo podemos expresar de forma matricial,
		$$\begin{bmatrix}
		1 & 1 & 3\\
		1 & 0 & 1 \\
		0 & 2 & 1
		\end{bmatrix} \stackrel{F.E.}{\sim} \begin{bmatrix}
		1 & 0 & 1\\
		0 & 1 & -2\\
		0 & 0 & 1
		\end{bmatrix}$$
		Cuya solución es
		$$\begin{array}{rcl}
		\alpha & = & 0\\
		\beta & = & 0\\
		\gamma & = & 0
		\end{array}$$
		$$q.e.d$$
	
\end{solucion}
\item Sea $L: P_2(\R) \ra \R^2$ una transformación lineal tal que
	$$L(1+x) = \begin{pmatrix}
	1\\
	1
	\end{pmatrix}, \quad L(1+x+x^2) = \begin{pmatrix}
	1\\
	-1
	\end{pmatrix}\ y \ L(1+2x) = \begin{pmatrix}
	1\\
	2
	\end{pmatrix}$$
	determine $L(a+bx+cx^2)$ para todo $a,b,c \in \R$.
\begin{solucion}
Recordemos la siguientes propiedades de las transformaciones lineales
		$$L(v_1) + L(v_2) = L(v_1 + v_2), \quad \alpha L(v) = L(\alpha v)$$
		Debemos buscar 
		$$L(1), \quad L(x), \quad L(x^2)$$
		Para esto, debemos jugar con la información que nos dan hasta encontrar cada uno de ellos.
		\begin{center}\rule{14.5cm}{0.1pt}\end{center}
		$$L(1+x+x^2) - L(1+x) = \begin{pmatrix}
		1\\-1
		\end{pmatrix} - \begin{pmatrix}
		1\\1
		\end{pmatrix}$$
		$$L(x^2) = \begin{pmatrix}
		0\\-2
		\end{pmatrix}$$
		
		\begin{center}\rule{14.5cm}{0.1pt}\end{center}
		$$L(1+2x) - L(1+x) = \begin{pmatrix}
		1\\2
		\end{pmatrix} - \begin{pmatrix}
		1\\1
		\end{pmatrix}$$
		$$L(x) = \begin{pmatrix}
		0\\1
		\end{pmatrix}$$
		
		\begin{center}\rule{14.5cm}{0.1pt}\end{center}
		$$L(1+x) - L(x) = \begin{pmatrix}
		1\\1
		\end{pmatrix} - \begin{pmatrix}
		0\\1
		\end{pmatrix}$$
		$$L(x) = \begin{pmatrix}
		1\\0
		\end{pmatrix}$$
		
		Luego, 
		$$L(a+bx+cx^2) = aL(1) + bL(x) + cL(x^2)$$
		$$L(a+bx+cx^2) = aL\begin{pmatrix}1\\0\end{pmatrix} + b\begin{pmatrix}0\\1\end{pmatrix} + c\begin{pmatrix}0\\-2\end{pmatrix}$$
		$$L(a+bx+cx^2) = \begin{pmatrix}a\\b-2c\end{pmatrix}$$
\end{solucion}
\end{preguntas}
\end{document}