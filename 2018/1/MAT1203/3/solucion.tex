\documentclass[12pt]{article}

\usepackage{fullpage}
\usepackage{graphicx}
\usepackage{amssymb}
\usepackage{amsmath}
\usepackage[none]{hyphenat}
\usepackage{parskip}
\usepackage[spanish]{babel}
\usepackage[utf8]{inputenc}
\usepackage{hyperref}
\usepackage{fancyhdr}
\usepackage{tasks}
\usepackage{mdframed}
\usepackage{xcolor}
\usepackage{pgfplots}
\usepackage[makeroom]{cancel}
\usepackage{multicol}
\usepackage[shortlabels]{enumitem}
\usepackage{tabto}

\setlength{\headheight}{10pt}
\setlength{\headsep}{10pt}
\pagestyle{fancy}
\rhead{\ayudantia \ - \alumno}

\newcommand*{\mybox}[2]{\colorbox{#1!30}{\parbox{.98\linewidth}{#2}}}

\newenvironment{solucion}
{\begin{mdframed}[backgroundcolor=black!10]
		{\bf Solución:}\\
	}
	{
	\end{mdframed}
}

\newenvironment{alternativas}[1]
{\begin{multicols}{#1}
		\begin{enumerate}[a)]
		}
		{
		\end{enumerate}
	\end{multicols}
}

\newenvironment{preguntas}
{\begin{enumerate}\itemsep12pt
	}
	{
	\end{enumerate}
}

\newcommand{\ayudantia}{{\sc Ayudantía 3}}
\newcommand{\tituloayu}{Conjunto solución, independencia lineal y transformaciones lineales}
\newcommand{\fecha}{21 de marzo de 2018}
\newcommand{\sigla}{MAT1203}
\newcommand{\nombre}{Álgebra Lineal}
\newcommand{\profesor}{Rodrigo Rubio Varas}
\newcommand{\ano}{2018}
\newcommand{\semestre}{1}
\newcommand{\mail}{mat1203@ifcastaneda.cl}
\newcommand{\alumno}{Ignacio Castañeda - \mail}

\newcommand{\ev}{\Big|}
\newcommand{\ra}{\rightarrow}
\newcommand{\lra}{\leftrightarrow}
\newcommand{\N}{\mathbb{N}}
\newcommand{\R}{\mathbb{R}}
\newcommand{\Exp}[1]{\mathcal{E}_{#1}}
\newcommand{\List}[1]{\mathcal{L}_{#1}}
\newcommand{\EN}{\Exp{\N}}
\newcommand{\LN}{\List{\N}}
\newcommand{\comment}[1]{}
\newcommand{\lb}{\\~\\}
\newcommand{\eop}{_{\square}}
\newcommand{\hsig}{\hat{\sigma}}

\begin{document}
\thispagestyle{empty}

\begin{minipage}{2cm}
	\includegraphics[width=2cm]{../../../../img/logo.pdf}
	\vspace{0.5cm}
\end{minipage}
\begin{minipage}{\linewidth}
	\begin{tabular}{lrl}
		{\scriptsize\sc Pontificia Universidad Catolica de Chile} & \hspace*{0.7in}Curso: &
		\sigla  - \nombre\\
		{\sc Facultad de Matemáticas}&
		Profesor: & \profesor \\
		{\sc Semestre \ano-\semestre} & Ayudante: & {Ignacio Castañeda}\\
		& {Mail:} & \texttt{\mail}
	\end{tabular}
\end{minipage}

\vspace{-10mm}
\begin{center}
	{\LARGE\bf \ayudantia}\\
	\vspace{0.1cm}
	{\tituloayu}\\
	\vspace{0.1cm}
	\fecha\\
	\vspace{0.4cm}
\end{center}

\begin{preguntas}
\item Sean los vectores 
	$$
	v_1 = \begin{pmatrix}
	1\\
	1\\
	2
	\end{pmatrix};\qquad
	v_2 = \begin{pmatrix}
	2\\
	2\\
	-3
	\end{pmatrix}; \qquad
	v_3 = \begin{pmatrix}
	0\\
	1\\
	3
	\end{pmatrix}$$
	determinar si $Gen\{v_1, v_2, v_3\} = R^3$.
\begin{solucion}
Para ver si estos tres vectores generan $\R^3$, tiene que pasar que todos sean L.I. entre si. Una forma de ver esto, es formar una matriz con los vectores y ver la cantidad de pivotes que hay, es decir
		$$\begin{bmatrix}
			\\
			v_1 & v_2 & v_3\\
			&&
		\end{bmatrix} \sim 
		\begin{bmatrix}
		1 & 2 & 0\\
		1 & 2 & 1\\
		2 & -3 & 3
		\end{bmatrix} \sim
		\begin{bmatrix}
		1 & 2 & 0\\
		0 & 0 & 1\\
		0 & -7 & 3
		\end{bmatrix} \sim
		\begin{bmatrix}
		1 & 2 & 0\\
		0 & -7 & 3\\
		0 & 0 & 1
		\end{bmatrix}$$
		Como la matriz tiene 3 pivotes, quiere decir que hay 3 vectores L.I., es decir, todos son linealmente independientes entre si, formando asi $\R^3$
\end{solucion}
\item Sea la matriz $A=
	\begin{bmatrix}
	3 & 5 & -4\\
	-3 & -2 & 4\\
	6 & 1 & -8
	\end{bmatrix}$
\begin{enumerate}[a)]
\item Determinar el conjunto solución de su sistema homogeneo.
\item Describir todas las soluciones de $Ax=b$ con $b=
		\begin{pmatrix}
		7\\
		-1\\
		-4
		\end{pmatrix}$ 
\end{enumerate}
\begin{solucion}

\begin{enumerate}[a)]
\item Determinar el conjunto solución de su sistema homogeneo.\\\\
			El sistema homogeneo corresponde a la ecuación $Ax=\vec{0}$, por lo que nuestra matriz aumentada sería
			$$\left[
			\begin{array}{ccc|c}
			3 & 5 & -4 & 0\\
			-3 & -2 & 4 & 0\\
			6 & 1 & -8 & 0
			\end{array}
			\right] \sim \left[
			\begin{array}{ccc|c}
			3 & 5 & -4 & 0\\
			0 & 3 & 0 & 0\\
			0 & -9 & 0 & 0
			\end{array}
			\right] \sim \left[
			\begin{array}{ccc|c}
			3 & 5 & -4 & 0\\
			0 & 3 & 0 & 0\\
			0 & 0 & 0 & 0
			\end{array}
			\right]$$
			$$\sim \left[
			\begin{array}{ccc|c}
			1 & \frac{5}{3} & -\frac{4}{3} & 0\\
			0 & 1 & 0 & 0\\
			0 & 0 & 0 & 0
			\end{array}
			\right] \sim \left[
			\begin{array}{ccc|c}
			1 & 0 & -\frac{4}{3} & 0\\
			0 & 1 & 0 & 0\\
			0 & 0 & 0 & 0
			\end{array}
			\right]$$
			Esto corresponde al sistema
			$$\begin{array}{rcl}
			x_1 - \frac{4}{3}x_3 & = & 0\\
			x_2 & = & 0\\
			0 & = & 0
			\end{array} \ra \begin{array}{rcl}
			x_1 & = & \frac{4}{3}x_3\\
			x_2 & = & 0\\
			x_3 & = & x_3
			\end{array} \ra x = \begin{pmatrix}
			\frac{4}{3}x_3\\
			0\\
			x_3
			\end{pmatrix} = x_3\begin{pmatrix}
			\frac{4}{3}\\
			0\\
			1
			\end{pmatrix}$$
			Luego, la solución del sistema homogeneo es
			$$S = Gen\left\{\begin{pmatrix}
			\frac{4}{3}\\
			0\\
			1
			\end{pmatrix}\right\}$$
\item Describir todas las soluciones de $Ax=b$ con $b=
			\begin{pmatrix}
			7\\
			-1\\
			-4
			\end{pmatrix}$ \\\\
			Para esto, hacemos lo mismo que en la parte a), pero aumentando la matriz por el vector $b$, es decir
			$$\left[
			\begin{array}{ccc|c}
			3 & 5 & -4 & 7\\
			-3 & -2 & 4 & -1\\
			6 & 1 & -8 & -4
			\end{array}
			\right] \sim \left[
			\begin{array}{ccc|c}
			3 & 5 & -4 & 7\\
			0 & 3 & 0 & 6\\
			0 & -9 & 0 & -18
			\end{array}
			\right] \sim \left[
			\begin{array}{ccc|c}
			3 & 5 & -4 & 7\\
			0 & 3 & 0 & 6\\
			0 & 0 & 0 & 0
			\end{array}
			\right]$$
			$$\sim \left[
			\begin{array}{ccc|c}
			1 & \frac{5}{3} & -\frac{4}{3} & \frac{7}{3}\\
			0 & 1 & 0 & 2\\
			0 & 0 & 0 & 0
			\end{array}
			\right] \sim \left[
			\begin{array}{ccc|c}
			1 & 0 & -\frac{4}{3} & -1\\
			0 & 1 & 0 & 2\\
			0 & 0 & 0 & 0
			\end{array}
			\right]$$
			Esto corresponde al sistema
			$$\begin{array}{rcl}
			x_1 - \frac{4}{3}x_3 & = & -1\\
			x_2 & = & 2\\
			0 & = & 0
			\end{array} \ra \begin{array}{rcl}
			x_1 & = & -1 + \frac{4}{3}x_3\\
			x_2 & = & 2\\
			x_3 & = & x_3
			\end{array}$$
			$$ x = \begin{pmatrix}
			-1 + \frac{4}{3}x_3\\
			2\\
			x_3
			\end{pmatrix} = \begin{pmatrix}
			-1\\2\\0
			\end{pmatrix} + x_3\begin{pmatrix}
			\frac{4}{3}\\
			0\\
			1
			\end{pmatrix}$$
			Luego, la solución del sistema $Ax = b$ es
			$$S = \begin{pmatrix}
			-1\\2\\0
			\end{pmatrix} + Gen\left\{\begin{pmatrix}
			\frac{4}{3}\\
			0\\
			1
			\end{pmatrix}\right\}$$
			Finalmente, podriamos amplificar el vector del generado por 3, para que quede más bonito, con lo que
			$$S = \begin{pmatrix}
			-1\\2\\0
			\end{pmatrix} + Gen\left\{\begin{pmatrix}
			4\\
			0\\
			3
			\end{pmatrix}\right\}$$
\end{enumerate}
\end{solucion}
\end{preguntas}
\end{document}