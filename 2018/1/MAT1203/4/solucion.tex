\documentclass[12pt]{article}

\usepackage{fullpage}
\usepackage{graphicx}
\usepackage{amssymb}
\usepackage{amsmath}
\usepackage[none]{hyphenat}
\usepackage{parskip}
\usepackage[spanish]{babel}
\usepackage[utf8]{inputenc}
\usepackage{hyperref}
\usepackage{fancyhdr}
\usepackage{tasks}
\usepackage{mdframed}
\usepackage{xcolor}
\usepackage{pgfplots}
\usepackage[makeroom]{cancel}
\usepackage{multicol}
\usepackage[shortlabels]{enumitem}
\usepackage{stackrel}

\setlength{\headheight}{10pt}
\setlength{\headsep}{10pt}
\pagestyle{fancy}
\rhead{\ayudantia \ - \alumno}

\newcommand*{\mybox}[2]{\colorbox{#1!30}{\parbox{.98\linewidth}{#2}}}

\newenvironment{solucion}
{\begin{mdframed}[backgroundcolor=black!10]
		{\bf Solución:}\\
	}
	{
	\end{mdframed}
}

\newenvironment{alternativas}[1]
{\begin{multicols}{#1}
		\begin{enumerate}[a)]
		}
		{
		\end{enumerate}
	\end{multicols}
}

\newenvironment{preguntas}
{\begin{enumerate}\itemsep12pt
	}
	{
	\end{enumerate}
}

\newcommand{\ayudantia}{{\sc Ayudantía 4}}
\newcommand{\tituloayu}{Transformaciones lineales y aplicaciones de sistemas lineales}
\newcommand{\fecha}{28 de marzo de 2018}
\newcommand{\sigla}{MAT1203}
\newcommand{\nombre}{Álgebra Lineal}
\newcommand{\profesor}{Rodrigo Rubio Varas}
\newcommand{\ano}{2018}
\newcommand{\semestre}{1}
\newcommand{\mail}{mat1203@ifcastaneda.cl}
\newcommand{\alumno}{Ignacio Castañeda - \mail}

\newcommand{\ev}{\Big|}
\newcommand{\ra}{\rightarrow}
\newcommand{\lra}{\leftrightarrow}
\newcommand{\N}{\mathbb{N}}
\newcommand{\R}{\mathbb{R}}
\newcommand{\Exp}[1]{\mathcal{E}_{#1}}
\newcommand{\List}[1]{\mathcal{L}_{#1}}
\newcommand{\EN}{\Exp{\N}}
\newcommand{\LN}{\List{\N}}
\newcommand{\comment}[1]{}
\newcommand{\lb}{\\~\\}
\newcommand{\eop}{_{\square}}
\newcommand{\hsig}{\hat{\sigma}}
\newcommand{\widesim}[2][1.5]{
	\mathrel{\overset{#2}{\scalebox{#1}[1]{$\sim$}}}
}
\newcommand{\wsim}{\widesim{}}

\begin{document}
\thispagestyle{empty}

\begin{minipage}{2cm}
	\includegraphics[width=2cm]{../../../../img/logo.pdf}
	\vspace{0.5cm}
\end{minipage}
\begin{minipage}{\linewidth}
	\begin{tabular}{lrl}
		{\scriptsize\sc Pontificia Universidad Catolica de Chile} & \hspace*{0.7in}Curso: &
		\sigla  - \nombre\\
		{\sc Facultad de Matemáticas}&
		Profesor: & \profesor \\
		{\sc Semestre \ano-\semestre} & Ayudante: & {Ignacio Castañeda}\\
		& {Mail:} & \texttt{\mail}
	\end{tabular}
\end{minipage}

\vspace{-10mm}
\begin{center}
	{\LARGE\bf \ayudantia}\\
	\vspace{0.1cm}
	{\tituloayu}\\
	\vspace{0.1cm}
	\fecha\\
	\vspace{0.4cm}
\end{center}

\begin{preguntas}
\item Sea $T$ una transformación lineal tal que
	$$T\begin{bmatrix}2 \\ 1\end{bmatrix} = \begin{bmatrix}1 \\ 0 \\ 1 \\ 0\end{bmatrix} 
	\quad y \quad
	T \begin{bmatrix}3 \\ 1\end{bmatrix}  =\begin{bmatrix}0 \\ 1 \\ 0 \\ 1\end{bmatrix}$$
	Determine la matriz que representa a $T$
\begin{solucion}
Estamos buscando la matriz $\begin{bmatrix}
		T\begin{bmatrix}
		1 \\ 0
		\end{bmatrix}
		&
		
		T\begin{bmatrix}
		0 \\ 1
		\end{bmatrix}
		\end{bmatrix}$.\\
		Entonces, lo que necesitamos es buscar $T\begin{bmatrix}
		1 \\ 0
		\end{bmatrix}$ y $T\begin{bmatrix}
			0 \\ 1
		\end{bmatrix}$ y para ello, debemos escribir cada uno como una combinación lineal de $T\begin{bmatrix}
		2 \\ 1
		\end{bmatrix}$ y $T\begin{bmatrix}
		3 \\ 1
		\end{bmatrix}$, es decir
		$$\begin{bmatrix}
		1 \\ 0
		\end{bmatrix} = \alpha \begin{bmatrix}
		2 \\ 1
		\end{bmatrix} +  \beta \begin{bmatrix}
		3 \\ 1
		\end{bmatrix} \quad y \quad
		\begin{bmatrix}
		0 \\ 1
		\end{bmatrix} = \gamma \begin{bmatrix}
		2 \\ 1
		\end{bmatrix} +  \delta \begin{bmatrix}
		3 \\ 1
		\end{bmatrix}$$
		O sea, debemos resolver los sistemas
		$$\begin{array}{rcl}
		2\alpha + 3\beta & = & 1\\
		\alpha + \beta & = & 0
		\end{array} \qquad \qquad \qquad
		\begin{array}{rcl}
		2\gamma + 3\delta & = & 0\\
		\gamma + \delta & = & 1
		\end{array}$$
		La solución de estos sistemas es
		$$\alpha = -1, \quad \beta = 1, \quad \gamma = 3, \quad \delta = -2$$
		Por lo que
		$$\begin{bmatrix}
		1 \\ 0
		\end{bmatrix} = \begin{bmatrix}
		3 \\ 1
		\end{bmatrix}- \begin{bmatrix}
		2 \\ 1
		\end{bmatrix}  \quad y \quad
		\begin{bmatrix}
		0 \\ 1
		\end{bmatrix} = 3 \begin{bmatrix}
		2 \\ 1
		\end{bmatrix} - 2 \begin{bmatrix}
		3 \\ 1
		\end{bmatrix}$$
		Luego,
		$$T\begin{bmatrix}
		1 \\ 0
		\end{bmatrix} = T\begin{bmatrix}
		3 \\ 1
		\end{bmatrix}- T\begin{bmatrix}
		2 \\ 1
		\end{bmatrix} = \begin{bmatrix}0 \\ 1 \\ 0 \\ 1\end{bmatrix} - \begin{bmatrix}1 \\0 \\ 1 \\ 0\end{bmatrix} = \begin{bmatrix}-1 \\ 1 \\ -1 \\ 1\end{bmatrix}$$
		y
		$$T\begin{bmatrix}
		0 \\ 1
		\end{bmatrix} = 3 T\begin{bmatrix}
		2 \\ 1
		\end{bmatrix} - 2 T\begin{bmatrix}
		3 \\ 1
		\end{bmatrix} = 3 \begin{bmatrix}1 \\ 0 \\ 1 \\ 0\end{bmatrix} - 2 \begin{bmatrix}0 \\ 1 \\ 0 \\ 1\end{bmatrix} = \begin{bmatrix}3 \\ -2 \\ 3 \\ -2\end{bmatrix}$$
		Finalmente, la matriz que representa $T$ es
		$$\begin{bmatrix}
		T\begin{bmatrix}
		1 \\ 0
		\end{bmatrix}
		&
		
		T\begin{bmatrix}
		0 \\ 1
		\end{bmatrix}
		\end{bmatrix} =
		\begin{bmatrix}
		-1 & 3 \\
		1 & -2 \\
		-1 & 3 \\
		1 & -2
		\end{bmatrix}$$
\end{solucion}
\item Sea la transformación lineal $T$, dada por la matriz
	$$T = \begin{bmatrix}
	1 & 2 & -1\\
	2 & 4 & -2
	\end{bmatrix}$$
	Determine la imagen del plano de ecuación $x_1 + x_3 = 1$ por T.
\begin{solucion}
En primer lugar, escribimos el plano en su forma paramétrica
		$$\begin{array}{rcl}
		x_1 & = &1-x_3\\
		x_2 & = &x_2\\
		x_3 & = &x_3
		\end{array}$$
		Vemos entonces que 3 puntos arbitrarios pertenecienctes al plano son
		$$P_1(1,0,0), \quad P_2(1,1,0), \quad P_3(0,0,1) $$
		Luego, dos vectores directores que definen al plano son
		$$\vec{d_1} = \vec{P_1P_2} = (0,1,0), \quad =\vec{d_2} = \vec{P_1P_3} = (-1,0,1)$$
		Con esto, podemos expresar el plano en su forma vectorial
		$$\vec{P}= \vec{P_1} + \alpha \vec{d_1}  + \beta \vec{d_2}$$
		Para encontrar la imagen del plano por $T$, basta con hacer
		$$T\vec{P}= T\vec{P_1} + \alpha T\vec{d_1}  + \beta T\vec{d_2}$$
		Como
		$$T\vec{P_1} = \begin{bmatrix}
		1 & 2 & -1\\
		2 & 4 & -2
		\end{bmatrix} \begin{bmatrix}
		1 \\ 0 \\ 0
		\end{bmatrix} = \begin{bmatrix}
		1 \\ 2
		\end{bmatrix}$$
		$$T\vec{d_1} = \begin{bmatrix}
		1 & 2 & -1\\
		2 & 4 & -2
		\end{bmatrix} \begin{bmatrix}
		0 \\ 1 \\ 0
		\end{bmatrix} = \begin{bmatrix}
		2 \\ 4
		\end{bmatrix}$$
		$$T\vec{d_2} = \begin{bmatrix}
		1 & 2 & -1\\
		2 & 4 & -2
		\end{bmatrix} \begin{bmatrix}
		-1 \\ 0 \\ 1
		\end{bmatrix} = \begin{bmatrix}
		-2 \\ -4
		\end{bmatrix}$$
		Tenemos que
		$$TP =  \begin{bmatrix}
		1 \\ 2
		\end{bmatrix} + \alpha  \begin{bmatrix}
		2 \\ 4
		\end{bmatrix} + 
		\beta  \begin{bmatrix}
		-2 \\ -4
		\end{bmatrix} =  \begin{bmatrix}
		1 + 2\alpha  - 2\beta \\
		2 + 4\alpha - 5\beta
		\end{bmatrix} = 
		(1+2\alpha - 2\beta)  \begin{bmatrix}
		1 \\ 2
		\end{bmatrix}$$
		En otras palabras, la imagen del plano por la traslación $T$ corresponde a los ponderados de $\begin{bmatrix}
			1 \\ 2
		\end{bmatrix}$, es decir
		$$T\vec{P} = Gen\left\{ \begin{bmatrix}
		1 \\ 2
		\end{bmatrix} \right\}$$
\end{solucion}
\item Determine si las siguientes afirmaciones son verdaderas o falsas.
\begin{enumerate}[a)]
\item Para todo $u, v \in \R^n$ se tiene que $||u+v|| < ||u||+||v||$
\item Si $u \in Gen\{v_1, v_2\}$ entonces $Gen\{u, v_1\} \subset Gen\{u, v_2\}$
\item Sea $A$ una matriz de $n \times m$ tal que $n < m $, entonces las columnas de A son L.D.
\item Si $A$ es una matriz de $2 \times 3$ tal que $Ax = \begin{pmatrix}
		1\\
		1
		\end{pmatrix}$ y $ Ax = \begin{pmatrix}
		1\\
		2
		\end{pmatrix}$ tienen solución, entonces las filas de $A$ son L.I.
\end{enumerate}
\begin{solucion}

\begin{enumerate}[a)]
\item Para todo $u, v \in \R^n$ se tiene que $||u+v|| < ||u||+||v||$\\\\
			Tomemos $u = v = \vec{0}$. Luego,
			$$||\vec{0} + \vec{0}|| < ||\vec{0}|| + ||\vec{0}||$$
			$$0 < 0$$
			Lo que es falso. Finalmente, la afirmación es FALSA
\item Si $u \in Gen\{v_1, v_2\}$ entonces $Gen\{u, v_1\} \subset Gen\{u, v_2\}$\\\\
			Tomemos
			$$u = \begin{pmatrix}
			0 \\ 0 
			\end{pmatrix}, \quad v_1 = \begin{pmatrix}
			1 \\ 0 
			\end{pmatrix}, \quad v_2 = \begin{pmatrix}
			0 \\ 1
			\end{pmatrix}$$
			Aqui se cumple que $u \in Gen\{v_1, v_2\}$. Sin embargo,
			$$Gen\left\{\begin{pmatrix}
			0 \\ 0 
			\end{pmatrix}, \begin{pmatrix}
			1 \\ 0 
			\end{pmatrix}\right\} \subset Gen\left\{\begin{pmatrix}
			0 \\ 0 
			\end{pmatrix}, \begin{pmatrix}
			0 \\ 1
			\end{pmatrix}\right\}$$
			$$Gen\left\{\begin{pmatrix}
			1 \\ 0 
			\end{pmatrix}\right\} \subset Gen\left\{\begin{pmatrix}
			0 \\ 1
			\end{pmatrix}\right\}$$
			Lo que claramente no es cierto. Luego, la afirmación es FALSA
\item Sea $A$ una matriz de $n \times m$ tal que $n < m $, entonces las columnas de A son L.D.\\\\
			Esto significa que la matriz esta compuesta por $m$ vectores en $\R^n$. Dicho esto, notemos que pueden haber a lo mas $n$ vectores L.I. (máximo $n$ pivotes) Como $m>n$, deben haber vectores L.D., por lo que la afirmación es VERDADERA.
\item Si $A$ es una matriz de $2 \times 3$ tal que $Ax = \begin{pmatrix}
			1\\
			1
			\end{pmatrix}$ y $ Ax = \begin{pmatrix}
			1\\
			2
			\end{pmatrix}$ tienen solución, entonces las filas de $A$ son L.I.\\\\
			Asumiendo que ambos sistemas tienen solución, digamos que las filas de $A$ no serán L.I., es decir, una será múltiplo de la otra. Esto se puede expresar como
			$$A = \begin{bmatrix}
			a_{11} & a_{12} & a_{13} \\
			a_{21} & a_{22} & a_{23}
			\end{bmatrix} = \begin{bmatrix}
			a_{11} & a_{12} & a_{13} \\
			\lambda a_{11} & \lambda a_{12} & \lambda a_{13}
			\end{bmatrix}$$
			Resolvamos ahora el sistema $Ax = \begin{pmatrix}
			b_1 \\ b_2
			\end{pmatrix}$
			$$\begin{array}{lcl}
			a_{11}x_1 + a_{12}x_2 + a_{13}x_3 & = & b_1 \\
			a_{21}x_1 + a_{22}x_2 + a_{23}x_3 & = & b_2			
			\end{array} \ra \begin{array}{lcl}
			a_{11}x_1 + a_{12}x_2 + a_{13}x_3 & = & b_1 \\
			\lambda a_{11}x_1 + \lambda a_{12}x_2 + \lambda a_{13}x_3 & = & \lambda b_1			
			\end{array}$$
			Aqui podemos ver que a lo más uno de los dos sistemas puede tener solución, lo que contradice nuestra hipotesis inicial. Por esto, las filas de $A$ deben ser L.I. Con esto concluimos que la afirmación es VERDADERA
\end{enumerate}
\end{solucion}
\end{preguntas}
\end{document}