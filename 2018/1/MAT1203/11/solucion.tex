\documentclass[12pt]{article}

\usepackage{fullpage}
\usepackage{graphicx}
\usepackage{amssymb}
\usepackage{amsmath}
\usepackage[none]{hyphenat}
\usepackage{parskip}
\usepackage[spanish]{babel}
\usepackage[utf8]{inputenc}
\usepackage{hyperref}
\usepackage{fancyhdr}
\usepackage{tasks}
\usepackage{mdframed}
\usepackage{xcolor}
\usepackage{pgfplots}
\usepackage[makeroom]{cancel}
\usepackage{multicol}
\usepackage[shortlabels]{enumitem}
\usepackage{stackrel}
\usepackage{tkz-tab}
\usepackage{xpatch}
\xpatchcmd{\tkzTabLine}{$0$}{$\bullet$}{}{}

\setlength{\headheight}{10pt}
\setlength{\headsep}{10pt}
\pagestyle{fancy}
\rhead{\ayudantia \ - \alumno}
\tikzset{t style/.style={style=solid}}

\newcommand*{\mybox}[2]{\colorbox{#1!30}{\parbox{.98\linewidth}{#2}}}

\newenvironment{solucion}
{\begin{mdframed}[backgroundcolor=black!10]
		{\bf Solución:}\\
	}
	{
	\end{mdframed}
}

\newenvironment{alternativas}[1]
{\begin{multicols}{#1}
		\begin{enumerate}[a)]
		}
		{
		\end{enumerate}
	\end{multicols}
}

\newenvironment{preguntas}
{\begin{enumerate}\itemsep12pt
	}
	{
	\end{enumerate}
}

\newcommand{\ayudantia}{{\sc Ayudantía 11}}
\newcommand{\tituloayu}{Vectores propios y diagonalización}
\newcommand{\fecha}{23 de mayo de 2018}
\newcommand{\sigla}{MAT1203}
\newcommand{\nombre}{Álgebra Lineal}
\newcommand{\profesor}{Rodrigo Rubio Varas}
\newcommand{\ano}{2018}
\newcommand{\semestre}{1}
\newcommand{\mail}{mat1203@ifcastaneda.cl}
\newcommand{\alumno}{Ignacio Castañeda - \mail}

\newcommand{\ev}{\Big|}
\newcommand{\ra}{\rightarrow}
\newcommand{\lra}{\leftrightarrow}
\newcommand{\N}{\mathbb{N}}
\newcommand{\R}{\mathbb{R}}
\newcommand{\Exp}[1]{\mathcal{E}_{#1}}
\newcommand{\List}[1]{\mathcal{L}_{#1}}
\newcommand{\EN}{\Exp{\N}}
\newcommand{\LN}{\List{\N}}
\newcommand{\comment}[1]{}
\newcommand{\lb}{\\~\\}
\newcommand{\eop}{_{\square}}
\newcommand{\hsig}{\hat{\sigma}}
\newcommand{\widesim}[2][1.5]{
	\mathrel{\overset{#2}{\scalebox{#1}[1]{$\sim$}}}
}
\newcommand{\wsim}{\widesim{}}
\newcommand{\lh}{\stackrel{L'H}{=}}

\begin{document}
\thispagestyle{empty}

\begin{minipage}{2cm}
	\includegraphics[width=2cm]{../../../../img/logo.pdf}
	\vspace{0.5cm}
\end{minipage}
\begin{minipage}{\linewidth}
	\begin{tabular}{lrl}
		{\scriptsize\sc Pontificia Universidad Catolica de Chile} & \hspace*{0.7in}Curso: &
		\sigla  - \nombre\\
		{\sc Facultad de Matemáticas}&
		Profesor: & \profesor \\
		{\sc Semestre \ano-\semestre} & Ayudante: & {Ignacio Castañeda}\\
		& {Mail:} & \texttt{\mail}
	\end{tabular}
\end{minipage}

\vspace{-10mm}
\begin{center}
	{\LARGE\bf \ayudantia}\\
	\vspace{0.1cm}
	{\tituloayu}\\
	\vspace{0.1cm}
	\fecha\\
	\vspace{0.4cm}
\end{center}

\begin{preguntas}
\item Sea $A$ una matriz invertible, y sea $\lambda$ un valor propio de $A$. Demuestre que $\lambda \neq 0$ y que $\dfrac{1}{\lambda}$ es un valor propio de $A^{-1}$.
\begin{solucion}

		P.D. $\lambda \neq 0 \wedge \dfrac{1}{\lambda} \text{ es VP de }A^{-1}$
		
		Digamos que $\lambda = 0$. Luego, como $\lambda = 0$ es un VP, tenemos que
		$$det(A-\lambda I) = 0$$
		$$det(A-0 I) = 0$$
		$$det(A) = 0$$
		Pero $det(A) \neq 0$, ya que $A$ es invertible. Entonces, $\lambda \neq 0$.
		
		Notemos también que
		$$A^{-1}x 
		= A^{-1}\left(\dfrac{1}{\lambda}(\lambda x)\right)
		= A^{-1}\dfrac{1}{\lambda}(\lambda x)
		= \dfrac{1}{\lambda}A^{-1}(A x)
		= \dfrac{1}{\lambda} x$$
		$$A^{-1}x 
		= \dfrac{1}{\lambda} x$$
		De donde se desprende que $\dfrac{1}{\lambda}$ es valor propio de $A^{-1}$.
\end{solucion}
\item Sea $A$ una matriz de $2\times 2$ de rango 1 tal que $A\begin{bmatrix} 1 \\ 2\end{bmatrix} = \begin{bmatrix} 2 \\ 4 \end{bmatrix}$. ¿Es $A$ diagonalizable? Justifique.
\begin{solucion}
Notemos que
		$$A\begin{bmatrix} 1 \\ 2\end{bmatrix} = \begin{bmatrix} 2 \\ 4 \end{bmatrix}
		= A\begin{bmatrix} 1 \\ 2\end{bmatrix} = 2\begin{bmatrix} 1 \\ 2 \end{bmatrix}$$
		Luego, $\lambda_1 = 2$ es valor propio de la matriz $A$ y $v_1 = \begin{bmatrix} 1 \\ 2\end{bmatrix}$ es el vector propio asociado.
		
		Como $A$ es de rango 1, $\exists u \neq 0$ tal que $Au = $. Para ese $u$ se cumple que
		$$Au = 0 \ra Au = 0 u$$
		Luego, $\lambda_2 = 0$ es valor propio de la matriz $A$ y $v_2 = u$ es el vector propio asociado.
		
		Finalmente, como la multiplicidad algebraica es igual a la multiplicidad geométrica, la matriz es diagonalizable.
\end{solucion}
\item Diagonalice la matriz
	$$M = \begin{bmatrix}
	1 & 0 & 0\\
	1 & 1 & 2 \\
	1 & 0 & 3
	\end{bmatrix}$$
	y encuentre una matriz $N$ tal que $N^3 = M$.
\begin{solucion}
Buscamos los valores propios,
		$$det(M-\lambda I) = \left|\begin{bmatrix}
		1-\lambda & 0 & 0\\
		1 & 1-\lambda & 2 \\
		1 & 0 & 3-\lambda
		\end{bmatrix}\right|
		= (1-\lambda)(1-\lambda)(3-\lambda) = 0$$
		$$\lambda_1 = 1 \ra \text{multiplicidad 2}$$
		$$\lambda_2 = 3 \ra \text{multiplicidad 1}$$
		Buscamos ahora los vectores propios,
		\begin{itemize}
			\item $\lambda_1 = 1$
			
			$$(A-I)x = 0$$
			$$\begin{bmatrix}
			0 & 0 & 0\\
			1 & 0 & 2 \\
			1 & 0 & 2
			\end{bmatrix} \ra v_1 = \begin{pmatrix}
			2 \\ 0 \\ -1
			\end{pmatrix}, \quad
			v_2 = \begin{pmatrix}
			0 \\ 1 \\ 0
			\end{pmatrix}$$
			
			\item $\lambda_2 = 3$
			
			$$(A-3I)x = 0$$
			$$\begin{bmatrix}
			-2 & 0 & 0\\
			1 & -2 & 2 \\
			1 & 0 & 0
			\end{bmatrix} \ra v_3 = \begin{pmatrix}
			0 \\ 1 \\ 1
			\end{pmatrix}$$
		\end{itemize}
			Recordemos ahora que para diagonalizar una matriz debemos expresarla de la forma
			$$M = PDP^{-1} \ra P = \begin{bmatrix}
			v_1 & v_2 & v_3
			\end{bmatrix}, \quad
			D = \begin{bmatrix}
			\lambda_1 & 0 & 0 \\
			0 & \lambda_2 & 0 \\
			0 & 0 & \lambda_3
			\end{bmatrix}$$
			Es importante recordar que cada vector propio debe ir asociado con su respectivo valor propio (misma columna).
			
			En nuestro ejercicio, debemos considerar $\lambda_1 = \lambda_2 = 1$ y $\lambda_3 = 3$. Entonces,
			$$P = \begin{bmatrix}
			2 & 0 & 0 \\
			0 & 1 & 1 \\
			-1 & 0 & 1
			\end{bmatrix}, \quad
			D = \begin{bmatrix}
			1 & 0 & 0 \\
			0 & 1 & 0 \\
			0 & 0 & 3
			\end{bmatrix}, \quad
			P^{-1} = \begin{bmatrix}
			\frac{1}{2} & 0 & 0 \\
			-\frac{1}{2} & 1 & -1 \\
			\frac{1}{2} & 0 & 1
			\end{bmatrix}$$
			Sea
			$$R = \sqrt[3]{D} = \begin{bmatrix}
			1 & 0 & 0 \\
			0 & 1 & 0 \\
			0 & 0 & \sqrt[3]{3}
			\end{bmatrix}$$
			y sea
			$$N = PRP^{-1}$$
			Notemos que	
			$$N^3 = (PRP^{-1}) = PR\cancel{P^{-1}P}R\cancel{P^{-1}P}RP^{-1} = PR^3P^{-1} = PDP^{-1} = M$$
			Finalmente, $N = P\sqrt[3]{D}P^{-1}$ es la matriz buscada.
\end{solucion}
\item Diagonalice $A =
	\begin{bmatrix} 
	-1 & -2 & 2 \\
	0 & -1 & 0 \\
	0 & -2 & 1
	\end{bmatrix}$ y diagonalice $B = A^{10} + A - I$ 
\begin{solucion}
Buscamos los valores propios,
		$$det(A-\lambda I) = \left|\begin{bmatrix}
		-1-\lambda & -2 & 2 \\
		0 & -1-\lambda & 0 \\
		0 & -2 & 1-\lambda
		\end{bmatrix}\right|
		= (1+\lambda)^2(1-\lambda) = 0$$
		$$\lambda_1 = -1 \ra \text{multiplicidad 2}$$
		$$\lambda_2 = 1 \ra \text{multiplicidad 1}$$
		Buscamos ahora los vectores propios,
		\begin{itemize}
			\item $\lambda_1 = -1$
			
			$$(A+I)x = 0 \ra v_1 = \begin{pmatrix}
			1 \\ 0 \\ 1
			\end{pmatrix}$$
			
			\item $\lambda_2 = 1$
			
			$$(A-I)x = 0 \ra v_2 = \begin{pmatrix}
			1 \\ 0 \\ 0
			\end{pmatrix}, \quad
			v_2 = \begin{pmatrix}
			0 \\ 1 \\ 1
			\end{pmatrix}$$
		\end{itemize}
		Luego,
		$$A = PDP^{-1} \ra P = \begin{bmatrix}
		1 & 1 & 0 \\
		0 & 0 & 1 \\
		1 & 0 & 1
		\end{bmatrix}, \quad
		D = \begin{bmatrix}
		1 & 0 & 0 \\
		0 & -1 & 0 \\
		0 & 0 & -1
		\end{bmatrix}, \quad
		P^{-1} = \begin{bmatrix}
		0 & -1 & 1 \\
		1 & 1 & -1 \\
		0 & 1 & 0
		\end{bmatrix}$$
		Reemplacemos ahora en $B$,
		$$\begin{array}{lcl}
		B & = & A^{10} + A - I \\
		 & = & (PDP^{-1})^{10} + PDP^{-1} - I \\
		 & = & (PDP^{-1})^{10} + PDP^{-1} - PIP^{-1} \\
		B & = & P(D^10 + D - I)P^{-1}
		\end{array}$$
		Además, sabemos que
		$$D^{10} = \begin{bmatrix}
		1^{10} & 0 & 0 \\
		0 & (-1)^{10} & 0 \\
		0 & 0 & (-1)^{10}
		\end{bmatrix}
		 = \begin{bmatrix}
		1 & 0 & 0 \\
		0 & 1 & 0 \\
		0 & 0 & 1
		\end{bmatrix} = I$$
		Por lo tanto,
		$$B = P(I + D - I)P^{-1} = PDP^{-1}$$
		Con lo que concluimos que $A=B$, por lo que la diagonalización de $A$ también es diagonalización de $B$.
\end{solucion}
\item La matriz $\begin{bmatrix} 1 & -2 \\ 1 & 3 \end{bmatrix}$ actúa sobre $\mathbb{C}^2$. Determine los valores propios y una base para cada espacio propio de $\mathbb{C}^2$.
\begin{solucion}
En primer lugar, buscamos los valores propios,
		$$det(A-\lambda I) = \left|\begin{bmatrix} 
		1-\lambda & -2 \\ 
		1 & 3-\lambda \end{bmatrix}\right| = (1-\lambda)(3-\lambda) + 2 = \lambda^2 - 4\lambda + 5$$
		$$\lambda_{1,2} = 2 \pm i$$
		Como los valores propios son un par de complejos conjugados, los vectores propios asociados a cada uno también lo serán, por lo que basta con buscar un solo vector propio.
		
		Para $\lambda = 2+i$,
		$$(A-(2+i)\lambda)x = 0 \ra 
		\begin{bmatrix} 
		-1-i & -2 \\ 
		1 & 1-i 
		\end{bmatrix} \sim 
		\begin{bmatrix} 
		-1-i & -2 \\ 
		0 & 0
		\end{bmatrix}$$
		$$(-1-i)x_1 - 2x_2 = 0 \ra \begin{array}{lcl}
		x_1 & = & x_1 \\
		x_2 & = & \dfrac{-1-i}{2}x_1
		\end{array}$$
		Luego,
		$$v_1 = \begin{pmatrix}
		2 \\
		-1-i
		\end{pmatrix}, \quad
		v_2 = \begin{pmatrix}
		2 \\
		-1+i
		\end{pmatrix} $$
		Por último, las bases de los espacios propios están formadas por los vectores propios asociados a cada valor propio, es decir
		$$E_{2+i} = Gen\left\{\begin{pmatrix}
		2 \\
		-1-i
		\end{pmatrix}\right\}, \quad 
		E_{2-i} = Gen\left\{\begin{pmatrix}
		2 \\
		-1+i
		\end{pmatrix}\right\}$$
\end{solucion}
\item Sea $A$ una matriz de $5 \times 5$ tal que $A^t = -A$ y $q$ el polinomio dado por $q(x) = 2-x^2+4x^3$. Demuestre que $2$ es valor propio de la matriz $q(A)$.
\begin{solucion}
Sabemos que
		$$det(A) = det(A^T) \wedge det(A^T) = det(-A)$$
		$$det(A) = det(-A)$$
		$$det(A) = -det(A)$$
		$$det(A) = 0$$
		Luego, existe $u \neq 0 | Au = 0$
		
		Notemos que
		$$\begin{array}{lcl}
		q(A) & = & 2I - A^2 + 4A^3\\
		q(A) u & = & (2I - A^2 + 4A^3)u\\
		& = & 2Iu - A^2u + 4A^3u \\
		& = & 2Iu - A(Au) + 4A^2(Au) \\
		& = & 2Iu - A0 + 4A^20 \\
		& = & 2Iu \\
		q(A) u & = & 2u
		\end{array}$$
		Luego, $\lambda = 2 $ es $VP$ de $q(A)$
\end{solucion}
\end{preguntas}
\end{document}