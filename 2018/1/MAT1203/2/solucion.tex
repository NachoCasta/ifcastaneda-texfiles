\documentclass[12pt]{article}

\usepackage{fullpage}
\usepackage{graphicx}
\usepackage{amssymb}
\usepackage{amsmath}
\usepackage[none]{hyphenat}
\usepackage{parskip}
\usepackage[spanish]{babel}
\usepackage[utf8]{inputenc}
\usepackage{hyperref}
\usepackage{fancyhdr}
\usepackage{tasks}
\usepackage{mdframed}
\usepackage{xcolor}
\usepackage{pgfplots}
\usepackage[makeroom]{cancel}
\usepackage{multicol}
\usepackage[shortlabels]{enumitem}
\usepackage{tabto}

\setlength{\headheight}{10pt}
\setlength{\headsep}{10pt}
\pagestyle{fancy}
\rhead{\ayudantia \ - \alumno}

\newcommand*{\mybox}[2]{\colorbox{#1!30}{\parbox{.98\linewidth}{#2}}}

\newenvironment{solucion}
{\begin{mdframed}[backgroundcolor=black!10]
		{\bf Solución:}\\
	}
	{
	\end{mdframed}
}

\newenvironment{alternativas}[1]
{\begin{multicols}{#1}
		\begin{enumerate}[a)]
		}
		{
		\end{enumerate}
	\end{multicols}
}

\newenvironment{preguntas}
{\begin{enumerate}\itemsep12pt
	}
	{
	\end{enumerate}
}

\newcommand{\ayudantia}{{\sc Ayudantía 2}}
\newcommand{\tituloayu}{Vectores y planos en el espacio.\\Sistemas de ecuaciones lineales y formas escalonadas.}
\newcommand{\fecha}{14 de marzo de 2018}
\newcommand{\sigla}{MAT1203}
\newcommand{\nombre}{Álgebra Lineal}
\newcommand{\profesor}{Rodrigo Rubio Varas}
\newcommand{\ano}{2018}
\newcommand{\semestre}{1}
\newcommand{\mail}{mat1203@ifcastaneda.cl}
\newcommand{\alumno}{Ignacio Castañeda - \mail}

\newcommand{\ev}{\Big|}
\newcommand{\ra}{\rightarrow}
\newcommand{\lra}{\leftrightarrow}
\newcommand{\N}{\mathbb{N}}
\newcommand{\R}{\mathbb{R}}
\newcommand{\Exp}[1]{\mathcal{E}_{#1}}
\newcommand{\List}[1]{\mathcal{L}_{#1}}
\newcommand{\EN}{\Exp{\N}}
\newcommand{\LN}{\List{\N}}
\newcommand{\comment}[1]{}
\newcommand{\lb}{\\~\\}
\newcommand{\eop}{_{\square}}
\newcommand{\hsig}{\hat{\sigma}}

\begin{document}
\thispagestyle{empty}

\begin{minipage}{2cm}
	\includegraphics[width=2cm]{../../../../img/logo.pdf}
	\vspace{0.5cm}
\end{minipage}
\begin{minipage}{\linewidth}
	\begin{tabular}{lrl}
		{\scriptsize\sc Pontificia Universidad Catolica de Chile} & \hspace*{0.7in}Curso: &
		\sigla  - \nombre\\
		{\sc Facultad de Matemáticas}&
		Profesor: & \profesor \\
		{\sc Semestre \ano-\semestre} & Ayudante: & {Ignacio Castañeda}\\
		& {Mail:} & \texttt{\mail}
	\end{tabular}
\end{minipage}

\vspace{-10mm}
\begin{center}
	{\LARGE\bf \ayudantia}\\
	\vspace{0.1cm}
	{\tituloayu}\\
	\vspace{0.1cm}
	\fecha\\
	\vspace{0.4cm}
\end{center}

\begin{preguntas}
\item Determinar el plano que pasa por los puntos $P_1(4,-1,-2)$, $P_2(0,0,1)$ y $P_3(2,-3,0)$.
\begin{solucion}
Sabemos que un plano se define de la forma
		$$Ax +By + Cz = D$$
		Reemplazando esto con los tres puntos que tenemos, obtendremos el siguiente sistema de ecuaciones
		$$
		\begin{array}{rcrr}
		4A-B-2C & = & D& \vline\\
		C & = & D & \vline\\
		2A-3B & = & D &\vline\\
		\hline
		\end{array}
		$$
		$$
		\begin{array}{rcrr}
		4A-B-2C & = & C& \vline\\
		2A-3B & = & C &\vline\\
		\hline
		\end{array}
		$$
		$$
		\begin{array}{rcrr}
		4A-B-3C & = & 0& \vline\\
		2A-3B -C& = & 0 &\vline\\
		\hline
		\end{array}
		$$
		$$(1) - 2(2)$$
		$$5B - C = 0 \ra 5B = C$$
		Como hay 4 variables y teníamos 3 ecuaciones, debemos elegir el valor de una de ellas de manera arbitraria.
		$$C = 5$$
		$$\ra D = 5$$
		$$\ra B = 1$$
		$$4A -1 -15 = 0 \ra 4A=16 \ra A=4$$
		Finalmente, el plano es
		$$\Pi:4x + y + 5z = 5$$
\end{solucion}
\item Dado $P_1(0,2,-3)$ y $P_2(1,0,3)$, determinar la recta que pasa por $P_1$ y $P_2$.
\begin{solucion}
Buscamos un vector director
		$$\vec{d} = P_2 - P_1 = (1, -2, 0)$$
		Luego, la recta que pasa por $P_1$ y $P_2$ esta definida por
		$$<x,y,z> = (0, 2, -3) + \lambda (1,-2,0)$$
\end{solucion}
\item Encontrar las ecuaciones de dos planos diferentes cuya intersección sea la recta que pasa por el punto $P_1(1,3,-2)$ y $P_2(2,0,4)$.
\begin{solucion}
Para realizar esto, basta con utilizar dos trios de puntos que contengan a los dos puntos dados y buscar dos planos. La intersección entre estos planos será la recta que pasa por ambos puntos. Para esto, definiremos puntos arbitrarios.\\
		Diremos que
		$$P_3(0, 0, 0), \quad P_4(1,0,0)$$
		Partamos usando los puntos $P_1$, $P_2$ y $P_3$. Sabemos que la ecuación del plano es de la forma
		$$Ax + By + Cz = D$$
		Reemplazando con estos tres puntos, obtenemos el siguiente sistema
		$$
		\begin{array}{rcrr}
		A + 3B - 2C & = & D& \vline\\
		2A + 4C & = & D & \vline\\
		0 & = & D &\vline\\
		\hline
		\end{array}
		$$
		$$
		\begin{array}{rcrr}
		A + 3B - 2C & = & 0& \vline\\
		2A + 4C & = & 0 & \vline\\
		\hline
		\end{array}
		$$
		$$(2) + 2(1)$$
		$$4A + 6B = 0 \ra A = -\dfrac{6B}{4} \ra A = -\dfrac{3B}{2}$$
		Le pondremos un valor arbitrario a $B$,
		$$B = 4$$
		$$\ra A = -6$$
		$$\ra -12 + 4C = 0 \ra C = 3$$
		Por ende,
		$$\Pi_1: -6x + 4y + 3z = 0$$
		Ahora realizamos lo mismo con los puntos $P_1$, $P_2$, $P_4$. Sea la ecuación del plano
		$$Ax + By + Cz = D$$
		Reemplazando con estos tres puntos, obtenemos el siguiente sistema
		$$
		\begin{array}{rcrr}
		A + 3B - 2C & = & D& \vline\\
		2A + 4C & = & D & \vline\\
		A & = & D &\vline\\
		\hline
		\end{array}
		$$
		$$
		\begin{array}{rcrr}
		3B - 2C & = & 0& \vline\\
		A + 4C & = & 0 & \vline\\
		\hline
		\end{array}
		$$
		$$
		\begin{array}{rcrr}
		3B & = & 2C& \vline\\
		A & = & -4C & \vline\\
		\hline
		\end{array}
		$$
		Le daremos un valor arbitrario a $C$
		$$C = 3$$
		$$ \ra B = 2$$
		$$ \ra A = -12 $$
		$$ \ra D = -12 $$
		Con lo que obtenemos el plano
		$$\Pi_2: -12x+2y+3z = -12$$
		Finalmente, los planos buscados son:
		$$\Pi_1: -6x + 4y + 3z = 0$$		
		$$\Pi_2: -12x+2y+3z = -12$$
\end{solucion}
\item Encontrar un plano que sea perpendicular al plano cuya ecuación es $3x -7y +2z = 5$ y que pase por el punto $P(0,2,-1)$
\begin{solucion}
En primer lugar, debemos encontrar el vector normal del plano, que esta dado por los coeficientes de su ecuación, es decir
		$$\vec{n} = \begin{pmatrix}
		3\\-7\\2
		\end{pmatrix}$$
		Si los vectores normales de dos planos son perpendiculares entre si, los planos también lo serán, por lo que basta encontrar un vector perpendicular a $\vec{n}$ y utilizarlo como vector normal para nuestro nuevo plano. Para encontrar un vector perpendicular, realizaremos el producto cruz entre $\vec{n}$ y un vector arbitrario elegido por nosotros
		$$\begin{pmatrix} 3 \\ -7 \\ 2 \end{pmatrix} \times \begin{pmatrix} 1 \\ 1 \\ 1 \end{pmatrix} = \begin{pmatrix} -9 \\ -1 \\ 10 \end{pmatrix}$$
		Para encontrar el plano asociado, basta que utilicemos los coeficientes del vector como coeficientes de la ecuación del plano
		$$-9x -y +10z = D$$
		Por último, para encontrar $D$, reemplazamos el punto que nos dan en la ecuación del plano
		$$-2 -10 = D \ra D = -12$$
		Finalmente,
		$$\Pi: -9x -y +10z = -12$$
\end{solucion}
\item Determinar la ecuación de un plano que pase por $P(4,2,1)$ y que sea paralelo al plano de ecuación $2x-5y+z=6$
\begin{solucion}
De forma análoga, para que dos planos sean paralelos, basta con que sus vectores normales sean paralelos. Dicho de otra forma, lo que va a cambiar será el parametro $D$. Es decir, nuestro plano es
		$$2x - 5y + z = D$$
		Reemplazando con el punto,
		$$8-10+1 = D \ra D = -1$$
		Luego, el plano buscado es
		$$\Pi: 2x - 5y + z = -1$$
\end{solucion}
\item Escribe los siguientes sistemas de ecuaciones en notación matricial.
\begin{tasks}(2)
\task $
		\begin{array}{rcr}
		x_1 -x_2 + 3x_3 & = & 2\\
		3x_2 - 2x_3 & = & 0\\
		4x_1 +2x_2 & = & -1
		\end{array}
		$
\task $
		\begin{array}{rcr}
		3x_1 & = & 1\\
		x_1-x_2& = & 3\\
		-2x_1+x_2 & = & -8\\
		x_2 & = & 0
		\end{array}
		$
\task $
		\begin{array}{rcr}
		2x_2 -x_3 +2x_4& = & 5\\
		x_1+x_2 - x_3 & = & 2\\
		-4x_1 +x_3-x_4& = & -4
		\end{array}
		$
\task $
		\begin{array}{rcr}
		x_1 +x_2-3x_3& = & 0\\
		x_2+4x_4& = & -1\\
		x_1-2x_2+x_3-5x_4 & = & 2\\
		x_3-x_4 & = & 5
		\end{array}
		$
\end{tasks}
\begin{solucion}

\begin{enumerate}[a)]
\item $
			\begin{array}{rcr}
			x_1 -x_2 + 3x_3 & = & 2\\
			3x_2 - 2x_3 & = & 0\\
			4x_1 +2x_2 & = & -1
			\end{array}
			\Longrightarrow 
			\left[
			\begin{array}{ccc|c}
				1 & -1 & 3 & 2\\
				0 & 3 & -2 & 0\\
				4 & 2 & 0 & 1
			\end{array}
			\right]
			$
\item $
			\begin{array}{rcr}
			3x_1 & = & 1\\
			x_1-x_2& = & 3\\
			-2x_1+x_2 & = & -8\\
			x_2 & = & 0
			\end{array}
			\Longrightarrow 
			\left[
			\begin{array}{cc|c}
			3 & 0 & 1\\
			1 &-1 & 3\\
			-2 & 1 & -8\\
			0 & 1 & 0
			\end{array}
			\right]
			$
\item $
			\begin{array}{rcr}
			2x_2 -x_3 +2x_4& = & 5\\
			x_1+x_2 - x_3 & = & 2\\
			-4x_1 +x_3-x_4& = & -4
			\end{array}
			\Longrightarrow 
			\left[
			\begin{array}{cccc|c}
			0 &2 &-1&2&5\\
			1 &1 & -1 & 0 &2\\
			-4 &0 & 1 & -1 &-4
			\end{array}
			\right]
			$
\item $
			\begin{array}{rcr}
			x_1 +x_2-3x_3& = & 0\\
			x_2+4x_4& = & -1\\
			x_1-2x_2+x_3-5x_4 & = & 2\\
			x_3-x_4 & = & 5
			\end{array}
			\Longrightarrow 
			\left[
			\begin{array}{cccc|c}
			1 & 1 & -3 & 0 & 0\\
			0 & 1 & 0 & 4 & -1\\
			1 & -2 & 1 & -5 & 2\\
			0 & 0 & 1 & -1 & 5
			\end{array}
			\right]
			$
\end{enumerate}
\end{solucion}
\item Dadas las siguientes matrices aumentadas, escribir el sistema de ecuaciones asociado.
\begin{tasks}(2)
\task $
		\begin{bmatrix}
		-4 & 2 & 1 &  0\\
		2  & 1 & 2 & -1\\
		-3 & 5 &-6& -2
		\end{bmatrix}
		$
\task $
		\begin{bmatrix}
		2 & 1 & -3 & 0\\
		1  & -2 & 0 & 12\\
		4 & 2 &1& 8
		\end{bmatrix}
		$
\end{tasks}
\begin{solucion}

\begin{enumerate}[a)]
\item $
			\begin{bmatrix}
			-4 & 2 & 1 &  0\\
			2  & 1 & 2 & -1\\
			-3 & 5 &-6& -2
			\end{bmatrix}
			\Longrightarrow
			\begin{array}{rcr}
			-4x_1 + 2x_2 + x_3& = & 0\\
			2x_1 + x_2 + 2x_3& = & -1\\
			-3x_1 + 5x_2 - 6x_3 & = & -2
			\end{array}
			$
\item $
			\begin{bmatrix}
			2 & 1 & -3 & 0\\
			1  & -2 & 0 & 12\\
			4 & 2 &1& 8
			\end{bmatrix}
			\Longrightarrow
			\begin{array}{rcr}
			2x_1 + x_2 - 3x_3& = & 0\\
			x_1-2x_2& = & 12\\
			4x_1+2x_2+x_3 & = & 8
			\end{array}
			$
\end{enumerate}
\end{solucion}
\item Determinar si el siguiente sistema es consistente o no
	$$
	\begin{array}{rcr}
	x_1 -6x_2& = & 5\\
	x_2-4x_3+x_4& = & 0\\
	-x_1+6x_2+x_3+5x_4& = & 3\\
	-x_2+5x_3+4x_4 & = & 0
	\end{array}
	$$
\begin{solucion}
En primer lugar, escribiremos el sistema en notación matricial
		$$
		\left[
		\begin{array}{cccc|c}
		1 & -6 & 0 & 0 & 5\\
		0 & 1 & -4 & 1 & 0\\
		-1& 6 & 1 & 5 & 3 \\
		0 & -1 & 5 & 4 &0
		\end{array}
		\right]$$
		A continuación, utilizando operaciones de fila, llevaremos esta matriz a su forma escalonada
		$$\left[
		\begin{array}{cccc|c}
			1 & -6 & 0 & 0 & 5\\
			0 & 1 & -4 & 1 & 0\\
			-1& 6 & 1 & 5 & 3 \\
			0 & -1 & 5 & 4 &0
		\end{array}
		\right] \stackbin[F_4+F_2]{F_3 +F_1}{\wsim}
		\left[
		\begin{array}{cccc|c}
		1 & -6 & 0 & 0 & 5\\
		0 & 1 & -4 & 1 & 0\\
		0 & 0 & 1 & 5 & 8 \\
		0 & 0 & 1 & 5 & 0
		\end{array}
		\right] \stackbin[]{F_4-F_3}{\wsim}
		\left[
		\begin{array}{cccc|c}
		1 & -6 & 0 & 0 & 5\\
		0 & 1 & -4 & 1 & 0\\
		0 & 0 & 1 & 5 & 8 \\
		0 & 0 & 0 & 0 & -8
		\end{array}
		\right] $$
		Aqui podemos apreciar que en el la última fila, todas las variables tienen coeficiente sero y la columna de coeficientes aumentados tiene valor, lo que no es posible, ya que esto significaría que se cumple la igualdad $0 = -8$, lo que no es cierto. Dicho esto, el sistema no es consistente.
\end{solucion}
\end{preguntas}
\end{document}