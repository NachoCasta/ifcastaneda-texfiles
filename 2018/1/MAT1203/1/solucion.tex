\documentclass[12pt]{article}

\usepackage{fullpage}
\usepackage{graphicx}
\usepackage{amssymb}
\usepackage{amsmath}
\usepackage[none]{hyphenat}
\usepackage{parskip}
\usepackage[spanish]{babel}
\usepackage[utf8]{inputenc}
\usepackage{hyperref}
\usepackage{fancyhdr}
\usepackage{tasks}
\usepackage{mdframed}
\usepackage{xcolor}
\usepackage{pgfplots}
\usepackage[makeroom]{cancel}
\usepackage{multicol}
\usepackage[shortlabels]{enumitem}
\usepackage{tabto}

\setlength{\headheight}{10pt}
\setlength{\headsep}{10pt}
\pagestyle{fancy}
\rhead{\ayudantia \ - \alumno}

\newcommand*{\mybox}[2]{\colorbox{#1!30}{\parbox{.98\linewidth}{#2}}}

\newenvironment{solucion}
{\begin{mdframed}[backgroundcolor=black!10]
		{\bf Solución:}\\
	}
	{
	\end{mdframed}
}

\newenvironment{alternativas}[1]
{\begin{multicols}{#1}
		\begin{enumerate}[a)]
		}
		{
		\end{enumerate}
	\end{multicols}
}

\newenvironment{preguntas}
{\begin{enumerate}\itemsep12pt
	}
	{
	\end{enumerate}
}

\newcommand{\ayudantia}{{\sc Ayudantía 1}}
\newcommand{\tituloayu}{Vectores, producto punto y producto cruz}
\newcommand{\fecha}{15 de marzo de 2018}
\newcommand{\sigla}{MAT1203}
\newcommand{\nombre}{Álgebra Lineal}
\newcommand{\profesor}{Rodrigo Rubio Varas}
\newcommand{\ano}{2018}
\newcommand{\semestre}{1}
\newcommand{\mail}{mat1203@ifcastaneda.cl}
\newcommand{\alumno}{Ignacio Castañeda - \mail}

\newcommand{\ev}{\Big|}
\newcommand{\ra}{\rightarrow}
\newcommand{\lra}{\leftrightarrow}
\newcommand{\N}{\mathbb{N}}
\newcommand{\R}{\mathbb{R}}
\newcommand{\Exp}[1]{\mathcal{E}_{#1}}
\newcommand{\List}[1]{\mathcal{L}_{#1}}
\newcommand{\EN}{\Exp{\N}}
\newcommand{\LN}{\List{\N}}
\newcommand{\comment}[1]{}
\newcommand{\lb}{\\~\\}
\newcommand{\eop}{_{\square}}
\newcommand{\hsig}{\hat{\sigma}}

\begin{document}
\thispagestyle{empty}

\begin{minipage}{2cm}
	\includegraphics[width=2cm]{../../../../img/logo.pdf}
	\vspace{0.5cm}
\end{minipage}
\begin{minipage}{\linewidth}
	\begin{tabular}{lrl}
		{\scriptsize\sc Pontificia Universidad Catolica de Chile} & \hspace*{0.7in}Curso: &
		\sigla  - \nombre\\
		{\sc Facultad de Matemáticas}&
		Profesor: & \profesor \\
		{\sc Semestre \ano-\semestre} & Ayudante: & {Ignacio Castañeda}\\
		& {Mail:} & \texttt{\mail}
	\end{tabular}
\end{minipage}

\vspace{-10mm}
\begin{center}
	{\LARGE\bf \ayudantia}\\
	\vspace{0.1cm}
	{\tituloayu}\\
	\vspace{0.1cm}
	\fecha\\
	\vspace{0.4cm}
\end{center}

\begin{preguntas}
\item Realiza las siguientes operaciones con los vectores dados
	$$
	v_1 = \begin{pmatrix}
	1\\
	7\\
	8
\end{pmatrix};\qquad
	v_2 = \begin{pmatrix}
	-4\\
	2\\
	-3
\end{pmatrix}; \qquad
	v_3 = \begin{pmatrix}
	0\\
	1\\
	3
\end{pmatrix}; \qquad
	v_4 = \begin{pmatrix}
	13\\
	-3\\
	1
\end{pmatrix},
	 $$
\begin{tasks}(4)
\task $v_1 + v_2$
\task $v_3 - v_4$
\task $v_2 \cdot v_3$
\task $v_1 \times v_2$
\task $v_2 \times v_1$
\task $(v_1 \times v_2) \cdot v_2$
\task $3 - ((2v_1) \cdot v_2)$
\task $v_2 - v_3 \times v_1$
\end{tasks}
\begin{solucion}

\begin{enumerate}[a)]
\item $v_1 + v_2 =\begin{pmatrix}
				1\\
				7\\
				8
				\end{pmatrix} + \begin{pmatrix}
				-4\\
				2\\
				-3
				\end{pmatrix} = \begin{pmatrix}
				-3\\
				9\\
				5
				\end{pmatrix}$
\item $v_3 - v_4 = \begin{pmatrix}
				0\\
				1\\
				3
				\end{pmatrix} - \begin{pmatrix}
				13\\
				-3\\
				1
				\end{pmatrix} =  \begin{pmatrix}
				-13\\
				4\\
				2
				\end{pmatrix}$
\item $v_2 \cdot v_3 = \begin{pmatrix}
				-4\\
				2\\
				-3
				\end{pmatrix} \cdot \begin{pmatrix}
				0\\
				1\\
				3
				\end{pmatrix} = -4 \cdot 0 + 2 \cdot 1 -3 \cdot 3 = -7$
\item $v_1 \times v_2 = \begin{pmatrix}
				1\\
				7\\
				8
			\end{pmatrix} \times \begin{pmatrix}
				-4\\
				2\\
				-3
			\end{pmatrix} = i \left| \begin{matrix} 7 & 8 \\ 2 & -3\end{matrix} \right| - j \left| \begin{matrix} 1 & 8 \\ 4 & -3\end{matrix} \right| + k \left| \begin{matrix} 1 & 7 \\ -4 & 2\end{matrix} \right|$\\
			$$= i (7 \cdot (-3) - 8 \cdot 2) - j (1 \cdot (-3) - 8 \cdot (-4)) + k (1 \cdot 2 - 7 \cdot(-4))$$
			$$= i (-21 -16)) - j(-3--32) + k(2 - - 28)$$
			$$-37 i - 29j + 30k = \begin{pmatrix}
			-37\\
			-29\\
			30
			\end{pmatrix}$$	
\item $v_2 \times v_1 =  \begin{pmatrix}
			-4\\
			2\\
			-3
			\end{pmatrix} \times \begin{pmatrix}
			1\\
			7\\
			8
			\end{pmatrix} = i \left| \begin{matrix} 2 & -3 \\ 7 & 8\end{matrix} \right| - j \left| \begin{matrix} 4 & -3 \\ 1 & 8\end{matrix} \right| + k \left| \begin{matrix} -4 & 2 \\ 1 & 7\end{matrix} \right|$\\
			$$= i (2 \cdot 8 - (-3) \cdot 7) - j ((-4)\cdot 8 - (-3)\cdot 1) + k ((-4) \cdot 7 -2 \cdot 1)$$
			$$= i (16 --21) + j(-32 --3) + k(-28-2)$$
			$$37 i + 29j - 30k = \begin{pmatrix}
			37\\
			29\\
			-30
			\end{pmatrix}$$	
\item $(v_1 \times v_2) \cdot v_2 = \begin{pmatrix}
			-37\\
			-29\\
			30
			\end{pmatrix} \cdot \begin{pmatrix}
			-4\\
			2\\
			-3
			\end{pmatrix} = (-37) \cdot (-4) + (-29) \cdot 2 + 30 \cdot (-3)$
			$$=148 +-58 - 90 = 0$$
\item $3 - ((2v_1) \cdot v_2) = 3 - \left(\left(2\begin{pmatrix}
			1\\
			7\\
			8
			\end{pmatrix}\right) \cdot \begin{pmatrix}
			-4\\
			2\\
			-3
			\end{pmatrix}\right) = 3 - \left(\begin{pmatrix}
			2\\
			14\\
			16
			\end{pmatrix} \cdot \begin{pmatrix}
			-4\\
			2\\
			-3
			\end{pmatrix}\right)$
			$$ = 3 - (2 \cdot (-4) + 14 \cdot 2 + 16 \cdot (-3))$$
			$$3 - (-8 + 28 -48)$$
\item $v_2 - v_3 \times v_1 = \begin{pmatrix}
			-4\\
			2\\
			-3
			\end{pmatrix} -  \begin{pmatrix}
			0\\
			1\\
			3
			\end{pmatrix} \times  \begin{pmatrix}
			1\\
			7\\
			8
			\end{pmatrix}$
			$$= -4i + 2j -3k - (i (8 - 21))- j(0 - 3) + k (0-1))$$
			$$ = -4i + 2j -3k +13i - 3j +k = 9i - j -2k $$
			$$ =\begin{pmatrix}
			9\\
			-1\\
			-2
			\end{pmatrix}$$
\end{enumerate}
\end{solucion}
\item Verificar si los siguientes puntos son o no colineares entre si
\begin{tasks}(2)
\task $ 
\begin{pmatrix}
	3\\
	2\\
	5
\end{pmatrix}; \quad
\begin{pmatrix}
	0\\
	1/2\\
	-1
\end{pmatrix}; \quad
\begin{pmatrix}
	5\\
	3\\
	9
\end{pmatrix}$
\task $ 
\begin{pmatrix}
	1\\
	1\\
	4
\end{pmatrix}; \quad
\begin{pmatrix}
	-2\\
	-3\\
	-8
\end{pmatrix}; \quad
\begin{pmatrix}
	4\\
	4\\
	16
\end{pmatrix}$
\end{tasks}
\begin{solucion}

\begin{enumerate}[a)]
\item $ 
			\begin{pmatrix}
			3\\
			2\\
			5
			\end{pmatrix}; \quad
			\begin{pmatrix}
			0\\
			1/2\\
			-1
			\end{pmatrix}; \quad
			\begin{pmatrix}
			5\\
			3\\
			9
			\end{pmatrix}$\\
			En primer lugar, debemos encontrar dos vectores directores entre dos pares de vectores distintos
			$$d_1 = \begin{pmatrix}
			3\\
			2\\
			5
			\end{pmatrix} - 
			\begin{pmatrix}
			0\\
			1/2\\
			-1
			\end{pmatrix} = \begin{pmatrix}
			3\\
			3/2\\
			6
			\end{pmatrix}$$
			$$d_2 = \begin{pmatrix}
			3\\
			2\\
			5
			\end{pmatrix} - 
			\begin{pmatrix}
			5\\
			3\\
			9
			\end{pmatrix} = \begin{pmatrix}
			-2\\
			-1\\
			-4
			\end{pmatrix}$$
			Dado que es posible representar $d_1$ de la forma $d_1 = \lambda d_2$, con $\lambda = -1,5$, los puntos son colineales.
\item $ 
			\begin{pmatrix}
			1\\
			1\\
			4
			\end{pmatrix}; \quad
			\begin{pmatrix}
			-2\\
			-3\\
			-8
			\end{pmatrix}; \quad
			\begin{pmatrix}
			4\\
			4\\
			16
			\end{pmatrix}$\\
			Igual que antes, buscaremos dos vectores directores
			$$d_1 = \begin{pmatrix}
			1\\
			1\\
			4
			\end{pmatrix} - 
			\begin{pmatrix}
			-2\\
			-3\\
			-8
			\end{pmatrix} = \begin{pmatrix}
			3\\
			4\\
			12
			\end{pmatrix}$$
			$$d_2 = \begin{pmatrix}
			1\\
			1\\
			4
			\end{pmatrix} - 
			\begin{pmatrix}
			4\\
			4\\
			16
			\end{pmatrix} = \begin{pmatrix}
			-3\\
			-3\\
			-12
			\end{pmatrix}$$
			Observando las dos primeras componentes de cada vector, vemos que es imposible representar uno en función del otro, por lo que los puntos no son colineales
\end{enumerate}
\end{solucion}
\item Sean los vectores
		$$
	v_1 = \begin{pmatrix}
	2\\
	1\\
	5
\end{pmatrix};\qquad
	v_2 = \begin{pmatrix}
	-1\\
	-3\\
	0
\end{pmatrix}$$
\begin{enumerate}[a)]
\item Buscar un vector $v_3$ que sea perpendicular a $v_1$ y $v_2$ y verificar que efectivamente sea perpendicular a ambos.
\item Buscar un vector paralelo a $(v_2-v_3)$ y verificar que lo sea
\item Encontrar un vector perpendicular a $(v_3 + 3v_1)$
\end{enumerate}
\begin{solucion}

\begin{enumerate}[a)]
\item Para que sea perpendicular a ambos, basta definir $v_3 = v_1 \times v_2$, es decir
			$$v_3 = \begin{pmatrix}
			2\\
			1\\
			5
			\end{pmatrix} \times \begin{pmatrix}
			-1\\
			-3\\
			0
			\end{pmatrix} = i (0 + 15) + j (0 + 5) + k(-6 + 1) =\begin{pmatrix}
			15\\
			-5\\
			-5
			\end{pmatrix} $$
			Para verificar basta utilizar el producto punto, donde obtendremos que
			$$v_3 \cdot v_1 = 0$$
			$$v_3 \cdot v_2 = 0$$
\item En primer lugar, calculemo el vector $v_2 - v_3$
			$$v_4 = v_2 - v_3 = \begin{pmatrix}
			-1\\
			-3\\
			0
			\end{pmatrix} - \begin{pmatrix}
			15\\
			-5\\
			-5
			\end{pmatrix} = \begin{pmatrix}
			-16\\
			2\\
			5
			\end{pmatrix}$$
			Para obtener un vector paralelo, podemos simplemente ponderar el vector por un escalar, como por ejemplo, 2
			$$v_5 = 2 v_4 = 2 \begin{pmatrix}
			-16\\
			2\\
			5
			\end{pmatrix} = \begin{pmatrix}
			-32\\
			4\\
			10
			\end{pmatrix}$$
			Para verificar utilizaremos el producto cruz, donde obtendremos que
			$$v_5 \times v_4 = 0$$
\item Primero calculemos el vector $v_3 + 3v_1$
			$$v_6 = v_3 + 3v_1 = \begin{pmatrix}
			15\\
			-5\\
			-5
			\end{pmatrix}  + 3 \begin{pmatrix}
			2\\
			1\\
			5
			\end{pmatrix} = \begin{pmatrix}
			21\\
			-2\\
			10
			\end{pmatrix} $$
			Para obtener un vector perpendicular, podemos hacer producto cruz con cualquier otro vector arbitrario. Para hacer los calculos simples, utilizaremos
			$$v_7 = \begin{pmatrix}
			1\\
			0\\
			0
			\end{pmatrix} $$
			Luego, 
			$$v_8 = v_6 \times v_7 = \begin{pmatrix}
			21\\
			-2\\
			10
			\end{pmatrix} \times \begin{pmatrix}
			1\\
			0\\
			0
			\end{pmatrix} = 10j + 2k = \begin{pmatrix}
			0\\
			-10\\
			2
			\end{pmatrix}$$
\end{enumerate}
\end{solucion}
\item Detereminar si los siguientes vectores son paralelos, perpendiculares u oblicuos
\begin{tasks}(2)
\task $$ 
			\begin{pmatrix}
			-1\\
			2\\
			5
		\end{pmatrix};\qquad
			\begin{pmatrix}
			2\\
			-4\\
			-10
		\end{pmatrix}$$
\task  $$ 
			\begin{pmatrix}
			2\\
			5\\
			10
		\end{pmatrix};\qquad
			\begin{pmatrix}
			-3\\
			8\\
			-11
		\end{pmatrix}$$
\end{tasks}
\begin{solucion}

\begin{enumerate}[a)]
\item $$ 
			\begin{pmatrix}
			-1\\
			2\\
			5
			\end{pmatrix};\qquad
			\begin{pmatrix}
			2\\
			-4\\
			-10
			\end{pmatrix}$$
			Si se fijan, es evidente que $v_2 = -2v_1$, por lo que comprobaremos de inmediato con el producto cruz para ver si son paralelos
			 $$ 
			\begin{pmatrix}
			-1\\
			2\\
			5
			\end{pmatrix} \times 
			\begin{pmatrix}
			2\\
			-4\\
			-10
			\end{pmatrix} = i (-20 + 20) -j (10 - 10) + k (4 - 4) = \vec{0}$$
			Por lo que ambos vectores son paralelos
\item $$ 
			\begin{pmatrix}
			2\\
			5\\
			10
			\end{pmatrix};\qquad
			\begin{pmatrix}
			-3\\
			8\\
			-11
			\end{pmatrix}$$
			Comenzaremo probando si son perpendiculares con el producto punto
			$$ 
			\begin{pmatrix}
			2\\
			5\\
			10
			\end{pmatrix} \cdot
			\begin{pmatrix}
			-3\\
			8\\
			-11
			\end{pmatrix} = -6 + 40 - 110 = -76$$
			Como es distinto de cero, vemos que no son perpendiculares.\\
			Ahora, utilizaremos el producto cruz para ver si son paralelos
			$$ 
			\begin{pmatrix}
			2\\
			5\\
			10
			\end{pmatrix} \times
			\begin{pmatrix}
			-3\\
			8\\
			-11
			\end{pmatrix} = i(-55 -80) - j (-22 +30) +k(16 + 15) = \begin{pmatrix}
			-135\\
			-8\\
			31
			\end{pmatrix}$$
			Como este no es el vector nulo, tampoco son paralelos. Con esto concluimos que ambos vectores son oblicuos.
\end{enumerate}
\end{solucion}
\end{preguntas}
\end{document}